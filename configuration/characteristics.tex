% Auteur de la thèse : prénom (1er argument obligatoire), nom (2e argument
% obligatoire) et éventuel courriel (argument optionnel). Les éventuels accents
% devront figurer et le nom /ne/ doit /pas/ être saisi en capitales
\author[mailto:adrien.ehrhardt@centraliens-lille.org]{Adrien}{Ehrhardt}
%
% Titre de la thèse dans la langue principale (argument obligatoire) et dans la
% langue secondaire (argument optionnel)
\title[Predictive models for big and biased data]{Modèles prédictifs pour données volumineuses et biaisées}
%
% (Facultatif) Sous-titre de la thèse dans la langue principale (argument
% obligatoire) et dans la langue secondaire (argument optionnel)
\subtitle[Application to Credit Scoring]{Application à l'amélioration du scoring en risque de crédit}
%
% Champ disciplinaire dans la langue principale (argument obligatoire) et dans
% la langue secondaire (argument optionnel)
\academicfield[Applied Mathematics]{Mathématiques et leurs interactions}
%
% (Facultatif) Spécialité dans la langue principale (argument obligatoire) et
% dans la langue secondaire (argument optionnel)
\speciality[Statistics]{Statistique}
%
% Date de la soutenance, au format {jour}{mois}{année} donnés sous forme de
% nombres
\date{}{}{}
%
% (Facultatif) Date de la soumission, au format {jour}{mois}{année} donnés sous
% forme de nombres
%\submissiondate{}{}{}
%
% (Facultatif) Sujet pour les méta-données du PDF
\subject[]{}
%
% (Facultatif) Nom (argument obligatoire) de la ComUE
\pres[logo=logos/logo_comue.png,url=http://www.cue-lillenorddefrance.fr/]{Lille Nord de France}
%
% Nom (argument obligatoire) de l'institut (principal en cas de cotutelle)
\institute[logo=logos/logo_lille1.png,url=https://www.univ-lille.fr/]{Université de Lille}

\company[logo=logos/logo-cacf-inria]{Crédit Agricole Consumer Finance - Inria Lille-Nord Europe}

%\company[logo=logos/logo-inria]{Inria Lille-Nord Europe}

%
% (Facultatif) En cas de cotutelle (normalement, seulement dans le cas de
% cotutelle internationale), nom (argument obligatoire) du second institut
% \coinstitute[logo=]{}
%
% (Facultatif) Nom (argument obligatoire) de l'école doctorale
\doctoralschool[url=http://edspi.univ-lille1.fr/]{Sciences pour l'Ingénieur}
%
% Nom (1er argument obligatoire) et adresse (2e argument obligatoire) du
% laboratoire (ou de l'unité) où la thèse a été préparée, à utiliser /autant de
% fois que nécessaire/
\laboratory[
logo=logos/logo-inria.jpg,
telephone=+33 (0)3 59 57 78 00,
fax=+33 (0)3 59 57 78 50,
email=contact-lille@inria.fr,
url=https://www.inria.fr/centre/lille
]{Équipe-projet M$\Theta$DAL}{%
Inria Lille Nord-Europe \\
40 Avenue Halley  \\
59650 Villeneuve-d'Ascq}
\laboratory[
logo=logos/logo-painleve.png,
telephone=(+33) 03 20 43 48 50,
url=https://math.univ-lille1.fr/
]{Laboratoire Paul Painlevé}{%
  CNRS U.M.R. 8524\\
  59655 Villeneuve d'Ascq Cedex\\
  France}
  
%
% Directeur(s) de thèse et membres du jury, saisis au moyen des commandes
% \supervisor, \cosupervisor, \comonitor, \referee, \committeepresident,
% \examiner, \guest, à utiliser /autant de fois que nécessaire/ et /seulement
% si nécessaire/. Toutes basées sur le même modèle, ces commandes ont
% 2 arguments obligatoires, successivement les prénom et nom de chaque
% personne. Si besoin est, on peut apporter certaines précisions en argument
% optionnel, essentiellement au moyen des clés suivantes :
% - « professor », « seniorresearcher », « associateprofessor »,
%   « associateprofessor* », « juniorresearcher », « juniorresearcher* » (qui
%   peuvent ne pas prendre de valeur) pour stipuler le corps auquel appartient
%   la personne ;
% - « affiliation » pour stipuler l'institut auquel est affiliée la personne ;
% - « female » pour stipuler que la personne est une femme pour que certains
%   mots clés soient accordés en genre.
%
\supervisor[professor,affiliation=Université de Lille]{Christophe}{Biernacki}
\comonitor[associateprofessor,affiliation=Université de Lille]{Philippe}{Heinrich}
\comonitor[associateprofessor,affiliation=Université de Lille]{Vincent}{Vandewalle}
% \comonitor[,affiliation=]{}{}
\referee[,affiliation=]{}{}
\referee[,affiliation=]{}{}
\committeepresident[,affiliation=]{}{}
\examiner[,affiliation=]{}{}
\examiner[,affiliation=]{}{}
\examiner[,affiliation=]{}{}
\guest[Crédit Agricole Consumer Finance]{Jérôme}{Beclin}
%
% (Facultatif) Mention du numéro d'ordre de la thèse (s'il est connu, ce numéro
% est à spécifier en argument optionnel)
% \ordernumber[]
%
% Préparation des mots clés dans la langue principale (1er argument) et dans la
% langue secondaire (2e argument)
%%%%%%%%%%%%%%%%%%%%%%%%%%%%%%%%%%%%%%%%%%%%%%%%%%%%%%%%%%%%%%%%%%%%%%%%%%%%%%%
\keywords{scoring, risque, crédit, prédiction, discrétisation, segmentation}{scoring, credit, risk, predictin, discretization, clustering}

