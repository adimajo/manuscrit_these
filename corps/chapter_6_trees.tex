\chapter{Tree-structure segmentation for logistic regression} \label{chap6}

\minitoc

\textcolor{red}{needs quote}

\textit{Nota Bene :} Ce chapitre s'inspire fortement ... \textcolor{red}{à adapter au moment de l'envoi du manuscrit}

\selectlanguage{english}




\section{Introduction}

\subsection{Context}

En matières de crédit à la consommation, les instituts financiers cherchent à automatiser la d´ecision de financement tout en ne s´electionnant que les clients susceptibles
de rembourser ledit cr´edit. Depuis une quarantaine d’ann´ees, le Credit Scoring consiste `a construire des mod`eles
de classification supervis´ee $p_{\glssymbol{bth}}$ `a partir des donn´ees demand´ees au clients x = (xj )
d
1
et de l’observation du
remboursement des clients pass´es $y \in \{0, 1\}$. Historiquement, des scores diff´erents sont d´evelopp´es sur des
march´es (e.g. grande distribution, ´electrom´enager, . . . ) et/ou des produits (e.g. renouvelable, amortissable, . . . )
et/ou des partenaires et/ou des profils clients diff´erents dans l’esprit de la figure 1. Ce d´ecoupage est historique
et rel`eve d’un a priori. On cherche ici `a rationnaliser cette pratique en consid´erant le cluster d’appartenance du
client comme un param`etre `a optimiser. Si l’on note K le nombre de scores `a construire (inconnu) et c = 1..K
chaque score, correspondant `a un cluster de clients, le m´elange de r´egressions logistiques s’´ecrit :
$p(y|x) = \sum_{c=1}^K p_{\glssymbol{bth}_c}(y|x, c)p(c|x)$,
o`u l’on restreint p(c|x) `a prendre la forme de la figure 1, de telle sorte que le m´elange n’est pas “flou” comme
pour un mod`ele de m´elange classique o`u la contribution de chaque classe est pond´er´ee par sa probabilit´e. La
difficult´e d’une approche directe r´eside dans cette contrainte discr`ete.


\tikzstyle{level 1}=[level distance=1.5cm, sibling distance=7cm]
\tikzstyle{level 2}=[level distance=1.5cm, sibling distance=4cm]
\tikzstyle{level 3}=[level distance=2cm, sibling distance=2cm]

\begin{figure}
\resizebox{\textwidth}{!}{
\centering
\begin{tikzpicture}
  [
    sibling distance        = 15em,
    level distance          = 5em,
    edge from parent/.style = {draw, -latex},
    every node/.style       = {font=\footnotesize},
    sloped
  ]
  \node [root] {\textcolor{black}{Clientèle}}
    child { node [dummy] {}
      child { node [dummy] {}
        child { node [env] {\textcolor{black}{$p_{\theta_1}(y|x)$}}
          edge from parent node [below] {Retraités} }
        child { node [env] {\textcolor{black}{$p_{\theta_2}(y|x)$}}
          edge from parent node [above] {Salariés} }
        child { node [env] {\textcolor{black}{$p_{\theta_3}(y|x)$}}
                edge from parent node [above] {Autres} }
        edge from parent node [above] {Crédit renouvelable} }
      child { node [env] {\textcolor{black}{$p_{\theta_4}(y|x)$}}
              edge from parent node [above, align=center]
                {Amortissable} }
              edge from parent node [above] {Electroménager} }
    child { node [dummy] {}
      child { node [dummy] {}
        child { node [env] {\textcolor{black}{$p_{\theta_5}(y|x)$}}
          edge from parent node [above] {Location} }
        child { node [env] {\textcolor{black}{$p_{\theta_6}(y|x)$}}
                edge from parent node [above] {Amortissable} }
        edge from parent node [above] {Fiat} }
      child { node [env] {\textcolor{black}{$p_{\theta_7}(y|x)$}}
              edge from parent node [above, align=center]
                {Kawasaki} }
              edge from parent node [above] {Automobile} };
\end{tikzpicture}
}
\caption{Cartographie simplifiée de la chaîne de construction des scores.} 
\label{fig:arbre}
\end{figure}




\subsection{In-house \textit{ad hoc} practice}





\section{Litterature review}


\subsection{Clustering methods}


\subsection{Direct approaches: logistic regression trees}



\section{Logistic regression trees as a combinatorial model selection problem}


\subsection{Cardinality example}


\subsection{Logistic regression tree selection}



\section{A mixture and latent feature-based relaxation}




\section{A stochastic estimation strategy}


\section{Numerical experiments}


\subsection{Empirical consistency on simulated data}

\subsection{Benchmark on \textit{Credit Scoring} data}




\bigskip

Ce chapitre.


\printbibliography[heading=subbibliography, title=References of Chapter 5]

