\chapter*{\nmu Introduction} \label{chap_intro}
% français

\epigraph{Anyone who has ever struggled with poverty knows how extremely expensive it is to be poor.}{James A. Baldwin}

Les cas d'application des travaux de ce manuscrit portent sur plusieurs problèmes connexes au \textit{Credit Scoring}.

Pour un particulier, le recours au crédit, c'est-à-dire à l'emprunt d'argent en échange d'une promesse de remboursement étalé dans le temps et assorti d'un intérêt, est possible depuis très longtemps, les plus anciennes traces ‘‘modernes'' de crédits bancaires se situant au XII$^\text{ème}$ siècle en Italie~\cite{thomas_wards_1828}. De nos jours, l'emprunt immobilier ou automobile, c'est-à-dire pour financer un lieu de résidence ou l'achat d'un véhicule, est répandu~\cite{la_tribune_2010}. Par opposition au crédit immobilier, on parle souvent de crédit à la consommation pour désigner le financement de biens et de services : automobile, électroménager, travaux, etc. De manière plus formelle, le crédit à la consommation est définie dans la loi \textnumero 2010-737 du 1$^\text{er}$ juillet 2010~\cite{noauthor_loi_2010} comme une :
\begin{displayquote}
Opération ou contrat de crédit, une opération ou un contrat par lequel un prêteur consent ou s’engage à consentir à l’emprunteur un crédit sous la forme d’un délai de paiement, d’un prêt, y compris sous forme de découvert ou de toute autre facilité de paiement similaire, à l’exception des contrats conclus en vue de la fourniture d’une prestation continue ou à exécution successive de services ou de biens de même nature et aux termes desquels l’emprunteur en règle le coût par paiements échelonnés pendant toute la durée de la fourniture.
\end{displayquote}

De nombreux acteurs bancaires proposent des crédits à la consommation, si bien qu'en 2017 environ 27,2 \% des ménages ont un crédit à la consommation~\cite{observatoire}. \glsfirst{cacf}, à l'origine de cette thèse CIFRE, est un acteur majeur du crédit à la consommation, à travers une marque spécialisée en France, Sofinco, et des partenaires distributeurs de crédit conso.

Parmi l'ensemble des demandeurs de crédit à la consommation, il est souhaitable, à plusieurs égards, de ne pas financer tous les crédits. Premièrement, si tant est que l'on puisse prêter un rôle sociétal à une entité bancaire, il paraît responsable de ne pas détériorer voire mettre en danger la santé financière de l'emprunteur. Pour ce faire, des contrôles automatiques permettent de refuser la clientèle dite fragile : taux d'endettement trop élevé, fichage bancaire pour incidents de paiements, \ldots Par ailleurs, d'un point de vue économique cette fois, un client se trouvant dans l'incapacité de rembourser le crédit souscrit sera vraisemblablement peu ou pas profitable pour l'institution financière du fait des coûts de traitements et de personnels de relance et procédure(s) judiciaire(s) qui peuvent aboutir à une annulation totale ou partielle de la dette du client engendrant une perte sèche pour l'organisme prêteur.

Dans ce cadre, le score vise à évaluer la propension d'un client à être ‘‘bon'' ou ‘‘mauvais'', selon des critères à définir ultérieurement, pour ainsi prendre une décision de financement ou de rejet de façon quantitative et objective. On donnera dans le chapitre~\ref{chap1} quelques éléments de contexte supplémentaires nécessaires à la bonne compréhension des cas d'application de cette thèse, un état de l'art de la pratique industrielle ainsi qu'un état de l'art académique des techniques d'apprentissage transposables au \textit{Credit Scoring}.

Le chapitre~\ref{chap2} est consacré à l'étude du problème de ‘‘Réintégration des refusés'' (ou \textit{Reject Inference}) qui peut être réinterprété comme un problème de biais d'échantillon comme on peut en trouver dans les sondages par exemple, sur la variable à prédire. En effet, le système d'acceptation en place ayant déjà pour but de refuser la clientèle risquée, la clientèle en portefeuille servant d'échantillon au statisticien pour dériver de nouvelles règles de classement est bien moins risquée que la population totale des demandeurs ; c'est en ce sens qu'elle est biaisée.

Ce problème d'échantillonnage résolu, il paraît naturel au statisticien de s'atteler à la modélisation : quelle relation existe-t-il entre les caractéristiques de l'emprunteur et la quantité de risque qu'il présente ? Le chapitre~\ref{chap1} aura mis en avant la nature des caractéristiques disponibles ainsi que certaines faiblesses statistiques de la procédure actuelle : les chapitres~\ref{chap4} et~\ref{chap5} présentent une nouvelle méthode de recherche et de sélection du meilleur modèle dans la famille imposée par le cas d'application.

Le chapitre~\ref{chap6} prend du recul sur les chapitres précédents et remet le problème du \textit{Credit Scoring} au niveau du système d'acceptation dans sa globalité. En effet, on verra au chapitre~\ref{chap1} que plusieurs scores sont en place sur des segments de clientèle différents ; un score spécifique peut être dédié aux crédits automobiles par exemple. Chacun de ces scores est optimal localement, sur son segment. En revanche, les techniques de construction des segments reposant sur des heuristiques relativement éloignés de l'objectif de prédiction, le système global est \textit{a priori} sous-optimal. On formalisera cette architecture de ``mélange d'experts'' pour proposer des solutions adaptées de la littérature statistique à l'optimisation globale du système d'acceptation.

Enfin, l'émergence récente du \textit{Big Data} n'échappe pas au monde du crédit à la consommation. Le chapitre~\ref{chap1} aura mis en évidence que les pratiques industrielles sont souvent peu formalisées, ce qui justifie la présente thèse, et ne seront pas adaptées à la grande dimension en termes de prédicteurs. En conséquence, pour répondre au double objectif d'utilisation à court-terme de données dites \textit{web} (\textit{e.g.}\ cookies, clics, logs) et d'éviter de se résoudre à des procédures \textit{ad hoc} qui nécessiteraient une formalisation ultérieure, des premières directions d'étude pour l'utilisation raisonnée de telles données dans le cadre du \textit{Credit Scoring} sont données dans la~\lnameref{ccl}.

\printbibliography[heading=subbibliography, title=Références de l'introduction]