\chapter{High dimensional data in \textit{Credit Scoring}} \label{chap7}



this term was first used by R. Bellman in the introduction of his book
“Dynamic programming” in 1957:
All [problems due to high dimension] may be subsumed under the
heading “the curse of dimensionality”. Since this is a curse, [...], there is
no need to feel discouraged about the possibility of obtaining significant
results despite it.
















\section{Le cas $d > n$}


\section{The curse of dimensionality}


\section{The blessings of dimensionality}


\section{La réduction de dimension}


\subsection{Par combinaison des variables d'origine}

\subsection{Par sélection des variables d'origine}


\bigskip

Ce chapitre.


\printbibliography[heading=subbibliography, title=Références du chapitre 6]