\chapter{High dimensional data in \textit{Credit Scoring}} \label{chap7}

\epigraph{All [problems due to high dimension] may be subsumed under the heading “the curse of dimensionality”. Since this is a curse, [...], there is no need to feel discouraged about the possibility of obtaining significant results despite it.}{R. Bellman, ``Dynamic programming'', 1957}

\minitoc



\section{Motivation}

\subsection{Industrial context}


\subsection{Two identified subproblems}


\section{Longitudinal data in high dimension}

\subsection{The $d > n$ setting}


\subsection{The curse of dimensionality}


\subsection{The blessings of dimensionality}


\subsection{Dimension reduction}


\subsubsection{By combining input features}



\subsubsection{By selecting input features}


\section{New data types in a supervised classification setting}


\bigskip

Ce chapitre.


\printbibliography[heading=subbibliography, title=Références du chapitre 6]