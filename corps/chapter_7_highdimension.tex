\chapter{High dimensional data in \textit{Credit Scoring}} \label{chap7}

\epigraph{All [problems due to high dimension] may be subsumed under the heading “the curse of dimensionality”. Since this is a curse, [...], there is no need to feel discouraged about the possibility of obtaining significant results despite it.}{R. Bellman, ``Dynamic programming'', 1957}

\minitoc



\section{Motivation}



\section{Données ``classiques''}

\subsection{Le cas $d > n$}


\subsection{The curse of dimensionality}


\subsection{The blessings of dimensionality}


\subsection{La réduction de dimension}


\subsubsection{Par combinaison des variables d'origine}

\subsubsection{Par sélection des variables d'origine}


\section{Traitement des nouvelles données}

\bigskip

Ce chapitre.


\printbibliography[heading=subbibliography, title=Références du chapitre 6]