% arara: pdflatex: {shell: yes}
% arara: biber 
% arara: makeglossaries
% arara: pdflatex: {shell: yes}
% arara: pdflatex: {shell: yes}
% arara: clean
% arara: lmkclean

% Document de classe yathesis
\documentclass[english]{yathesis}
%
% Chargement manuel de packages (pas déjà chargés par la classe yathesis)
\usepackage[T1]{fontenc}
\usepackage[utf8]{inputenc}
\usepackage{kpfonts}
\usepackage{booktabs}
\usepackage{siunitx}
\usepackage{pgfplots}
\usepackage{caption}
\usepackage{microtype}
\usepackage{varioref}
\usepackage[backend=biber,safeinputenc,refsection=chapter]{biblatex}
\usepackage{hyperref}
\usepackage{csquotes}
\usepackage{textcomp}
\usepackage{bm}
\usepackage{nameref}
\usepackage{amsthm}
\usepackage{dsfont}
\usepackage{enumerate}
\usepackage{subcaption}
\usepackage{multicol}
\usepackage[french]{minitoc}
\usepackage{animate}
\usepackage[notransparent]{svg}
\usepackage{multirow}
\usepackage{makecell}
\usepackage{listings}
\usepackage{cancel}
\usetikzlibrary{arrows,intersections,positioning,shapes.arrows,babel}
\usepackage[xindy,acronyms,symbols]{glossaries}
\usepackage[english]{algorithm2e}
%
% (Facultatif) Génération de l'index (obligatoire si un package d'index, par
% exemple « imakeidx », est chargé)
% \makeindex
%
% (Facultatif) Spécification de la ou des ressources bibliographiques
% (obligatoire si le package « biblatex » est chargé)
%\addbibresource{auxiliaires/bibliographie.bib}
% \addbibresource{auxiliaires/}
\addbibresource{bibtex_files/introduction/introduction.bib}
\addbibresource{bibtex_files/chapitre1/chapitre1.bib}
\addbibresource{bibtex_files/chapitre2/chapitre2.bib}
\addbibresource{bibtex_files/chapitre4/chapitre4.bib}
\addbibresource{bibtex_files/chapitre5/chapitre5.bib}
\addbibresource{bibtex_files/chapitre6/chapitre6.bib}
\addbibresource{bibtex_files/appendix/appendix.bib}
%
% (Facultatif) Génération du glossaire (obligatoire si le package « glossaries »
% est chargé)
\makeglossaries
%
% (Facultatif) Configuration des styles du glossaire et de la liste d'acronymes
% (à n'utiliser que si le package « glossaries » est chargé)
\setglossarystyle{indexhypergroup}
%\setacronymstyle{long-sc-short}
%
% (Facultatif) Spécification de la ou des ressources terminologiques
\loadglsentries{auxiliaires/glossaries.tex}
% \loadglsentries{auxiliaires/}
% \loadglsentries{auxiliaires/}
%
% Les réglages figurant habituellement dans le préambule, notamment concernant
% la bibliographie et l'éventuel index, peuvent être saisis dans le fichier
% « thesis.cfg » (situé dans le sous-dossier « configuration ») qui est
% automatiquement importé par la classe yathesis.
%
% Importation manuelle du fichier de macros personnelles
\defbibheading{subbibliography}[\refname]{\subsection*{#1}}

\newcommand{\ag}{\ensuremath{\bm{\alpha}}}

\newcommand{\be}{\ensuremath{\bm{e}}}

\newcommand{\bE}{\ensuremath{\bm{E}}}

\newcommand{\E}{\ensuremath{{E}}}

\newcommand{\f}{\ensuremath{\text{f}}}

\newcommand{\nf}{\ensuremath{\text{nf}}}

\newcommand{\pd}{\ensuremath{p_{\text{data}}}}

\newcommand{\ps}{p_{\text{sample}}}

\DeclareMathOperator*{\argmin}{argmin}

\DeclareMathOperator*{\argmax}{argmax}

\newcommand{\bc}{\boldsymbol{c}} 
\newcommand{\bbe}{\mathbf{e}}
\newcommand\q{{\bm{q}}}
\newcommand\Q{\mathcal{Q}}
\newcommand{\bdelta}{\bm{\delta}}

\newcommand{\bqk}{\bm{\mathfrak{q}}}
\newcommand{\bbqk}{\mathfrak{\mathbf{q}}}
\newcommand{\qk}{\mathfrak{q}}

\newcommand{\tr}{t}

\newcommand{\appropto}{\mathrel{\vcenter{
  \offinterlineskip\halign{\hfil$##$\cr
    \propto\cr\noalign{\kern2pt}\sim\cr\noalign{\kern-2pt}}}}}


\newcommand{\yslant}{0.5}
\newcommand{\xslant}{-0.6}


\tikzset{style green/.style={
    set fill color=green!50!lime!60,
    set border color=white,
  },
  style cyan/.style={
    set fill color=cyan!90!blue!60,
    set border color=white,
  },
  style orange/.style={
    set fill color=orange!80!red!60,
    set border color=white,
  },
  hor/.style={
    above left offset={-0.15,0.31},
    below right offset={0.15,-0.125},
    #1
  },
  ver/.style={
    above left offset={-0.1,0.3},
    below right offset={0.15,-0.15},
    #1
  }
}

\newcommand{\tikzmark}[2][minimum width=6cm,minimum height=1.5cm]{
\tikz[remember picture, overlay]
\node[anchor=west,
inner sep=0pt,
outer sep=6pt,
xshift=-0.5em,
yshift=-3ex,
#1](#2){};
}


\tikzset{
    myarrow/.style={
        draw,
        fill=orange,
        single arrow,
        minimum height=3.5ex,
        single arrow head extend=1ex
    }
}


\tikzset{
  treenode/.style = {shape=rectangle, rounded corners,
                     draw, align=left,
                     top color=white, bottom color=blue!20},
  root/.style     = {treenode, font=\Large, bottom color=red!30},
  env/.style      = {treenode, font=\ttfamily\normalsize},
  dummy/.style    = {circle,draw}
}


\def\mybar#1#2#3{%%
\begin{tabular}{@{}l@{}} {\scriptsize #1} \vspace*{-0.2cm} \\ {\scriptsize #2} \vspace*{-0.2cm} \\ {\scriptsize #3}\end{tabular} & \resizebox{.05\textwidth}{0.5cm}{\begin{tabular}{@{}l@{}}{\color{green}\rule[-6pt]{#1bp}{10pt}} \\ {\color{orange}\rule[0pt]{#2bp}{10pt}} \vspace*{-0.1cm} \\ {\color{red}\rule[10pt]{#3bp}{10pt}} \end{tabular}}}

\def\myobar#1#2#3{%%
\begin{tabular}{@{}l@{}} {\scriptsize #1} \vspace*{-0.2cm} \\ {\scriptsize #2} \vspace*{-0.2cm} \\ {\scriptsize #3}\end{tabular} & \resizebox{.05\textwidth}{0.5cm}{\begin{tabular}{@{}l@{}}{\color{orange}\rule[-6pt]{#1bp}{10pt}} \\ {\color{green}\rule[0pt]{#2bp}{10pt}} \vspace*{-0.1cm} \\ {\color{orange}\rule[10pt]{#3bp}{10pt}}\end{tabular}}}


\definecolor{light-gray}{gray}{0.95}
\newcommand{\code}[1]{\colorbox{light-gray}{\texttt{#1}}}


\newcommand{\myGlobalTransformation}[2]
{
    \pgftransformcm{1}{0}{0.4}{0.5}{\pgfpoint{#1cm}{#2cm}}
}


\newcommand{\myGlobalTransformationbis}[2]
{
    \pgftransformcm{1}{0}{0.4}{0.3}{\pgfpoint{#1cm}{#2cm}}
}

\newcommand{\gridThreeD}[3]
{
    \begin{scope}
        \myGlobalTransformation{#1}{#2};
        \draw [#3,step=7cm] grid (7,7);
    \end{scope}
}

\tikzstyle myBG=[line width=3pt,opacity=1.0]

\newcommand{\drawLinewithBG}[2]
{
    \draw[white,myBG]  (#1) -- (#2);
    \draw[black,very thick] (#1) -- (#2);
}

\newcommand{\graphLinesHorizontal}
{
    \drawLinewithBG{1,1}{7,1};
    \drawLinewithBG{1,3}{7,3};
    \drawLinewithBG{1,5}{7,5};
    \drawLinewithBG{1,7}{7,7};
}

\newcommand{\graphLinesVertical}
{
    %swaps x and y coordinate (hence vertical lines):
    \pgftransformcm{0}{1}{1}{0}{\pgfpoint{0cm}{0cm}}
    \graphLinesHorizontal;
}

\newcommand{\graphThreeDnodes}[2]
{
    \begin{scope}
        \myGlobalTransformation{#1}{#2};
        \foreach \x in {1,3,5,7} {
            \foreach \y in {1,3,5,7} {
                \node at (\x,\y) [circle,fill=black,scale=0.3] {};
                %this way circle of nodes will not be transformed
            }
        }
    \end{scope}
}

\newcommand\rstyle{%
  \NoAutoSpacing
  \lstset{%
    language=R,
    alsoletter={_},
otherkeywords={install_github},             % Add keywords here
%    basicstyle=\small,
    keywordstyle=\color{blue},
    stringstyle=\color{green},
%    numbers=left,
%    numberstyle=\tiny,
%    numbersep=5pt,
  numbers=left,                   % where to put the line-numbers
  numberstyle=\tiny\color{blue},  % the style that is used for the line-numbers
  stepnumber=1,                   % the step between two line-numbers. If it is 1, each line
                                  % will be numbered
  numbersep=5pt,                  % how far the line-numbers are from the code
  showspaces=false,               % show spaces adding particular underscores
  showstringspaces=false,         % underline spaces within strings
  showtabs=false,                 % show tabs within strings adding particular underscores
  breaklines=true,                % sets automatic line breaking
  breakatwhitespace=false,        % sets if automatic breaks should only happen at whitespace
    frame=single%
  }%
}

\lstnewenvironment{rlisting}
{\rstyle}{}

\newcommand\rinline[1]{{\rstyle\lstinline!#1!}}

\newcommand\pystyle{%
  \NoAutoSpacing
  \lstset{%
    language=R,
    alsoletter={_},
    alsoletter={.},
otherkeywords={install_github},             % Add keywords here
%    basicstyle=\small,
    keywordstyle=\color{blue},
    stringstyle=\color{green},
%    numbers=left,
%    numberstyle=\tiny,
%    numbersep=5pt,
  numbers=left,                   % where to put the line-numbers
  numberstyle=\tiny\color{blue},  % the style that is used for the line-numbers
  stepnumber=1,                   % the step between two line-numbers. If it is 1, each line
                                  % will be numbered
  numbersep=5pt,                  % how far the line-numbers are from the code
  showspaces=false,               % show spaces adding particular underscores
  showstringspaces=false,         % underline spaces within strings
  showtabs=false,                 % show tabs within strings adding particular underscores
  breaklines=true,                % sets automatic line breaking
  breakatwhitespace=false,        % sets if automatic breaks should only happen at whitespace
    frame=single%
  }%
}

\lstnewenvironment{pylisting}
{\pystyle}{}

\newcommand\pyinline[1]{{\pystyle\lstinline!#1!}}






\newcommand\bashstyle{%
  \NoAutoSpacing
  \lstset{%
    language=bash,
    alsoletter={_},
otherkeywords={install_github},             % Add keywords here
%    basicstyle=\small,
%    keywordstyle=\color{blue},
%    stringstyle=\color{green},
%    numbers=left,
%    numberstyle=\tiny,
%    numbersep=5pt,
  numbers=left,                   % where to put the line-numbers
%  numberstyle=\tiny\color{blue},  % the style that is used for the line-numbers
  stepnumber=1,                   % the step between two line-numbers. If it is 1, each line
                                  % will be numbered
  numbersep=5pt,                  % how far the line-numbers are from the code
  showspaces=false,               % show spaces adding particular underscores
  showstringspaces=false,         % underline spaces within strings
  showtabs=false,                 % show tabs within strings adding particular underscores
  breaklines=true,                % sets automatic line breaking
  breakatwhitespace=false,        % sets if automatic breaks should only happen at whitespace
    frame=single%
  }%
}

\lstnewenvironment{bashlisting}
{\bashstyle}{}

\newcommand\bashinline[1]{{\bashstyle\lstinline!#1!}}
%
%%%%%%%%%%%%%%%%%%%%%%%%%%%%%%%%%%%%%%%%%%%%%%%%%%%%%%%%%%%%%%%%%%%%%%%%%%%%%%%
%%%%%%%%%%%%%%%%%%%%%%%%%%%%%%%%%%%%%%%%%%%%%%%%%%%%%%%%%%%%%%%%%%%%%%%%%%%%%%%
% Début du document
%%%%%%%%%%%%%%%%%%%%%%%%%%%%%%%%%%%%%%%%%%%%%%%%%%%%%%%%%%%%%%%%%%%%%%%%%%%%%%%
%%%%%%%%%%%%%%%%%%%%%%%%%%%%%%%%%%%%%%%%%%%%%%%%%%%%%%%%%%%%%%%%%%%%%%%%%%%%%%%
\begin{document}
%
%%%%%%%%%%%%%%%%%%%%%%%%%%%%%%%%%%%%%%%%%%%%%%%%%%%%%%%%%%%%%%%%%%%%%%%%%%%%%%%
% Caractéristiques du document
%%%%%%%%%%%%%%%%%%%%%%%%%%%%%%%%%%%%%%%%%%%%%%%%%%%%%%%%%%%%%%%%%%%%%%%%%%%%%%%
%
% Préparation des pages de couverture et de titre
%%%%%%%%%%%%%%%%%%%%%%%%%%%%%%%%%%%%%%%%%%%%%%%%%%%%%%%%%%%%%%%%%%%%%%%%%%%%%%%
% Les caractéristiques de la thèse sont saisies dans le fichier
% « characteristics.tex » (situé dans le dossier « configuration »).
%
% Production des pages de couverture et de titre
%%%%%%%%%%%%%%%%%%%%%%%%%%%%%%%%%%%%%%%%%%%%%%%%%%%%%%%%%%%%%%%%%%%%%%%%%%%%%%%
\maketitle
%
%%%%%%%%%%%%%%%%%%%%%%%%%%%%%%%%%%%%%%%%%%%%%%%%%%%%%%%%%%%%%%%%%%%%%%%%%%%%%%%
% Début de la partie liminaire de la thèse
%%%%%%%%%%%%%%%%%%%%%%%%%%%%%%%%%%%%%%%%%%%%%%%%%%%%%%%%%%%%%%%%%%%%%%%%%%%%%%%
%
% (Facultatif) Production de la page de clause de non-responsabilité
\makedisclaimer
%
% (Facultatif) Production de la page de mots clés
\makekeywords
%
% (Facultatif) Production de la page affichant les logo, nom et coordonnées du
% ou des laboratoires (ou unités de recherche) où la thèse a été préparée
\makelaboratory
%
% (Facultatif) Dédicace(s)
% Dédicace(s)
\dedication{À ma famille,}
\dedication{À mes amis,}
\dedication{À mes lecteurs.}
% Production de la page de dédicace(s)
\makededications


%
% (Facultatif) Épigraphe(s)
% Épigraphes(s)
\frontepigraph{The task of the human brain remains what it has always been; that of discovering new data to be analyzed, and of devising new concepts to be tested.}{Isaac Asimov, \textit{I, Robot}}
\frontepigraph{J'respecte \textsf{R}.}{Damso}
% Production de la page de d'épigraphe(s)
\makefrontepigraphs
%
% Résumés succincts
% Résumés (de 1700 caractères maximum, espaces compris) dans la
% langue principale (1re occurrence de l'environnement « abstract »)
% et, facultativement, dans la langue secondaire (2e occurrence de
% l'environnement « abstract »)
\begin{abstract}
Cette thèse se place dans le cadre des modèles d'apprentissage automatique de classification binaire. Le cas d'application est le scoring de risque de crédit. En particulier, les méthodes proposées ainsi que les approches existantes sont illustrées par des données réelles de Crédit Agricole Consumer Finance, acteur majeur en Europe du crédit à la consommation, à l'origine de cette thèse grâce à un financement CIFRE.

%, plus précisément dans un contexte de classification~: on cherche à prédire l'appartenance à une catégorie de chaque observation de variables aléatoires dites explicatives dans un corpus. On restreint la problématique à la classification binaire (deux catégories possibles) et on dispose de la vraie catégorie pour une partie de ce corpus : on parle alors d'apprentissage semi-supervisé. L'``apprentissage'' consiste à établir par calcul une fonction de lien entre ces observations et leur catégorie respective, de manière à pouvoir prédire cette catégorie pour de nouvelles observations. On limite la famille de fonctions possibles à une famille de modèles et la tâche du statisticien est alors de choisir le modèle qui prédit ``le mieux possible'', c'est-à-dire de minimiser la perte d'information inhérente au caractère stochastique de la relation entre variables explicatives et la catégorie à prédire.

%Le ratio risque/récompense désigne en finance la logique selon laquelle un investissement peu risqué ne pourra être que faiblement rentable tandis qu'un investissement risqué a un rendement plus élevé mais est exposé à une perte. Les établissements financiers spécialisés en crédit à la consommation transposent ce principe en deux heuristiques : premièrement, le taux d'intérêt des crédits est adapté en fonction des clients et des produits ; deuxièmement, les clients demandeurs sont sélectionnés selon leur solvabilité. Ce mécanisme d'acceptation/rejet de la clientèle est composé de plusieurs règles de décision dont un score, le modèle précédemment décrit correspondant, en termes ``métier'', à une notation liée aux caractéristiques socio-démographiques de clients passés témoignant de la probabilité de défaut d'un nouveau client. La construction de ce score, qu'on désigne généralement par \textit{Credit Scoring}, repose sur des techniques statistiques et des heuristiques industrielles dont certaines ont été examinées dans cette thèse.

%Après une première partie décrivant l'évolution et le contexte industriels actuels ainsi que la littérature académique associée à la classification supervisée, 
Premièrement, on s'intéresse 
%dans une deuxième partie 
à la problématique de
%une contribution importante de cette thèse : la 
``réintégration des refusés'' ou comment tirer partie des informations collectées sur les clients refusés non étiquetés.  Ce problème industriel est reformulé en un problème statistique rigoureux par l'interprétation de l'absence d'étiquettes comme une perte d'information.
% que l'on peut compenser par un ajout ``industriel'' (financer une partie des clients supposés mauvais) ou un ajout ``statistique'' (modéliser le mécanisme de génération des données manquantes).

Une autre pratique industrielle est la discrétisation des variables continues et le regroupement des modalités de variables catégorielles avant la modélisation 
%par régression logistique 
pour des raisons pratiques (interprétabilité) et théoriques (performance de prédiction). Pour ce faire, des heuristiques internes, manuelles et chronophages, sont utilisées.
%, qui sont manuelles et prennent donc beaucoup de temps au statisticien.
% puisque cela implique de chercher dans un espace de modèles très grand. 
Sa ré-interprétation comme un problème à variables latentes a permis de proposer une nouvelle approche de résolution, basée sur un réseau de neurone utilisé comme graphe de calcul, et qui permet d'obtenir des garanties statistiques.

%On verra ensuite en troisième partie l'apport de la méthode proposée dans cette thèse pour la discrétisation (resp.\ le regroupement de modalités) des variables quantitatives (resp.\ qualitatives) constitutives du score. La ré-interprétation comme un problème à variables manquantes a permis de proposer une nouvelle approche de résolution dont les résultats sont significatifs en termes de performance et de gain de temps pour le statisticien.

De plus, il est courant d'introduire également des interactions afin d'améliorer la performance prédictive des modèles.
%, comme la présence simultanée d'une catégorie de revenus et d'un emploi particulier. 
Cette pratique est également manuelle et chronophage, c'est pourquoi on propose un algorithme de Metropolis-Hastings garantissant de trouver les meilleures interactions et rendant le mécanisme de construction des scores quasi-automatique.

%Outre la discrétisation et le regroupement de modalités, il est courant en \textit{Credit Scoring} d'introduire également des interactions, c'est-à-dire de choisir des produits de variables prédictives plutôt qu'un effet additif, comme la présence simultanée d'une catégorie de revenus et d'un emploi particulier. Nous ajouterons en quatrième partie une résolution du problème de sélection de ces interactions à l'algorithme développé en troisième partie, rendant le mécanisme de construction des scores quasi-automatique.

Nous prendrons ensuite du recul pour constater que le système d'acceptation est rarement constitué d'un seul score mais plutôt d'un arbre de scores
%, ou d'un mélange d'experts dans la littérature statistique
, chacun relatif à un segment de population particulier. Cette structure découle du développement historique de l'entreprise et est donc certainement sous-optimale. Nous proposerons des pistes de réflexion pour optimiser le système d'acceptation dans son entièreté qui montre de bons résultats empiriques et représente une direction de recherche future.

Enfin, nous conclurons cette thèse en explicitant les enjeux et défis de la grande dimension (en termes de prédicteurs) dans la pratique du \textit{Credit Scoring} dans la mesure où l'industrie souhaite mesurer l'apport de l'utilisation de données massives et non structurées, encore inutilisées.
%L'ensemble des travaux est illustré par des données réelles de \gls{cacf}, établissement bancaire spécialiste du crédit à la consommation à l'origine de cette thèse CIFRE.

\medskip

\end{abstract}

\begin{abstract}
This manuscript deals with model-based statistical learning in the binary classification setting. As an application, credit scoring is widely examined with a special attention on its specificities. Proposed and existing approaches are illustrated on real data from Crédit Agricole Consumer Finance, a financial institute specialized in consumer loans which financed this PhD through a CIFRE funding.

First, we consider the so-called reject inference problem, which aims at taking advantage of rejected credit applicants for which no repayment performance can be observed (\textit{i.e.}\ unlabelled observations). This industrial problem led to a research one by reinterpreting unlabelled observations as an information loss.

Next, the discretization of continuous features or grouping of levels of categorical features, which are empirical and time-consuming for practitioners, are seen here as a latent variable problem. This results in a new cost-effective and automatic processing which involves a particular neural network architecture and gives precise statistical guarantees.

Third, interactions of covariates may be introduced in the problem in order to improve the performance. This task, up to now manually processed by practitioners and combinatorial, is performed here with a Metropolis-Hastings sampling procedure which finds the best interactions.

Finally, we look at the scoring system as a whole. It generally consists of a tree-like structure composed of many scorecards, which is often not optimized but rather imposed by the company’s culture and / or history. We propose some lines of approach to optimize it which lead to good empirical results and new research directions.

This manuscript is concluded by a discussion on how curses and blessings of dimensionality (in the number of features) might affect the practice of Credit Scoring, since the industry is pushing forward the use of high-dimensional unstructured data.

%This manuscript is centred on model-based statistical learning in a binary classification setting with a special attention to \textit{Credit Scoring} and its related constraints. In particular, proposed methods and comparisons with existing approaches are illustrated on real data from Crédit Agricole Consumer Finance, a financial institute specialized in consumer loans which financed this PhD through a CIFRE funding.
%: the membership of partially labelled observations of random variables
% (semi-supervised learning), called predictive features and forming a training set, 
% to one of two categories. 
%We restrict the problem to binary classification (only two possible categories) and we are provided with the true category for some observations in the training set: this setting corresponds to semi-supervised learning. 
%``Learning'' consists in calculating a link function between these features and the labels,
% of observations in the training set, such that it enables us to apply its prediction to new data. This function 
% which is searched in a restricted space of models.
 % and the practitioner subsequently chooses the model that predicts ``best'', \textit{i.e.}\ which minimizes an information loss criterion attached to the intrinsic stochasticity of the link between predictive features and the label to predict.

%The risk-reward is a well known finance paradigm: the higher the risk of an investment, the higher the expected reward. 
%When it comes to consumer loans, 
%two heuristics are usually employed: first, the interest rate of the loan depends on the client's risk of defaulting and second, 
%applicants get rejected depending on their estimated creditworthiness. This acceptance / rejection mechanism is composed of several business rules, among which 
%the score, \textit{i.e.}\ 
%the aforementioned model which predicts
%outputs a numeric value depending upon socio-demographic characteristics of past clients and their repayment behaviour which tends to reflect 
% the propensity of new clients to pay back. The construction of such scores relies on both statistical learning and \textit{ad hoc}, industrial recipes, which are partly tackled in this manuscript.

%The first chapter focuses on the industrial context of \textit{Credit Scoring} and its associated academic literature of supervised classification. In the second chapter
%First, we consider the so-called ``Reject Inference'' problem, which aims at taking into account information about previously rejected clients (for which no repayment performance was observed). This industrial problem leads to a rigorous statistical formulation by reinterpreting unlabelled observations as an information loss that can be compensated by an ``industrial'' addition (financing some of these supposedly bad applicants) or a ``statistical'' addition (the cost of modelling its missingness mechanism).

%Another industrial practice of \textit{Credit Scoring} is to discretize (resp.\ group) continuous features (resp.\ levels of categorical features) before performing logistic regression for practical (interpretability) and theoretical (performance) reasons. To do so, in-house heuristics are usually used, which are manual and subsequently take a lot of the practitioner's time since it involves searching through a huge model space. We reinterpret this practice as a latent variable problem and propose a new resolution, based on a particular neural network architecture which acts as a computational graph and enables us to give precise statistical guarantees. The results are satisfactory both in performance and in saving the practitioner's time, on benchmark and \textit{Credit Scoring} data.

%Moreover, interactions among covariates, \textit{e.g.}\ the simultaneous presence of a category of wages and a type of job, might be introduced in the logistic regression model for better performance. We build on what was proposed for discretization and grouping of factor levels to automatically search for the best interactions among covariates. This task was even more manual and combinatorial, and less formalized in the existing literature. Our proposal relies on a Metropolis-Hastings sampling procedure which is guaranteed to find the best interactions.

%Finally, we take a step back and look at the acceptance system as a whole: it is generally composed of many scorecards, in a tree-like structure which is often not optimized over but rather imposed by the company's culture and / or history. We propose some guidelines to optimize the whole scoring system which leads to good empirical results and future research directions.

%This manuscript is concluded by a discussion on how curse and blessing of dimensionality (in the number of features) might affect the practice of \textit{Credit Scoring}, since the industry is pushing forward to use high-dimensional unstructured data.

\medskip

\end{abstract}

\makeabstract
%
% (Facultatif) Chapitre de remerciements
\chapter{Remerciements}

Aboutissement d'un travail personnel, cette thèse n'en est pas moins une réussite collective et la contribution de nombreuses personnes, injustement absente de la page de couverture de ce manuscrit, doit ici être extensivement mentionnée.

% Collègues
Tout d'abord, je suis persuadé que le principal facteur de succès d'une thèse CIFRE est l'implication de l'entreprise d'accueil, de la conception du sujet à l'usage des fruits du travail de recherche. A ce titre, je remercie Crédit Agricole Consumer Finance de m'avoir permis de réaliser cette thèse dans de très bonnes conditions. En particulier, j'ai eu la chance d'interagir avec des managers réceptifs à la démarche de recherche et qui m'ont fait confiance : un grand merci à Jérôme Beclin et Nicolas Borde. Je me dois également de saluer la probité intellectuelle de Sébastien Beben ; nos riches échanges de début de thèse constituent sans doute le carburant de ce doctorat.

% Labo
Haut-lieu de la recherche publique française, Inria m'a permis, en acceptant d'être le laboratoire d'accueil de cette CIFRE, de compléter ma formation d'ingénieur généraliste centralien en tentant de combler le vide technique ressenti en fin de cursus, ce qui m'avait motivé à poursuivre en thèse. Je vous laisse le soin, chers lecteurs, d'apprécier l'éventuelle réussite de cet objectif initial. Je remercie chaleureusement le centre de Lille et plus particulièrement l'équipe-projet M$\Theta$DAL pour m'avoir permis de (re)connaître la beauté des mathématiques. Contributeurs directs et véritables artisans de ce travail de recherche, mes trois co-directeurs de thèse ont constitué le moteur de cette thèse ; merci à Christophe Biernacki dont j'espère garder la rigueur scientifique ; merci à Philippe Heinrich, pour m'avoir montré qu'un problème bien posé est déjà à moitié résolu ; merci à Vincent Vandewalle, dont les éclairages passionés, à grands coups de feutre virevoltant sur le tableau ou scripts \textsf{R} envoyés au milieu de la nuit, ont pour la plupart donné la vitesse initiale à chaque partie de ce manuscrit.

% Famille
Enfin, il convient de saluer la part de responsabilité de mes proches dans ce travail et les quelques mots qui suivront sont bien peu de choses en comparaison de tout ce qu'ont pu apporter ma famille et mes amis dans ma formation intellectuelle au sens large. Un immense merci revient tout d'abord à mes parents, ils m'ont tout donné. Merci également à tous mes amis, de nombreux rires restent à partager. Merci bien sûr à ma femme Valentine
%, dont j'ai mis la patience à rude épreuve avec toutes ces années d'étude mais 
qui m'a toujours soutenu ; je suis très fier de tout ce que nous avons accompli. Puisse-t-on traverser la vie comme ces dix dernières années.

% Lecteurs
Cher lecteur, merci de parcourir le manuscrit que tu tiens entre les mains ; sans toi, il n'existerait pas. Bonne lecture.
%
% (Facultatif) Chapitre d'avertissement
% \include{liminaires/avertissement}
%
% (Facultatif) Liste des acronymes
%\printacronyms
%
%
% (Facultatif) Chapitre d'avant-propos
% \include{liminaires/avant-propos}
%
% Sommaire
\dominitoc
\tableofcontents[depth=section,name=Sommaire]
%
% (Facultatif) Liste des tableaux
\listoftables
%
% (Facultatif) Table des figures
\listoffigures
%
% (Facultatif) Table des listings (nécessite que le package « listings » soit
% chargé)
% \lstlistoflistings
% (Facultatif) Glossaire (si souhaité distinct de la liste des acronymes) :
\printglossary
% (Facultatif) Liste des symboles
\printsymbols

%\printglossary[type=symbols,style=long,title={Liste des symboles}]

%
%%%%%%%%%%%%%%%%%%%%%%%%%%%%%%%%%%%%%%%%%%%%%%%%%%%%%%%%%%%%%%%%%%%%%%%%%%%%%%%
% Début de la partie principale (du « corps ») de la thèse
%%%%%%%%%%%%%%%%%%%%%%%%%%%%%%%%%%%%%%%%%%%%%%%%%%%%%%%%%%%%%%%%%%%%%%%%%%%%%%%
\mainmatter
%
% Chapitre d'introduction (générale)
%%%%%%%%%%%%%%%%%%%%%%%%%%%%%%%%%%%%%%%%%%%%%%%%%%%%%%%%%%%%%%%%%%%%%%%%%%%%%%%
\chapter*{Introduction} \label{chap_intro}
% français

Les cas d'application des travaux de ce manuscrit portent sur plusieurs problèmes connexes au \textit{Credit Scoring}.

Pour un particulier, le recours au crédit, c'est-à-dire à l'emprunt d'argent en échange d'une promesse de remboursement étalé dans le temps et assorti d'un intérêt, est possible depuis très longtemps, les plus anciennes traces ’’modernes'' de crédits bancaires se situant au XII$^\text{ème}$ siècle en Italie~\cite{thomas_wards_1828}. De nos jours, l'emprunt immobilier ou automobile, c'est-à-dire pour financer un lieu de résidence ou l'achat d'un véhicule, est répandu~\cite{la_tribune_2010}. Par opposition au crédit immobilier, on parle souvent de crédit à la consommation pour désigner le financement de biens et de services : automobile, électroménager, travaux, etc. De manière plus formelle, le crédit à la consommation est définie dans la loi \textnumero 2010-737 du 1$^\text{er}$ juillet 2010~\cite{noauthor_loi_2010} comme une :
\begin{displayquote}
Opération ou contrat de crédit, une opération ou un contrat par lequel un prêteur consent ou s’engage à consentir à l’emprunteur un crédit sous la forme d’un délai de paiement, d’un prêt, y compris sous forme de découvert ou de toute autre facilité de paiement similaire, à l’exception des contrats conclus en vue de la fourniture d’une prestation continue ou à exécution successive de services ou de biens de même nature et aux termes desquels l’emprunteur en règle le coût par paiements échelonnés pendant toute la durée de la fourniture.
\end{displayquote}

De nombreux acteurs bancaires proposent des crédits à la consommation, si bien qu'en 2013 environ 26,6 \% des ménages ont un crédit à la consommation~\cite{}. \gls{cacf} est un acteur majeur du crédit à la consommation, à travers une marque spécialisée en France, Sofinco, et des partenaires distributeurs de crédit conso.

Parmi l'ensemble des demandeurs de crédit à la consommation, il est souhaitable, à plusieurs égards, de ne pas financer tous les crédits. Premièrement, si tant est que l'on puisse prêter un rôle sociétal à une entité bancaire, il paraît responsable de ne pas détériorer voire mettre en danger la santé financière de l'emprunteur. Pour ce faire, des contrôles automatiques permettent de refuser la clientèle dite fragile : taux d'endettement trop élevé, fichage bancaire pour incidents de paiements, \ldots Par ailleurs, d'un point de vue économique cette fois, un client se trouvant dans l'incapacité de rembourser le crédit souscrit sera vraisemblablement peu ou pas profitable pour l'institution financière du fait des coûts de traitements et de personnels de relance et procédure(s) judiciaire(s) qui peuvent aboutir à une annulation totale ou partielle de la dette du client engendrant une perte sèche pour l'organisme prêteur.

Dans ce cadre, le \textit{Credit Scoring} vise à évaluer la propension d'un client à être ‘‘bon'' ou ‘‘mauvais'', selon des critères à définir ultérieurement, pour ainsi prendre une décision de financement ou de rejet de façon quantitative et objective. On donnera dans le chapitre~\ref{chap1} quelques éléments de contexte supplémentaires nécessaires à la bonne compréhension des cas d'application de cette thèse, un état de l'art de la pratique industrielle ainsi qu'un état de l'art académique des techniques d'apprentissage transposables au \textit{Credit Scoring}.

Le chapitre~\ref{chap2} est consacré à l'étude du problème de ‘‘Réintégration des refusés'' (ou \textit{Reject Inference}) qui peut être réinterprété comme un biais d'échantillon 

Ce problème d'échantillonnage résolu, il paraît naturel au statisticien de s'atteler à la modélisation : quelle relation existe-t-il entre les caractéristiques de l'emprunteur et la quantité de risque qu'il présente ? Le chapitre~\ref{chap1} aura mis en avant certaines faiblesses statistiques de la procédure actuelle : le chapitre~\ref{chap3} met en oeuvre une nouvelle méthode de recherche et sélection du meilleur modèle dans la famille imposée par le cas d'application.

\textcolor{red}{à compléter avec le(s) dernier(s) chapitre(s)}

\printbibliography[heading=subbibliography, title=Références de l'introduction]
%
% Chapitres ordinaires (avec parties éventuelles)
%%%%%%%%%%%%%%%%%%%%%%%%%%%%%%%%%%%%%%%%%%%%%%%%%%%%%%%%%%%%%%%%%%%%%%%%%%%%%%%
%
% Première partie éventuelle
% \part{...}
%
% Premier chapitre
\chapter{Apprendre des demandes de crédit à la consommation} \label{chap1}

Ce chapitre est destiné à poser les bases de l'apprentissage statistique dans le cadre des crédits à la consommation. On introduira dans une première partie une partie de la terminologie consacrée aux crédits à la consommation avant de s'attarder plus en détails, dans une seconde partie, sur l'état de l'art industriel du \textit{Credit Scoring} à travers une étude bibliographique et la pratique de \gls{cacf}. On clotûrera le chapitre par une troisième partie, la plus traditionnelle pour débuter un manuscrit de thèse, à savoir l'état de l'art académique sur l'apprentissage statistique, en nous limitant bien entendu aux cas d'usage spécifiques aux crédits à la consommation mis en exergue dans les deux premières parties de ce chapitre.

\section{Le marché du crédit à la consommation : quels enjeux ?} \label{chap1:sec1}

S'agissant d'une thèse CIFRE, il apparaît comme nécessaire de planter le décor industriel de la problématique. Dans cette première partie, on verra succintement le coeur du métier de \gls{cacf}, les produits qu'elles proprosent et l'environnement dans lequel elle s'insert.

\subsection{Qu'est-ce qu'un crédit à la consommation ?}

La définition légale en a été donnée en~\ref{chap_intro}. En pratique, on peut distinguer trois produits de crédit à la consommation.


Le premier d'entre eux, le \gls{creditclassique} est le produit historique. De la même manière qu'un crédit immobilier, le client emprunte une somme fixe qui lui est attribuée au financement et qu'il rembourse selon un échéancier (taux et nombre de mensualités fixes) défini à l'avance. D'un point de vue statistique, le traitement est relativement simple : que ce soit à l'octroi, pour déterminer le risque du client, ou au cours de la vie du dossier, pour provisionner les pertes potentielles, tout est connu à l'avance. Il suffit en quelque sorte de vérifier le paiement de la mensualité à la date prévue. Il convient également de préciser que certains crédits classiques sont dits \glspl{creditaffecte}, c'est-à-dire qu'ils financent un bien précis et identifié, de sorte que le prêt transite directement de l'organisme prêteur au vendeur (concessionnaire par exemple). Par ailleurs, la mise en défaut du crédit entraîne généralement une procédure de recouvrement de la dette qui peut se solder, dans le cas d'un \gls{creditaffecte}, par la récupération du bien par un huissier. Là encore, d'un point de vue statistique, il paraît indispensable de consigner les caractéristiques du bien sous-jacent afin d'intégrer sa valeur résiduelle récupérable en cas de défaut.


Le second produit, développé à partir de 1965 en France et ayant connu une forte croissance depuis~\cite{ducourant2009credit} mais néanmoins bien moins répandu en Europe qu'aux Etats-Unis par exemple~\cite{credit_cards_country}, est le \gls{creditrenouvelable}. Un capital dit accordé ou autorisé est attribué au demandeur qui peut utiliser tout ou partie de ce montant et le rembourse à un taux et par mensualités dépendants tous deux de la proportion du capital consommé. Au fur et à mesure du remboursement du capital emprunté, le capital ‘‘empruntable'', c'est-à-dire la différence entre le capital accordé et le capital emprunté puis remboursé, se reconstitue et de nouvelles utilisations sont possibles, toujours dans la limite du capital accordé au départ. D'un point de vue statistique à nouveau, plusieurs problèmes se posent du fait du caractère intrinsèquement aléatoire de l'utilisation ou non de tout ou partie de la ligne de crédit accordée. Plus précisément, ce produit présente un risque important porté par deux facteurs : premièrement, le taux élevé attire des clients risqués, au taux de défaut plus élevé que pour un crédit classique par exemple ; deuxièmement, ces crédits portent un risque dit de hors-bilan très fort, puisqu'à tout moment, l'ensemble des crédits accordés mais non utilisés et donc non comptabilisés ‘‘au bilan'' c'est-à-dire comme une dette du client envers l'établissement bancaire, peuvent être utilisés et faire défaut. La mauvaise quantification de ce risque est à présent reconnu comme un important catalyseur de la récente crise financière \cite{karim2013off}.


Enfin, la \gls{location} a récemment connu un essor important~\cite{peden_2018}. D'abord concentrée sur le secteur automobile, elle se développe actuellement pour les produits électroniques (smartphones notamment) et même plus récemment pour des produits plus insolites comme les matelas~\cite{dicharry_2017}. Comme le \gls{creditaffecte}, il est important de prendre en compte les données du bien loué afin d'évaluer le risque que porte ce produit, la difficulté supplémentaire reposant sur l'éventualité de l'exercice de l'option d'achat.

De cette partie, deux considérations statistiques doivent retenir notre attention : d'abord, ces différents produits nécessitent des traitements différents dans la mesure où leur risque est intrinsèquement différent ; ensuite, les données disponibles pour chacun de ces produits diffèrent : par exemple, les données du produit financé ne sont disponibles que pour les \glspl{creditaffecte} et les \glspl{location}. Cette dernière notion de ‘‘blocs'' de variables est au coeur du chapitre~\ref{chap6}.

%\subsection{Quels sont les acteurs de ce marché ?}
%Au même titre que la partie précédente, cette partie ne se veut pas une analyse du marché du crédit à la consommation mais 

\subsection{Crédit Agricole Consumer Finance}

\gls{cacf} opère dans de nombreux pays. En France, c'est principalement à travers la marque Sofinco que sont commercialisés les crédits à la consommation pour lesquels il existe une relation directe entre \gls{cacf} et le client (dite B2C), par exemple lorsqu'un demandeur se rend directement sur le site internet \href{https://www.sofinco.fr}{sofinco.fr}.

Par ailleurs, de nombreux crédits à la consommation sont distribués à travers un réseau de partenaires, qui jouent le rôle d'intermédiaires (on parle alors de B2B) : concessionnaires automobile, distributeurs d'électroménager, etc.

Enfin, \gls{cacf} faisant partie du groupe Crédit Agricole, de nombreuses Caisses Régionales distribuent des crédits à la consommation à leur clientèle bancarisé, par l'intermédiaire des gestionnaires de compte.

Là encore, on constate que les spécificités des canaux de distribution des crédits impactent grandement la collecte des données et leur traitement statistique. En effet, les informations collectées sur le client, le produit et éventuellement l'apporteur d'affaires sont différentes selon le canal.

Dans la partie suivante, la méthodologie présentée est spécifique à \gls{cacf} ; il pourra néanmoins être admis que, dans les grandes lignes, cette méthodologie est similaire à la concurrence d'une part, et à la pratique d'autres pays (européens du moins) puisque la législation sur la protection et le traitement des données est sensiblement similaire (du fait de l'entrée en vigueur récente de la GDPR) et les établissements bancaires possèdent généralement des filiales dans plusieurs pays d'Europe et y font appliquer la même méthodologie.

% disclaimer
%Par ailleurs, les données réelles utilisées pour illustrer les méthodologies statistiques présentées dans ce manuscrit sont la propriété de \gls{cacf}.

\section{Le \textit{Credit Scoring} : état de l'art de la pratique industrielle} \label{chap1:sec2}

Cette partie vise à présenter la pratique actuelle en matière de \textit{Credit Scoring} et pose un certain nombre de questions statistiques dont certaines ont été traitées dans cette thèse, d'autres trouvent des réponses (parfois partielles) dans la litérature et dont certaines références sont données à titre informatif mais ne sont pas développées dans ce manuscrit ; enfin, certaines questions ne trouvent \textit{a priori} pas de réponse immédiate dans la litérature et sont autant de matière à de futurs travaux dans le domaine !

\subsection{Collecte des données}

La partie précédente a mis en exergue la pluralité des sources de données. La figure~\ref{fig:souscription} présente par exemple le formulaire de souscription en vigueur pour un crédit automobile auprès de Sofinco \textit{via} leur site web. Dans cet exemple, des données socio-démographiques et du véhicule à financer sont demandées. Pour un client, elles sont notées $\glssymbol{bx} = (\glssymbol{x}_j)_1^D$ dans la suite. Ces informations sont de nature continue $x_j \in \glssymbol{R}$ ou catégorielle $x_j \in \glssymbol{N}_{o_j}$.


\begin{figure}
\centering \includegraphics[width=15cm]{figures/chapitre1/souscription.png}
\caption{\label{fig:souscription} Formulaire de souscription d'un crédit automobile Sofinco.}
\end{figure}

% parler des degrés de certitude de l'information

% Montrer l'équivalence formulaire de souscription / alimentation des tables SAS

% Introduire x / données structurées

\subsection{Préparation des données}


% Parler de discrétisation comme d'une technique de gestion des manquants et des outliers + côté pratique et historique

% Sélection de variables basée sur les corrélations et l'expertise métier


\subsection{Critère à modéliser}

% Profitabilité du crédit

% Notion finance -> notion risque

% Temporalité du crédit

% Horizon risque

% Impayés consécutifs

% Passage en pertes

\subsection{Données d'apprentissage}

% Introduire x avec des trous

% Introduire y avec des trous

% Passage aux données complètes par sélection / discrétisation / suppression des non financés

\subsection{L'apprentissage d'une règle de décision}

% Régression logistique (et proc logistic SAS)

% Procédure stepwise

% Calibration des scores

\subsection{La métrique de performance}

% Gini et AUC

% Problèmes à l'utilisation de ces métriques

% Difficultées liées à l'utilisation d'une matrice de confusion

\subsection{Suivi temporel de la performance du score}

% Population drift

% Quand refondre une grille ?

% Un problème qui peut se réinterpréter comme de l'apprentissage par renforcement

\section{Apprentissage statistique} \label{chap1:sec3}


\subsection{Mécanisme de génération des données}


\subsection{Apprentissage semi-supervisé}


\subsection{Apprentissage supervisé}


\subsubsection{Sélection de variables}

% Nombreuse réf. + thèse Vidal

\subsubsection{Variable cible}

% SEME

% Autres références biblio ?

\subsubsection{Choix de modèle}



%%%%%%%%%%%%%%%%%%%%%%%%%%%%%%%%%%%%%%%%%%%%%%%%%%

Ce chapitre a permis .


\printbibliography[heading=subbibliography, title=Références du chapitre 1]
%
% Deuxième chapitre
\selectlanguage{english}

\chapter{Reject Inference} \label{chap2}

\epigraph{Sounds good, doesn't work.}{Donald J.\ Trump}

\minitoc

\textit{Nota Bene :} ce chapitre s'inspire fortement de l'article [...]

\bigskip




\section{Consumer loans: acceptance and financing process}



\begin{figure}[ht]
\begin{minipage}[b]{0.45\linewidth}
\center \includegraphics[width=5cm]{figures/chapitre2/schema.png}
\caption{Simplified Acceptance mechanism in~Crédit Agricole Consumer Finance}
\label{fig:figure1}

\end{minipage}%
\hfil \begin{minipage}[b]{0.5\linewidth}

\center \begin{tikzpicture}

    \foreach \start/\end/\middle/\percent/\anchor/\name in {
      0/90/30/financed/above/accepted,
      90/180/150/not-taken-up/above/accepted}
  {
    \draw[fill=green, thick] (0,0) -- (\end:2.4cm) arc (\end:\start:2.4cm)
      node at (\middle:1.3cm) {\percent};
    \draw (\middle:2.4cm) -- (\middle:3cm) node[\anchor] {\name};
  };
    
    \foreach \start/\end/\middle/\percent/\anchor/\name in {
      180/240/200/score/below/rejected,
      240/300/270/rules/below/rejected,
      300/360/340/operator/below/rejected}
  {
    \draw[fill=red, thick] (0,0) -- (\end:2.4cm) arc (\end:\start:2.4cm)
      node at (\middle:1.3cm) {\percent};
    \draw (\middle:2.4cm) -- (\middle:3cm) node[\anchor] {\name};
  };
\end{tikzpicture}
\caption{Simplified Acceptance status in Crédit Agricole Consumer Finance - scale relations not respected}
\label{fig:figure2}

\end{minipage}
\end{figure}



\printbibliography[heading=subbibliography, title=References of Chapter 2]
%
% Troisième chapitre
%\chapter{Target feature in \textit{Credit Scoring}}

% cf SEME


%
% Deuxième partie éventuelle
% \part{...}
%
% Quatrième chapitre
\chapter{Supervised multivariate quantization} \label{chap4}

\selectlanguage{english}

\epigraph{All models are wrong, but some are useful.}{Georges Box, ``Empirical Model-Building and Response Surfaces'', 1978.}

\minitoc

%\textit{Nota Bene :} Ce chapitre s'inspire fortement ... \textcolor{red}{à adapter au moment de l'envoi du manuscrit}

\bigskip

To improve prediction accuracy and interpretability of \gls{lr}-based scorecards, a preprocessing step quantizing both continuous and categorical data is usually performed: continuous features are discretized by assigning factor levels to intervals and, if numerous, levels of categorical features are grouped. However, a better predictive accuracy can be reached by embedding this quantization estimation step directly into the predictive estimation step itself. By doing so, the predictive loss has to be optimized on a huge and intractable discontinuous quantization set. To overcome this difficulty, we introduced a specific two-step optimization strategy: first, the optimization problem is relaxed by approximating discontinuous quantization functions by smooth functions; second, the resulting relaxed optimization problem is solved either \textit{via} a particular neural network and a stochastic gradient descent or an \gls{sem} algorithm. The strategy gives then access to good candidates for the original optimization problem after a straightforward \textit{maximum a posteriori} procedure to obtain cutpoints. The good performances of this approach, which we call \textit{glmdisc}, are illustrated on simulated and real data from the UCI library and \gls{cacf}. The results show that practitioners finally have an automatic all-in-one tool that answers their recurring needs of quantization for predictive tasks.
 
\section{Motivation} \label{sec:motivation}

As stated in~\cite{hosmer2013applied} and illustrated in this manuscript, in many application contexts (\textit{Credit Scoring}, biostatistics, {\it etc.}), \gls{lr} is widely used for its simplicity, decent performance and interpretability in predicting a binary outcome given predictors of different types (categorical, continuous). However, to achieve  higher interpretability, continuous predictors are sometimes discretized so as to produce a ``scorecard'', \textit{i.e.}\ a table assigning a grade to an applicant in \textit{Credit Scoring} (or a patient in biostatistics, {\it etc.}) depending on its predictors being in a given interval, as exemplified in Table~\ref{tab:ex_scorecard}.

\begin{table}
\centering
\caption{\label{tab:ex_scorecard} Example of a final scorecard on quantized data.}
\begin{tabular}{p{3cm}|p{3cm}|p{2cm}}
Feature & Level & Points \\
\hline
\hline
\multirow{3}{*}{Age} & 18-25 & 10 \\
 & 25-45 & 20 \\
 & 45-$+\infty$ & 30 \\
 \hline
\multirow{3}{*}{Wages} & $-\infty$-1000 & 15 \\
 & 1000-2000 & 25 \\
 & 2000-$+\infty$ & 35 \\
 \dots & \dots & \dots \\
\end{tabular}
\end{table}


Discretization is also an opportunity for reducing the (possibly large) modeling bias which can appear in \gls{lr} as a result of the linearity assumption on the continuous predictors in the model which was discussed in Section~\ref{chap1:sec3}. Indeed, this restriction can be overcome by approximating the true predictive mapping with a step function where the tuning of the steps and their sizes allows more flexibility. However, the resulting increase of the number of parameters can lead to an increase in variance (overfitting) as shown in \cite{yang2009discretization}. Thus, a precise tuning of the discretization procedure is required. Likewise when dealing with categorical features which take numerous levels, their respective regression coefficients suffer from high variance. A straightforward solution formalized by \cite{maj2015delete} is to merge their factor levels which leads to less coefficients and therefore less variance. We showcase this phenomenon on simple simulated data in the next section. On \textit{Credit Scoring} data, a typical example is the number of children (although not continuous strictly speaking). The log-odd ratio of clients' creditworthiness w.r.t.\ their number of children is often visually ``quadratic'', \textit{i.e.}\ the risk is lower for clients having 1 to 3 children, a bit higher for 0 child, and then it grows steadily with the number of children above 4. This can be visually represented by a spline, see~\ref{fig:nbenf_spline}. As using a spline is not very interpretable, this is not done in practice. Without quantizing the number of children, a linear relationship is assumed as displayed on Figure~\ref{fig:nbenf_cont}. When quantizing this feature, a piecewise constant relationship is assumed, see Figure~\ref{fig:nbenf_disc}. In this example, it is visually unclear which is best, such that there is a need to formalize the problem.

Another potential motivation for quantization is optimal data compression: as will be shown rigorously in subsequent sections, quantization aims at ``squeezing'' as much predictive information in the original features about the class as possible. Taking an informatics point of view, quantization of a continuous feature is equivalent to discarding a float column (taking \textit{e.g.}\ 32 bits per observation) by overwriting it with its quantized version (which would either be one column of unsigned 8 bits integers - ``interval'' encoding without order - or several 1 bit columns - one-hot / dummy encoding). The same thought process is applicable to quantizations of categorical features. In the end, the ``raw'' data can be compressed by a factor of $32 / 8 = 4$ without losing its predictive power, which, in an era of Big Data, is useful both in terms of data storage and of computational power to process these data since by 2040, the energy needs for calculations will exceed the global energy production (see~\cite{villani2018donner} p.\ 123).

From now on, the generic term quantization will stand for both discretization of continuous features and level grouping of categorical ones. Its aim is to improve the prediction accuracy. Such a quantization can be seen as a special case of \textit{representation learning}~\cite{bengio2013representation}, but suffers from a highly combinatorial optimization problem whatever the predictive criterion used to select the best quantization. The present work proposes a strategy to overcome these combinatorial issues by invoking a relaxed alternative of the initial quantization problem leading to a simpler estimation problem since it can be easily optimized by either a specific neural network or an \gls{sem} algorithm. These relaxed versions serve as a plausible quantization provider related to the initial criterion after a classical thresholding (\textit{maximum a posteriori}) procedure.

The outline of this chapter is the following. After some introductory examples, we illustrate cases where quantization is either beneficial or detrimental depending on the data generating mechanism. In the subsequent section, we formalize both continuous and categorical quantization. Selecting the best quantization in a predictive setting is reformulated as a model selection problem on a huge discrete space which size is precisely derived. In Section~\ref{sec:proposal}, a particular neural network architecture is used to optimize a relaxed version of this criterion and propose good quantization candidates. At first, an \gls{sem} procedure was proposed to solve the quantization problem and is reported in Section~\ref{sec:sem}. Section~\ref{sec:experiments} is dedicated to numerical experiments on both simulated and real data from the field of Credit Scoring, highlightening the good results offered by the use of the two new methods without any human intervention. A final section concludes the chapter by stating also new challenges.

\begin{figure}[!ht]
\begin{subfigure}[t]{.9\textwidth}
\centering \includegraphics[width = .8\textwidth]{figures/chapitre4/nbenf_spline.png}
\caption{\label{fig:nbenf_spline} Risk of \gls{cacf} clients w.r.t.\ their number of children and output of a spline regression.}
\end{subfigure}
\begin{subfigure}[t]{.9\textwidth}
\centering \includegraphics[width = .8\textwidth]{figures/chapitre4/nbenf_spline_cont.png}
\caption{\label{fig:nbenf_cont} When the \gls{lr} is used without quantization, it amounts to assuming the \textcolor{green}{green} linear relationship.}
\end{subfigure}
\begin{subfigure}[t]{.9\textwidth}
\centering \includegraphics[width = .8\textwidth]{figures/chapitre4/nbenf_spline_disc.png}
\caption{\label{fig:nbenf_disc} When the \gls{lr} is used with quantization, \textit{e.g.}\ more or less than 3 children, it amounts to assuming the risk is similar for all levels and equals the \textcolor{green}{green} steps.}
\end{subfigure}
\caption{Relationship of the creditworthiness of a client w.r.t.\ his / her number of children, all else being equal.}
\label{fig:splines}
\end{figure}


\begin{figure}[!ht]
\begin{subfigure}[t]{0.5\textwidth}
\centering \resizebox{\textwidth}{!}{% Created by tikzDevice version 0.12 on 2019-03-11 11:52:46
% !TEX encoding = UTF-8 Unicode
\begin{tikzpicture}[x=1pt,y=1pt]
\definecolor{fillColor}{RGB}{255,255,255}
\path[use as bounding box,fill=fillColor,fill opacity=0.00] (0,0) rectangle (505.89,505.89);
\begin{scope}
\path[clip] (  0.00,  0.00) rectangle (505.89,505.89);
\definecolor{drawColor}{RGB}{255,255,255}
\definecolor{fillColor}{RGB}{255,255,255}

\path[draw=drawColor,line width= 1.8pt,line join=round,line cap=round,fill=fillColor] (  0.00,  0.00) rectangle (505.89,505.89);
\end{scope}
\begin{scope}
\path[clip] ( 30.05, 48.75) rectangle (500.39,500.39);
\definecolor{fillColor}{RGB}{255,255,255}

\path[fill=fillColor] ( 30.05, 48.75) rectangle (500.39,500.39);
\definecolor{drawColor}{gray}{0.92}

\path[draw=drawColor,line width= 0.9pt,line join=round] ( 30.05,134.58) --
	(500.39,134.58);

\path[draw=drawColor,line width= 0.9pt,line join=round] ( 30.05,268.08) --
	(500.39,268.08);

\path[draw=drawColor,line width= 0.9pt,line join=round] ( 30.05,401.59) --
	(500.39,401.59);

\path[draw=drawColor,line width= 1.8pt,line join=round] ( 30.05, 67.82) --
	(500.39, 67.82);

\path[draw=drawColor,line width= 1.8pt,line join=round] ( 30.05,201.33) --
	(500.39,201.33);

\path[draw=drawColor,line width= 1.8pt,line join=round] ( 30.05,334.84) --
	(500.39,334.84);

\path[draw=drawColor,line width= 1.8pt,line join=round] ( 30.05,468.35) --
	(500.39,468.35);

\path[draw=drawColor,line width= 1.8pt,line join=round] ( 40.06, 48.75) --
	( 40.06,500.39);

\path[draw=drawColor,line width= 1.8pt,line join=round] ( 56.74, 48.75) --
	( 56.74,500.39);

\path[draw=drawColor,line width= 1.8pt,line join=round] ( 73.42, 48.75) --
	( 73.42,500.39);

\path[draw=drawColor,line width= 1.8pt,line join=round] ( 90.09, 48.75) --
	( 90.09,500.39);

\path[draw=drawColor,line width= 1.8pt,line join=round] (106.77, 48.75) --
	(106.77,500.39);

\path[draw=drawColor,line width= 1.8pt,line join=round] (123.45, 48.75) --
	(123.45,500.39);

\path[draw=drawColor,line width= 1.8pt,line join=round] (140.13, 48.75) --
	(140.13,500.39);

\path[draw=drawColor,line width= 1.8pt,line join=round] (156.81, 48.75) --
	(156.81,500.39);

\path[draw=drawColor,line width= 1.8pt,line join=round] (173.49, 48.75) --
	(173.49,500.39);

\path[draw=drawColor,line width= 1.8pt,line join=round] (190.17, 48.75) --
	(190.17,500.39);

\path[draw=drawColor,line width= 1.8pt,line join=round] (206.85, 48.75) --
	(206.85,500.39);

\path[draw=drawColor,line width= 1.8pt,line join=round] (223.52, 48.75) --
	(223.52,500.39);

\path[draw=drawColor,line width= 1.8pt,line join=round] (240.20, 48.75) --
	(240.20,500.39);

\path[draw=drawColor,line width= 1.8pt,line join=round] (256.88, 48.75) --
	(256.88,500.39);

\path[draw=drawColor,line width= 1.8pt,line join=round] (273.56, 48.75) --
	(273.56,500.39);

\path[draw=drawColor,line width= 1.8pt,line join=round] (290.24, 48.75) --
	(290.24,500.39);

\path[draw=drawColor,line width= 1.8pt,line join=round] (306.92, 48.75) --
	(306.92,500.39);

\path[draw=drawColor,line width= 1.8pt,line join=round] (323.60, 48.75) --
	(323.60,500.39);

\path[draw=drawColor,line width= 1.8pt,line join=round] (340.27, 48.75) --
	(340.27,500.39);

\path[draw=drawColor,line width= 1.8pt,line join=round] (356.95, 48.75) --
	(356.95,500.39);

\path[draw=drawColor,line width= 1.8pt,line join=round] (373.63, 48.75) --
	(373.63,500.39);

\path[draw=drawColor,line width= 1.8pt,line join=round] (390.31, 48.75) --
	(390.31,500.39);

\path[draw=drawColor,line width= 1.8pt,line join=round] (406.99, 48.75) --
	(406.99,500.39);

\path[draw=drawColor,line width= 1.8pt,line join=round] (423.67, 48.75) --
	(423.67,500.39);

\path[draw=drawColor,line width= 1.8pt,line join=round] (440.35, 48.75) --
	(440.35,500.39);

\path[draw=drawColor,line width= 1.8pt,line join=round] (457.03, 48.75) --
	(457.03,500.39);

\path[draw=drawColor,line width= 1.8pt,line join=round] (473.70, 48.75) --
	(473.70,500.39);

\path[draw=drawColor,line width= 1.8pt,line join=round] (490.38, 48.75) --
	(490.38,500.39);
\definecolor{drawColor}{RGB}{0,0,0}

\path[draw=drawColor,line width= 1.8pt,line join=round] (239.37,247.82) --
	(241.04,247.82);

\path[draw=drawColor,line width= 1.8pt,line join=round] (240.20,247.82) --
	(240.20,192.71);

\path[draw=drawColor,line width= 1.8pt,line join=round] (239.37,192.71) --
	(241.04,192.71);

\path[draw=drawColor,line width= 1.8pt,line join=round] (356.12,234.41) --
	(357.79,234.41);

\path[draw=drawColor,line width= 1.8pt,line join=round] (356.95,234.41) --
	(356.95,186.76);

\path[draw=drawColor,line width= 1.8pt,line join=round] (356.12,186.76) --
	(357.79,186.76);

\path[draw=drawColor,line width= 1.8pt,line join=round] (105.94,314.77) --
	(107.61,314.77);

\path[draw=drawColor,line width= 1.8pt,line join=round] (106.77,314.77) --
	(106.77,203.67);

\path[draw=drawColor,line width= 1.8pt,line join=round] (105.94,203.67) --
	(107.61,203.67);

\path[draw=drawColor,line width= 1.8pt,line join=round] (372.80,237.67) --
	(374.47,237.67);

\path[draw=drawColor,line width= 1.8pt,line join=round] (373.63,237.67) --
	(373.63,191.07);

\path[draw=drawColor,line width= 1.8pt,line join=round] (372.80,191.07) --
	(374.47,191.07);

\path[draw=drawColor,line width= 1.8pt,line join=round] (339.44,233.68) --
	(341.11,233.68);

\path[draw=drawColor,line width= 1.8pt,line join=round] (340.27,233.68) --
	(340.27,183.07);

\path[draw=drawColor,line width= 1.8pt,line join=round] (339.44,183.07) --
	(341.11,183.07);

\path[draw=drawColor,line width= 1.8pt,line join=round] (172.65,268.58) --
	(174.32,268.58);

\path[draw=drawColor,line width= 1.8pt,line join=round] (173.49,268.58) --
	(173.49,179.67);

\path[draw=drawColor,line width= 1.8pt,line join=round] (172.65,179.67) --
	(174.32,179.67);

\path[draw=drawColor,line width= 1.8pt,line join=round] (439.51,253.90) --
	(441.18,253.90);

\path[draw=drawColor,line width= 1.8pt,line join=round] (440.35,253.90) --
	(440.35,128.85);

\path[draw=drawColor,line width= 1.8pt,line join=round] (439.51,128.85) --
	(441.18,128.85);

\path[draw=drawColor,line width= 1.8pt,line join=round] (289.40,245.02) --
	(291.07,245.02);

\path[draw=drawColor,line width= 1.8pt,line join=round] (290.24,245.02) --
	(290.24,195.15);

\path[draw=drawColor,line width= 1.8pt,line join=round] (289.40,195.15) --
	(291.07,195.15);

\path[draw=drawColor,line width= 1.8pt,line join=round] (256.05,280.02) --
	(257.72,280.02);

\path[draw=drawColor,line width= 1.8pt,line join=round] (256.88,280.02) --
	(256.88,220.93);

\path[draw=drawColor,line width= 1.8pt,line join=round] (256.05,220.93) --
	(257.72,220.93);

\path[draw=drawColor,line width= 1.8pt,line join=round] ( 55.90,226.67) --
	( 57.57,226.67);

\path[draw=drawColor,line width= 1.8pt,line join=round] ( 56.74,226.67) --
	( 56.74,143.54);

\path[draw=drawColor,line width= 1.8pt,line join=round] ( 55.90,143.54) --
	( 57.57,143.54);

\path[draw=drawColor,line width= 1.8pt,line join=round] (306.08,275.80) --
	(307.75,275.80);

\path[draw=drawColor,line width= 1.8pt,line join=round] (306.92,275.80) --
	(306.92,201.53);

\path[draw=drawColor,line width= 1.8pt,line join=round] (306.08,201.53) --
	(307.75,201.53);

\path[draw=drawColor,line width= 1.8pt,line join=round] (322.76,260.22) --
	(324.43,260.22);

\path[draw=drawColor,line width= 1.8pt,line join=round] (323.60,260.22) --
	(323.60,202.05);

\path[draw=drawColor,line width= 1.8pt,line join=round] (322.76,202.05) --
	(324.43,202.05);

\path[draw=drawColor,line width= 1.8pt,line join=round] ( 89.26,240.48) --
	( 90.93,240.48);

\path[draw=drawColor,line width= 1.8pt,line join=round] ( 90.09,240.48) --
	( 90.09,173.73);

\path[draw=drawColor,line width= 1.8pt,line join=round] ( 89.26,173.73) --
	( 90.93,173.73);

\path[draw=drawColor,line width= 1.8pt,line join=round] (422.83,249.20) --
	(424.50,249.20);

\path[draw=drawColor,line width= 1.8pt,line join=round] (423.67,249.20) --
	(423.67, 97.43);

\path[draw=drawColor,line width= 1.8pt,line join=round] (422.83, 97.43) --
	(424.50, 97.43);

\path[draw=drawColor,line width= 1.8pt,line join=round] (272.73,234.60) --
	(274.39,234.60);

\path[draw=drawColor,line width= 1.8pt,line join=round] (273.56,234.60) --
	(273.56,174.25);

\path[draw=drawColor,line width= 1.8pt,line join=round] (272.73,174.25) --
	(274.39,174.25);

\path[draw=drawColor,line width= 1.8pt,line join=round] (139.30,335.85) --
	(140.96,335.85);

\path[draw=drawColor,line width= 1.8pt,line join=round] (140.13,335.85) --
	(140.13,166.73);

\path[draw=drawColor,line width= 1.8pt,line join=round] (139.30,166.73) --
	(140.96,166.73);

\path[draw=drawColor,line width= 1.8pt,line join=round] (389.48,219.90) --
	(391.14,219.90);

\path[draw=drawColor,line width= 1.8pt,line join=round] (390.31,219.90) --
	(390.31,167.91);

\path[draw=drawColor,line width= 1.8pt,line join=round] (389.48,167.91) --
	(391.14,167.91);

\path[draw=drawColor,line width= 1.8pt,line join=round] (189.33,363.13) --
	(191.00,363.13);

\path[draw=drawColor,line width= 1.8pt,line join=round] (190.17,363.13) --
	(190.17,195.23);

\path[draw=drawColor,line width= 1.8pt,line join=round] (189.33,195.23) --
	(191.00,195.23);

\path[draw=drawColor,line width= 1.8pt,line join=round] (122.62,352.06) --
	(124.29,352.06);

\path[draw=drawColor,line width= 1.8pt,line join=round] (123.45,352.06) --
	(123.45,147.31);

\path[draw=drawColor,line width= 1.8pt,line join=round] (122.62,147.31) --
	(124.29,147.31);

\path[draw=drawColor,line width= 1.8pt,line join=round] (206.01,360.21) --
	(207.68,360.21);

\path[draw=drawColor,line width= 1.8pt,line join=round] (206.85,360.21) --
	(206.85,151.65);

\path[draw=drawColor,line width= 1.8pt,line join=round] (206.01,151.65) --
	(207.68,151.65);

\path[draw=drawColor,line width= 1.8pt,line join=round] (222.69,304.30) --
	(224.36,304.30);

\path[draw=drawColor,line width= 1.8pt,line join=round] (223.52,304.30) --
	(223.52,191.96);

\path[draw=drawColor,line width= 1.8pt,line join=round] (222.69,191.96) --
	(224.36,191.96);

\path[draw=drawColor,line width= 1.8pt,line join=round] (406.16,222.39) --
	(407.82,222.39);

\path[draw=drawColor,line width= 1.8pt,line join=round] (406.99,222.39) --
	(406.99,124.41);

\path[draw=drawColor,line width= 1.8pt,line join=round] (406.16,124.41) --
	(407.82,124.41);

\path[draw=drawColor,line width= 1.8pt,line join=round] ( 72.58,267.52) --
	( 74.25,267.52);

\path[draw=drawColor,line width= 1.8pt,line join=round] ( 73.42,267.52) --
	( 73.42,159.62);

\path[draw=drawColor,line width= 1.8pt,line join=round] ( 72.58,159.62) --
	( 74.25,159.62);

\path[draw=drawColor,line width= 1.8pt,line join=round] (155.98,388.62) --
	(157.64,388.62);

\path[draw=drawColor,line width= 1.8pt,line join=round] (156.81,388.62) --
	(156.81,166.28);

\path[draw=drawColor,line width= 1.8pt,line join=round] (155.98,166.28) --
	(157.64,166.28);

\path[draw=drawColor,line width= 1.8pt,line join=round] (456.19,479.86) --
	(457.86,479.86);

\path[draw=drawColor,line width= 1.8pt,line join=round] (457.03,479.86) --
	(457.03,179.07);

\path[draw=drawColor,line width= 1.8pt,line join=round] (456.19,179.07) --
	(457.86,179.07);

\path[draw=drawColor,line width= 1.8pt,line join=round] ( 39.22,466.63) --
	( 40.89,466.63);

\path[draw=drawColor,line width= 1.8pt,line join=round] ( 40.06,466.63) --
	( 40.06,167.09);

\path[draw=drawColor,line width= 1.8pt,line join=round] ( 39.22,167.09) --
	( 40.89,167.09);

\path[draw=drawColor,line width= 1.8pt,line join=round] (489.55,272.67) --
	(491.22,272.67);

\path[draw=drawColor,line width= 1.8pt,line join=round] (490.38,272.67) --
	(490.38,184.79);

\path[draw=drawColor,line width= 1.8pt,line join=round] (489.55,184.79) --
	(491.22,184.79);

\path[draw=drawColor,line width= 1.8pt,line join=round] (472.87,377.02) --
	(474.54,377.02);

\path[draw=drawColor,line width= 1.8pt,line join=round] (473.70,377.02) --
	(473.70, 69.28);

\path[draw=drawColor,line width= 1.8pt,line join=round] (472.87, 69.28) --
	(474.54, 69.28);
\definecolor{fillColor}{RGB}{0,0,0}

\path[draw=drawColor,line width= 1.2pt,line join=round,line cap=round,fill=fillColor] (240.20,220.27) circle (  1.96);

\path[draw=drawColor,line width= 1.2pt,line join=round,line cap=round,fill=fillColor] (356.95,210.59) circle (  1.96);

\path[draw=drawColor,line width= 1.2pt,line join=round,line cap=round,fill=fillColor] (106.77,259.22) circle (  1.96);

\path[draw=drawColor,line width= 1.2pt,line join=round,line cap=round,fill=fillColor] (373.63,214.37) circle (  1.96);

\path[draw=drawColor,line width= 1.2pt,line join=round,line cap=round,fill=fillColor] (340.27,208.38) circle (  1.96);

\path[draw=drawColor,line width= 1.2pt,line join=round,line cap=round,fill=fillColor] (173.49,224.12) circle (  1.96);

\path[draw=drawColor,line width= 1.2pt,line join=round,line cap=round,fill=fillColor] (440.35,191.38) circle (  1.96);

\path[draw=drawColor,line width= 1.2pt,line join=round,line cap=round,fill=fillColor] (290.24,220.09) circle (  1.96);

\path[draw=drawColor,line width= 1.2pt,line join=round,line cap=round,fill=fillColor] (256.88,250.48) circle (  1.96);

\path[draw=drawColor,line width= 1.2pt,line join=round,line cap=round,fill=fillColor] ( 56.74,185.11) circle (  1.96);

\path[draw=drawColor,line width= 1.2pt,line join=round,line cap=round,fill=fillColor] (306.92,238.67) circle (  1.96);

\path[draw=drawColor,line width= 1.2pt,line join=round,line cap=round,fill=fillColor] (323.60,231.13) circle (  1.96);

\path[draw=drawColor,line width= 1.2pt,line join=round,line cap=round,fill=fillColor] ( 90.09,207.10) circle (  1.96);

\path[draw=drawColor,line width= 1.2pt,line join=round,line cap=round,fill=fillColor] (423.67,173.32) circle (  1.96);

\path[draw=drawColor,line width= 1.2pt,line join=round,line cap=round,fill=fillColor] (273.56,204.43) circle (  1.96);

\path[draw=drawColor,line width= 1.2pt,line join=round,line cap=round,fill=fillColor] (140.13,251.29) circle (  1.96);

\path[draw=drawColor,line width= 1.2pt,line join=round,line cap=round,fill=fillColor] (390.31,193.90) circle (  1.96);

\path[draw=drawColor,line width= 1.2pt,line join=round,line cap=round,fill=fillColor] (190.17,279.18) circle (  1.96);

\path[draw=drawColor,line width= 1.2pt,line join=round,line cap=round,fill=fillColor] (123.45,249.69) circle (  1.96);

\path[draw=drawColor,line width= 1.2pt,line join=round,line cap=round,fill=fillColor] (206.85,255.93) circle (  1.96);

\path[draw=drawColor,line width= 1.2pt,line join=round,line cap=round,fill=fillColor] (223.52,248.13) circle (  1.96);

\path[draw=drawColor,line width= 1.2pt,line join=round,line cap=round,fill=fillColor] (406.99,173.40) circle (  1.96);

\path[draw=drawColor,line width= 1.2pt,line join=round,line cap=round,fill=fillColor] ( 73.42,213.57) circle (  1.96);

\path[draw=drawColor,line width= 1.2pt,line join=round,line cap=round,fill=fillColor] (156.81,277.45) circle (  1.96);

\path[draw=drawColor,line width= 1.2pt,line join=round,line cap=round,fill=fillColor] (457.03,329.46) circle (  1.96);

\path[draw=drawColor,line width= 1.2pt,line join=round,line cap=round,fill=fillColor] ( 40.06,316.86) circle (  1.96);

\path[draw=drawColor,line width= 1.2pt,line join=round,line cap=round,fill=fillColor] (490.38,228.73) circle (  1.96);

\path[draw=drawColor,line width= 1.2pt,line join=round,line cap=round,fill=fillColor] (473.70,223.15) circle (  1.96);
\definecolor{drawColor}{RGB}{255,0,0}

\path[draw=drawColor,line width= 1.8pt,line join=round] ( 30.05,201.33) -- (500.39,201.33);
\definecolor{drawColor}{gray}{0.20}

\path[draw=drawColor,line width= 1.8pt,line join=round,line cap=round] ( 30.05, 48.75) rectangle (500.39,500.39);
\end{scope}
\begin{scope}
\path[clip] (  0.00,  0.00) rectangle (505.89,505.89);
\definecolor{drawColor}{gray}{0.30}

\node[text=drawColor,anchor=base east,inner sep=0pt, outer sep=0pt, scale=  0.88] at ( 25.10, 64.79) {-2};

\node[text=drawColor,anchor=base east,inner sep=0pt, outer sep=0pt, scale=  0.88] at ( 25.10,198.30) {0};

\node[text=drawColor,anchor=base east,inner sep=0pt, outer sep=0pt, scale=  0.88] at ( 25.10,331.81) {2};

\node[text=drawColor,anchor=base east,inner sep=0pt, outer sep=0pt, scale=  0.88] at ( 25.10,465.32) {4};
\end{scope}
\begin{scope}
\path[clip] (  0.00,  0.00) rectangle (505.89,505.89);
\definecolor{drawColor}{gray}{0.20}

\path[draw=drawColor,line width= 1.8pt,line join=round] ( 27.30, 67.82) --
	( 30.05, 67.82);

\path[draw=drawColor,line width= 1.8pt,line join=round] ( 27.30,201.33) --
	( 30.05,201.33);

\path[draw=drawColor,line width= 1.8pt,line join=round] ( 27.30,334.84) --
	( 30.05,334.84);

\path[draw=drawColor,line width= 1.8pt,line join=round] ( 27.30,468.35) --
	( 30.05,468.35);
\end{scope}
\begin{scope}
\path[clip] (  0.00,  0.00) rectangle (505.89,505.89);
\definecolor{drawColor}{gray}{0.20}

\path[draw=drawColor,line width= 1.8pt,line join=round] ( 40.06, 46.00) --
	( 40.06, 48.75);

\path[draw=drawColor,line width= 1.8pt,line join=round] ( 56.74, 46.00) --
	( 56.74, 48.75);

\path[draw=drawColor,line width= 1.8pt,line join=round] ( 73.42, 46.00) --
	( 73.42, 48.75);

\path[draw=drawColor,line width= 1.8pt,line join=round] ( 90.09, 46.00) --
	( 90.09, 48.75);

\path[draw=drawColor,line width= 1.8pt,line join=round] (106.77, 46.00) --
	(106.77, 48.75);

\path[draw=drawColor,line width= 1.8pt,line join=round] (123.45, 46.00) --
	(123.45, 48.75);

\path[draw=drawColor,line width= 1.8pt,line join=round] (140.13, 46.00) --
	(140.13, 48.75);

\path[draw=drawColor,line width= 1.8pt,line join=round] (156.81, 46.00) --
	(156.81, 48.75);

\path[draw=drawColor,line width= 1.8pt,line join=round] (173.49, 46.00) --
	(173.49, 48.75);

\path[draw=drawColor,line width= 1.8pt,line join=round] (190.17, 46.00) --
	(190.17, 48.75);

\path[draw=drawColor,line width= 1.8pt,line join=round] (206.85, 46.00) --
	(206.85, 48.75);

\path[draw=drawColor,line width= 1.8pt,line join=round] (223.52, 46.00) --
	(223.52, 48.75);

\path[draw=drawColor,line width= 1.8pt,line join=round] (240.20, 46.00) --
	(240.20, 48.75);

\path[draw=drawColor,line width= 1.8pt,line join=round] (256.88, 46.00) --
	(256.88, 48.75);

\path[draw=drawColor,line width= 1.8pt,line join=round] (273.56, 46.00) --
	(273.56, 48.75);

\path[draw=drawColor,line width= 1.8pt,line join=round] (290.24, 46.00) --
	(290.24, 48.75);

\path[draw=drawColor,line width= 1.8pt,line join=round] (306.92, 46.00) --
	(306.92, 48.75);

\path[draw=drawColor,line width= 1.8pt,line join=round] (323.60, 46.00) --
	(323.60, 48.75);

\path[draw=drawColor,line width= 1.8pt,line join=round] (340.27, 46.00) --
	(340.27, 48.75);

\path[draw=drawColor,line width= 1.8pt,line join=round] (356.95, 46.00) --
	(356.95, 48.75);

\path[draw=drawColor,line width= 1.8pt,line join=round] (373.63, 46.00) --
	(373.63, 48.75);

\path[draw=drawColor,line width= 1.8pt,line join=round] (390.31, 46.00) --
	(390.31, 48.75);

\path[draw=drawColor,line width= 1.8pt,line join=round] (406.99, 46.00) --
	(406.99, 48.75);

\path[draw=drawColor,line width= 1.8pt,line join=round] (423.67, 46.00) --
	(423.67, 48.75);

\path[draw=drawColor,line width= 1.8pt,line join=round] (440.35, 46.00) --
	(440.35, 48.75);

\path[draw=drawColor,line width= 1.8pt,line join=round] (457.03, 46.00) --
	(457.03, 48.75);

\path[draw=drawColor,line width= 1.8pt,line join=round] (473.70, 46.00) --
	(473.70, 48.75);

\path[draw=drawColor,line width= 1.8pt,line join=round] (490.38, 46.00) --
	(490.38, 48.75);
\end{scope}
\begin{scope}
\path[clip] (  0.00,  0.00) rectangle (505.89,505.89);
\definecolor{drawColor}{gray}{0.30}

\node[text=drawColor,rotate= 90.00,anchor=base east,inner sep=0pt, outer sep=0pt, scale=  0.88] at ( 46.12, 43.80) {CSP10};

\node[text=drawColor,rotate= 90.00,anchor=base east,inner sep=0pt, outer sep=0pt, scale=  0.88] at ( 62.80, 43.80) {CSP21};

\node[text=drawColor,rotate= 90.00,anchor=base east,inner sep=0pt, outer sep=0pt, scale=  0.88] at ( 79.48, 43.80) {CSP22};

\node[text=drawColor,rotate= 90.00,anchor=base east,inner sep=0pt, outer sep=0pt, scale=  0.88] at ( 96.16, 43.80) {CSP23};

\node[text=drawColor,rotate= 90.00,anchor=base east,inner sep=0pt, outer sep=0pt, scale=  0.88] at (112.83, 43.80) {CSP31};

\node[text=drawColor,rotate= 90.00,anchor=base east,inner sep=0pt, outer sep=0pt, scale=  0.88] at (129.51, 43.80) {CSP32};

\node[text=drawColor,rotate= 90.00,anchor=base east,inner sep=0pt, outer sep=0pt, scale=  0.88] at (146.19, 43.80) {CSP34};

\node[text=drawColor,rotate= 90.00,anchor=base east,inner sep=0pt, outer sep=0pt, scale=  0.88] at (162.87, 43.80) {CSP35};

\node[text=drawColor,rotate= 90.00,anchor=base east,inner sep=0pt, outer sep=0pt, scale=  0.88] at (179.55, 43.80) {CSP36};

\node[text=drawColor,rotate= 90.00,anchor=base east,inner sep=0pt, outer sep=0pt, scale=  0.88] at (196.23, 43.80) {CSP42};

\node[text=drawColor,rotate= 90.00,anchor=base east,inner sep=0pt, outer sep=0pt, scale=  0.88] at (212.91, 43.80) {CSP43};

\node[text=drawColor,rotate= 90.00,anchor=base east,inner sep=0pt, outer sep=0pt, scale=  0.88] at (229.58, 43.80) {CSP45};

\node[text=drawColor,rotate= 90.00,anchor=base east,inner sep=0pt, outer sep=0pt, scale=  0.88] at (246.26, 43.80) {CSP46};

\node[text=drawColor,rotate= 90.00,anchor=base east,inner sep=0pt, outer sep=0pt, scale=  0.88] at (262.94, 43.80) {CSP47};

\node[text=drawColor,rotate= 90.00,anchor=base east,inner sep=0pt, outer sep=0pt, scale=  0.88] at (279.62, 43.80) {CSP48};

\node[text=drawColor,rotate= 90.00,anchor=base east,inner sep=0pt, outer sep=0pt, scale=  0.88] at (296.30, 43.80) {CSP52};

\node[text=drawColor,rotate= 90.00,anchor=base east,inner sep=0pt, outer sep=0pt, scale=  0.88] at (312.98, 43.80) {CSP53};

\node[text=drawColor,rotate= 90.00,anchor=base east,inner sep=0pt, outer sep=0pt, scale=  0.88] at (329.66, 43.80) {CSP54};

\node[text=drawColor,rotate= 90.00,anchor=base east,inner sep=0pt, outer sep=0pt, scale=  0.88] at (346.34, 43.80) {CSP55};

\node[text=drawColor,rotate= 90.00,anchor=base east,inner sep=0pt, outer sep=0pt, scale=  0.88] at (363.01, 43.80) {CSP56};

\node[text=drawColor,rotate= 90.00,anchor=base east,inner sep=0pt, outer sep=0pt, scale=  0.88] at (379.69, 43.80) {CSP60};

\node[text=drawColor,rotate= 90.00,anchor=base east,inner sep=0pt, outer sep=0pt, scale=  0.88] at (396.37, 43.80) {CSP64};

\node[text=drawColor,rotate= 90.00,anchor=base east,inner sep=0pt, outer sep=0pt, scale=  0.88] at (413.05, 43.80) {CSP69};

\node[text=drawColor,rotate= 90.00,anchor=base east,inner sep=0pt, outer sep=0pt, scale=  0.88] at (429.73, 43.80) {CSP81};

\node[text=drawColor,rotate= 90.00,anchor=base east,inner sep=0pt, outer sep=0pt, scale=  0.88] at (446.41, 43.80) {CSP82};

\node[text=drawColor,rotate= 90.00,anchor=base east,inner sep=0pt, outer sep=0pt, scale=  0.88] at (463.09, 43.80) {CSP84};

\node[text=drawColor,rotate= 90.00,anchor=base east,inner sep=0pt, outer sep=0pt, scale=  0.88] at (479.76, 43.80) {CSP85};

\node[text=drawColor,rotate= 90.00,anchor=base east,inner sep=0pt, outer sep=0pt, scale=  0.88] at (496.44, 43.80) {CSP87};
\end{scope}
\begin{scope}
\path[clip] (  0.00,  0.00) rectangle (505.89,505.89);
\definecolor{drawColor}{RGB}{0,0,0}

\node[text=drawColor,anchor=base,inner sep=0pt, outer sep=0pt, scale=  1.10] at (265.22,  7.44) {CSP levels};
\end{scope}
\begin{scope}
\path[clip] (  0.00,  0.00) rectangle (505.89,505.89);
\definecolor{drawColor}{RGB}{0,0,0}

\node[text=drawColor,rotate= 90.00,anchor=base,inner sep=0pt, outer sep=0pt, scale=  1.10] at ( 13.08,274.57) {Associated LR coefficient};
\end{scope}
\end{tikzpicture}
}
\caption{Having a lot of levels means having lots of coefficients, few of which are significant.}
\label{fig:csp_estim}
\end{subfigure}
\begin{subfigure}[t]{0.5\textwidth}
\centering \resizebox{\textwidth}{!}{% Created by tikzDevice version 0.12 on 2019-03-11 09:44:28
% !TEX encoding = UTF-8 Unicode
\begin{tikzpicture}[x=1pt,y=1pt]
\definecolor{fillColor}{RGB}{255,255,255}
\path[use as bounding box,fill=fillColor,fill opacity=0.00] (0,0) rectangle (505.89,505.89);
\begin{scope}
\path[clip] (  0.00,  0.00) rectangle (505.89,505.89);
\definecolor{drawColor}{RGB}{255,255,255}
\definecolor{fillColor}{RGB}{255,255,255}

\path[draw=drawColor,line width= 1.8pt,line join=round,line cap=round,fill=fillColor] (  0.00,  0.00) rectangle (505.89,505.89);
\end{scope}
\begin{scope}
\path[clip] ( 38.36, 44.35) rectangle (500.39,500.39);
\definecolor{fillColor}{RGB}{255,255,255}

\path[fill=fillColor] ( 38.36, 44.35) rectangle (500.39,500.39);
\definecolor{drawColor}{gray}{0.92}

\path[draw=drawColor,line width= 0.9pt,line join=round] ( 38.36, 60.90) --
	(500.39, 60.90);

\path[draw=drawColor,line width= 0.9pt,line join=round] ( 38.36,166.79) --
	(500.39,166.79);

\path[draw=drawColor,line width= 0.9pt,line join=round] ( 38.36,272.69) --
	(500.39,272.69);

\path[draw=drawColor,line width= 0.9pt,line join=round] ( 38.36,378.58) --
	(500.39,378.58);

\path[draw=drawColor,line width= 0.9pt,line join=round] ( 38.36,484.48) --
	(500.39,484.48);

\path[draw=drawColor,line width= 1.8pt,line join=round] ( 38.36,113.85) --
	(500.39,113.85);

\path[draw=drawColor,line width= 1.8pt,line join=round] ( 38.36,219.74) --
	(500.39,219.74);

\path[draw=drawColor,line width= 1.8pt,line join=round] ( 38.36,325.64) --
	(500.39,325.64);

\path[draw=drawColor,line width= 1.8pt,line join=round] ( 38.36,431.53) --
	(500.39,431.53);

\path[draw=drawColor,line width= 1.8pt,line join=round] (124.99, 44.35) --
	(124.99,500.39);

\path[draw=drawColor,line width= 1.8pt,line join=round] (269.38, 44.35) --
	(269.38,500.39);

\path[draw=drawColor,line width= 1.8pt,line join=round] (413.76, 44.35) --
	(413.76,500.39);
\definecolor{drawColor}{RGB}{0,0,0}

\path[draw=drawColor,line width= 1.8pt,line join=round] (117.77,479.66) --
	(132.21,479.66);

\path[draw=drawColor,line width= 1.8pt,line join=round] (124.99,479.66) --
	(124.99,150.57);

\path[draw=drawColor,line width= 1.8pt,line join=round] (117.77,150.57) --
	(132.21,150.57);

\path[draw=drawColor,line width= 1.8pt,line join=round] (262.16,258.96) --
	(276.59,258.96);

\path[draw=drawColor,line width= 1.8pt,line join=round] (269.38,258.96) --
	(269.38, 65.08);

\path[draw=drawColor,line width= 1.8pt,line join=round] (262.16, 65.08) --
	(276.59, 65.08);

\path[draw=drawColor,line width= 1.8pt,line join=round] (406.54,438.77) --
	(420.98,438.77);

\path[draw=drawColor,line width= 1.8pt,line join=round] (413.76,438.77) --
	(413.76,170.98);

\path[draw=drawColor,line width= 1.8pt,line join=round] (406.54,170.98) --
	(420.98,170.98);
\definecolor{fillColor}{RGB}{0,0,0}

\path[draw=drawColor,line width= 1.2pt,line join=round,line cap=round,fill=fillColor] (124.99,315.11) circle (  1.96);

\path[draw=drawColor,line width= 1.2pt,line join=round,line cap=round,fill=fillColor] (269.38,162.02) circle (  1.96);

\path[draw=drawColor,line width= 1.2pt,line join=round,line cap=round,fill=fillColor] (413.76,304.87) circle (  1.96);
\definecolor{drawColor}{RGB}{255,0,0}

\path[draw=drawColor,line width= 1.8pt,line join=round] ( 38.36,113.85) -- (500.39,113.85);
\definecolor{drawColor}{gray}{0.20}

\path[draw=drawColor,line width= 1.8pt,line join=round,line cap=round] ( 38.36, 44.35) rectangle (500.39,500.39);
\end{scope}
\begin{scope}
\path[clip] (  0.00,  0.00) rectangle (505.89,505.89);
\definecolor{drawColor}{gray}{0.30}

\node[text=drawColor,anchor=base east,inner sep=0pt, outer sep=0pt, scale=  0.88] at ( 33.41,110.82) {0.00};

\node[text=drawColor,anchor=base east,inner sep=0pt, outer sep=0pt, scale=  0.88] at ( 33.41,216.71) {0.25};

\node[text=drawColor,anchor=base east,inner sep=0pt, outer sep=0pt, scale=  0.88] at ( 33.41,322.61) {0.50};

\node[text=drawColor,anchor=base east,inner sep=0pt, outer sep=0pt, scale=  0.88] at ( 33.41,428.50) {0.75};
\end{scope}
\begin{scope}
\path[clip] (  0.00,  0.00) rectangle (505.89,505.89);
\definecolor{drawColor}{gray}{0.20}

\path[draw=drawColor,line width= 1.8pt,line join=round] ( 35.61,113.85) --
	( 38.36,113.85);

\path[draw=drawColor,line width= 1.8pt,line join=round] ( 35.61,219.74) --
	( 38.36,219.74);

\path[draw=drawColor,line width= 1.8pt,line join=round] ( 35.61,325.64) --
	( 38.36,325.64);

\path[draw=drawColor,line width= 1.8pt,line join=round] ( 35.61,431.53) --
	( 38.36,431.53);
\end{scope}
\begin{scope}
\path[clip] (  0.00,  0.00) rectangle (505.89,505.89);
\definecolor{drawColor}{gray}{0.20}

\path[draw=drawColor,line width= 1.8pt,line join=round] (124.99, 41.60) --
	(124.99, 44.35);

\path[draw=drawColor,line width= 1.8pt,line join=round] (269.38, 41.60) --
	(269.38, 44.35);

\path[draw=drawColor,line width= 1.8pt,line join=round] (413.76, 41.60) --
	(413.76, 44.35);
\end{scope}
\begin{scope}
\path[clip] (  0.00,  0.00) rectangle (505.89,505.89);
\definecolor{drawColor}{gray}{0.30}

\node[text=drawColor,rotate= 90.00,anchor=base east,inner sep=0pt, outer sep=0pt, scale=  0.88] at (131.05, 39.40) {CSP2};

\node[text=drawColor,rotate= 90.00,anchor=base east,inner sep=0pt, outer sep=0pt, scale=  0.88] at (275.44, 39.40) {CSP3};

\node[text=drawColor,rotate= 90.00,anchor=base east,inner sep=0pt, outer sep=0pt, scale=  0.88] at (419.82, 39.40) {CSP4};
\end{scope}
\begin{scope}
\path[clip] (  0.00,  0.00) rectangle (505.89,505.89);
\definecolor{drawColor}{RGB}{0,0,0}

\node[text=drawColor,anchor=base,inner sep=0pt, outer sep=0pt, scale=  1.10] at (269.38,  7.44) {CSP levels};
\end{scope}
\begin{scope}
\path[clip] (  0.00,  0.00) rectangle (505.89,505.89);
\definecolor{drawColor}{RGB}{0,0,0}

\node[text=drawColor,rotate= 90.00,anchor=base,inner sep=0pt, outer sep=0pt, scale=  1.10] at ( 13.08,272.37) {Associated LR coefficient};
\end{scope}
\end{tikzpicture}
}
\caption{By grouping levels, fewer coefficients are obtained, which variance is significantly smaller and are thus significant.}
\label{fig:csp_estim_disc}
\end{subfigure}
\caption{\gls{lr} coefficients of the levels of the job of borrowers.}
\label{fig:csp}
\end{figure}

\section{Illustration of the bias-variance quantization tradeoff} \label{sec:bias_variance_quant}
 

The previous section motivated the use of quantization on a practical level. On a theoretical level, at least in terms of probability theory, quantization is equivalent to throwing away information: for continuous features, it is only known that they belong to a certain interval and for categorical features, their granularity among the original levels is lost.

However, two things must appear clearly: first, we are in a ``statistical'' setting, \textit{i.e.}\ finite-dimensional setting, where variance of estimation can play a big role, as was developed in Section~\ref{subsec:gradient}, which partly justifies the need to regroup categorical levels. Second, we are in a predictive setting, with an imposed classification model $p_{\gls{bth}}$. We focus on \gls{lr}, for which continuous features get a single coefficient: their relationship with the logit transform of the probability of an event (bad borrower) is assumed to be linear which can yield model bias. Thus, having several coefficients per feature, which can be achieved with a  variety of techniques (\textit{e.g.}\ splines), can yield a lower model bias (when the true model is not linear, which is generally the case for \textit{Credit Scoring} data) at the cost of increased variance of estimation.

This phenomenon can be very simply captured by a small simulation: in the misspecified model setting, where the logit transform is assumed to stem from a sinusoidal transformation of $\gls{x}$ on $[0;1]$, it can clearly be seen from Figure~\ref{fig:sinus_lin} that a standard linear \gls{lr} performs poorly. Discretizing the feature $\gls{x}$ results, using a very simple unsupervised heuristic named \textit{equal-length} (described in-depth in Appendix~\ref{app1:equal_length}), in good results (\textit{i.e.}\ visually mild bias / low variance) so long as the number of intervals, and subsequently of \gls{lr} coefficients, is low (see Animation on Figure~\ref{fig:anim_sinus} or still on Figure~\ref{fig:sinus_deb}). When the number of intervals gets large, the bias gets low (the sinus is well approximated by the little step functions), but the variance gets bigger (see Animation on Figure~\ref{fig:anim_sinus} or still on Figure~\ref{fig:sinus_fin}).


%\textcolor{red}{décommenter animation}

\begin{figure}[!ht]
\begin{animateinline}[poster=first, controls=all, palindrome, autopause, autoresume, width=\textwidth, height=7cm]{3}
\multiframe{99}{i=2+1}{\input{R_CODE_FIGURES/chapitre4/disc_plot\i.tex}}%
\end{animateinline}
\caption{\label{fig:anim_sinus} Animation of logistic regression fits on data generated by a sinus with a number of discretization steps in the \textit{equal-length} algorithm ranging from 2 to 100.}
\end{figure}

\begin{figure}[!ht]
%\vspace*{-1cm}
\begin{subfigure}[t]{\textwidth}
\centering \resizebox{.8\textwidth}{!}{\input{R_CODE_FIGURES/chapitre4/linear_plot.tex}}
%\vspace*{-1cm}
\caption{\label{fig:sinus_lin} Linear logistic regression (in \textcolor{red}{red}) fit on data generated by a sinus (in \textcolor{green}{green}).}
\end{subfigure}
%\vspace*{-1cm}
\begin{subfigure}[t]{\textwidth}
\centering \resizebox{.8\textwidth}{!}{\input{R_CODE_FIGURES/chapitre4/disc_plot5.tex}}
%\vspace*{-1cm}
\caption{\label{fig:sinus_deb} Logistic regression fit on data generated by a sinus with 5 discretization steps in the \textit{equal-length} algorithm.}
\end{subfigure}
%\vspace*{-1cm}
\begin{subfigure}[t]{\textwidth}
\centering \resizebox{.8\textwidth}{!}{\input{R_CODE_FIGURES/chapitre4/disc_plot100.tex}}
%\vspace*{-1cm}
\caption{\label{fig:sinus_fin} Logistic regression fit on data generated by a sinus with 100 discretization steps in the \textit{equal-length} algorithm.}
\end{subfigure}
\caption{Motivational example: achieving a good bias-variance trade-off.}
\label{fig:sin_trois}
\end{figure}
 
As the number of intervals is directly linked to the number of coefficient, and to a notion of ``complexity'' of the resulting \gls{lr} model, the bias-variance tradeoff (introduced in Chapter~\ref{chap1}) plays a key role in choosing an appropriate step size, and, as will be seen in the next section which was not possible for the simple \textit{equal-length} algorithm, appropriate step locations (hereafter called cutpoints). Again, this can be witnessed visually by looking at a model selection criterion, \textit{e.g.}\ expected Gini on a test set (which was also introduced in Chapter~\ref{chap1}), for different values of the number of intervals on Figure~\ref{fig:bic_sin}: as was visually concluded from Figure~\ref{fig:anim_sinus}, somewhere around 10-15 intervals seem the most satisfactory. Of course, as the model was misspecified, the flexibility brought by discretization was beneficial. The same phenomenon can b ewitnessed for categorical features on Figure~\ref{fig:csp} with real data from \gls{cacf}. On Figure~\ref{fig:csp_estim}, the \gls{lr} coefficients of the raw job types are displayed: none are significant and estimation variance is large. Grouping these levels results in narrower confidence intervals and significant \gls{lr} parameters as can be seen in Figure~\ref{fig:csp_estim_disc}.
%As can be witnessed from Figure~\ref{fig:bic_sin} with a well-specified model (in \textcolor{blue}{blue}). 
We formalize these empirical findings in the next section.



\begin{figure}[!ht]
\centering \resizebox{.8\textwidth}{!}{% Created by tikzDevice version 0.12 on 2019-03-15 11:11:12
% !TEX encoding = UTF-8 Unicode
\begin{tikzpicture}[x=1pt,y=1pt]
\definecolor{fillColor}{RGB}{255,255,255}
\path[use as bounding box,fill=fillColor,fill opacity=0.00] (0,0) rectangle (578.16,231.26);
\begin{scope}
\path[clip] ( 49.20, 61.20) rectangle (552.96,182.06);
\definecolor{drawColor}{RGB}{0,255,0}

\path[draw=drawColor,line width= 0.4pt,line join=round,line cap=round] ( 67.86,177.59) --
	( 72.62, 93.38) --
	( 77.38, 77.21) --
	( 82.14, 70.94) --
	( 86.90, 68.38) --
	( 91.66, 67.09) --
	( 96.42, 67.10) --
	(101.18, 67.43) --
	(105.93, 65.68) --
	(110.69, 68.26) --
	(115.45, 66.76) --
	(120.21, 66.44) --
	(124.97, 66.53) --
	(129.73, 68.14) --
	(134.49, 69.69) --
	(139.25, 71.88) --
	(144.01, 70.50) --
	(148.77, 71.02) --
	(153.53, 71.36) --
	(158.29, 73.59) --
	(163.05, 74.22) --
	(167.81, 74.83) --
	(172.57, 75.02) --
	(177.33, 75.88) --
	(182.09, 77.79) --
	(186.85, 78.37) --
	(191.61, 79.06) --
	(196.37, 79.37) --
	(201.13, 81.28) --
	(205.89, 81.19) --
	(210.65, 82.61) --
	(215.41, 83.76) --
	(220.17, 83.51) --
	(224.93, 85.71) --
	(229.69, 86.14) --
	(234.45, 87.57) --
	(239.20, 88.43) --
	(243.96, 89.23) --
	(248.72, 89.48) --
	(253.48, 90.31) --
	(258.24, 92.56) --
	(263.00, 91.17) --
	(267.76, 93.39) --
	(272.52, 94.73) --
	(277.28, 96.00) --
	(282.04, 96.42) --
	(286.80, 97.60) --
	(291.56, 97.85) --
	(296.32, 98.32) --
	(301.08,101.41) --
	(305.84,101.76) --
	(310.60,102.22) --
	(315.36,103.44) --
	(320.12,103.64) --
	(324.88,104.19) --
	(329.64,104.13) --
	(334.40,107.13) --
	(339.16,108.00) --
	(343.92,109.53) --
	(348.68,109.82) --
	(353.44,110.05) --
	(358.20,111.95) --
	(362.96,112.74) --
	(367.71,112.13) --
	(372.47,115.37) --
	(377.23,115.37) --
	(381.99,115.72) --
	(386.75,117.40) --
	(391.51,118.68) --
	(396.27,118.00) --
	(401.03,119.57) --
	(405.79,120.07) --
	(410.55,121.41) --
	(415.31,122.79) --
	(420.07,124.14) --
	(424.83,124.99) --
	(429.59,125.09) --
	(434.35,124.68) --
	(439.11,127.12) --
	(443.87,128.18) --
	(448.63,129.76) --
	(453.39,131.28) --
	(458.15,130.59) --
	(462.91,132.47) --
	(467.67,131.11) --
	(472.43,133.57) --
	(477.19,135.51) --
	(481.95,133.72) --
	(486.71,137.04) --
	(491.47,138.57) --
	(496.23,138.16) --
	(500.98,138.90) --
	(505.74,137.75) --
	(510.50,142.49) --
	(515.26,141.71) --
	(520.02,142.20) --
	(524.78,142.13) --
	(529.54,143.26) --
	(534.30,143.22);
\end{scope}
\begin{scope}
\path[clip] (  0.00,  0.00) rectangle (578.16,231.26);
\definecolor{drawColor}{RGB}{0,0,0}

\path[draw=drawColor,line width= 0.4pt,line join=round,line cap=round] ( 58.34, 61.20) -- (534.30, 61.20);

\path[draw=drawColor,line width= 0.4pt,line join=round,line cap=round] ( 58.34, 61.20) -- ( 58.34, 55.20);

\path[draw=drawColor,line width= 0.4pt,line join=round,line cap=round] (153.53, 61.20) -- (153.53, 55.20);

\path[draw=drawColor,line width= 0.4pt,line join=round,line cap=round] (248.72, 61.20) -- (248.72, 55.20);

\path[draw=drawColor,line width= 0.4pt,line join=round,line cap=round] (343.92, 61.20) -- (343.92, 55.20);

\path[draw=drawColor,line width= 0.4pt,line join=round,line cap=round] (439.11, 61.20) -- (439.11, 55.20);

\path[draw=drawColor,line width= 0.4pt,line join=round,line cap=round] (534.30, 61.20) -- (534.30, 55.20);

\node[text=drawColor,anchor=base,inner sep=0pt, outer sep=0pt, scale=  1.00] at ( 58.34, 39.60) {0};

\node[text=drawColor,anchor=base,inner sep=0pt, outer sep=0pt, scale=  1.00] at (153.53, 39.60) {20};

\node[text=drawColor,anchor=base,inner sep=0pt, outer sep=0pt, scale=  1.00] at (248.72, 39.60) {40};

\node[text=drawColor,anchor=base,inner sep=0pt, outer sep=0pt, scale=  1.00] at (343.92, 39.60) {60};

\node[text=drawColor,anchor=base,inner sep=0pt, outer sep=0pt, scale=  1.00] at (439.11, 39.60) {80};

\node[text=drawColor,anchor=base,inner sep=0pt, outer sep=0pt, scale=  1.00] at (534.30, 39.60) {100};

\path[draw=drawColor,line width= 0.4pt,line join=round,line cap=round] ( 49.20, 83.18) -- ( 49.20,173.12);

\path[draw=drawColor,line width= 0.4pt,line join=round,line cap=round] ( 49.20, 83.18) -- ( 43.20, 83.18);

\path[draw=drawColor,line width= 0.4pt,line join=round,line cap=round] ( 49.20,105.67) -- ( 43.20,105.67);

\path[draw=drawColor,line width= 0.4pt,line join=round,line cap=round] ( 49.20,128.15) -- ( 43.20,128.15);

\path[draw=drawColor,line width= 0.4pt,line join=round,line cap=round] ( 49.20,150.64) -- ( 43.20,150.64);

\path[draw=drawColor,line width= 0.4pt,line join=round,line cap=round] ( 49.20,173.12) -- ( 43.20,173.12);

\node[text=drawColor,rotate= 90.00,anchor=base,inner sep=0pt, outer sep=0pt, scale=  1.00] at ( 34.80, 83.18) {13000};

\node[text=drawColor,rotate= 90.00,anchor=base,inner sep=0pt, outer sep=0pt, scale=  1.00] at ( 34.80,128.15) {13400};

\node[text=drawColor,rotate= 90.00,anchor=base,inner sep=0pt, outer sep=0pt, scale=  1.00] at ( 34.80,173.12) {13800};

\path[draw=drawColor,line width= 0.4pt,line join=round,line cap=round] ( 49.20, 61.20) --
	(552.96, 61.20) --
	(552.96,182.06) --
	( 49.20,182.06) --
	( 49.20, 61.20);
\end{scope}
\begin{scope}
\path[clip] (  0.00,  0.00) rectangle (578.16,231.26);
\definecolor{drawColor}{RGB}{0,0,0}

\node[text=drawColor,anchor=base,inner sep=0pt, outer sep=0pt, scale=  1.00] at (301.08, 15.60) {Number of bins in \textit{equal-freq}};

\node[text=drawColor,rotate= 90.00,anchor=base,inner sep=0pt, outer sep=0pt, scale=  1.00] at ( 10.80,121.63) {BIC};
\end{scope}
\begin{scope}
\path[clip] ( 49.20, 61.20) rectangle (552.96,182.06);
\definecolor{drawColor}{RGB}{255,0,0}

\path[draw=drawColor,line width= 0.4pt,line join=round,line cap=round] ( 67.86,177.84) --
	( 72.62,177.84) --
	( 77.38,177.84) --
	( 82.14,177.84) --
	( 86.90,177.84) --
	( 91.66,177.84) --
	( 96.42,177.84) --
	(101.18,177.84) --
	(105.93,177.84) --
	(110.69,177.84) --
	(115.45,177.84) --
	(120.21,177.84) --
	(124.97,177.84) --
	(129.73,177.84) --
	(134.49,177.84) --
	(139.25,177.84) --
	(144.01,177.84) --
	(148.77,177.84) --
	(153.53,177.84) --
	(158.29,177.84) --
	(163.05,177.84) --
	(167.81,177.84) --
	(172.57,177.84) --
	(177.33,177.84) --
	(182.09,177.84) --
	(186.85,177.84) --
	(191.61,177.84) --
	(196.37,177.84) --
	(201.13,177.84) --
	(205.89,177.84) --
	(210.65,177.84) --
	(215.41,177.84) --
	(220.17,177.84) --
	(224.93,177.84) --
	(229.69,177.84) --
	(234.45,177.84) --
	(239.20,177.84) --
	(243.96,177.84) --
	(248.72,177.84) --
	(253.48,177.84) --
	(258.24,177.84) --
	(263.00,177.84) --
	(267.76,177.84) --
	(272.52,177.84) --
	(277.28,177.84) --
	(282.04,177.84) --
	(286.80,177.84) --
	(291.56,177.84) --
	(296.32,177.84) --
	(301.08,177.84) --
	(305.84,177.84) --
	(310.60,177.84) --
	(315.36,177.84) --
	(320.12,177.84) --
	(324.88,177.84) --
	(329.64,177.84) --
	(334.40,177.84) --
	(339.16,177.84) --
	(343.92,177.84) --
	(348.68,177.84) --
	(353.44,177.84) --
	(358.20,177.84) --
	(362.96,177.84) --
	(367.71,177.84) --
	(372.47,177.84) --
	(377.23,177.84) --
	(381.99,177.84) --
	(386.75,177.84) --
	(391.51,177.84) --
	(396.27,177.84) --
	(401.03,177.84) --
	(405.79,177.84) --
	(410.55,177.84) --
	(415.31,177.84) --
	(420.07,177.84) --
	(424.83,177.84) --
	(429.59,177.84) --
	(434.35,177.84) --
	(439.11,177.84) --
	(443.87,177.84) --
	(448.63,177.84) --
	(453.39,177.84) --
	(458.15,177.84) --
	(462.91,177.84) --
	(467.67,177.84) --
	(472.43,177.84) --
	(477.19,177.84) --
	(481.95,177.84) --
	(486.71,177.84) --
	(491.47,177.84) --
	(496.23,177.84) --
	(500.98,177.84) --
	(505.74,177.84) --
	(510.50,177.84) --
	(515.26,177.84) --
	(520.02,177.84) --
	(524.78,177.84) --
	(529.54,177.84) --
	(534.30,177.84);
\definecolor{drawColor}{RGB}{0,0,0}

\path[draw=drawColor,line width= 0.4pt,line join=round,line cap=round] (105.93,161.88) rectangle (230.86,125.88);
\definecolor{drawColor}{RGB}{0,255,0}

\path[draw=drawColor,line width= 0.8pt,line join=round,line cap=round] (114.93,149.88) -- (132.93,149.88);
\definecolor{drawColor}{RGB}{255,0,0}

\path[draw=drawColor,line width= 0.8pt,line join=round,line cap=round] (114.93,137.88) -- (132.93,137.88);
\definecolor{drawColor}{RGB}{0,0,0}

\node[text=drawColor,anchor=base west,inner sep=0pt, outer sep=0pt, scale=  1.00] at (141.93,146.44) {Quantized data LR};

\node[text=drawColor,anchor=base west,inner sep=0pt, outer sep=0pt, scale=  1.00] at (141.93,134.44) {Linear LR};
\end{scope}
\end{tikzpicture}
}
\caption{\label{fig:bic_sin} BIC of the resulting logistic regression on quantized data in \textcolor{green}{green} with a varying number of bins in the \textit{equal-length} algorithm and the linear logistic regression Gini in \textcolor{red}{red} (both misspecified).}
\end{figure}


 
\section{Quantization as a combinatorial challenge} \label{sec:model_selection}

\subsection{Quantization: definition}

\paragraph{General principle}

The quantization procedure consists in turning a $d$-dimensional raw vector of continuous and/or categorical features $\gls{bx} = (\gls{x}_1, \ldots, \gls{x}_d)$ into a $d$-dimensional categorical vector \textit{via} a component wise mapping $\q=(\gls{qj})_1^d$:
\[\q(\gls{bx})=(\q_1(\gls{x}_1),\ldots,\q_d(\gls{x}_d)),\]
where each of the $\gls{qj}$'s is a vector of $\gls{mj}$ dummies: 
\begin{align}\label{eq:qj}
& q_{j,h}(\cdot) =  1 \text{ if } \gls{x}_j \in C_{j,h}, 0 \text{ otherwise, } 1 \leq h \leq \gls{mj}, \\
\equiv & \gls{qj}(\cdot) = \gls{ehmj} \nonumber
\end{align}
where $\gls{mj}$ is an integer, $\gls{ehmj}$ is the $h^{\text{th}}$ basis vector of $\gls{R}^{\gls{mj}}$ and the sets $C_{j,h}$ are defined with respect to each feature type as is described just below.
\paragraph{Raw continuous features} If $\gls{x}_j$ is a continuous component of $\gls{bx}$, quantization $\gls{qj}$ has to perform a discretization of $\gls{x}_j$ and the $C_{j,h}$'s, $1 \leq h \leq \gls{mj}$, are contiguous intervals  
\begin{equation}\label{eq:Cjhcont}
C_{j,h}=(c_{j,h-1},c_{j,h}]
\end{equation}
where $c_{j,1},\ldots,c_{j,\gls{mj}-1}$ are increasing numbers called cutpoints, $c_{j,0}=-\infty$, $c_{j,\gls{mj}}=\infty$. For example, the quantization of the unit segment in thirds would be defined as $\gls{mj}=3$, $c_{j,1} = 1/3$, $c_{j,2} = 2/3$ and subsequently $\gls{qj}(0.1) = (1,0,0)$. This is visually exemplified on Figure~\ref{fig:disc_cont}.
\paragraph{Raw categorical features} If $\gls{x}_j$ is a categorical component of $\gls{bx}$, quantization $\gls{qj}$ consists in grouping levels of $\gls{x}_j$ and the $C_{j,h}$s form a partition of the set $\gls{NO}$.
%, say $\{1,\ldots,l_j\}$, of levels of $\gls{x}_j$: 
\begin{equation*}
%\bigsqcup_{h=1}^{\gls{mj}}C_{j,h}=\{1,\ldots,\gls{lj}\}.
\bigsqcup_{h=1}^{\gls{mj}}C_{j,h}=\gls{NO}.
\end{equation*}
For example, the grouping of levels encoded as ``1'' and ``2'' would yield $C_{j,1} = \{1,2\}$ such that $\gls{qj}(1) = \gls{qj}(2) = (1,0,\ldots,0)$. Note that it is assumed that there are no empty buckets, \textit{i.e.}\ $\nexists j, h$ s.t.\ $C_{j,h} = \varnothing$ and that there is no overlap, \textit{i.e.}\ $\nexists j,h,h'$ s.t.\ $C_{j,h} \cup C_{j,h'} = \varnothing$. This is visually exemplified on Figure~\ref{fig:disc_disc}.

\begin{figure}[!ht]
\begin{multicols}{2}

\begin{minipage}{0.45\textwidth}
\centering
\begin{tikzpicture}[scale=0.2,every node/.style={scale=0.7}]
\draw[->,line width=0.08cm] (-5,0)--(26,0) node[right]{$\gls{x}$};

\node at (0,-1) {$C_1$};
\node at (8,-1) {$C_2$};
\node at (16,-1) {$C_3$};

\node at (4,-1) {$c_1$};
\node at (12,-1) {$c_2$};

\node [red,circle, fill] at (4,0) {};
\node [red,circle, fill] at (12,0) {};

\node at (-1,1) {$\q(x) = (1,0,0)$};
\node at (8,1) {$\q(x) = (0,1,0)$};
\node at (20,1) {$\q(x) = (0,0,1)$};
\end{tikzpicture}
\caption{\label{fig:disc_cont} Quantization (discretization) of a continuous feature.}
\end{minipage}

\columnbreak

\begin{minipage}{0.45\textwidth}
\centering
\begin{tikzpicture}[scale=0.2,every node/.style={scale=0.7}]
\tikzset{vertex/.style = {shape=circle,draw,scale=0.7,minimum size=1cm}}
\tikzset{edge/.style = {->,> = latex'}}

% Boules E^j
\node [vertex] (q1) at (3,2.5) {(1,0)};
\node [vertex] (q2) at (15,2.5) {(0,1)};

% Boules X^J
\node [vertex] (x1) at (-4,0) {0};
\node [vertex] (x2) at (1.8,0) {1};
\node [vertex] (x3) at (9.5,0) {2};
\node [vertex] (x4) at (17,0) {3};
\node [vertex] (x5) at (24,0) {4};

% Labels
\node at (-7,2.5) {$\q(\gls{x})=$};
\node at (-7,0.2) {$\gls{x}=$};

% Flèches
\draw[edge,line width=0.03cm] (x1) to (q1);
\draw[edge,line width=0.03cm] (x3) to (q1);
\draw[edge,line width=0.03cm] (x4) to (q1);
\draw[edge,line width=0.03cm] (x2) to (q2);
\draw[edge,line width=0.03cm] (x5) to (q2);

\end{tikzpicture}
\caption{\label{fig:disc_disc} Quantization (factor levels merging) of categorical feature.}
\end{minipage}
\end{multicols}
\end{figure}


\paragraph{Notations for the quantization family}

In both continuous and categorical cases, keep in mind that $\gls{mj}$ is the dimension of $\gls{qj}$. For notational convenience, the (global) order of the quantization $\q$ is set as 
\[|\q|=\sum_{j=1}^d \gls{mj}.\]
The space where quantizations $\q$ live (resp. $\gls{qj}$) will be denoted by $\gls{bQm}$ in the sequel (resp. $\gls{bQjmj}$), when the number of levels $\gls{bm} = (\gls{mj})_1^d$ is fixed. Since it is not known, the full model space is $\Q = \cup_{m \in \gls{N}_\star^{d}} \gls{bQm}$.

\subparagraph{Equivalence of quantizations} \label{par:equiv}

Let $\q^1$ and $\q^2$ in $\Q$ such that $\q^1 \mathcal{R}_{\gls{T}_n} \q^2 \equiv \forall i,j \; \q^1_j(\gls{xij}) = \q^2_j(\gls{xij})$. See Figure~\ref{fig:equiv} for an example.

\subparagraph{Lemma} Relation $\mathcal{R}_{\gls{T}_n}$ defines an equivalence relation on $\Q$.

\begin{proof}
Relation $\mathcal{R}_{\gls{T}_n}$ is trivially reflexive and symmetric because of the reflexive and symmetric nature of the equality relation in $\gls{R}$: $\forall i,j \; \q^1_j(\gls{xij}) = \q^1_j(\gls{xij})$ and $\forall i,j \; \q^1(\gls{xij}) = \q^2(\gls{xij})$. Similarly, let $\q^3 \in \Q$ such that $\q^1 \mathcal{R}_{\gls{T}_n} \q^3  \equiv \forall i,j \; \q^1_j(\gls{xij}) = \q^3_j(\gls{xij})$. Again, we immediately get $\forall i,j \; \q^2_j(\gls{xij}) = \q^3_j(\gls{xij})$, \textit{i.e.}\ $\q^2 \mathcal{R}_{\gls{T}_n} \q^3$ which proves the transitivity of $\mathcal{R}_{\gls{T}_n}$.
\end{proof}

 \begin{figure}[!ht]
     \centering
     \begin{tikzpicture}[scale=0.3]
 \draw[->,line width=0.1cm] (-5,0)--(24,0) node[right]{$\gls{x}_j$};

 \node [red,circle,fill] at (3,0) {};

 \node [blue,circle, fill] at (0.5,0) {};
 \node [blue,circle, fill] at (2,0) {};

 \node [blue,circle, fill] at (5.5,0) {};
 \node [blue,circle, fill] at (7,0) {};
 \node [blue,circle, fill] at (9,0) {};

 \node at (3,1.5) {$c^1_1$};

 \node at (-1.1,1.5) {$\q^1_j(\gls{x}_j) = (1,0)$};
 \node at (8.3,1.5) {$\q^1_j(\gls{x}_j) = (0,1)$};
 \end{tikzpicture}

 \begin{tikzpicture}[scale=0.3]
 \draw[->,line width=0.1cm] (-5,0)--(24,0) node[right]{$x_j$};

 \node [red,circle,fill] at (4,0) {};

 \node [blue,circle, fill] at (0.5,0) {};
 \node [blue,circle, fill] at (2,0) {};

 \node [blue,circle, fill] at (5.5,0) {};
 \node [blue,circle, fill] at (7,0) {};
 \node [blue,circle, fill] at (9,0) {};

 \node at (4,1.5) {$c^2_1$};

 \node at (-1.1,1.5) {$\q^2_j(x_j) = (1,0)$};
 \node at (8.3,1.5) {$\q^2_j(x_j) = (0,1)$};

 \end{tikzpicture}

     \caption{On the sample $\gls{bbx}$ (blue points), the two discretization functions $\q^1$ and $\q^2$ (which respective unique cutpoint $c^1_1$ and $c^2_1$ are displayed in red) take the same value and are thus equivalent w.r.t. $\mathcal{R}_{\gls{T}_n}$.}
     \label{fig:equiv}
 \end{figure}


\subparagraph{Cardinality of the quantization family in the continuous case} ~\label{par:cardinality}

For a continuous feature $x_j$, let $\gls{qj} \in \gls{bQjmj}$ and cutpoints $\boldsymbol{c}_j$. Without any loss of generality, \textit{i.e.}\ up to a relabelling on individuals $i$, it can be assumed that there are $\gls{mj}+1$ observations $x_{1,j},\dots,x_{\gls{mj}+1,j}$ s.t.\ $x_{1,j} < c_{j,1} < x_{2,j} < \dots < c_{\gls{mj}-1,1} < x_{\gls{mj}+1,j}$. Indeed, if for example there exists $k < \gls{mj} - 1$ s.t.\ $c_{j,k} < \dots < c_{j,\gls{mj}-1}$ and $\max_{1 \leq i \leq n} x_{i,j} < c_{j,k}$, then discretization $\q^{\text{bis}}_j \in \gls{bQj}$ with $k+1$ cutpoints $(-\infty,c_{j,1},\dots,c_{j,k-1},+\infty)$ is equivalent w.r.t.\ $\mathcal{R}_{\gls{T}_n}$ to $\gls{qj}$: $\forall i, \; \gls{qj}(x_{i,j}) = \q^{\text{bis}}_j(x_{i,j})$. A similar proof can be conducted with cutpoints below the minimum of $\gls{bx}_j$ or with several cutpoints in-between consecutive values of the observations. Subsequently, there are $\binom{n-1}{\gls{mj}-1}$ ways to construct $\bm{c}_j$, \textit{i.e.}\ equivalence classes $[\gls{qj}]$ for a fixed $\gls{mj} \leq n$. The number of intervals $\gls{mj}$ can range from $2$ (binarization) to $n$ (each $x_{i,j}$ is in its own interval, thus $\gls{qj}(x_{i,j}) \neq \gls{qj}(x_{i',j})$ for $i \neq i'$), so that the number of admissible discretization of $\gls{bbx}_j$ is $|\gls{bQj}| = \sum_{i=2}^{n}$ ${n-1}\choose{i-1}$. Note that $|\gls{bQj}|$ depends on the number of observations $n$; we shall go back to this property in the following section.


\subparagraph{Cardinality of the quantization family in the categorical case}

For a continuous feature $x_j$, let $\gls{qj} \in \gls{bQj}$ with $\gls{mj}$ groups. The number of re-arrangements of $\gls{lj}$ labelled elements into $\gls{mj}$ unlabelled groups is given by the Stirling number of the second kind $S(\gls{lj},\gls{mj}) = \frac{1}{\gls{mj}!} \sum_{i=0}^{\gls{mj}} (-1)^{\gls{mj}-i} {\gls{mj} \choose i} i^{\gls{lj}}$. As $\gls{mj}$ is unknown and must be searched over the range $\{1,\dots,\gls{lj}\}$. Thus for categorical features, model space $\gls{bQj}$ is also discrete; subsequently, $\Q = \prod_{j=1}^d \gls{bQj}$ is discrete.







\paragraph{Literature review}

The current practice of quantization is prior to any predictive task, thus ignoring its consequences on the final predictive ability. It consists in optimizing a heuristic criterion, often totally unrelated (unsupervised methods) or at least explicitly (supervised methods) to prediction, and mostly univariate (each feature is quantized irrespective of other features' values). The cardinality of the quantization space $\Q$ was calculated explicitely w.r.t.\ $d$, $(\gls{mj})_1^d$ and, for categorical features, $\gls{lj}$: it is huge, so that a greedy approach is intractable and such heuristics are needed, as will be detailed in the next section.
Many algorithms have thus been designed and a review of approximatively 200 discretization strategies, gathering both criteria and related algorithms, can be found in~\cite{ramirez2016data}, preceded by other enlightening review articles such as~\cite{dougherty1995supervised,liu2002discretization}. They classify discretization methods by distinguishing, among other criteria and as said previously, unsupervised and supervised methods ($\bm{y}$ is used to discretize $\gls{bx}$), for which model-specific (assumptions on $p_{\gls{bth}}$) or model-free approaches are distinguished, univariate and multivariate methods (features $\gls{Xnotj} = (X_{1},\ldots,X_{j-1},X_{j+1},\ldots,X_{d})$ may influence the quantization scheme of $\gls{Xj}$) and other criteria as can be seen from Figure~\ref{fig:taxonomy} reproduced from~\cite{ramirez2016data} with permission. For factor levels grouping, We found no such taxonomy, but some discretization methods, \textit{e.g.}\ $\chi^2$ independence test-based methods can be naturally extended to this type of quantization, which is for example what the CHAID algorithm, proposed in~\cite{kass1980exploratory} and applied to each categorical feature, relies on. A simple idea is to use Group LASSO~\cite{meier2008group} which attempts to shrink to zero all coefficients of a categorical feature to avoid situations where a few levels enter the model, which is arguably less interpretable. Another idea would be to use Fused LASSO~\cite{tibshirani2005sparsity}, which seeks to shrink the pairwise absolute difference of selected coefficients, and apply it to all pairs of levels: the levels for which the difference would be shrunk to zero would be grouped. A combination of both approaches would allow both selection and grouping\footnote{See \url{https://stats.stackexchange.com/questions/60100/penalized-methods-for-categorical-data-combining-levels-in-a-factor}}.
For benchmarking purposes, and following results found in the taxonomy of~\cite{ramirez2016data}, we used the MDLP~\cite{fayyad1993multi} discretization method, described in-depth in Appendix~\ref{app1:mdlp}, which is a popular supervised univariate discretization method, and I implemented an extension of the discretization method ChiMerge~\cite{kerber1992chimerge} to categorical features, performing pairwise $\chi^2$ independence tests rather than only pairs of contiguous intervals, which we called ChiCollapse and describe in-depth in Appendix~\ref{app1:chicollapse}. Note that various refinements of ChiMerge have been proposed in the literature, Chi2~\cite{liu1995chi2}, ConMerge~\cite{wang1998concurrent}, ModifiedChi2~\cite{tay2002modified}, and ExtendedChi2~\cite{su2005extended}, which seek to correct for multiple hypothesis testing~\cite{shaffer1995multiple} and automize the choice of the confidence parameter $\alpha$ in the $\chi^2$ tests, but adapting them to categorical features for benchmarking purposes would have been too time-consuming. A similar measure, called Zeta, has been proposed in place of $\chi^2$ in~\cite{ho1997zeta} and subsequent refinement~\cite{ho1998efficient}: it is the classification error achievable by using only two contiguous intervals; if it is low, the two intervals are dissimilar w.r.t.\ the prediction task, if not, they can be merged.

\begin{figure}[!ht]
\includegraphics[width=\textwidth]{figures/chapitre4/taxonomy.PNG}
\caption{Taxonomy of discretization methods.}
\label{fig:taxonomy}
\end{figure}



\subsection{Quantization embedded in a predictive process} \label{subsec:embedding}

In what follows, focus is given to \gls{lr} since it is a requirement from \gls{cacf} but it is applicable to any other supervised (binary) classification model.

\paragraph{Logistic regression on quantized data}

Quantization is a widespread preprocessing step to perform a learning task consisting in predicting, say, a binary variable $\gls{y}\in\{0,1\}$, from a quantized predictor  $\q(\gls{bx})$, through, say, a parametric conditional distribution $p_{\gls{bth}}(\gls{y}|\q(\gls{bx}))$ like \gls{lr}; the whole process can be visually represented as a dependence structure among $\gls{bX}$, its quantization $\bm{Q}$ (which notation as a random variable will be made clearer in Section~\ref{sec:sem}) and the target $\gls{Y}$ on Figure~\ref{fig:dep}. Considering quantized data instead of raw data has a double benefit. First, the quantization order $|\q|$ acts as a tuning parameter for controlling the model's flexibility and thus the bias/variance trade-off of the estimate of the parameter $\gls{bth}$ (or of its predictive accuracy) for a given dataset. This claim becomes clearer with the example of logistic regression I focus on, as a still very popular model for many practitioners. On quantized data, Equation~\eqref{eq:logit} becomes:
\begin{equation}
    \label{eq:reglogq}
\ln \left( \dfrac{p_{\gls{bth}}(1|\q(\gls{bx}))}{1 - p_{\gls{bth}}(1|\q(\gls{bx}))} \right) = \theta_0 + \sum_{j=1}^d \gls{qj}(\gls{x}_j)' \gls{bth}_j,
\end{equation}
where $\gls{bth} = (\theta_{0},(\gls{bth}_j)_1^d) \in \gls{R}^{|\q|+1}$ and $\gls{bth}_j = (\theta_{j}^{1},\dots,\theta_{j}^{\gls{mj}})$ with $\theta_{j}^{\gls{mj}} = 0$, $j=1 \ldots d$, for identifiability reasons (see Section~\ref{subsec:apprentissage}).
Second, at the practitioner level, the previous tuning of $|\q|$ through each feature's quantization order $\gls{mj}$, especially when it is quite low, allows an easier interpretation of the most important predictor values involved in the predictive process. Denoting the $n$-sample as in previous chapters by $(\gls{bbx},\gls{bby})$, with $\gls{bbx}=(\gls{bx}_1,\ldots,\gls{bx}_n)$ and $\gls{bby}=(\gls{y}_1,\ldots,\gls{y}_n)$, the log-likelihood 
\begin{equation}
\label{eq:lq}
\ell_{\q}(\gls{bth} ; (\gls{bbx},\gls{bby}))=\sum_{i=1}^n \ln p_{\gls{bth}}(\gls{y}_i|\q(\gls{bx}_i))
\end{equation}
provides a Maximum Likelihood estimator $\hat{\gls{bth}}_{\q}$ of $\gls{bth}$ for a given quantization $\q$. For the rest of the chapter and consistently with the manuscript, the approach is exemplified with \gls{lr} as $p_{\gls{bth}}$ but it can be applied to any other predictive model, as will be recalled in the concluding section.

\begin{figure}[!ht]
\centering
\begin{minipage}{0.45\textwidth}
\centering
\begin{tikzpicture}
\tikzset{vertex/.style = {shape=circle,draw,minimum size=1.5em}}
\tikzset{edge/.style = {->,> = latex'}}

% vertices
\node[vertex] (x1) at  (0,1.5) {$\gls{X}_1$};
\node[vertex] (xj) at  (0,0) {$\gls{X}_j$};
\node[vertex] (xd) at  (0,-1.5) {$\gls{X}_{d}$};


\node[vertex] (q1) at  (2.5,1.5) {$\bm{\mathfrak{Q}}_1$};
\node[vertex] (qj) at  (2.5,0) {$\bm{\mathfrak{Q}}_j$};
\node[vertex] (qd) at  (2.5,-1.5) {$\bm{\mathfrak{Q}}_{d}$};

\node[vertex] (y) at (5,0) {$\gls{Y}$};

%edges

\draw[edge] (x1) to (q1);
\draw[edge] (xj) to (qj);
\draw[edge] (xd) to (qd);
\draw[edge] (q1) to (y);
\draw[edge] (qj) to (y);
\draw[edge] (qd) to (y);

\draw[dashed] (x1) to (xj);
\draw[dashed] (xj) to (xd);

\draw[dashed] (q1) to (qj);
\draw[dashed] (qj) to (qd);
\end{tikzpicture}
\caption{\label{fig:dep}Dependence structure between $\gls{X}_j$,$\bm{\mathfrak{Q}}_j$ and $\gls{Y}$} 
\end{minipage}
\end{figure}

\paragraph{Quantization as a model selection problem} \label{par:model_selec}

As dicussed in the previous section, and emphasized in the literature review, quantization is often a preprocessing step; however, quantization can be embedded directly in the predictive model. Continuing our logistic example, a standard information criteria such as the BIC (see Section~\ref{subsubsec:choix_modele}) can be used to select the best quantization:
\begin{equation}
    \label{eq:BICq}
    \hat{\q}=\argmin_{\q \in \Q} \text{BIC}(\hat{\gls{bth}}_{\q})
\end{equation}
where the ``complexity'' parameter $\nu$ depends on ${\q}$ and is traditionally the number of continuous parameters to be estimated in the $\gls{bth}$-parameter space. We shall insist here on the fact that choosing the BIC as our gold standard to compare quantizations is only a matter of consistency throughout the chapters. The practitioner can swap this criterion with any other penalized criterion on training data such as AIC~\cite{akaike1973information} or, as \textit{Credit Scoring} people like, the Gini index on a test set. Note however that, regardless of the criterion used, an exhaustive search of $\hat{\q}\in\Q$ is an intractable task due to its highly combinatorial nature as was explicitly formulated in the previous section. Anyway, the optimization~(\ref{eq:BICq}) requires a new specific strategy that is describe in the next section.

\paragraph{Remark on model selection consistency} \label{par:consistency}

In high-dimensional spaces and among models with a wildly varying number of parameters, classical model selection tools like BIC can have disappointing asymptotic properties, as emphasized in~\cite{chen2008extended}, where a modified BIC criterion, taking into account the number of models per parameter size, is proposed.
Moreover in essence, as is apparent from the $\hat{\gls{bth}}_{\q}$ symbol, and supplemental to the \gls{lr} coefficients $\gls{bth}$, the inherent parameters of $\q$, in the continuous case, which are the $c_{j,h}$ (see Equation~\eqref{eq:Cjhcont}) shall be accounted for in the penalization term $\nu$: they are estimated indirectly in the subsequent section.
In addition, in this setting, the BIC criterion relies on the Laplace approximation~\cite{lebarbier} which requires the likelihood to be twice differentiable in the parameters. However, as $\q$ consists in a collection of step functions of parameters $C_{j,h}$, this is not the case. For continuous features, since it is nevertheless almost everywhere differentiable, for the properties of the BIC criterion to hold, it suffices that there exists a neighbourhood $V_{j,h}$ around true parameters $c_{j,h}^\star$ where there is no observation: $\not\exists i, \: x_{i,j} \in V_{j,h}$.

For categorical features, the Laplace approximation~\cite{lebarbier} does not work and there is no way, in general, to approximate the integral (\textit{i.e.}\ the sum over the discrete parameter space) by ``counting'' the number of parameters as in the continuous case~\cite{vincent_disc}.

Lastly, for the asymptotic properties of BIC to hold in the case of nested models (which is not \textit{stricto sensu} the case here since for any global quantization order $|\q|$ there are a lot of possible univariate quantization orders $|\gls{qj}|$), any multiplicative constant to the number of parameters is appropriate. Indeed, suppose two nested models $M_i$, $M_t$ have parameters $|\q|_i > |\q|_t$ respectively; then, we have:
\[ \text{BIC}_i - \text{BIC}_t \approx - \chi^2_{|\q|_i - |\q|_t} + C(|\q|_i - |\q|_t)\ln n, \]
where $C=1$ for the BIC criterion but could be replaced by any other $C \in \gls{R}^+_\star$ since for $n \to + \infty$, $C(|\q|_i - |\q|_t)\ln n$ will dominate and reject the overly parametrized model $M_i$.

For all these reasons, the BIC criterion which penalizes only on the logistic regression parameters $\gls{bth}$ is used in the remainder of this manuscript.

\section{The proposed neural network based quantization}
\label{sec:proposal}

\subsection{A relaxation of the optimization problem} \label{subsec:relaxation}

In this section, we propose to relax the constraints on $\gls{qj}$ to simplify the search of $\hat{\q}$. Indeed, the derivatives of $\gls{qj}$ are zero almost everywhere and consequently a gradient descent cannot be directly applied to find an optimal quantization.

\paragraph{Smooth approximation of the quantization mapping}

A classical approach is to replace the binary functions $q_{j,h}$ (see Equation (\ref{eq:qj}))  by smooth parametric ones  with a simplex condition, namely with $\ag_j=(\ag_{j,1},\ldots, \ag_{j,\gls{mj}})$:
%\begin{equation}
\begin{equation*}
    %\label{eq:qaj}
    {\q_{\ag_j}(\cdot)=\left(q_{\ag_{j,h}}(\cdot)\right)_{h=1}^{\gls{mj}} \text{ with } \sum_{h=1}^{\gls{mj}}q_{\ag_{j,h}}(\cdot)=1 \text{ and } 0 \leq q_{\ag_{j,h}}(\cdot) \leq 1,}
\end{equation*}
%\end{equation}
where functions $q_{\ag_{j,h}}(\cdot)$, properly defined hereafter for both continuous and categorical features, represent a fuzzy quantization in that, here, each level $h$ is weighted by $q_{\ag_{j,h}}(\cdot)$ instead of being selected once and for all as in (\ref{eq:qj}). The resulting fuzzy quantization for all components depends on the global parameter $\ag = (\ag_1, \ldots, \ag_d)$ and is denoted by $\q_{\ag}(\cdot)=\left(\q_{\ag_j}(\cdot)\right)_{j=1}^d$. 

%From a deterministic point of view, denoting by $\tilde{\Q}$ the space of $\q_{\ag}$, we have $\Q \subset \widetilde{\Q}$. From a statistical point of view, under standard regularity conditions and with a suitable estimation procedure (see later for the proposed estimation procedure), we have consistency of $(\q_{\hat{\ag}}, \hat{\gls{bth}})$ towards $(\q,\gls{bth})$. From an empirical point of view, we will see in Section~\ref{sec:experiments} and in particular in Figure~\ref{fig:MAP}, that this smooth approximation $\q_{\ag}$ converges towards ``hard'' quantizations\footnote{Up to a permutation on the labels $h=1 \ldots m_j$ to recover the ordering in $C_{j,h}$ (see Eq. (\ref{eq:Cjhcont})).} $\q$.



 {\bf For continuous features}, we set for $\ag_{j,h} = (\alpha^0_{j,h},\alpha^1_{j,h}) \in \gls{R}^2$
\begin{equation} \label{eq:softmax}
q_{\ag_{j,h}}(\cdot) = \frac{\exp(\alpha^0_{j,h} + \alpha^1_{j,h}  \cdot)}{\sum_{g=1}^{\gls{mj}} \exp(\alpha^0_{j,g} + \alpha^1_{j,g}  \cdot)}
\end{equation}
where $\ag_{j,\gls{mj}}$ is set to $(0,0)$ for identifiability reasons.




{\bf For categorical features}, we set for $\ag_{j,h}=\left(\alpha_{j,h}(1),\ldots, \alpha_{j,h}(\gls{lj})\right) \in \gls{R}^{\gls{lj}}$
\[q_{\ag_{j,h}}(\cdot) = \frac{\exp\left(\alpha_{j,h}(\cdot)\right)}{\sum_{g=1}^{\gls{mj}} \exp\left(\alpha_{j,g}(\cdot)\right)}\]
where $\gls{lj}$ is the number of levels of the categorical feature $\gls{x}_j$.



\paragraph{Parameter estimation}

With this new fuzzy quantization, the \gls{lr} for the predictive task is then expressed as
\begin{equation}
    \label{eq:reglogqa}
    \ln \left( \dfrac{p_{\gls{bth}}(1|\q_{\ag} (\gls{bx}))}{1 - p_{\gls{bth}}(1|\q_{\ag} (\gls{bx}))} \right) = \theta_0 + \sum_{j=1}^d { \q_{\ag_{j}}(\gls{x}_j)' \gls{bth}_j },
\end{equation}
where $\q$ has been replaced by $\q_{\ag}$ from Equation~(\ref{eq:reglogq}).
Note that as $\q_{\ag}$ is a sound approximation of $\q$ (see above), this \gls{lr} in $\q_{\ag}$ is consequently a good approximation of the \gls{lr} in $\q$ from Equation~(\ref{eq:reglogq}). The relevant log-likelihood is here 
\begin{equation}
    \label{eq:lqa}
    \ell_{\q_{\ag}}(\gls{bth} ; (\gls{bbx},\gls{bby}))=\sum_{i=1}^n \ln p_{\gls{bth}}(y_i|\q_{\bm{\alpha}}(\bm{x}_i))
\end{equation}
and can be used as a tractable substitute for (\ref{eq:lq}) to solve the original optimization problem (\ref{eq:BICq}), where now both $\ag$ and $\gls{bth}$ have to be estimated, which is discussed in the next section. We wish to maximize the log-likelihood (\ref{eq:reglogqa}) which would yield parameters $(\hat{\ag},\hat{\gls{bth}})$; these are consistent if the model is well-specified (\textit{i.e.}\ there is a ``true'' quantization under classical regularity conditions). Denoting by $A$ the space of $\ag$ and $\Q_A$ the space of $\q_{\ag}$, to ``push'' $\Q_A$ further into $\Q$, $\hat{\q}$ is deduced from a \textit{maximum a posteriori} procedure applied to $\q_{\hat{\ag}}$:
\begin{align}
    \label{eq:ht}
    & \hat{q}_{j,h}(\gls{x}_j) = 1 \text{ if } h = \argmax_{1 \leq h' \leq \gls{mj}} q_{\hat{\ag}_{j,h'}}(\gls{x}_j), 0 \text{ otherwise.} \\
    \equiv & \hat{\q}_j(\gls{x}_j) = \gls{ehmj} \nonumber
\end{align}
If there are several levels $h$ that satisfy (\ref{eq:ht}), we simply take the level that corresponds to smaller values of $\gls{x}_j$ to be in accordance with the definition of $C_{j,h}$ in Equation~(\ref{eq:Cjhcont}). This {\it maximum a posteriori} principle will be exemplified in Figure~\ref{fig:MAP} on simulated data. These approximations are justified by the following arguments. 

From a deterministic point of view, we have $\Q \subset \Q_A$: First, the \textit{maximum a posteriori} step~(\ref{eq:ht}) produces contiguous intervals (\textit{i.e.}\ there exists $C_{j,h}$; $1 \leq j \leq d$, $1 \leq h \leq \gls{mj}$, s.t.\ ${\hat{\q}}$ can be written as in~\ref{eq:qj}) \cite{same2011model}. Second, in the continuous case, the higher $\alpha_{j,h}^1$, the less smooth the transition from one quantization $h$ to its ``neighbor''\footnotemark[1] $h+1$, whereas $\dfrac{\alpha_{j,h}^0}{\alpha_{j,h}^1}$ controls the point in $\gls{R}$ where the transition occurs \cite{chamroukhi2009regression}. Concerning the categorical case, the rationale is even simpler as $q_{\lambda \ag_j}(\gls{x}_j) \to \gls{ehmj} \text{ if } h = \argmax_{h'} q_{\alpha_{j,h'}}(\gls{x}_j)$ as $\lambda \to +\infty$~\cite{reverdy2016parameter}.

From a statistical point of view, 
%as $\ag$ needs to diverge to infinity for $\q_{{\ag}}$ to approximate $\q$, the maximum likelihood estimator of $\ag$ does not converge. However, 
under standard regularity conditions and with a suitable estimation procedure (see later for the proposed estimation procedure), the maximum likelihood framework would ensure the consistency of $(\q_{\hat{\ag}}, \hat{\gls{bth}})$ towards $(\q,\gls{bth})$ if $\ag^\star$ s.t.\ $\q_{\ag^\star} = \q$ was an interior point of the parameter space $A$. However, as emphasized in the previous paragraph, ``$\ag^\star = + \infty$'' such that the maximum likelihood parameter is on the edge of the parameter space which hinders asymptotic properties (\textit{e.g.}\ normality) in some settings~\cite{10.2307/2289471}, but not ``convergence'' on which we focus here. We did not investigate this issue further since numerical experiments showed consistency: from an empirical point of view, we will see in Section~\ref{sec:experiments} and in particular in Figure~\ref{fig:MAP}, that the smooth approximation $\q_{\hat{\ag}}$ converges towards ``hard'' quantizations\footnotemark[1] $\q$.

Moreover, and as is usual, the log-likelihood $\ell_{\q_{\ag}}(\gls{bth},(\gls{bbx},\gls{bby}))$ cannot be directly maximized w.r.t.\ $(\ag,\gls{bth})$, so that we need an iterative procedure. To this end, the next section introduces a neural network of particular architecture.

\footnotetext[1]{Up to a permutation on the labels $h=1 \ldots \gls{mj}$ to recover the ordering in $C_{j,h}$ (see Equation (\ref{eq:Cjhcont})).}

\subsection{A neural network-based estimation strategy} \label{sec:estim}

\paragraph{Neural network architecture}

To estimate parameters $\ag$ and $\gls{bth}$ in the model (\ref{eq:reglogqa}), a particular neural network architecture can be used. We shall insist that this network is only a way to use common deep learning frameworks, namely Tensorflow~\cite{tensorflow2015-whitepaper} through the high-level API Keras~\cite{chollet2015keras} instead of building a gradient ascent algorithm from scratch to optimize~\eqref{eq:lqa}. The most obvious part is the output layer that must produce $p_{\gls{bth}}(1|\q_{\ag}(\gls{bx}))$ which is equivalent to a densely connected layer with a sigmoid activation (the reciprocal function of logit).

For a continuous feature $\gls{x}_j$ of $\gls{bx}$, the combined use of $\gls{mj}$ neurons including affine transformations and softmax activation obviously yields $\q_{\ag_{j}}(\gls{x}_j)$. Similarly, an input categorical feature $\gls{x}_j$ with $\gls{lj}$ levels is equivalent to $\gls{lj}$ binary input neurons (presence or absence of the factor level). These $\gls{lj}$ neurons are densely connected to $\gls{mj}$ neurons without any bias term and a softmax activation. The softmax outputs are next aggregated via the summation in model (\ref{eq:reglogqa}), say $\Sigma_{\gls{bth}}$ for short, and then the sigmoid function $\sigma$ gives the final output. All in all, the proposed model is straightforward to optimize with a simple neural network, as shown in Figure~\ref{fig:nn}.


\def\layersep{2.5cm}

\begin{figure}[!ht]
\centering
\begin{tikzpicture}[shorten >=1pt,->,draw=black!50, node distance=\layersep]
    \tikzstyle{every pin edge}=[<-,shorten <=1pt]
    \tikzstyle{neuron}=[circle,fill=black!25,minimum size=17pt,inner sep=0pt]
    \tikzstyle{input neuron}=[neuron, fill=green!50];
    \tikzstyle{output neuron}=[neuron, fill=red!50];
    \tikzstyle{hidden neuron}=[neuron, fill=blue!50];
    \tikzstyle{annot} = [text width=4em, text centered]
    \tikzstyle{annotrectangle} = [text width=8em, text centered]


        \node[input neuron, pin=left:continuous value $\gls{x}_j$] (I-1) at (0,-1) {};
        
        \node[input neuron, pin=left:categorical value $1$] (I-2) at (0,-2) {};
        \node[input neuron, pin=left:$\vdots$] (I-3) at (0,-3) {};
        \node[input neuron, pin=left:categorical value $\gls{lj}$] (I-4) at (0,-4) {};

    % Draw the hidden layer nodes
    \foreach \name / \y in {1,...,2}
        \path[yshift=0.5cm]
            node[hidden neuron] (H-\name) at (\layersep,-\y cm) {soft};

    \foreach \name / \y in {3,...,4}
        \path[yshift=0.5cm]
            node[hidden neuron] (H-\name) at (\layersep,-\y cm) {soft};
            
    % Draw the sum layer node 
    
    \node[neuron, right of=H-2] (S) {$\Sigma_{\gls{bth}}$};

    % Draw the output layer node
    
    \node[output neuron,pin={[pin edge={->}]right:output}, right of=S] (O) {$\sigma$};
    
    %\node[output neuron,pin={[pin edge={->}]right:Output}, right of=H-2] (O) {$\sigma(\cdot)$};
    
    

    % Connect every node in the input layer with every node in the
    % hidden layer.
%    \foreach \source in {1,...,4}
        \foreach \dest in {1,2}
            \path (I-1) edge (H-\dest);

        \foreach \dest in {3,4}
            \path (I-2) edge (H-\dest);
        \foreach \dest in {3,4}
            \path (I-3) edge (H-\dest);
        \foreach \dest in {3,4}
            \path (I-4) edge (H-\dest);

        % \foreach \dest in {5,6}
        %     \path (I-3) edge (H-\dest);

    % Connect every node in the hidden layer with the output layer
    \foreach \source in {1,...,4}
        \path (H-\source) edge (S);
        
    % connect Sigma with sigma
    \path (S) edge (O);

    % Annotate the layers
    \node[annot,above of=H-1, node distance=1cm] (hl) {softmax layer};
    \node[annot,above of=I-1,node distance=1cm] {weights $\ag_j$};
    %\node[annot,right of=hl] (s) {};
    \node[annot, below of=O, node distance=1cm] (s) {sigmoid function};
    \node[annot, below of=S,node distance=1cm] {summation function};
    
    \draw [orange] (2,0) rectangle (3,-1.9);
    % \draw [red] (2,-2) rectangle (3,-4);
    
    \node[annotrectangle,right of=H-1, node distance=1.5cm] {soft outputs $\q_{\ag_j}(\gls{x}_j)$}; 

\end{tikzpicture}
\caption{Proposed shallow architecture to maximize (\ref{eq:lqa}).}
\label{fig:nn}
\end{figure}


\paragraph{Stochastic gradient ascent as a quantization provider}

By relying on stochastic gradient ascent, the smoothed likelihood (\ref{eq:lqa}) can be maximized over $\left(\ag, \gls{bth} \right)$. Due to its convergence properties~\cite{bottou2010large}, the results should be close to the maximizers of the original likelihood (\ref{eq:lq}) if the model is well-specified, when there is a true underlying quantization. However, in the misspecified model case, there is no such guarantee. Therefore, to be more conservative, we evaluate at each training epoch $(s)$ the quantization $\hat{\q}^{(s)}$ resulting from the \textit{maximum a posteriori} procedure explicited in Equation~(\ref{eq:ht}), then classically estimate the \gls{lr} parameter \textit{via} maximum likelihood, as done in Equation~(\ref{eq:lq}):
\begin{equation} \label{eq:lr_param_q}
\hat{\gls{bth}}^{(s)} = \argmax_{\gls{bth}} \ell_{\hat{q}^{(s)}}(\gls{bth}; (\gls{bbx},\gls{bby}))
\end{equation}
and the resulting $\mbox{BIC}(\hat{\gls{bth}}^{(s)})$ as in (\ref{eq:BICq}). If $S$ is a given maximum number of iterations of the stochastic gradient ascent algorithm, the quantization retained at the end is then determined by the optimal epoch
\begin{equation} \label{eq:opt_epoch}
s_\star=\argmin_{s\in \{1,\ldots, S\}} \mbox{BIC}(\hat{\gls{bth}}^{(s)}).
\end{equation}
You can think of $S$ as a computational budget: contrary to classical early stopping rules (\textit{e.g.}\ based on validation loss) used in neural network fitting, this network only acts as a stochastic quantization provider for~\eqref{eq:opt_epoch} which will naturally prevent overfitting. We reiterate that, in~\eqref{eq:opt_epoch}, the BIC can be swapped for the user's favourite model choice criterion.

Lots of optimization algorithms for neural networks have been proposed, which all come with their hyperparameters. As, in the general case, $\ell_{\q_{\ag}}(\gls{bth} ; (\gls{bbx},\gls{bby}))$ of Equation~\eqref{eq:lqa} is not guaranteed to be convex, there might be several local maxima, such that all these optimization methods might diverge, converge to a different maximum, or at least converge in very different numbers of epochs, as can be examplified in Animation~\ref{fig:anim_sgd}\footnote{Reproduced from \url{https://github.com/wassname/viz_torch_optim}}. We chose the RMSProp method, which showed good results and is one of the standard methods.

\begin{figure}[!ht]
\begin{animateinline}[poster=first, controls=all, palindrome, autopause, autoresume, width=\textwidth]{3}
\multiframe{300}{i=1+1}{\includegraphics{figures/chapitre4/optimization_methods/viz-\i.png}}%
\end{animateinline}
\caption{\label{fig:anim_sgd} Animation of several optimization methods (the $\star$ denotes the global maximum).}
\end{figure}

 
\paragraph{Choosing an appropriate number of levels}

Concerning now the number of intervals or factor levels $\boldsymbol{m} = (\gls{mj})_1^d$, they have also to be estimated since in practice they are unknown. Looping over all candidates $\boldsymbol{m}$ is intractable. But in practice, by relying on the \textit{maximum a posteriori} procedure developed in Equation~(\ref{eq:ht}), a lot of unseen factor levels might be dropped, \textit{e.g.}\ if $q_{\ag_{j,h}}(\gls{xij}) \ll 1$ for all training observations $\gls{xij}$, the level $h$ ``vanishes'', \textit{i.e.}\ $\hat{q}_{j,h} = 0$. In practice, we recommend to start with a user-chosen $\gls{bm}=\boldsymbol{m}_{\max}$ and we will see in the experiments of Section~\ref{sec:experiments} that the proposed approach is able to explore small values of $\boldsymbol{m}$ and to select a value $\hat{\boldsymbol{m}}$ drastically smaller than $\boldsymbol{m}_{\max}$. This phenomenon, which reduces the computational burden of the quantization task, is also illustrated in Section~\ref{sec:experiments}.

The full algorithm is described in Appendix~\ref{app1:glmdiscNN}.


\section{An \gls{sem} approach} \label{sec:sem}
 
 
In what follows, the quantization $\q(\gls{bx})$ is seen as a latent feature denoted by $\bm{\mathfrak{Q}}$. The same notations can be introduced for this new feature:  $\bqk$ is an observation of $\bm{\mathfrak{Q}}$, $\bqk_j$ will designate the $j^{\text{th}}$ vectorial component of $\bqk$, $\bbqk$ will designate the $n$-sample, and so on. In the following section, we translate earlier assumptions in probabilitics terms. In the subsequent section, we make good use of these assumptions to provide a continuous relaxation of the quantization problem, as was empirically argumented in Section~\ref{subsec:relaxation}. This relaxation is equivalent to the one proposed in Section~\ref{sec:proposal}, although its use differs drastically, as will be emphasized in Section~\ref{subsec:stoch}.

\subsection{Probabilistic assumptions regarding the quantization latent feature}

Firstly, only the well-specified model case is considered, which translates, with this new latent feature, as a probabilistic assumption:
\begin{equation} \label{hyp:true}
\exists \gls{bthstar}, \bqk^\star \text{s.t.\ } Y \sim p_{\gls{bthstar}}(\cdot | \bqk^\star)
\end{equation}
Secondly, the result of the quantization is assumed to be ``self-contained'' w.r.t.\ the predictive information in $\gls{bx}$, \textit{i.e.}\ it is assumed that all available information about $y$ in $\gls{bx}$ has been ``squeezed'' by quantizing the data:
\begin{equation} \label{hyp:squeeze}
\forall \gls{bx},y,\: p(y|\gls{bx},\bqk) = p(y|\bqk)
\end{equation}
Thirdly, the component-wise nature of the quantization can be stated as:
\begin{equation} \label{hyp:component}
\forall \gls{bx},\bqk,\: p(\bqk|\gls{bx}) = \prod_{j=1}^d p(\bqk_j | \gls{x}_j)
\end{equation}



\subsection{Continuous relaxation of the quantization as seen as fuzzy assignment} \label{subsec:fuzzy}

If we consider the deterministic discretization scheme defined in Section~\ref{sec:model_selection}, we have, analogous to Equation~\eqref{eq:qj}:
$$
p(\bqk_j = \gls{ehmj} | \gls{x}_j) = 1 \text{ if } \gls{x}_j \in C_{j,h},
$$
which is a step function. Rewriting $p(y| \gls{bx})$ by integrating over these new latent features,
% and using hypotheses~\ref{hyp:true}, \ref{hyp:squeeze} and \ref{hyp:component} respectively, 
we get:
\begin{align*}
p(y | \gls{bx}) & = p_{\gls{bthstar}}(y | \bqk^\star) & \text{ (using \eqref{hyp:true}) } \\
& = \sum_{\bqk \in \Q} p(y, \bqk | \gls{bx}) \\
& = \sum_{\bqk \in \Q} p(y | \bqk, \gls{bx}) p(\bqk | \gls{bx}) \\
& = \sum_{\bqk \in \Q} p(y | \bqk) p(\bqk | \gls{bx}) & \text{ (using \eqref{hyp:squeeze}) } \\
& = \sum_{\bqk \in \Q} p(y | \bqk) \prod_{j=1}^d p(\bqk_j | \gls{x}_j) & \text{ (using \eqref{hyp:component}) }
\end{align*}
The well-specified model hypothesis~\eqref{hyp:true} yields for all $\gls{x}_j$, $p(\bqk_j^\star | \gls{x}_j) = 1$. Conversely, for $\bqk \in \Q$ such that $\bqk \cancel{\mathcal{R}_{\gls{T}_n}} \bqk^\star$, there exists a feature $j$ and an observation $\gls{xij}$ such that $p(\bqk_j | \gls{x}_j) = 0$. Consequently, the above sum, over all training observations in $\gls{T}_n$, reduces to:
\begin{align*}
p(\gls{bby} | \gls{bbx}) & = \prod_{i=1}^n p(y_i | \gls{bx}_i) \\
 & = \sum_{\bqk \in \Q} \prod_{i=1}^n p(y_i | \bqk_i) \prod_{j=1}^d p(\bqk_{i,j} | \gls{xij}) \\
 & = \prod_{i=1}^n p(y_i | \bqk_i^\star) \prod_{j=1}^d p(\bqk_{i,j}^\star | \gls{xij})
\end{align*}
Thus, we have :
\[ \bqk^\star = \argmax_{\bqk \in \Q} \prod_{i=1}^n p(y_i | \bqk_i ) \prod_{j=1}^d p(\bqk_{i,j} | \gls{xij}). \]
This new formulation of the best quantization is still intractable since it requires to evaluate all quantizations in $\Q$ just like in~\eqref{eq:BICq}, although all terms except $\bqk^\star$ contribute to $0$ in the above $\argmax$. In the misspecfied model-case however, there is no such guarantee but it can still be claimed that the best candidate $\bqk^\star$ in terms of criterion~\eqref{eq:BICq} dominates the sum.

Our goal in the next section is to generate good candidates $\bbqk$ as in Section~\ref{sec:estim}. Among other things detailed later on, models for $p(y | \bqk)$ and $p(\bqk_j | \gls{x}_j)$ shall be proposed. 
A stochastic ``quantization provider'' is designed as in the previous section. Following arguments of the preceding paragraph, its empirical distribution of generated candidates shall be dominated by $\q^\star$,
%If the resulting MCMC is efficient, $\q^\star$ will be the mode of the empirical distribution of generated candidates,
which, as in Section~\ref{sec:proposal} with the neural network approach, can be selected with the BIC criterion~\eqref{eq:BICq}. Using \eqref{hyp:true}, it seems most natural to use a \gls{lr} for $p( y | \bqk_j)$. Following Section~\ref{sec:proposal} and as was empirically argumented in Section~\ref{subsec:relaxation}, the instrumental distribution $p(\bqk_j | \gls{x}_j)$ will take a similar form as $\q_{\ag}$. 

{\bf For a continuous feature}, we resort to a polytomous logistic regression, similar to the softmax function of Equation~\eqref{eq:softmax} without the over-parametrization (one level per feature $j$, say $\gls{mj}$, is considered reference):
\[ p_{\ag_{j,h}}(\bqk_j = \gls{ehmj} | \gls{x}_j) = \begin{cases} \frac{1}{\sum_{h'=1}^{\gls{mj}-1} \exp(\alpha_{j,h'}^0 + \alpha_{j,h'}^1 \gls{x}_j)} \text{ if } h = \gls{mj}, \\ \frac{\alpha_{j,h}^0 + \alpha_{j,h}^1 \gls{x}_j}{\sum_{h'=1}^{\gls{mj}-1} \exp(\alpha_{j,h'}^0 + \alpha_{j,h'}^1 \gls{x}_j)} \text{ otherwise.} \end{cases} \]
{\bf For categorical features}, simple contingency tables are employed:
\[ p_{\ag_{j,h}^o}(\bqk_j = \gls{ehmj} | \gls{x}_j) = \frac{|\bbqk_{j,h}|}{|\{\gls{x}_j=o\}|} \text{ for } 1 \leq o \leq \gls{lj} \]
Similarly, $p_{\ag_j}(\bqk_j | \gls{x}_j)$ are no more step functions but smooth functions as in Figure~\ref{fig:MAP}.

\paragraph{Remark on polytomous logistic regressions}

Since the resulting latent categorical feature can be interpreted as an ordered categorical features (the \textit{maximum a posterior} operation yields contiguous intervals as argued in Section~\ref{subsec:relaxation}), ordinal ``parallel'' \gls{lr}~\cite{o2006logistic} could be used (provided levels $h$ are reordered). This particular model is of the form:
\[ \ln \frac{p(\bqk_{j} = \bm{e}_{h+1}^{\gls{mj}} | \gls{x}_j)}{p(\bqk_{j} = \gls{ehmj} | \gls{x}_j)} = \alpha_{j,h,0} + \alpha_{j} \gls{x}_j, 1 \leq h < \gls{mj}, \]
which restricts the number of parameters since all levels $h$ share the same slope $\alpha_j$. Its advantages lie in the fact that it might lead to sharper door functions quicker, and that it has fewer parameters to estimate, thus reducing \textit{de facto} the estimation variance of each ``soft'' quantization $p_{\ag_j}$. However, it makes it harder for levels to ``vanish'' which would require to iterate over the number of levels per feature $\gls{mj}$ which we wanted to avoid (see Paragraph~\nameref{par:choosing_sem} in the next section). In practice, it yielded similar results to polytomous \gls{lr} such that they remain a parameter of the \textsf{R} package \textit{glmdisc} (see Appendix~\ref{app2}).

\subsection{Stochastic search of the best quantization} \label{subsec:stoch}

We parametrized $p(y|\gls{bx})$ as:
\begin{equation}
p(y | \gls{bx}, \gls{bth}, \ag) = \sum_{\bqk \in \Q} p_{\gls{bth}}(y | \bqk) \prod_{j=1}^d p_{\ag_j}(\bqk_j | \gls{bbx}_j)
\end{equation}
A straightforward way to maximize the likelihood of $p(y | \gls{bx}, \gls{bth}, \ag)$ in $(\gls{bth}, \ag)$ (not to be mistaken with~\eqref{eq:lqa}), as was done in Section~\ref{sec:proposal}, to deduce $\q^\star$ from $\ag$ \textit{via} the $\argmax$ operation (see Section~\ref{subsec:relaxation} and Equation~\eqref{eq:ht}), is to use an \gls{em} algorithm~\cite{dempster1977maximum}.

However, maximizing this likelihood directly is intractable as the Expectation step requires to sum over $\bqk \in \Q$. Classically, the \gls{em} can be replaced by the \acrlong{sem}~\cite{celeux1985sem} algorithm: the expectation (the sum over $\bqk \in \Q$) is approximated by the empirical distribution of draws $\bqk^{(1)}, \dots, \bqk^{(S)}$ from $p_{\gls{bth}}(y | \cdot) \prod_{j=1}^d p_{\ag_j}(\cdot | \gls{x}_j)$.

\subsubsection{\gls{sem} as a quantization provider}

As the parameters $\ag$ of $\q_{\ag}$ were initialized randomly in the neural network approach, the latent features observations $\bbqk^{(0)}$ are initialized randomly. At step $s$, the \gls{sem} algorithm allows us to compute the \gls{mle} ${\gls{bth}}^{(s)}$ (resp. ${\ag}^{(s)}$) of $\gls{bth}$ (resp. $\ag$) given $\bbqk^{(s)}$ by maximizing the following likelihoods (M-steps):
\begin{alignat}{2}
{\gls{bth}}^{(s)} & = \argmax_{\gls{bth}} \ell(\gls{bth}; \bbqk^{(s)}, \gls{bby}) && = \argmax_{\gls{bth}} \sum_{i=1}^n \ln p_{\gls{bth}}(y_i | \bqk^{(s)}_i), \nonumber \\
{\ag_j}^{(s)} & = \argmax_{\ag_j} \ell(\ag_j; \gls{bbx}_j, \bbqk^{(s)}_j) && = \argmax_{\ag_j} \sum_{i=1}^n \ln p_{\ag_j}(\bqk^{(s)}_{i,j} | \gls{xij}) \text{ for } 1 \leq j \leq d. \label{eq:mle_ag}
\end{alignat}

As the \gls{lr} $p_{\gls{bth}}(y | \bqk)$ is multivariate, it is hard to sample simultaneously all latent features. We have to resort to the Gibbs-sampler~\cite{casella1992explaining}: $\bqk_j$ is sampled while holding latent features $\bqk_{-\{j\}}$ fixed (S-step):
\begin{equation} \label{eq:q_draw}
\bqk_j^{(s+1)} \sim p_{\hat{\gls{bth}}^{(s)}}(y | \bqk_{-\{j\}}^{(s)}, \cdot) p_{\hat{\ag}_j^{(s)}}(\cdot | \gls{x}_j)
\end{equation}
This process is repeated for all features $1 \leq j \leq d$.

This \gls{sem} provides parameters ${\ag}^{(1)}, \dots, {\ag}^{(S)}$ which can be used to produce $\hat{\q}^{(1)}, \dots, \hat{\q}^{(S)}$ following the \textit{maximum a posteriori} scheme from Equation~\eqref{eq:ht}, adapated to this new formulation:
\[ \hat{\q}_j^{(s)}(\cdot) = \argmax_{h} p_{\ag_j^{(s)}}(\gls{ehmj} | \cdot ) .\]
The \gls{lr} parameters $\hat{\gls{bth}}^{(s)}$ on quantized data are obtained similarly as in~\eqref{eq:lr_param_q}. The best proposed quantization $\q^\star$ is thus chosen among them \textit{via e.g.}\ the BIC criterion as in Equation~\eqref{eq:opt_epoch}.


\subsubsection{Validity of the approach}

The pseudo-completed sample $(\gls{bbx}, \bbqk^{(s)}, \gls{bby})$ allows to compute $({\gls{bth}}^{(s)},{\ag}^{(s)})$ which do not converge to the \gls{mle} of $p(\gls{bby} | \gls{bbx}, \gls{bth}, \ag)$, for the simple reason that, being random in essence, it does not converge pointwise. From its authors, the \gls{sem} is however expected to be directed by the \gls{em} dynamics~\cite{celeux_sem} and its empirical distribution converges to the target distribution $p(\gls{bby} | \gls{bbx}, \gls{bth}, \ag)$ provided such a distribution exists and is unique. This existence is guaranteed by remarking that for all features $j$, $ p(\bqk_j | \gls{x}_j, \bqk^{(s)}_{-\{j\}} y, \gls{bth}, \ag) \propto p_{\gls{bth}}(y | \bqk^{(s)}_{-\{j\}}, \bqk_j) p_{\ag_j}(\bqk_j | \gls{x}_j) > 0 $ by definition of the \gls{lr} and polytomous logistic regressions or the contingency tables respectively. The uniqueness is not guaranteed since levels can disappear and there is an absorbing state (the empty model): this point is detailed in the next Section.

In its original purpose~\cite{celeux_sem}, the \gls{sem} was employed either to find good starting points for the \gls{em} (\textit{e.g.}\ to avoid local maxima) or to propose an estimator of the \gls{mle} of the target distribution as the mean or the mode of the resulting empirical distribution, eventually after a burn-in phase. As, in our setting, we are not directly interested in the \gls{mle} but only to the best quantization in the sense of Equation~\eqref{eq:BICq}. The best proposed quantization $\q^\star$ is thus chosen among them \textit{via} the BIC criterion as in Equation~\eqref{eq:opt_epoch} as stated in the previous paragraph.

\subsubsection{Choosing an appropriate number of levels} \label{par:choosing_sem}

Contrary to the neural network approach developed in Section~\ref{sec:proposal}, the \gls{sem} algorithm alternates between drawing $\bbqk^{(s)}$ and fitting ${\gls{bth}}^{(s)}$ and ${\ag}^{(s)}$  at each step $s$. Therefore, additionally to the phenomenon of ``vanishing'' levels caused by the \textit{maximum a posteriori} procedure similar to the neural network approach, if a level $h$ of $\bqk$ is not drawn, following Equation~\eqref{eq:q_draw}, at step $s$, then at step $s+1$ when adjusting parameters $\ag_j$ by maximum likelihood from Equation~\eqref{eq:mle_ag}, this level will have disappeared and cannot be drawn again. A Reversible-Jump MCMC approach would be needed~\cite{green1995reversible} to ``resuscitate'' these levels, which is not needed in the neural network approach because its architecture is fixed in advance. As a consequence, with a design matrix of fixed size $n$, there is a non-zero probability that for any given feature, any of its levels collapses at each step such that $\gls{mj}^{(s+1)} \leftarrow \gls{mj}^{(s)} - 1$.

The MCMC has thus an absorbing state for which all features are quantized into one level (the empty model with no features) which is reached in a finite number of steps (although very high if $n$ is sufficiently large as is the case with \textit{Credit Scoring} data). The \gls{sem} algorithm is an effective way to start from a high number of levels per feature $\gls{bm}_{\max}$ and explore smaller values.

The full algorithm is described in Appendix~\ref{app1:glmdiscSEM}.

\section{Numerical experiments} \label{sec:experiments}

This section is divided into three complementary parts to assess the validity of our proposal, that is called hereafter \textit{glmdisc}-NN and \textit{glmdisc}-SEM, designating respectively the approaches developed in Sections~\ref{sec:proposal} and~\ref{sec:sem}. First, simulated data are used to evaluate its ability to recover the true data generating mechanism. Second, the predictive quality of the new learned representation approach is illustrated on several classical benchmark datasets from the UCI library. Third, we use it on \textit{Credit Scoring} datasets provided by \gls{cacf}. The code of all experiments, excluding the confidential real data, can be retrieved following the guidelines in Appendix~\ref{app2}.


\subsection{Simulated data: empirical consistency and robustness}

Focus is here given on discretization of continuous features (similar experiments could be conducted on categorical ones). Two continuous features $\gls{x}_1$ and $\gls{x}_2$ are sampled from the uniform distribution on $[0,1]$ and discretized as exemplified on Figure~\ref{fig:exp_sim} by using
\[\q_1(\cdot)=\q_2(\cdot) = (\mathds{1}_{]-\infty,1/3]}(\cdot),\mathds{1}_{]1/3,2/3]}(\cdot),\mathds{1}_{]2/3,\infty[}(\cdot)).\]
Here, following (\ref{eq:Cjhcont}), we have $d=2$ and $m_1=m_2=3$ and the cutpoints are $c_{j,1}=1/3$ and $c_{j,2}=2/3$ for $j=1,2$. Setting $\gls{bth}=(0,-2,2,0,-2,2,0)$, the target feature $y$ is then sampled from $p_{\gls{bth}}(\cdot | \q(\gls{bbx}))$ via the logistic model (\ref{eq:reglogq}).

\begin{figure}[!ht]
\centering
\begin{tikzpicture}
      \draw[->] (-1,0) -- (9,0) node[right] {$\gls{x}$};
      \draw[->] (0,-1) -- (0,3) node[above] {$p(\gls{x})$};
      \draw[scale=1,domain=0.5:7,smooth,variable=\y,red,thick]  plot ({\y},2.5);
      \draw[scale=1,domain=-1:0.5,smooth,variable=\y,red,thick]  plot ({\y},0);
      \draw[scale=1,domain=7:8.5,smooth,variable=\y,red,thick]  plot ({\y},0);
      \draw[scale=1,domain=0:2.5,smooth,variable=\x,red]  plot (0.5,{\x});
      \draw[scale=1,domain=0:2.5,smooth,variable=\x,red]  plot (7,{\x});
      
      \draw[scale=1,domain=-0.2:2.8,smooth,variable=\x,blue]  plot (2.67,{\x});
      \draw[scale=1,domain=-0.2:2.8,smooth,variable=\x,blue]  plot (4.83,{\x});

		\node[scale=0.7] (q1) at  (1.2,2.7) {\small $\q(\gls{x})=(1,0,0)$};
		\node[scale=0.7] (q2) at  (3.5,2.7) {\small $\q(\gls{x})=(0,1,0)$};
		\node[scale=0.7] (q3) at  (6,2.7) {\small $\q(\gls{x})=(0,0,1)$};

		\node[scale=0.7] (x1) at  (0.5,-0.5) {$0$};
		\node[scale=0.7] (x2) at  (2.67,-0.5) {$c_1=1/3$};
		\node[scale=0.7] (x3) at  (4.83,-0.5) {$c_2=2/3$};
		\node[scale=0.7] (x4) at  (7,-0.5) {$1$};

\end{tikzpicture}
\caption{\label{fig:exp_sim} Pdf of the simulated continuous data $\gls{x}$ and the true quantization $\q$.}
\end{figure}


From the \textit{glmdisc} algorithm, we studied three cases:
\begin{enumerate}[(a)]
    \item First, the quality of the cutoff estimator $\hat{c}_{j,2}$ of $c_{j,2} = 2/3$ is assessed when the starting maximum number of intervals per discretized continuous feature is set to its true value $m_1=m_2= 3$;
    \item Second, I estimated the number of intervals $\hat{m}_1$ of $m_1=3$ when the starting maximum number of intervals per discretized continuous feature is set to $m_{\text{max}} = 10$; 
    \item Last, I added a third feature $\gls{x}_3$ also drawn uniformly on $[0,1]$ but uncorrelated to $\gls{y}$ and estimated the number $\hat{m}_3$ of discretization intervals selected for $\gls{x}_3$. The reason is that a non-predictive feature which is discretized or grouped into a single value is \textit{de facto} excluded from the model, and this is a positive side effect.
\end{enumerate}
From a statistical point of view, experiment (a) assesses the empirical consistency of the estimation of $C_{j,h}$, whereas experiments (b) and (c) focus on the consistency of the estimation of $\gls{mj}$. The results are summarized in Table~\ref{tab:estim_precision} where 95\% confidence intervals (CI~\cite{sun2014fast}) are given, with a varying sample size. Note in particular that the slight underestimation in (b) is a classical consequence of the BIC criterion on small samples. 

\begin{table}[ht]
    \centering
    \caption{For \textit{glmdisc}-NN and \textit{glmdisc}-SEM and different sample sizes $n$, (A) CI of $\hat{c}_{j,2}$ for $c_{j,2} = 2/3$. (B) Bar plot of $\hat{m} = 2, 3, 4$ (resp.) for $m_1=3$. (C) Bar plot of $\hat{m}_3 = 1, 2, 3$ (resp.) for $m_3=1$.}
    \label{tab:estim_precision}
\begin{tabular}{lllllll}
Algorithm & $n$ & (a) $\hat{c}_{j,2}$ & (b) & $\hat{m}_1$ & (c) & $\hat{m}_3$ \\
\hline
\hline
\textit{glmdisc}-NN & $1{,}000$ & $[0.656,0.666]$ & \myobar{9}{90}{1} & \mybar{60}{32}{8} \\
\textit{glmdisc}-SEM & $1{,}000$ & $[0.664,0.669]$ & \myobar{2}{53}{44} & \mybar{34}{56}{10} \\
\hline
\textit{glmdisc}-NN & $10{,}000$ & $[0.666,0.666]$ & \myobar{0}{100}{0} & \mybar{88}{12}{0} \\
\textit{glmdisc}-SEM & $10{,}000$ & $[0.666,0.666]$ & \myobar{2}{69}{30} & \mybar{30}{48}{22}
\end{tabular}
\end{table}

To complement these experiments on simulated data following a well-specified model, a similar study can be done for categorical features: 10 levels are drawn uniformly and 3 groups of levels, which share the same log-odd ratio, are created. The same phenomenon as in Table~\ref{tab:estim_precision} is witnessed: the empirical distribution of the estimated number of groups of levels is peaked at its true value of 3.

Finally, it was argued in Section~\ref{sec:model_selection} that by considering all features when quantizing the data, relying on a multivariate approach could yield better results than classical univariate techniques in presence of correlation. This claim is verified in Table~\ref{tab:sim_false} where multivariate heteroskedastic gaussian data is simulated on which the log odd ratio of $y$ depends linearly (misspecified model setting for the quantized \gls{lr}). The proposed approach yields significantly better results.

\begin{table}[ht]
    \centering
    \caption{Gini of the resulting misspecified \gls{lr} from quantized data using ChiMerge, MDLP and \textit{glmdisc}-SEM: the multivariate approach is able to capture information about the correlation structure.}
    \label{tab:sim_false}
\begin{tabular}{llll}
 & ChiMerge & MDLP & \textit{glmdisc}-SEM \\
\hline
Performance & 50.1 (1.6) & 77.1 (0.9) & \textbf{80.6} (0.6)
\end{tabular}
\end{table}



 \newlength\figureheight
 \newlength\figurewidth
 \setlength\figureheight{4cm}
 \setlength\figurewidth{14cm}
 
  \begin{figure}[!ht]
    \centering
    \begin{subfigure}[t]{\textwidth}
        \centering
        % This file was created by matplotlib2tikz v0.6.18.
\begin{tikzpicture}

\definecolor{color0}{rgb}{0,0.75,0.75}
\definecolor{color1}{rgb}{0.75,0,0.75}
\definecolor{color2}{rgb}{0.75,0.75,0}

\begin{axis}[
height=\figureheight,
legend cell align={left},
legend entries={{${q}_{\bm{\alpha}_{0,0}}$},{${q}_{\bm{\alpha}_{0,1}}$},{${q}_{\bm{\alpha}_{0,2}}$},{${q}_{\bm{\alpha}_{0,3}}$},{${q}_{\bm{\alpha}_{0,4}}$},{$c_{0,1}$},{$c_{0,2}$},{$\hat{c}_{0,2}$}},
legend style={at={(0.91,0.5)}, anchor=east, draw=white!80.0!black},
tick align=outside,
tick pos=left,
title={Continuous feature 0 at iteration 5},
width=\figurewidth,
x grid style={white!69.01960784313725!black},
xlabel={$x_1$},
xmin=0, xmax=1,
y grid style={white!69.01960784313725!black},
ylabel={${q}_{\bm{\alpha}_{0,h}}$},
ymin=0, ymax=1
]
\addlegendimage{no markers, red}
\addlegendimage{no markers, color0}
\addlegendimage{no markers, color1}
\addlegendimage{no markers, color2}
\addlegendimage{no markers, black}
\addlegendimage{no markers, green!50.0!black}
\addlegendimage{no markers, blue}
\addlegendimage{no markers, blue}
\addplot [thick, red]
table [row sep=\\]{%
0.0011059794751318	0.990096211433411 \\
0.00144600828615515	0.990088164806366 \\
0.00151620394328711	0.990086495876312 \\
0.00161381928831428	0.990084171295166 \\
0.00193644961641948	0.99007648229599 \\
0.0024363370221504	0.990064680576324 \\
0.00384167794354173	0.990030884742737 \\
0.00421063900202368	0.990022242069244 \\
0.00632372037002726	0.989971518516541 \\
0.00947686783375368	0.989895045757294 \\
0.0099356428556221	0.989883720874786 \\
0.0101317191976503	0.989879012107849 \\
0.0102519961026396	0.989876091480255 \\
0.0116572936797861	0.989841759204865 \\
0.0122763055741271	0.989826560020447 \\
0.0128807282915644	0.989811718463898 \\
0.0131006826676461	0.989806354045868 \\
0.0132105085232722	0.989803791046143 \\
0.0150477352877949	0.989758253097534 \\
0.0151511241047927	0.989755690097809 \\
0.0168553495563415	0.989713549613953 \\
0.0191895991753259	0.989655137062073 \\
0.0192724193499449	0.989653170108795 \\
0.0202131188025382	0.989629566669464 \\
0.0251886934986562	0.989503502845764 \\
0.0255409812646521	0.989494621753693 \\
0.0256196546755231	0.989492535591125 \\
0.0263800727822662	0.989473164081573 \\
0.026825697968402	0.989461719989777 \\
0.0272388337061176	0.989451229572296 \\
0.0276781997826389	0.989439904689789 \\
0.0277969734855988	0.989436745643616 \\
0.027888291941232	0.98943430185318 \\
0.032801074197002	0.989307105541229 \\
0.0337126623182875	0.989283204078674 \\
0.0349109578825388	0.989251792430878 \\
0.0367757145561887	0.989202558994293 \\
0.0368887253476963	0.989199638366699 \\
0.0376301111814882	0.989180088043213 \\
0.0383020767002249	0.989162087440491 \\
0.0385765124363792	0.989154756069183 \\
0.0386281055032889	0.989153444766998 \\
0.0397134557815922	0.989124536514282 \\
0.0400530872540832	0.989115476608276 \\
0.0408224228009035	0.989094793796539 \\
0.042416614098386	0.989051878452301 \\
0.042542997860721	0.989048600196838 \\
0.0425634328976269	0.989048182964325 \\
0.0426938593122377	0.989044547080994 \\
0.0436353215690566	0.989019095897675 \\
0.0476491609572692	0.988909900188446 \\
0.0476764809468383	0.988909184932709 \\
0.0480632979275277	0.988898575305939 \\
0.0489335324060199	0.988874673843384 \\
0.0494953541501341	0.988859176635742 \\
0.0498656975684827	0.988848924636841 \\
0.0507299142819697	0.988825023174286 \\
0.0520149216066462	0.988789319992065 \\
0.053491737186121	0.98874819278717 \\
0.0545456118282759	0.988718867301941 \\
0.0549368100196436	0.988707900047302 \\
0.0549667939602374	0.988706946372986 \\
0.0555159955251522	0.988691568374634 \\
0.0574839546417889	0.988636016845703 \\
0.0576927502423773	0.988630056381226 \\
0.0580609228746025	0.988619565963745 \\
0.0610070220733733	0.988535583019257 \\
0.0613718310828713	0.988525211811066 \\
0.0624819644863142	0.988493263721466 \\
0.0632024977130776	0.988472521305084 \\
0.0641707318537266	0.988444447517395 \\
0.064803290117067	0.988426208496094 \\
0.0661546388560771	0.988386809825897 \\
0.0675945492445934	0.988344788551331 \\
0.0701208744758773	0.988270401954651 \\
0.0704495440664726	0.988260746002197 \\
0.0707913481046274	0.988250613212585 \\
0.0709503961374458	0.988245725631714 \\
0.0723079949898875	0.988205313682556 \\
0.0729582139062805	0.988186001777649 \\
0.0736427592129116	0.988165497779846 \\
0.0738431652021745	0.988159537315369 \\
0.074094042941424	0.988152086734772 \\
0.0741988699821797	0.988148748874664 \\
0.0745344402647278	0.988138735294342 \\
0.0761651718345726	0.988089501857758 \\
0.0762565104011357	0.988086819648743 \\
0.0764247618188199	0.988081812858582 \\
0.0780473992355428	0.988032579421997 \\
0.0784939963588281	0.988018870353699 \\
0.0787288834363844	0.988011837005615 \\
0.0796528614743472	0.987983345985413 \\
0.0821291956626848	0.987907350063324 \\
0.0836539892117674	0.987860023975372 \\
0.0837976822096664	0.987855732440948 \\
0.0840863722935196	0.987846791744232 \\
0.0847664004367794	0.987825334072113 \\
0.0848642352310873	0.987822473049164 \\
0.0852150851090396	0.987811505794525 \\
0.085891869345856	0.98779034614563 \\
0.087068315817913	0.987753331661224 \\
0.0885849846877879	0.98770546913147 \\
0.0903431131516136	0.987649619579315 \\
0.0909278592293694	0.987630903720856 \\
0.0914296081009306	0.987614989280701 \\
0.0922298300096035	0.987589418888092 \\
0.0926517333242025	0.987575769424438 \\
0.0930086769471365	0.987564265727997 \\
0.0936368650628759	0.987544059753418 \\
0.0939832817677504	0.987532913684845 \\
0.0963578674192209	0.987455725669861 \\
0.0969604240071604	0.987436056137085 \\
0.0974174263218397	0.987421154975891 \\
0.0997464409700664	0.98734450340271 \\
0.100249930959155	0.987327754497528 \\
0.100820744388044	0.987308919429779 \\
0.101229398304142	0.98729532957077 \\
0.102271393895604	0.987260580062866 \\
0.102325882258744	0.987258732318878 \\
0.103474916599901	0.987220406532288 \\
0.104369461810087	0.987190425395966 \\
0.104849386693897	0.987174034118652 \\
0.105274572558957	0.987159729003906 \\
0.105437385176067	0.987154245376587 \\
0.105718642159513	0.987144768238068 \\
0.106451679001465	0.987119913101196 \\
0.106754108203929	0.987109661102295 \\
0.106889477031479	0.987105011940002 \\
0.108308586301563	0.987056612968445 \\
0.108355506097126	0.987054944038391 \\
0.110864090628752	0.986968576908112 \\
0.112062699234041	0.986927092075348 \\
0.112686863552251	0.986905515193939 \\
0.113787419472266	0.986867070198059 \\
0.114114908535732	0.986855447292328 \\
0.114926076298882	0.98682701587677 \\
0.115710925433917	0.986799418926239 \\
0.116193807044869	0.986782312393188 \\
0.116352662276253	0.986776649951935 \\
0.116979956596317	0.986754596233368 \\
0.117247317403657	0.986745059490204 \\
0.118793828877862	0.986689925193787 \\
0.119352906137166	0.986669838428497 \\
0.120388240876593	0.986632704734802 \\
0.122357154089364	0.986561357975006 \\
0.125712799722249	0.986438751220703 \\
0.126988316690772	0.986391663551331 \\
0.130289689058528	0.986268401145935 \\
0.130830575131366	0.986247956752777 \\
0.13178708265069	0.986211776733398 \\
0.132217070962463	0.98619544506073 \\
0.132641072460891	0.986179351806641 \\
0.133694684584262	0.986139118671417 \\
0.135586813424106	0.986066520214081 \\
0.136970534315695	0.986012995243073 \\
0.137167897059254	0.986005425453186 \\
0.13731957841644	0.985999643802643 \\
0.138770949832641	0.985942959785461 \\
0.141579671494468	0.985832333564758 \\
0.142337772861435	0.985802233219147 \\
0.142852169410084	0.985781610012054 \\
0.14313843453823	0.985770225524902 \\
0.143709892765619	0.985747277736664 \\
0.143996587198751	0.985735833644867 \\
0.144338718134467	0.985722303390503 \\
0.145474509131844	0.985676288604736 \\
0.146058924442183	0.985652804374695 \\
0.147496370200373	0.98559433221817 \\
0.147934500473159	0.985576450824738 \\
0.148464696403164	0.985554695129395 \\
0.149007159183894	0.985532462596893 \\
0.151290695331351	0.985438108444214 \\
0.153507740604538	0.985345423221588 \\
0.153927498261582	0.98532772064209 \\
0.153953160827713	0.985326647758484 \\
0.154179436777791	0.985317170619965 \\
0.155107736993068	0.985277831554413 \\
0.15519005879533	0.985274434089661 \\
0.155418402417591	0.985264718532562 \\
0.156078407887877	0.985236644744873 \\
0.158100170599977	0.98514997959137 \\
0.158283533549941	0.985142171382904 \\
0.158599777942073	0.98512852191925 \\
0.159258263288539	0.985100209712982 \\
0.159678910282026	0.985081911087036 \\
0.162540881855109	0.984956681728363 \\
0.164077384668961	0.984888792037964 \\
0.164957348092599	0.984849691390991 \\
0.165030574098427	0.984846472740173 \\
0.165280388241905	0.984835386276245 \\
0.167273692684132	0.984745621681213 \\
0.168043854724866	0.984710872173309 \\
0.168088628486329	0.984708905220032 \\
0.168662857989897	0.984682738780975 \\
0.169151147716595	0.984660446643829 \\
0.169673214248392	0.984636664390564 \\
0.170135609058708	0.984615385532379 \\
0.171767974673073	0.984540283679962 \\
0.1728518498484	0.984490036964417 \\
0.172984599941167	0.984483897686005 \\
0.176370551959836	0.98432457447052 \\
0.176588273575192	0.984314322471619 \\
0.176656553576852	0.984311044216156 \\
0.17794804219768	0.984249413013458 \\
0.178429717033185	0.984226286411285 \\
0.179458814233081	0.984176635742188 \\
0.180595104426831	0.984121680259705 \\
0.18064479669491	0.984119236469269 \\
0.182294707450728	0.984038591384888 \\
0.182819109355123	0.984012842178345 \\
0.189141614514094	0.983695864677429 \\
0.190723527970293	0.983614504337311 \\
0.192508826701743	0.983521819114685 \\
0.200712508340217	0.983083128929138 \\
0.202130410523186	0.983005046844482 \\
0.202705868873484	0.982973217964172 \\
0.202975660615516	0.982958137989044 \\
0.2033001668374	0.982939958572388 \\
0.204138047670012	0.982893168926239 \\
0.204584283026216	0.982868194580078 \\
0.205975712225296	0.982789635658264 \\
0.208693969096644	0.982634127140045 \\
0.209097012308874	0.982610762119293 \\
0.210129299396579	0.982550919055939 \\
0.210470524326312	0.982531011104584 \\
0.210726758187617	0.982516050338745 \\
0.210879765452585	0.982507050037384 \\
0.213404555935633	0.982357859611511 \\
0.214779680955682	0.982275724411011 \\
0.215976334819965	0.982203483581543 \\
0.21605980843853	0.982198417186737 \\
0.21626681121775	0.982185900211334 \\
0.217333435243771	0.982120931148529 \\
0.217860184045886	0.982088625431061 \\
0.218810300034316	0.982030212879181 \\
0.219036830922259	0.982016265392303 \\
0.219186045259484	0.982006967067719 \\
0.219509161852637	0.981986999511719 \\
0.219517283100912	0.981986403465271 \\
0.220007941128277	0.981955945491791 \\
0.221124031401732	0.981886148452759 \\
0.221655686766872	0.981852829456329 \\
0.222787328400317	0.981781363487244 \\
0.222812240344664	0.981779634952545 \\
0.224413114839207	0.98167759180069 \\
0.225712117798195	0.98159384727478 \\
0.226064824403414	0.981570899486542 \\
0.22778297811216	0.981458723545074 \\
0.227786137279108	0.981458604335785 \\
0.228255816092263	0.981427669525146 \\
0.22886379454373	0.981387257575989 \\
0.229286896531361	0.981359362602234 \\
0.231570483736477	0.981206357479095 \\
0.232839821990075	0.98112028837204 \\
0.234699792231533	0.98099285364151 \\
0.236598748956898	0.980860710144043 \\
0.236914864918852	0.980838418006897 \\
0.238193170258512	0.980748295783997 \\
0.238267801131912	0.980743169784546 \\
0.241280255564163	0.98052704334259 \\
0.241532477911657	0.980508804321289 \\
0.242724887440966	0.980421721935272 \\
0.243487528482621	0.980365574359894 \\
0.243851755033328	0.980338871479034 \\
0.243931127736832	0.980332911014557 \\
0.245492487219521	0.980216801166534 \\
0.245961870318215	0.980181634426117 \\
0.247220233087027	0.980086445808411 \\
0.248029547774932	0.980025112628937 \\
0.248195183805212	0.980012357234955 \\
0.252597476547261	0.97967004776001 \\
0.254049623190913	0.97955459356308 \\
0.254469169693275	0.979520857334137 \\
0.254777550536903	0.9794961810112 \\
0.254877558399671	0.9794881939888 \\
0.257253649731868	0.979295134544373 \\
0.257671137054733	0.979260742664337 \\
0.26187840669548	0.978908538818359 \\
0.264182169008257	0.978710412979126 \\
0.265462287195374	0.978598713874817 \\
0.267638564660905	0.978406071662903 \\
0.268216851612881	0.978354394435883 \\
0.269980014659659	0.978195011615753 \\
0.270837940398052	0.978116631507874 \\
0.271133018969332	0.97808963060379 \\
0.271522794293612	0.978053689002991 \\
0.272730212426688	0.977941691875458 \\
0.27325151773824	0.977893173694611 \\
0.273458501731612	0.977873682975769 \\
0.275787496734588	0.977653026580811 \\
0.275790346826632	0.977652788162231 \\
0.275963502468307	0.977636277675629 \\
0.276441724078765	0.977590441703796 \\
0.277788622824976	0.977460145950317 \\
0.278249554181787	0.977415263652802 \\
0.278902433397602	0.977351307868958 \\
0.280228184178203	0.977220237255096 \\
0.282170737525181	0.977025628089905 \\
0.284775883532131	0.976759314537048 \\
0.285319633849948	0.976702868938446 \\
0.285945143849347	0.976637721061707 \\
0.287293498153222	0.976496040821075 \\
0.288098109983407	0.976410686969757 \\
0.288474372887691	0.976370573043823 \\
0.288733693686359	0.976342856884003 \\
0.291563977198166	0.97603577375412 \\
0.293976605993451	0.975768089294434 \\
0.294014528290392	0.97576367855072 \\
0.294718175832172	0.975684463977814 \\
0.294831463459233	0.975671648979187 \\
0.294887681144015	0.975665271282196 \\
0.299064488228799	0.975183427333832 \\
0.299223468133874	0.975164771080017 \\
0.299853491506122	0.975090265274048 \\
0.301624334890634	0.974878787994385 \\
0.302100714632492	0.974821269512177 \\
0.302855829384862	0.974729657173157 \\
0.303468685820656	0.974654912948608 \\
0.304335545384868	0.97454822063446 \\
0.305081848845274	0.974455714225769 \\
0.305263786725066	0.974433064460754 \\
0.305511643802704	0.974402070045471 \\
0.30635774546767	0.974296033382416 \\
0.307937429467832	0.974095642566681 \\
0.308578187230734	0.974013686180115 \\
0.313991998160918	0.973299562931061 \\
0.315973848719028	0.973028957843781 \\
0.317413649622303	0.97282874584198 \\
0.319657848506035	0.972511410713196 \\
0.319781920771261	0.972493708133698 \\
0.320496004308145	0.972391188144684 \\
0.320601310305949	0.972375869750977 \\
0.322246867517597	0.972136557102203 \\
0.323306124649751	0.971980512142181 \\
0.323664358184811	0.971927285194397 \\
0.323805398711091	0.971906304359436 \\
0.324816846999054	0.971755027770996 \\
0.325263683488062	0.97168755531311 \\
0.327533460856979	0.971340835094452 \\
0.328385869951453	0.971208572387695 \\
0.329844054701458	0.970979809761047 \\
0.33074904970515	0.970836043357849 \\
0.332974695502457	0.970476984977722 \\
0.333600133199531	0.970374703407288 \\
0.336001356191777	0.969975829124451 \\
0.338057278352338	0.969626247882843 \\
0.33940563378102	0.969393193721771 \\
0.339522610836627	0.969372808933258 \\
0.341295751839904	0.969060778617859 \\
0.343025583329651	0.968751013278961 \\
0.344156227439929	0.968545436859131 \\
0.344485478041551	0.968485295772552 \\
0.346631556569637	0.968087196350098 \\
0.348247004908186	0.967781722545624 \\
0.348257869791207	0.967779576778412 \\
0.350134953366802	0.96741795539856 \\
0.352271571140202	0.966997742652893 \\
0.352403536611892	0.966971576213837 \\
0.356607161458692	0.966115713119507 \\
0.357473355861859	0.965934634208679 \\
0.359032837138255	0.965604305267334 \\
0.360085447455855	0.965378224849701 \\
0.360587038984981	0.965269505977631 \\
0.3618814720936	0.964986562728882 \\
0.362541325151431	0.964840710163116 \\
0.363971114014582	0.964521467685699 \\
0.36446377318254	0.964410364627838 \\
0.364734654416428	0.964348793029785 \\
0.364782079325344	0.964338064193726 \\
0.36626622804169	0.96399849653244 \\
0.36745676602671	0.963722169399261 \\
0.367555756327517	0.963698923587799 \\
0.367766921022286	0.963649451732635 \\
0.368015883464081	0.9635910987854 \\
0.368183587302196	0.963551580905914 \\
0.368646745255496	0.963442385196686 \\
0.368824340831231	0.96340024471283 \\
0.36908666014638	0.963337957859039 \\
0.370227788452905	0.963064968585968 \\
0.371079364739364	0.962859034538269 \\
0.371457393668236	0.962767004966736 \\
0.372521553247257	0.96250593662262 \\
0.372816229090882	0.962433159351349 \\
0.373696444358129	0.962214291095734 \\
0.375236418301489	0.961826264858246 \\
0.375355278624935	0.961796045303345 \\
0.375592850161798	0.961735367774963 \\
0.375621310337888	0.961728096008301 \\
0.37623340048021	0.961571455001831 \\
0.376649481158299	0.961464464664459 \\
0.37682934901581	0.96141791343689 \\
0.377022374188538	0.961368024349213 \\
0.378777395062222	0.960908889770508 \\
0.379219181502585	0.960792005062103 \\
0.380639994297166	0.960412204265594 \\
0.381635433931239	0.960142314434052 \\
0.382416367624347	0.959928631782532 \\
0.383276747323035	0.959691107273102 \\
0.383441774868681	0.959645390510559 \\
0.384589480637987	0.959324300289154 \\
0.38762147240542	0.958456873893738 \\
0.387652450706726	0.958447873592377 \\
0.38818745089578	0.95829164981842 \\
0.388531576179462	0.958190679550171 \\
0.391108514236782	0.957422435283661 \\
0.391741157295481	0.957230687141418 \\
0.396723908263121	0.955670952796936 \\
0.39803082540264	0.955247402191162 \\
0.398171926406937	0.955201148986816 \\
0.398666017137866	0.955039083957672 \\
0.398819673578751	0.954988658428192 \\
0.4012576168302	0.954174697399139 \\
0.402752737045505	0.953664541244507 \\
0.403290070599278	0.953479111194611 \\
0.404563584871068	0.953034937381744 \\
0.406532702382757	0.952336013317108 \\
0.406578018270242	0.95231956243515 \\
0.413273296435021	0.949822723865509 \\
0.413592377461711	0.949698925018311 \\
0.41401191876112	0.949535548686981 \\
0.414776267969876	0.949235856533051 \\
0.414785799829368	0.94923210144043 \\
0.415238430177184	0.949053406715393 \\
0.416465974154779	0.948564112186432 \\
0.416689013292801	0.948474586009979 \\
0.418297500020148	0.947821736335754 \\
0.422720878338663	0.945963680744171 \\
0.424343600824592	0.945258438587189 \\
0.425257001696199	0.944855868816376 \\
0.42710227803989	0.944029569625854 \\
0.427962517509508	0.943638324737549 \\
0.429309169026115	0.943018615245819 \\
0.429443446115563	0.942956387996674 \\
0.429705476003957	0.942834436893463 \\
0.431003390699289	0.942225277423859 \\
0.432539042653788	0.941492736339569 \\
0.433068569337496	0.941237270832062 \\
0.433545956268458	0.941005527973175 \\
0.433750136573834	0.94090610742569 \\
0.433778478107147	0.940892279148102 \\
0.434834104829426	0.940374195575714 \\
0.435898870411909	0.939845383167267 \\
0.43750740037395	0.939034342765808 \\
0.437737100851006	0.938917279243469 \\
0.439197385289126	0.93816602230072 \\
0.440921894924801	0.937262892723083 \\
0.442219378180834	0.936571538448334 \\
0.446052885401706	0.934469044208527 \\
0.447759455363191	0.933503150939941 \\
0.448771556188603	0.932921826839447 \\
0.448918332177756	0.932836830615997 \\
0.449861491210787	0.932287991046906 \\
0.449976789412666	0.93222051858902 \\
0.450129821782654	0.932130753993988 \\
0.453058505166632	0.930383682250977 \\
0.455515296612047	0.928872764110565 \\
0.461612686527044	0.924937844276428 \\
0.462183229417738	0.924555659294128 \\
0.46315332653893	0.923900187015533 \\
0.463567773958142	0.923618018627167 \\
0.464569005654222	0.922930955886841 \\
0.46546966357775	0.922306180000305 \\
0.465627783083443	0.922195732593536 \\
0.466166033904132	0.921818792819977 \\
0.466888847202291	0.921308994293213 \\
0.468332203161028	0.920278251171112 \\
0.469094139229819	0.919727563858032 \\
0.469464236056835	0.919458329677582 \\
0.469806075154756	0.919208645820618 \\
0.470776441432414	0.918494880199432 \\
0.471475988022769	0.917975306510925 \\
0.473199954060133	0.916677832603455 \\
0.474832247453521	0.915426313877106 \\
0.475230428162318	0.91511744260788 \\
0.476638306671274	0.914014756679535 \\
0.478988899075304	0.912134945392609 \\
0.479018256118887	0.912111043930054 \\
0.479279919995254	0.911898910999298 \\
0.479532435695615	0.911693453788757 \\
0.479763011578347	0.911505103111267 \\
0.480041882522935	0.911276876926422 \\
0.480923824850651	0.910550653934479 \\
0.481629462985493	0.909964323043823 \\
0.481771508198592	0.90984570980072 \\
0.482503736852496	0.909231722354889 \\
0.483064995688331	0.908757507801056 \\
0.48384062770339	0.908097505569458 \\
0.484085120492796	0.907888293266296 \\
0.484352733912412	0.907658696174622 \\
0.484550939846654	0.907488167285919 \\
0.485022749278845	0.907080709934235 \\
0.485322068953861	0.906821250915527 \\
0.48610300375999	0.906139969825745 \\
0.486253849387163	0.906007766723633 \\
0.489154935058063	0.903421461582184 \\
0.489792628987292	0.902841866016388 \\
0.490458183784177	0.902232706546783 \\
0.492282643359321	0.900540173053741 \\
0.493928358105999	0.898984253406525 \\
0.495701335790147	0.897277057170868 \\
0.496106252426634	0.896882474422455 \\
0.496164487189195	0.89682549238205 \\
0.498739819509518	0.894274115562439 \\
0.500164108812496	0.892832458019257 \\
0.500739446514191	0.89224374294281 \\
0.500830786926564	0.892149984836578 \\
0.502541719930356	0.890376448631287 \\
0.503498564874392	0.889370381832123 \\
0.504534081188611	0.888270020484924 \\
0.504637823699037	0.888159155845642 \\
0.506860898199194	0.885753095149994 \\
0.507491092240963	0.885060489177704 \\
0.508120202534381	0.88436484336853 \\
0.510856303638406	0.881284594535828 \\
0.512327455201249	0.879591703414917 \\
0.51318641739938	0.878591060638428 \\
0.513457954653542	0.878272831439972 \\
0.515863112336536	0.875415325164795 \\
0.516616766484905	0.87450510263443 \\
0.517831355926389	0.873023450374603 \\
0.518931921097757	0.871664762496948 \\
0.521152065672689	0.868876934051514 \\
0.524519468497769	0.864526629447937 \\
0.524576690058109	0.86445140838623 \\
0.525883589956303	0.862721800804138 \\
0.526973717032038	0.861261427402496 \\
0.528680408273136	0.858943045139313 \\
0.530600365407266	0.856287717819214 \\
0.530809662396022	0.855995059013367 \\
0.53166475254574	0.854793667793274 \\
0.532158485575632	0.85409539937973 \\
0.535204926183725	0.849710404872894 \\
0.535356594523575	0.84948867559433 \\
0.537004451507049	0.847058653831482 \\
0.537089668720381	0.846931874752045 \\
0.537190935482585	0.846781194210052 \\
0.538117362047647	0.845395267009735 \\
0.538453506118723	0.844889163970947 \\
0.539359907716354	0.843516826629639 \\
0.539560791003075	0.84321129322052 \\
0.541180326502456	0.840724527835846 \\
0.547271589683928	0.83102285861969 \\
0.548822201570025	0.828463673591614 \\
0.549624872437239	0.8271244764328 \\
0.549819206435189	0.826798737049103 \\
0.550184928091146	0.826184332370758 \\
0.552139674398207	0.822864472866058 \\
0.552982438656419	0.821414768695831 \\
0.5537594353206	0.820068418979645 \\
0.553976139314498	0.819691359996796 \\
0.554051563436848	0.819559931755066 \\
0.55426017856816	0.819195747375488 \\
0.555396555661942	0.817200899124146 \\
0.555501425327998	0.817015826702118 \\
0.556360039382527	0.815493404865265 \\
0.557883446271028	0.812763690948486 \\
0.558670348456701	0.811339199542999 \\
0.559194391744592	0.81038510799408 \\
0.559473603788983	0.809875071048737 \\
0.560198442614008	0.808544635772705 \\
0.560853933269278	0.80733448266983 \\
0.561152915237902	0.806780099868774 \\
0.561401591015197	0.806317985057831 \\
0.562479951006079	0.804302573204041 \\
0.563005908106818	0.803312599658966 \\
0.563050000205171	0.803229629993439 \\
0.563426857102662	0.802517175674438 \\
0.563560604448239	0.802263855934143 \\
0.564062038149409	0.801311492919922 \\
0.565336310962797	0.798872768878937 \\
0.565855772748177	0.797870874404907 \\
0.566858287901599	0.795925199985504 \\
0.566881879040954	0.795879185199738 \\
0.568231608554438	0.793233156204224 \\
0.569223717564581	0.791268885135651 \\
0.569387586449175	0.790942788124084 \\
0.569573989003313	0.790571570396423 \\
0.569720721150053	0.790278792381287 \\
0.572004618116245	0.785676896572113 \\
0.572005154945678	0.785675764083862 \\
0.572151528516064	0.78537791967392 \\
0.57219889974288	0.785281360149384 \\
0.572780347928331	0.784094095230103 \\
0.573600339080632	0.782410204410553 \\
0.576593284261927	0.776169419288635 \\
0.576676798911885	0.775993168354034 \\
0.576819659872325	0.775691270828247 \\
0.577653861143233	0.773922085762024 \\
0.577917937736817	0.773359715938568 \\
0.57793430586259	0.773324608802795 \\
0.581759386894494	0.765045762062073 \\
0.582475032298807	0.763469696044922 \\
0.582896249817215	0.762538135051727 \\
0.589571735768374	0.747377812862396 \\
0.592388904473667	0.740757286548615 \\
0.593611787705259	0.737842679023743 \\
0.594151694891475	0.736547887325287 \\
0.595008948927732	0.734482228755951 \\
0.596105779635777	0.731821835041046 \\
0.597490139381406	0.728435873985291 \\
0.598951438337811	0.724827587604523 \\
0.599201054717089	0.724207758903503 \\
0.600228747033593	0.721645355224609 \\
0.600693928524281	0.720479786396027 \\
0.601589358474053	0.718226373195648 \\
0.602529748251813	0.715845942497253 \\
0.602916831603936	0.714862167835236 \\
0.603866403426713	0.712438344955444 \\
0.606129689323002	0.706603169441223 \\
0.607808363354841	0.702223360538483 \\
0.60811641039762	0.701415002346039 \\
0.608424173673929	0.700605571269989 \\
0.610203299896366	0.695898711681366 \\
0.61052401101773	0.695045053958893 \\
0.611007548691541	0.693754911422729 \\
0.61345268097321	0.687177002429962 \\
0.614300601551049	0.684874773025513 \\
0.616415105925723	0.679087281227112 \\
0.617595625529809	0.675827443599701 \\
0.619695383178102	0.669979572296143 \\
0.620626131680421	0.667367160320282 \\
0.620884093030541	0.666640996932983 \\
0.622959094549987	0.660766005516052 \\
0.624820249131202	0.655446112155914 \\
0.627396892203872	0.648004114627838 \\
0.629679734717881	0.641338348388672 \\
0.63108091324942	0.637214243412018 \\
0.631412061821856	0.636235952377319 \\
0.632209138475077	0.633875846862793 \\
0.632608841524959	0.632689476013184 \\
0.635174872896091	0.625027120113373 \\
0.63680487509045	0.620120048522949 \\
0.638872902814914	0.613851308822632 \\
0.639214862746441	0.61281031370163 \\
0.640549343973947	0.608735322952271 \\
0.6407383188268	0.608156979084015 \\
0.642765742598199	0.601926982402802 \\
0.642772962045468	0.6019047498703 \\
0.644393148200581	0.596896350383759 \\
0.646791193304363	0.589437663555145 \\
0.647126395490269	0.588391005992889 \\
0.647747012654458	0.586449921131134 \\
0.648487744871255	0.584129333496094 \\
0.65164263639371	0.574192881584167 \\
0.651778799711512	0.573762118816376 \\
0.652422089544704	0.57172566652298 \\
0.652778066297422	0.570597231388092 \\
0.653037870316759	0.569773197174072 \\
0.653946378314801	0.566887378692627 \\
0.656181358570071	0.559763491153717 \\
0.657515298781176	0.555495321750641 \\
0.657916355325804	0.554209887981415 \\
0.65821939898974	0.553238093852997 \\
0.660347901626978	0.54639595746994 \\
0.663462409580861	0.536340057849884 \\
0.665002009135555	0.531352043151855 \\
0.665125585501828	0.530951082706451 \\
0.665840946327609	0.528629779815674 \\
0.669598702136104	0.516403913497925 \\
0.669671107753933	0.516167879104614 \\
0.670317902411108	0.514058887958527 \\
0.67055566811489	0.513283312320709 \\
0.670890712943537	0.512190341949463 \\
0.671190867447461	0.51121062040329 \\
0.672572121540986	0.506700158119202 \\
0.674425718831603	0.500641465187073 \\
0.67545414417829	0.497277617454529 \\
0.67596832728122	0.495595216751099 \\
0.676862814397186	0.492667943239212 \\
0.677008756097011	0.492190420627594 \\
0.677760880297266	0.489728361368179 \\
0.680496299332769	0.480773240327835 \\
0.6805627271546	0.480555653572083 \\
0.682518497342478	0.474153965711594 \\
0.682667061493577	0.473667800426483 \\
0.683323391450582	0.471520334482193 \\
0.683455311069743	0.471088707447052 \\
0.686099541255968	0.462443053722382 \\
0.686972247003404	0.459592133760452 \\
0.687287921742946	0.458561450242996 \\
0.688979470872378	0.453042447566986 \\
0.689844777380402	0.450222134590149 \\
0.69066753173656	0.447542697191238 \\
0.690924484351316	0.446706414222717 \\
0.691684985973077	0.444232285022736 \\
0.694075688265434	0.436468929052353 \\
0.694253786667502	0.435891538858414 \\
0.696056771003981	0.430054098367691 \\
0.696659969942023	0.42810469865799 \\
0.69712902782289	0.426590025424957 \\
0.697436633040946	0.425597339868546 \\
0.697470314030711	0.425488620996475 \\
0.698295232645012	0.422829121351242 \\
0.698478216988194	0.422239512205124 \\
0.699563853841568	0.418746262788773 \\
0.700079752305739	0.417088478803635 \\
0.700291212685059	0.416409581899643 \\
0.701162742599685	0.413613885641098 \\
0.702327231437584	0.409885942935944 \\
0.702639966218633	0.408886581659317 \\
0.703077231783761	0.407489717006683 \\
0.703198302674555	0.407103270292282 \\
0.703418311750763	0.40640127658844 \\
0.704913184821873	0.401640087366104 \\
0.705985343895602	0.398235261440277 \\
0.70676366586065	0.395769059658051 \\
0.70953480835097	0.387027144432068 \\
0.709597763991885	0.386829346418381 \\
0.710916839332999	0.382691383361816 \\
0.710971707887323	0.382519632577896 \\
0.711527088935361	0.380782157182693 \\
0.711625741549748	0.380473852157593 \\
0.712030078150365	0.379211097955704 \\
0.713838731517111	0.373580664396286 \\
0.71495171630821	0.370131552219391 \\
0.715877253359645	0.367272347211838 \\
0.717907954554698	0.361029505729675 \\
0.718267373832691	0.359928876161575 \\
0.719482266109768	0.356219351291656 \\
0.720072506203388	0.354422748088837 \\
0.72063303964824	0.35272017121315 \\
0.721414113641739	0.350353419780731 \\
0.721834558021562	0.349082291126251 \\
0.721858215436245	0.349010795354843 \\
0.722606600061919	0.346753358840942 \\
0.723634632256085	0.343662649393082 \\
0.724295951475884	0.341681331396103 \\
0.724780388322819	0.340233087539673 \\
0.725967182906895	0.336696922779083 \\
0.728562998007155	0.329022765159607 \\
0.729378939538106	0.326628237962723 \\
0.729672766087416	0.325767874717712 \\
0.730156271089799	0.324354827404022 \\
0.730418624607748	0.323589235544205 \\
0.731079272213317	0.321665316820145 \\
0.73140958111192	0.320705652236938 \\
0.734582373950397	0.311561495065689 \\
0.735036686489948	0.310263335704803 \\
0.736495162798249	0.306115239858627 \\
0.737637169481189	0.302887976169586 \\
0.738132551177952	0.301493614912033 \\
0.738772404713377	0.299698114395142 \\
0.73919080617973	0.29852694272995 \\
0.739634766317876	0.297287374734879 \\
0.739671405613807	0.297185093164444 \\
0.740318229014976	0.29538431763649 \\
0.741017797721675	0.293443560600281 \\
0.741923938399663	0.290940582752228 \\
0.742870439596604	0.288338959217072 \\
0.743500813750507	0.286613702774048 \\
0.745235425982657	0.281897068023682 \\
0.745794924415534	0.280385494232178 \\
0.746651135832285	0.278081715106964 \\
0.748242417565033	0.273829787969589 \\
0.749080152481028	0.27160707116127 \\
0.750738597109713	0.267239719629288 \\
0.754059260293618	0.258626043796539 \\
0.754427277039572	0.257682293653488 \\
0.755971311279052	0.253746628761292 \\
0.7564660147832	0.252493858337402 \\
0.756964107658398	0.25123655796051 \\
0.757301980310619	0.250385910272598 \\
0.757561417579673	0.249734044075012 \\
0.759460973532791	0.244995072484016 \\
0.759943823499425	0.243799924850464 \\
0.76007481161266	0.243476375937462 \\
0.76520949758525	0.231018230319023 \\
0.769369251223668	0.22125019133091 \\
0.77094746139779	0.217620730400085 \\
0.771797041083953	0.21568451821804 \\
0.772164013593626	0.214852020144463 \\
0.772721858147557	0.213590756058693 \\
0.774553209660584	0.209487617015839 \\
0.774686774792601	0.209190681576729 \\
0.774763255366	0.209020838141441 \\
0.776676196809448	0.204802736639977 \\
0.781826328102722	0.193756580352783 \\
0.783936140597273	0.189361944794655 \\
0.784010287514813	0.189208775758743 \\
0.784889735175701	0.187400385737419 \\
0.785018616770813	0.18713641166687 \\
0.78602927274531	0.185076639056206 \\
0.786818186718577	0.183480724692345 \\
0.787974469502378	0.181160762906075 \\
0.791139177359944	0.174926340579987 \\
0.791586708117001	0.174058318138123 \\
0.791796196079504	0.173653185367584 \\
0.79182158322754	0.173604100942612 \\
0.792072748806228	0.173119336366653 \\
0.792682852079258	0.171946510672569 \\
0.792705401732446	0.171903118491173 \\
0.794569839509874	0.168357774615288 \\
0.795023193386171	0.167504355311394 \\
0.795367478837485	0.166858613491058 \\
0.795527938976809	0.166558355093002 \\
0.795634872431087	0.166358456015587 \\
0.79745955888423	0.162976890802383 \\
0.798224122257701	0.161576196551323 \\
0.799881016007139	0.158573821187019 \\
0.801765379467602	0.155213922262192 \\
0.80362962295172	0.151946619153023 \\
0.803814553638313	0.151625588536263 \\
0.806469107208753	0.147077634930611 \\
0.806697504962046	0.146691605448723 \\
0.807337366299355	0.145614445209503 \\
0.808888043138527	0.143031120300293 \\
0.809190014741277	0.142532423138618 \\
0.809307640373056	0.142338573932648 \\
0.810501764417227	0.140382960438728 \\
0.811889729396757	0.138137951493263 \\
0.811942350832027	0.138053387403488 \\
0.814196326909986	0.134472921490669 \\
0.815151250895423	0.132979646325111 \\
0.815544311745005	0.132369130849838 \\
0.816375455024602	0.131085768342018 \\
0.818069790025891	0.128501832485199 \\
0.820698196177955	0.124578781425953 \\
0.821263012964588	0.123749099671841 \\
0.823441468430582	0.12059336155653 \\
0.824103550036923	0.119648009538651 \\
0.825253713607837	0.118020720779896 \\
0.827170035611177	0.115351937711239 \\
0.827687372625058	0.114640332758427 \\
0.828681002627947	0.113284409046173 \\
0.831688136352904	0.10926491767168 \\
0.832015103492291	0.10883554071188 \\
0.832618539764885	0.108046777546406 \\
0.833979270360146	0.106286577880383 \\
0.835071309810646	0.104892037808895 \\
0.835699577265585	0.104097053408623 \\
0.836498037763016	0.103094324469566 \\
0.836836957998254	0.102671273052692 \\
0.838046127292981	0.101174354553223 \\
0.838525323246066	0.100586548447609 \\
0.839239765753767	0.0997155755758286 \\
0.839402685897079	0.0995179638266563 \\
0.839999299682593	0.0987970679998398 \\
0.840281951751549	0.0984571576118469 \\
0.840483203711437	0.0982158184051514 \\
0.840691788412589	0.0979661270976067 \\
0.842685722208791	0.0956081002950668 \\
0.842993851211431	0.0952482372522354 \\
0.844759374999568	0.0932093486189842 \\
0.844835827371022	0.0931220427155495 \\
0.844992041004939	0.0929436609148979 \\
0.84621071404422	0.0915626734495163 \\
0.84647818555566	0.0912620201706886 \\
0.847551016932997	0.0900650173425674 \\
0.848091115090324	0.0894677713513374 \\
0.84863665263528	0.0888680964708328 \\
0.85175414414149	0.0855099707841873 \\
0.852055625215632	0.0851913616061211 \\
0.853183518340707	0.0840088054537773 \\
0.855762763379044	0.0813601985573769 \\
0.855845256518328	0.0812767520546913 \\
0.856101518575613	0.0810180678963661 \\
0.856353280691239	0.080764576792717 \\
0.857431392977156	0.0796872824430466 \\
0.858507268631162	0.0786252692341805 \\
0.858727032795181	0.0784099102020264 \\
0.859611972462117	0.0775482207536697 \\
0.860598063349272	0.0765981748700142 \\
0.861479766548793	0.0757577568292618 \\
0.863021652496581	0.0743081793189049 \\
0.863191163941701	0.0741504058241844 \\
0.863243196088782	0.0741020143032074 \\
0.863264684077014	0.0740820914506912 \\
0.86433033865504	0.0730978772044182 \\
0.865386426798682	0.072134368121624 \\
0.865574805896198	0.0719637870788574 \\
0.867009964664233	0.0706759169697762 \\
0.868402477420643	0.0694465637207031 \\
0.868462536448662	0.0693939700722694 \\
0.8687736544668	0.0691222622990608 \\
0.870769922748437	0.0674017891287804 \\
0.87137568194098	0.0668875798583031 \\
0.871475162483056	0.066803477704525 \\
0.872823623463892	0.0656731948256493 \\
0.873231322955811	0.0653348714113235 \\
0.87330913025223	0.0652705356478691 \\
0.874554061243274	0.0642485767602921 \\
0.874752616263773	0.0640869736671448 \\
0.875835647990412	0.0632120370864868 \\
0.878375805455995	0.0612033754587173 \\
0.879016462457315	0.060706228017807 \\
0.879915986267341	0.0600145347416401 \\
0.880187041340505	0.0598075799643993 \\
0.880228906915535	0.0597756505012512 \\
0.880706157108953	0.0594130419194698 \\
0.881313907623959	0.0589542053639889 \\
0.88357711634013	0.0572745576500893 \\
0.88575628070969	0.0556997396051884 \\
0.885860150516388	0.0556256771087646 \\
0.885973756819001	0.055544774979353 \\
0.886578379643548	0.0551162138581276 \\
0.88662976023068	0.0550799295306206 \\
0.887201064437371	0.054678063839674 \\
0.88721221948908	0.0546702556312084 \\
0.889489342121342	0.0530958771705627 \\
0.889610563902722	0.0530132614076138 \\
0.891226627532086	0.0519235245883465 \\
0.891601333881282	0.0516739264130592 \\
0.894395711960869	0.0498476177453995 \\
0.894941283658575	0.049498226493597 \\
0.895002834569702	0.0494589321315289 \\
0.895316145165804	0.0492595136165619 \\
0.89572498422586	0.0490003749728203 \\
0.900281097465164	0.0461988039314747 \\
0.901776757070804	0.0453127138316631 \\
0.902902077475763	0.0446566566824913 \\
0.903042041253622	0.0445757023990154 \\
0.903417484940991	0.044359240680933 \\
0.905006674071476	0.043453898280859 \\
0.905428107857581	0.0432167798280716 \\
0.90623158203924	0.0427681431174278 \\
0.909863735264334	0.0407948344945908 \\
0.914192347433306	0.0385566279292107 \\
0.914860066850241	0.0382219776511192 \\
0.916826528028795	0.0372525826096535 \\
0.918089828672593	0.036642350256443 \\
0.918205626409754	0.0365868918597698 \\
0.918453743387651	0.0364683046936989 \\
0.918779069418784	0.0363134145736694 \\
0.919195929572793	0.0361158698797226 \\
0.920730394218805	0.0353975892066956 \\
0.920881992386684	0.0353273823857307 \\
0.921716310403637	0.034943301230669 \\
0.921720421578646	0.0349414348602295 \\
0.923498857953819	0.0341362096369267 \\
0.923553028604257	0.0341119728982449 \\
0.924006080008399	0.0339098572731018 \\
0.92420639372191	0.0338208824396133 \\
0.924470669810756	0.033703800290823 \\
0.924724469351036	0.0335917286574841 \\
0.926231702262142	0.0329335816204548 \\
0.926990949332903	0.0326067730784416 \\
0.933531060787661	0.0299183428287506 \\
0.934434631069947	0.029564194381237 \\
0.934570551019964	0.0295112747699022 \\
0.935534942058006	0.0291384011507034 \\
0.936268298594132	0.0288579314947128 \\
0.936269242169234	0.0288575738668442 \\
0.938329910665083	0.0280833467841148 \\
0.94030185596857	0.0273613706231117 \\
0.940807389024333	0.0271792020648718 \\
0.941016354925932	0.0271042343229055 \\
0.942120305962364	0.0267115645110607 \\
0.944292950605489	0.02595479413867 \\
0.944561491811014	0.025862742215395 \\
0.944673077956372	0.0258245673030615 \\
0.945708241295843	0.0254731141030788 \\
0.947179815483775	0.0249814670532942 \\
0.947693224914561	0.024812113493681 \\
0.950107910712063	0.0240305103361607 \\
0.953012070755755	0.0231222026050091 \\
0.953264426846652	0.0230448730289936 \\
0.953432175112961	0.0229936074465513 \\
0.954059232722306	0.0228029489517212 \\
0.954894864067499	0.0225512962788343 \\
0.955427443299792	0.0223923027515411 \\
0.958107244064232	0.021608853712678 \\
0.95825981412274	0.0215650480240583 \\
0.961834544341432	0.0205634534358978 \\
0.962013308437074	0.0205145832151175 \\
0.962557230525089	0.0203665737062693 \\
0.964107437971182	0.0199504364281893 \\
0.96546229210349	0.0195935647934675 \\
0.966544858040119	0.0193128902465105 \\
0.966587760026518	0.0193018466234207 \\
0.966792601928706	0.0192492213100195 \\
0.968404850237238	0.018839854747057 \\
0.968813010049556	0.0187375675886869 \\
0.969045613941432	0.018679516389966 \\
0.969871084100612	0.0184749495238066 \\
0.970530837594512	0.0183130577206612 \\
0.971272888982341	0.0181325878947973 \\
0.971527190360033	0.0180711448192596 \\
0.977559419065177	0.0166721772402525 \\
0.978837143824273	0.0163897853344679 \\
0.979500241452307	0.0162451043725014 \\
0.979698366054342	0.0162021163851023 \\
0.981925106253984	0.0157265737652779 \\
0.98193841332347	0.0157237946987152 \\
0.985466980517735	0.0149983437731862 \\
0.985624420362111	0.0149667505174875 \\
0.985660230866112	0.0149595765396953 \\
0.985903288996567	0.0149109559133649 \\
0.986891736095765	0.0147148352116346 \\
0.988177000983844	0.014463622123003 \\
0.988188066807458	0.0144614763557911 \\
0.988339255736614	0.0144322160631418 \\
0.988532835659511	0.0143948271870613 \\
0.988627149659547	0.0143766487017274 \\
0.991317832181511	0.0138673530891538 \\
0.991792763797732	0.0137793282046914 \\
0.99271508471038	0.0136099439114332 \\
0.992808132462744	0.0135929780080914 \\
0.99282017193431	0.0135907810181379 \\
0.994749345227338	0.0132435532286763 \\
0.99866338972896	0.0125657673925161 \\
};
\addplot [thick, color0]
table [row sep=\\]{%
0.0011059794751318	0.00577228888869286 \\
0.00144600828615515	0.0057767303660512 \\
0.00151620394328711	0.00577764911577106 \\
0.00161381928831428	0.005778927821666 \\
0.00193644961641948	0.00578314997255802 \\
0.0024363370221504	0.00578969763591886 \\
0.00384167794354173	0.00580814201384783 \\
0.00421063900202368	0.00581299560144544 \\
0.00632372037002726	0.00584086263552308 \\
0.00947686783375368	0.00588269112631679 \\
0.0099356428556221	0.00588879827409983 \\
0.0101317191976503	0.00589141016826034 \\
0.0102519961026396	0.00589301669970155 \\
0.0116572936797861	0.00591178191825747 \\
0.0122763055741271	0.0059200688265264 \\
0.0128807282915644	0.0059281699359417 \\
0.0131006826676461	0.0059311231598258 \\
0.0132105085232722	0.00593259884044528 \\
0.0150477352877949	0.00595731055364013 \\
0.0151511241047927	0.00595870427787304 \\
0.0168553495563415	0.00598172284662724 \\
0.0191895991753259	0.00601339060813189 \\
0.0192724193499449	0.00601451983675361 \\
0.0202131188025382	0.00602733623236418 \\
0.0251886934986562	0.00609554909169674 \\
0.0255409812646521	0.00610040547326207 \\
0.0256196546755231	0.00610149558633566 \\
0.0263800727822662	0.00611199252307415 \\
0.026825697968402	0.00611815927550197 \\
0.0272388337061176	0.00612387619912624 \\
0.0276781997826389	0.00612996472045779 \\
0.0277969734855988	0.00613161409273744 \\
0.027888291941232	0.00613287696614861 \\
0.032801074197002	0.00620138738304377 \\
0.0337126623182875	0.00621418794617057 \\
0.0349109578825388	0.00623104581609368 \\
0.0367757145561887	0.0062573654577136 \\
0.0368887253476963	0.00625896407291293 \\
0.0376301111814882	0.00626946799457073 \\
0.0383020767002249	0.00627899635583162 \\
0.0385765124363792	0.00628289440646768 \\
0.0386281055032889	0.00628362316638231 \\
0.0397134557815922	0.00629906309768558 \\
0.0400530872540832	0.00630389992147684 \\
0.0408224228009035	0.00631487229838967 \\
0.042416614098386	0.00633766874670982 \\
0.042542997860721	0.00633947970345616 \\
0.0425634328976269	0.00633977307006717 \\
0.0426938593122377	0.00634163990616798 \\
0.0436353215690566	0.00635514641180634 \\
0.0476491609572692	0.00641306024044752 \\
0.0476764809468383	0.00641345605254173 \\
0.0480632979275277	0.00641906587406993 \\
0.0489335324060199	0.00643169926479459 \\
0.0494953541501341	0.00643987162038684 \\
0.0498656975684827	0.00644526071846485 \\
0.0507299142819697	0.00645785918459296 \\
0.0520149216066462	0.00647663464769721 \\
0.053491737186121	0.00649828230962157 \\
0.0545456118282759	0.00651377113536 \\
0.0549368100196436	0.006519531365484 \\
0.0549667939602374	0.00651996955275536 \\
0.0555159955251522	0.0065280687995255 \\
0.0574839546417889	0.00655715120956302 \\
0.0576927502423773	0.00656024552881718 \\
0.0580609228746025	0.00656570261344314 \\
0.0610070220733733	0.00660953437909484 \\
0.0613718310828713	0.00661498587578535 \\
0.0624819644863142	0.0066315894946456 \\
0.0632024977130776	0.00664239097386599 \\
0.0641707318537266	0.0066569303162396 \\
0.064803290117067	0.00666644796729088 \\
0.0661546388560771	0.00668681552633643 \\
0.0675945492445934	0.00670859590172768 \\
0.0701208744758773	0.0067469677887857 \\
0.0704495440664726	0.00675197783857584 \\
0.0707913481046274	0.00675718672573566 \\
0.0709503961374458	0.00675961282104254 \\
0.0723079949898875	0.00678036082535982 \\
0.0729582139062805	0.00679031992331147 \\
0.0736427592129116	0.00680082337930799 \\
0.0738431652021745	0.00680389627814293 \\
0.074094042941424	0.00680775195360184 \\
0.0741988699821797	0.00680935895070434 \\
0.0745344402647278	0.00681452266871929 \\
0.0761651718345726	0.00683964928612113 \\
0.0762565104011357	0.00684105977416039 \\
0.0764247618188199	0.00684365490451455 \\
0.0780473992355428	0.00686876149848104 \\
0.0784939963588281	0.00687569146975875 \\
0.0787288834363844	0.00687933573499322 \\
0.0796528614743472	0.00689369719475508 \\
0.0821291956626848	0.00693231960758567 \\
0.0836539892117674	0.00695620849728584 \\
0.0837976822096664	0.00695846462622285 \\
0.0840863722935196	0.00696299830451608 \\
0.0847664004367794	0.00697368942201138 \\
0.0848642352310873	0.00697522889822721 \\
0.0852150851090396	0.0069807511754334 \\
0.085891869345856	0.00699141481891274 \\
0.087068315817913	0.00700999330729246 \\
0.0885849846877879	0.00703401537612081 \\
0.0903431131516136	0.00706196157261729 \\
0.0909278592293694	0.00707127572968602 \\
0.0914296081009306	0.00707928603515029 \\
0.0922298300096035	0.00709207309409976 \\
0.0926517333242025	0.00709882332012057 \\
0.0930086769471365	0.00710454117506742 \\
0.0936368650628759	0.00711461016908288 \\
0.0939832817677504	0.00712016690522432 \\
0.0963578674192209	0.00715839024633169 \\
0.0969604240071604	0.00716812256723642 \\
0.0974174263218397	0.00717551400884986 \\
0.0997464409700664	0.00721327820792794 \\
0.100249930959155	0.00722147058695555 \\
0.100820744388044	0.00723076984286308 \\
0.101229398304142	0.0072374283336103 \\
0.102271393895604	0.00725444545969367 \\
0.102325882258744	0.00725533813238144 \\
0.103474916599901	0.0072741499170661 \\
0.104369461810087	0.00728882802650332 \\
0.104849386693897	0.00729671493172646 \\
0.105274572558957	0.00730370730161667 \\
0.105437385176067	0.00730638718232512 \\
0.105718642159513	0.0073110219091177 \\
0.106451679001465	0.00732310535386205 \\
0.106754108203929	0.00732809724286199 \\
0.106889477031479	0.0073303310200572 \\
0.108308586301563	0.00735380547121167 \\
0.108355506097126	0.0073545821942389 \\
0.110864090628752	0.00739626213908195 \\
0.112062699234041	0.00741625390946865 \\
0.112686863552251	0.00742668705061078 \\
0.113787419472266	0.0074451151303947 \\
0.114114908535732	0.00745061039924622 \\
0.114926076298882	0.00746423518285155 \\
0.115710925433917	0.00747743900865316 \\
0.116193807044869	0.0074855680577457 \\
0.116352662276253	0.00748824886977673 \\
0.116979956596317	0.00749883288517594 \\
0.117247317403657	0.00750334979966283 \\
0.118793828877862	0.00752951996400952 \\
0.119352906137166	0.00753900222480297 \\
0.120388240876593	0.00755659071728587 \\
0.122357154089364	0.00759015558287501 \\
0.125712799722249	0.00764768850058317 \\
0.126988316690772	0.00766966911032796 \\
0.130289689058528	0.00772684486582875 \\
0.130830575131366	0.00773624749854207 \\
0.13178708265069	0.00775291072204709 \\
0.132217070962463	0.00776041299104691 \\
0.132641072460891	0.00776781560853124 \\
0.133694684584262	0.00778624089434743 \\
0.135586813424106	0.00781943555921316 \\
0.136970534315695	0.0078437989577651 \\
0.137167897059254	0.00784728582948446 \\
0.13731957841644	0.00784996058791876 \\
0.138770949832641	0.00787561479955912 \\
0.141579671494468	0.00792548805475235 \\
0.142337772861435	0.0079390024766326 \\
0.142852169410084	0.00794818066060543 \\
0.14313843453823	0.00795329827815294 \\
0.143709892765619	0.0079635176807642 \\
0.143996587198751	0.00796864461153746 \\
0.144338718134467	0.00797477271407843 \\
0.145474509131844	0.00799514725804329 \\
0.146058924442183	0.00800565350800753 \\
0.147496370200373	0.00803154241293669 \\
0.147934500473159	0.00803945306688547 \\
0.148464696403164	0.00804902613162994 \\
0.149007159183894	0.00805884134024382 \\
0.151290695331351	0.00810027401894331 \\
0.153507740604538	0.00814068969339132 \\
0.153927498261582	0.00814836099743843 \\
0.153953160827713	0.00814883410930634 \\
0.154179436777791	0.00815297290682793 \\
0.155107736993068	0.00816997699439526 \\
0.15519005879533	0.00817149132490158 \\
0.155418402417591	0.00817567855119705 \\
0.156078407887877	0.00818780157715082 \\
0.158100170599977	0.00822502840310335 \\
0.158283533549941	0.00822841562330723 \\
0.158599777942073	0.00823425222188234 \\
0.159258263288539	0.00824642740190029 \\
0.159678910282026	0.00825421698391438 \\
0.162540881855109	0.00830737501382828 \\
0.164077384668961	0.00833604950457811 \\
0.164957348092599	0.00835250876843929 \\
0.165030574098427	0.008353884331882 \\
0.165280388241905	0.00835856329649687 \\
0.167273692684132	0.00839599594473839 \\
0.168043854724866	0.00841050874441862 \\
0.168088628486329	0.00841135066002607 \\
0.168662857989897	0.008422183804214 \\
0.169151147716595	0.00843140762299299 \\
0.169673214248392	0.00844127777963877 \\
0.170135609058708	0.00845002755522728 \\
0.171767974673073	0.0084809921681881 \\
0.1728518498484	0.00850160978734493 \\
0.172984599941167	0.00850414019078016 \\
0.176370551959836	0.00856887456029654 \\
0.176588273575192	0.00857305619865656 \\
0.176656553576852	0.00857436470687389 \\
0.17794804219768	0.00859919749200344 \\
0.178429717033185	0.00860847253352404 \\
0.179458814233081	0.00862833019345999 \\
0.180595104426831	0.00865029916167259 \\
0.18064479669491	0.00865125935524702 \\
0.182294707450728	0.00868326518684626 \\
0.182819109355123	0.00869345758110285 \\
0.189141614514094	0.00881726294755936 \\
0.190723527970293	0.00884849112480879 \\
0.192508826701743	0.00888386182487011 \\
0.200712508340217	0.00904811266809702 \\
0.202130410523186	0.00907678995281458 \\
0.202705868873484	0.00908844918012619 \\
0.202975660615516	0.00909391883760691 \\
0.2033001668374	0.00910050515085459 \\
0.204138047670012	0.00911752879619598 \\
0.204584283026216	0.00912661384791136 \\
0.205975712225296	0.0091549726203084 \\
0.208693969096644	0.00921061728149652 \\
0.209097012308874	0.00921889673918486 \\
0.210129299396579	0.00924012809991837 \\
0.210470524326312	0.00924715213477612 \\
0.210726758187617	0.00925243645906448 \\
0.210879765452585	0.00925558991730213 \\
0.213404555935633	0.00930778589099646 \\
0.214779680955682	0.00933632627129555 \\
0.215976334819965	0.00936122704297304 \\
0.21605980843853	0.00936296954751015 \\
0.21626681121775	0.00936728436499834 \\
0.217333435243771	0.00938954763114452 \\
0.217860184045886	0.0094005586579442 \\
0.218810300034316	0.00942045170813799 \\
0.219036830922259	0.00942520145326853 \\
0.219186045259484	0.00942833069711924 \\
0.219509161852637	0.00943510606884956 \\
0.219517283100912	0.00943528115749359 \\
0.220007941128277	0.00944558344781399 \\
0.221124031401732	0.0094690565019846 \\
0.221655686766872	0.00948025472462177 \\
0.222787328400317	0.00950413756072521 \\
0.222812240344664	0.00950466003268957 \\
0.224413114839207	0.00953853968530893 \\
0.225712117798195	0.00956611055880785 \\
0.226064824403414	0.00957360304892063 \\
0.22778297811216	0.00961020775139332 \\
0.227786137279108	0.00961027666926384 \\
0.228255816092263	0.00962030328810215 \\
0.22886379454373	0.00963329710066319 \\
0.229286896531361	0.0096423514187336 \\
0.231570483736477	0.00969134271144867 \\
0.232839821990075	0.00971866399049759 \\
0.234699792231533	0.00975883472710848 \\
0.236598748956898	0.00979999452829361 \\
0.236914864918852	0.00980685837566853 \\
0.238193170258512	0.00983467604964972 \\
0.238267801131912	0.00983630958944559 \\
0.241280255564163	0.00990213546901941 \\
0.241532477911657	0.00990766566246748 \\
0.242724887440966	0.00993384886533022 \\
0.243487528482621	0.00995062571018934 \\
0.243851755033328	0.00995864812284708 \\
0.243931127736832	0.00996039714664221 \\
0.245492487219521	0.00999484956264496 \\
0.245961870318215	0.0100052300840616 \\
0.247220233087027	0.0100330924615264 \\
0.248029547774932	0.0100510530173779 \\
0.248195183805212	0.0100547382608056 \\
0.252597476547261	0.0101529518142343 \\
0.254049623190913	0.010185532271862 \\
0.254469169693275	0.0101949628442526 \\
0.254777550536903	0.0102018974721432 \\
0.254877558399671	0.0102041503414512 \\
0.257253649731868	0.0102577386423945 \\
0.257671137054733	0.0102671850472689 \\
0.26187840669548	0.010362739674747 \\
0.264182169008257	0.0104153864085674 \\
0.265462287195374	0.0104447333142161 \\
0.267638564660905	0.0104947928339243 \\
0.268216851612881	0.0105081340298057 \\
0.269980014659659	0.0105488803237677 \\
0.270837940398052	0.0105687575414777 \\
0.271133018969332	0.0105755990371108 \\
0.271522794293612	0.010584644973278 \\
0.272730212426688	0.0106127075850964 \\
0.27325151773824	0.0106248399242759 \\
0.273458501731612	0.0106296604499221 \\
0.275787496734588	0.0106840264052153 \\
0.275790346826632	0.0106840953230858 \\
0.275963502468307	0.0106881437823176 \\
0.276441724078765	0.010699350386858 \\
0.277788622824976	0.0107309278100729 \\
0.278249554181787	0.0107417553663254 \\
0.278902433397602	0.0107571100816131 \\
0.280228184178203	0.0107883308082819 \\
0.282170737525181	0.0108342235907912 \\
0.284775883532131	0.0108960065990686 \\
0.285319633849948	0.0109089408069849 \\
0.285945143849347	0.0109238233417273 \\
0.287293498153222	0.0109559874981642 \\
0.288098109983407	0.0109752118587494 \\
0.288474372887691	0.0109842056408525 \\
0.288733693686359	0.0109904110431671 \\
0.291563977198166	0.0110583202913404 \\
0.293976605993451	0.0111164636909962 \\
0.294014528290392	0.0111173838376999 \\
0.294718175832172	0.0111343860626221 \\
0.294831463459233	0.0111371222883463 \\
0.294887681144015	0.0111384829506278 \\
0.299064488228799	0.011239861138165 \\
0.299223468133874	0.0112437307834625 \\
0.299853491506122	0.0112590901553631 \\
0.301624334890634	0.0113023379817605 \\
0.302100714632492	0.0113139916211367 \\
0.302855829384862	0.0113324895501137 \\
0.303468685820656	0.0113475173711777 \\
0.304335545384868	0.0113687943667173 \\
0.305081848845274	0.0113871367648244 \\
0.305263786725066	0.0113916136324406 \\
0.305511643802704	0.0113977119326591 \\
0.30635774546767	0.0114185493439436 \\
0.307937429467832	0.011457527987659 \\
0.308578187230734	0.0114733688533306 \\
0.313991998160918	0.0116078220307827 \\
0.315973848719028	0.0116573208943009 \\
0.317413649622303	0.0116933761164546 \\
0.319657848506035	0.0117497267201543 \\
0.319781920771261	0.0117528410628438 \\
0.320496004308145	0.0117708155885339 \\
0.320601310305949	0.0117734707891941 \\
0.322246867517597	0.0118149565532804 \\
0.323306124649751	0.0118417209014297 \\
0.323664358184811	0.0118507826700807 \\
0.323805398711091	0.0118543477728963 \\
0.324816846999054	0.0118799563497305 \\
0.325263683488062	0.0118912821635604 \\
0.327533460856979	0.0119489151984453 \\
0.328385869951453	0.011970610357821 \\
0.329844054701458	0.0120077766478062 \\
0.33074904970515	0.0120308762416244 \\
0.332974695502457	0.0120878051966429 \\
0.333600133199531	0.01210383977741 \\
0.336001356191777	0.0121655017137527 \\
0.338057278352338	0.0122184464707971 \\
0.33940563378102	0.0122532481327653 \\
0.339522610836627	0.0122562637552619 \\
0.341295751839904	0.0123021146282554 \\
0.343025583329651	0.0123469457030296 \\
0.344156227439929	0.0123762935400009 \\
0.344485478041551	0.0123848523944616 \\
0.346631556569637	0.0124406833201647 \\
0.348247004908186	0.0124827977269888 \\
0.348257869791207	0.0124830789864063 \\
0.350134953366802	0.0125321112573147 \\
0.352271571140202	0.0125880427658558 \\
0.352403536611892	0.0125914961099625 \\
0.356607161458692	0.0127019071951509 \\
0.357473355861859	0.0127247171476483 \\
0.359032837138255	0.0127658229321241 \\
0.360085447455855	0.0127936052158475 \\
0.360587038984981	0.0128068430349231 \\
0.3618814720936	0.0128410561010242 \\
0.362541325151431	0.0128585137426853 \\
0.363971114014582	0.0128963626921177 \\
0.36446377318254	0.0129094207659364 \\
0.364734654416428	0.0129166012629867 \\
0.364782079325344	0.0129178557544947 \\
0.36626622804169	0.0129572236910462 \\
0.36745676602671	0.0129888327792287 \\
0.367555756327517	0.0129914674907923 \\
0.367766921022286	0.0129970703274012 \\
0.368015883464081	0.0130036855116487 \\
0.368183587302196	0.0130081437528133 \\
0.368646745255496	0.0130204521119595 \\
0.368824340831231	0.0130251739174128 \\
0.36908666014638	0.0130321532487869 \\
0.370227788452905	0.0130625171586871 \\
0.371079364739364	0.0130851808935404 \\
0.371457393668236	0.0130952522158623 \\
0.372521553247257	0.0131236091256142 \\
0.372816229090882	0.0131314685568213 \\
0.373696444358129	0.0131549322977662 \\
0.375236418301489	0.0131960399448872 \\
0.375355278624935	0.0131992120295763 \\
0.375592850161798	0.0132055561989546 \\
0.375621310337888	0.0132063124328852 \\
0.37623340048021	0.01322266086936 \\
0.376649481158299	0.0132337780669332 \\
0.37682934901581	0.0132385846227407 \\
0.377022374188538	0.0132437441498041 \\
0.378777395062222	0.013290673494339 \\
0.379219181502585	0.0133024854585528 \\
0.380639994297166	0.0133405150845647 \\
0.381635433931239	0.0133671732619405 \\
0.382416367624347	0.0133880916982889 \\
0.383276747323035	0.0134111503139138 \\
0.383441774868681	0.0134155778214335 \\
0.384589480637987	0.0134463449940085 \\
0.38762147240542	0.0135276783257723 \\
0.387652450706726	0.0135285127907991 \\
0.38818745089578	0.0135428672656417 \\
0.388531576179462	0.0135521050542593 \\
0.391108514236782	0.0136212967336178 \\
0.391741157295481	0.0136382915079594 \\
0.396723908263121	0.0137721505016088 \\
0.39803082540264	0.0138072660192847 \\
0.398171926406937	0.0138110546395183 \\
0.398666017137866	0.0138243259862065 \\
0.398819673578751	0.0138284545391798 \\
0.4012576168302	0.0138939404860139 \\
0.402752737045505	0.0139340832829475 \\
0.403290070599278	0.0139485104009509 \\
0.404563584871068	0.0139826843515038 \\
0.406532702382757	0.0140355071052909 \\
0.406578018270242	0.0140367168933153 \\
0.413273296435021	0.014215980656445 \\
0.413592377461711	0.0142245087772608 \\
0.41401191876112	0.014235713519156 \\
0.414776267969876	0.0142561253160238 \\
0.414785799829368	0.0142563823610544 \\
0.415238430177184	0.0142684616148472 \\
0.416465974154779	0.0143012078478932 \\
0.416689013292801	0.0143071562051773 \\
0.418297500020148	0.0143500091508031 \\
0.422720878338663	0.0144675290212035 \\
0.424343600824592	0.0145105021074414 \\
0.425257001696199	0.0145346615463495 \\
0.42710227803989	0.0145833762362599 \\
0.427962517509508	0.0146060483530164 \\
0.429309169026115	0.0146414870396256 \\
0.429443446115563	0.014645017683506 \\
0.429705476003957	0.0146518955007195 \\
0.431003390699289	0.0146859716624022 \\
0.432539042653788	0.0147262001410127 \\
0.433068569337496	0.0147400461137295 \\
0.433545956268458	0.0147525183856487 \\
0.433750136573834	0.0147578474134207 \\
0.433778478107147	0.0147585943341255 \\
0.434834104829426	0.0147861214354634 \\
0.435898870411909	0.014813844114542 \\
0.43750740037395	0.0148556269705296 \\
0.437737100851006	0.0148615743964911 \\
0.439197385289126	0.0148993749171495 \\
0.440921894924801	0.0149438735097647 \\
0.442219378180834	0.0149772213771939 \\
0.446052885401706	0.0150752225890756 \\
0.447759455363191	0.0151185467839241 \\
0.448771556188603	0.0151441432535648 \\
0.448918332177756	0.0151478545740247 \\
0.449861491210787	0.0151716424152255 \\
0.449976789412666	0.0151745388284326 \\
0.450129821782654	0.0151783972978592 \\
0.453058505166632	0.0152517799288034 \\
0.455515296612047	0.0153128206729889 \\
0.461612686527044	0.0154621200636029 \\
0.462183229417738	0.0154759231954813 \\
0.46315332653893	0.0154993003234267 \\
0.463567773958142	0.0155092626810074 \\
0.464569005654222	0.0155332610011101 \\
0.46546966357775	0.0155547605827451 \\
0.465627783083443	0.015558528713882 \\
0.466166033904132	0.0155713278800249 \\
0.466888847202291	0.015588459558785 \\
0.468332203161028	0.0156225273385644 \\
0.469094139229819	0.015640415251255 \\
0.469464236056835	0.015649076551199 \\
0.469806075154756	0.0156570579856634 \\
0.470776441432414	0.0156796742230654 \\
0.471475988022769	0.0156959015876055 \\
0.473199954060133	0.0157356467097998 \\
0.474832247453521	0.015772944316268 \\
0.475230428162318	0.0157819893211126 \\
0.476638306671274	0.0158138275146484 \\
0.478988899075304	0.0158663783222437 \\
0.479018256118887	0.0158670358359814 \\
0.479279919995254	0.0158728417009115 \\
0.479532435695615	0.0158784314990044 \\
0.479763011578347	0.0158835276961327 \\
0.480041882522935	0.0158896762877703 \\
0.480923824850651	0.0159090850502253 \\
0.481629462985493	0.0159245170652866 \\
0.481771508198592	0.0159276202321053 \\
0.482503736852496	0.0159435477107763 \\
0.483064995688331	0.0159557163715363 \\
0.48384062770339	0.0159724354743958 \\
0.484085120492796	0.0159776899963617 \\
0.484352733912412	0.0159834399819374 \\
0.484550939846654	0.0159876700490713 \\
0.485022749278845	0.0159977525472641 \\
0.485322068953861	0.0160041321069002 \\
0.48610300375999	0.0160207077860832 \\
0.486253849387163	0.0160238929092884 \\
0.489154935058063	0.0160845369100571 \\
0.489792628987292	0.0160976815968752 \\
0.490458183784177	0.0161113310605288 \\
0.492282643359321	0.0161483585834503 \\
0.493928358105999	0.0161812696605921 \\
0.495701335790147	0.0162161644548178 \\
0.496106252426634	0.0162240471690893 \\
0.496164487189195	0.0162251833826303 \\
0.498739819509518	0.0162746161222458 \\
0.500164108812496	0.0163014065474272 \\
0.500739446514191	0.0163121093064547 \\
0.500830786926564	0.0163138043135405 \\
0.502541719930356	0.0163452196866274 \\
0.503498564874392	0.0163625217974186 \\
0.504534081188611	0.0163810402154922 \\
0.504637823699037	0.0163828823715448 \\
0.506860898199194	0.016421789303422 \\
0.507491092240963	0.0164326261729002 \\
0.508120202534381	0.0164433475583792 \\
0.510856303638406	0.0164889618754387 \\
0.512327455201249	0.0165127646178007 \\
0.51318641739938	0.0165264140814543 \\
0.513457954653542	0.0165306944400072 \\
0.515863112336536	0.0165678132325411 \\
0.516616766484905	0.0165791492909193 \\
0.517831355926389	0.0165971107780933 \\
0.518931921097757	0.0166130438446999 \\
0.521152065672689	0.016644224524498 \\
0.524519468497769	0.0166889578104019 \\
0.524576690058109	0.0166896767914295 \\
0.525883589956303	0.0167061574757099 \\
0.526973717032038	0.0167195256799459 \\
0.528680408273136	0.0167397446930408 \\
0.530600365407266	0.0167614407837391 \\
0.530809662396022	0.0167637411504984 \\
0.53166475254574	0.0167729891836643 \\
0.532158485575632	0.0167782176285982 \\
0.535204926183725	0.0168088134378195 \\
0.535356594523575	0.0168102625757456 \\
0.537004451507049	0.0168254692107439 \\
0.537089668720381	0.0168262366205454 \\
0.537190935482585	0.0168271362781525 \\
0.538117362047647	0.016835231333971 \\
0.538453506118723	0.016838101670146 \\
0.539359907716354	0.0168456453830004 \\
0.539560791003075	0.0168472733348608 \\
0.541180326502456	0.0168599393218756 \\
0.547271589683928	0.0168992169201374 \\
0.548822201570025	0.016907038167119 \\
0.549624872437239	0.0169107280671597 \\
0.549819206435189	0.0169115830212831 \\
0.550184928091146	0.0169131550937891 \\
0.552139674398207	0.016920693218708 \\
0.552982438656419	0.0169234741479158 \\
0.5537594353206	0.0169257875531912 \\
0.553976139314498	0.016926396638155 \\
0.554051563436848	0.0169266052544117 \\
0.55426017856816	0.0169271621853113 \\
0.555396555661942	0.016929879784584 \\
0.555501425327998	0.0169301144778728 \\
0.556360039382527	0.0169317908585072 \\
0.557883446271028	0.0169340223073959 \\
0.558670348456701	0.0169347990304232 \\
0.559194391744592	0.0169351696968079 \\
0.559473603788983	0.0169353205710649 \\
0.560198442614008	0.0169355589896441 \\
0.560853933269278	0.0169355887919664 \\
0.561152915237902	0.0169355403631926 \\
0.561401591015197	0.0169354677200317 \\
0.562479951006079	0.0169348586350679 \\
0.563005908106818	0.0169343817979097 \\
0.563050000205171	0.0169343333691359 \\
0.563426857102662	0.0169339049607515 \\
0.563560604448239	0.0169337391853333 \\
0.564062038149409	0.0169330462813377 \\
0.565336310962797	0.0169307980686426 \\
0.565855772748177	0.0169296693056822 \\
0.566858287901599	0.016927158460021 \\
0.566881879040954	0.0169270914047956 \\
0.568231608554438	0.0169229749590158 \\
0.569223717564581	0.0169194340705872 \\
0.569387586449175	0.0169188044965267 \\
0.569573989003313	0.0169180724769831 \\
0.569720721150053	0.0169174820184708 \\
0.572004618116245	0.0169070698320866 \\
0.572005154945678	0.016907062381506 \\
0.572151528516064	0.0169063191860914 \\
0.57219889974288	0.0169060621410608 \\
0.572780347928331	0.0169029776006937 \\
0.573600339080632	0.0168983433395624 \\
0.576593284261927	0.0168787203729153 \\
0.576676798911885	0.0168781112879515 \\
0.576819659872325	0.0168770551681519 \\
0.577653861143233	0.0168707240372896 \\
0.577917937736817	0.0168686471879482 \\
0.57793430586259	0.0168685205280781 \\
0.581759386894494	0.0168345868587494 \\
0.582475032298807	0.0168274343013763 \\
0.582896249817215	0.0168231017887592 \\
0.589571735768374	0.0167423486709595 \\
0.592388904473667	0.0167013239115477 \\
0.593611787705259	0.0166822113096714 \\
0.594151694891475	0.0166735127568245 \\
0.595008948927732	0.0166593920439482 \\
0.596105779635777	0.0166407451033592 \\
0.597490139381406	0.0166162811219692 \\
0.598951438337811	0.0165893379598856 \\
0.599201054717089	0.0165846161544323 \\
0.600228747033593	0.0165648348629475 \\
0.600693928524281	0.0165556874126196 \\
0.601589358474053	0.0165377501398325 \\
0.602529748251813	0.016518434509635 \\
0.602916831603936	0.0165103431791067 \\
0.603866403426713	0.0164901446551085 \\
0.606129689323002	0.0164399780333042 \\
0.607808363354841	0.0164009351283312 \\
0.60811641039762	0.0163935963064432 \\
0.608424173673929	0.016386216506362 \\
0.610203299896366	0.0163425002247095 \\
0.61052401101773	0.016334431245923 \\
0.611007548691541	0.0163221582770348 \\
0.61345268097321	0.0162580776959658 \\
0.614300601551049	0.016235064715147 \\
0.616415105925723	0.0161759275943041 \\
0.617595625529809	0.0161418057978153 \\
0.619695383178102	0.0160791836678982 \\
0.620626131680421	0.0160506181418896 \\
0.620884093030541	0.0160426143556833 \\
0.622959094549987	0.015976894646883 \\
0.624820249131202	0.0159158725291491 \\
0.627396892203872	0.0158281810581684 \\
0.629679734717881	0.0157473832368851 \\
0.63108091324942	0.0156963430345058 \\
0.631412061821856	0.015684125944972 \\
0.632209138475077	0.0156544633209705 \\
0.632608841524959	0.015639454126358 \\
0.635174872896091	0.015541004948318 \\
0.63680487509045	0.0154765909537673 \\
0.638872902814914	0.0153927784413099 \\
0.639214862746441	0.0153787015005946 \\
0.640549343973947	0.0153231509029865 \\
0.6407383188268	0.0153152076527476 \\
0.642765742598199	0.0152287734672427 \\
0.642772962045468	0.0152284661307931 \\
0.644393148200581	0.0151578281074762 \\
0.646791193304363	0.0150507474318147 \\
0.647126395490269	0.0150355491787195 \\
0.647747012654458	0.0150072472169995 \\
0.648487744871255	0.0149732092395425 \\
0.65164263639371	0.0148251038044691 \\
0.651778799711512	0.0148185957223177 \\
0.652422089544704	0.014787744730711 \\
0.652778066297422	0.014770582318306 \\
0.653037870316759	0.0147580169141293 \\
0.653946378314801	0.0147138210013509 \\
0.656181358570071	0.0146033819764853 \\
0.657515298781176	0.0145363211631775 \\
0.657916355325804	0.0145159922540188 \\
0.65821939898974	0.0145005863159895 \\
0.660347901626978	0.014391154050827 \\
0.663462409580861	0.0142272990196943 \\
0.665002009135555	0.0141447065398097 \\
0.665125585501828	0.0141380298882723 \\
0.665840946327609	0.0140992673113942 \\
0.669598702136104	0.0138920964673162 \\
0.669671107753933	0.0138880452141166 \\
0.670317902411108	0.0138517804443836 \\
0.67055566811489	0.0138384019955993 \\
0.670890712943537	0.0138195212930441 \\
0.671190867447461	0.0138025684282184 \\
0.672572121540986	0.0137240784242749 \\
0.674425718831603	0.0136175956577063 \\
0.67545414417829	0.0135579518973827 \\
0.67596832728122	0.0135279865935445 \\
0.676862814397186	0.013475626707077 \\
0.677008756097011	0.0134670594707131 \\
0.677760880297266	0.013422766700387 \\
0.680496299332769	0.0132600367069244 \\
0.6805627271546	0.0132560487836599 \\
0.682518497342478	0.0131381005048752 \\
0.682667061493577	0.0131290918216109 \\
0.683323391450582	0.013089207932353 \\
0.683455311069743	0.0130811706185341 \\
0.686099541255968	0.0129190031439066 \\
0.686972247003404	0.0128650180995464 \\
0.687287921742946	0.0128454426303506 \\
0.688979470872378	0.0127400383353233 \\
0.689844777380402	0.0126858130097389 \\
0.69066753173656	0.012634071521461 \\
0.690924484351316	0.0126178758218884 \\
0.691684985973077	0.0125698409974575 \\
0.694075688265434	0.0124178966507316 \\
0.694253786667502	0.0124065196141601 \\
0.696056771003981	0.0122909620404243 \\
0.696659969942023	0.0122521389275789 \\
0.69712902782289	0.012221897020936 \\
0.697436633040946	0.0122020421549678 \\
0.697470314030711	0.0121998619288206 \\
0.698295232645012	0.0121465036645532 \\
0.698478216988194	0.0121346488595009 \\
0.699563853841568	0.01206417940557 \\
0.700079752305739	0.0120306117460132 \\
0.700291212685059	0.0120168374851346 \\
0.701162742599685	0.0119599793106318 \\
0.702327231437584	0.0118837943300605 \\
0.702639966218633	0.0118632987141609 \\
0.703077231783761	0.0118346074596047 \\
0.703198302674555	0.0118266576901078 \\
0.703418311750763	0.0118122072890401 \\
0.704913184821873	0.011713813059032 \\
0.705985343895602	0.0116430260241032 \\
0.70676366586065	0.0115915415808558 \\
0.70953480835097	0.0114075886085629 \\
0.709597763991885	0.0114033967256546 \\
0.710916839332999	0.0113155078142881 \\
0.710971707887323	0.0113118477165699 \\
0.711527088935361	0.0112747801467776 \\
0.711625741549748	0.0112681919708848 \\
0.712030078150365	0.0112411910668015 \\
0.713838731517111	0.0111201982945204 \\
0.71495171630821	0.0110456142574549 \\
0.715877253359645	0.0109835173934698 \\
0.717907954554698	0.0108470870181918 \\
0.718267373832691	0.010822918266058 \\
0.719482266109768	0.0107411835342646 \\
0.720072506203388	0.0107014449313283 \\
0.72063303964824	0.010663696564734 \\
0.721414113641739	0.0106110963970423 \\
0.721834558021562	0.010582766495645 \\
0.721858215436245	0.0105811720713973 \\
0.722606600061919	0.0105307446792722 \\
0.723634632256085	0.0104614552110434 \\
0.724295951475884	0.01041688490659 \\
0.724780388322819	0.0103842345997691 \\
0.725967182906895	0.0103042460978031 \\
0.728562998007155	0.0101293530315161 \\
0.729378939538106	0.0100744189694524 \\
0.729672766087416	0.0100546348839998 \\
0.730156271089799	0.0100220991298556 \\
0.730418624607748	0.0100044468417764 \\
0.731079272213317	0.00996000878512859 \\
0.73140958111192	0.00993780232965946 \\
0.734582373950397	0.00972477067261934 \\
0.735036686489948	0.00969432387501001 \\
0.736495162798249	0.00959667563438416 \\
0.737637169481189	0.0095203360542655 \\
0.738132551177952	0.00948725733906031 \\
0.738772404713377	0.00944456737488508 \\
0.73919080617973	0.0094166724011302 \\
0.739634766317876	0.00938709639012814 \\
0.739671405613807	0.00938465446233749 \\
0.740318229014976	0.00934160035103559 \\
0.741017797721675	0.0092950901016593 \\
0.741923938399663	0.00923492014408112 \\
0.742870439596604	0.00917218346148729 \\
0.743500813750507	0.00913045462220907 \\
0.745235425982657	0.00901591125875711 \\
0.745794924415534	0.00897904578596354 \\
0.746651135832285	0.00892272591590881 \\
0.748242417565033	0.00881833862513304 \\
0.749080152481028	0.00876353681087494 \\
0.750738597109713	0.00865539256483316 \\
0.754059260293618	0.00844028126448393 \\
0.754427277039572	0.00841656606644392 \\
0.755971311279052	0.00831734109669924 \\
0.7564660147832	0.00828564912080765 \\
0.756964107658398	0.0082537904381752 \\
0.757301980310619	0.00823220517486334 \\
0.757561417579673	0.00821564625948668 \\
0.759460973532791	0.008094840683043 \\
0.759943823499425	0.00806425139307976 \\
0.76007481161266	0.00805596634745598 \\
0.76520949758525	0.00773406866937876 \\
0.769369251223668	0.00747787253931165 \\
0.77094746139779	0.0073818052187562 \\
0.771797041083953	0.00733035709708929 \\
0.772164013593626	0.00730819441378117 \\
0.772721858147557	0.00727457273751497 \\
0.774553209660584	0.00716477306559682 \\
0.774686774792601	0.00715680420398712 \\
0.774763255366	0.00715224491432309 \\
0.776676196809448	0.00703864404931664 \\
0.781826328102722	0.00673792604357004 \\
0.783936140597273	0.00661695888265967 \\
0.784010287514813	0.00661272834986448 \\
0.784889735175701	0.00656271260231733 \\
0.785018616770813	0.00655540311709046 \\
0.78602927274531	0.00649825390428305 \\
0.786818186718577	0.00645385729148984 \\
0.787974469502378	0.00638912711292505 \\
0.791139177359944	0.00621407711878419 \\
0.791586708117001	0.00618957495316863 \\
0.791796196079504	0.00617812527343631 \\
0.79182158322754	0.00617673853412271 \\
0.792072748806228	0.00616303365677595 \\
0.792682852079258	0.00612982781603932 \\
0.792705401732446	0.00612859940156341 \\
0.794569839509874	0.00602785591036081 \\
0.795023193386171	0.00600352324545383 \\
0.795367478837485	0.00598508911207318 \\
0.795527938976809	0.00597651302814484 \\
0.795634872431087	0.00597080029547215 \\
0.79745955888423	0.00587389664724469 \\
0.798224122257701	0.0058336085639894 \\
0.799881016007139	0.00574695225805044 \\
0.801765379467602	0.00564948143437505 \\
0.80362962295172	0.00555419269949198 \\
0.803814553638313	0.00554480217397213 \\
0.806469107208753	0.00541124632582068 \\
0.806697504962046	0.00539986463263631 \\
0.807337366299355	0.00536806648597121 \\
0.808888043138527	0.00529156858101487 \\
0.809190014741277	0.00527676101773977 \\
0.809307640373056	0.00527100404724479 \\
0.810501764417227	0.00521280383691192 \\
0.811889729396757	0.00514575093984604 \\
0.811942350832027	0.0051432210020721 \\
0.814196326909986	0.00503572775050998 \\
0.815151250895423	0.00499069737270474 \\
0.815544311745005	0.00497225252911448 \\
0.816375455024602	0.00493341451510787 \\
0.818069790025891	0.00485494872555137 \\
0.820698196177955	0.00473511731252074 \\
0.821263012964588	0.00470966147258878 \\
0.823441468430582	0.00461248867213726 \\
0.824103550036923	0.00458326656371355 \\
0.825253713607837	0.00453284056857228 \\
0.827170035611177	0.00444980338215828 \\
0.827687372625058	0.00442758901044726 \\
0.828681002627947	0.00438517658039927 \\
0.831688136352904	0.00425878074020147 \\
0.832015103492291	0.00424521742388606 \\
0.832618539764885	0.00422027334570885 \\
0.833979270360146	0.00416446151211858 \\
0.835071309810646	0.00412009935826063 \\
0.835699577265585	0.00409475527703762 \\
0.836498037763016	0.00406272523105145 \\
0.836836957998254	0.00404919171705842 \\
0.838046127292981	0.00400120811536908 \\
0.838525323246066	0.00398232322186232 \\
0.839239765753767	0.00395429925993085 \\
0.839402685897079	0.00394793227314949 \\
0.839999299682593	0.00392468646168709 \\
0.840281951751549	0.00391371548175812 \\
0.840483203711437	0.00390591868199408 \\
0.840691788412589	0.00389784900471568 \\
0.842685722208791	0.00382141815498471 \\
0.842993851211431	0.00380971864797175 \\
0.844759374999568	0.00374325504526496 \\
0.844835827371022	0.00374040030874312 \\
0.844992041004939	0.00373457022942603 \\
0.84621071404422	0.0036893526557833 \\
0.84647818555566	0.00367948855273426 \\
0.847551016932997	0.00364015158265829 \\
0.848091115090324	0.00362048065289855 \\
0.84863665263528	0.00360070448368788 \\
0.85175414414149	0.00348943681456149 \\
0.852055625215632	0.0034788332413882 \\
0.853183518340707	0.00343940733000636 \\
0.855762763379044	0.00335068115964532 \\
0.855845256518328	0.003347875084728 \\
0.856101518575613	0.00333917746320367 \\
0.856353280691239	0.00333064771257341 \\
0.857431392977156	0.00329433428123593 \\
0.858507268631162	0.00325843901373446 \\
0.858727032795181	0.00325114885345101 \\
0.859611972462117	0.00322193512693048 \\
0.860598063349272	0.00318964943289757 \\
0.861479766548793	0.00316102267242968 \\
0.863021652496581	0.00311149517074227 \\
0.863191163941701	0.00310609256848693 \\
0.863243196088782	0.00310443551279604 \\
0.863264684077014	0.00310375192202628 \\
0.86433033865504	0.00306999171152711 \\
0.865386426798682	0.00303685432299972 \\
0.865574805896198	0.00303097767755389 \\
0.867009964664233	0.00298652215860784 \\
0.868402477420643	0.0029439392965287 \\
0.868462536448662	0.00294211250729859 \\
0.8687736544668	0.00293267983943224 \\
0.870769922748437	0.00287277204915881 \\
0.87137568194098	0.00285481032915413 \\
0.871475162483056	0.00285186781547964 \\
0.872823623463892	0.00281227682717144 \\
0.873231322955811	0.00280040130019188 \\
0.87330913025223	0.00279814004898071 \\
0.874554061243274	0.00276218471117318 \\
0.874752616263773	0.0027564880438149 \\
0.875835647990412	0.00272560049779713 \\
0.878375805455995	0.00265436805784702 \\
0.879016462457315	0.00263666780665517 \\
0.879915986267341	0.00261199404485524 \\
0.880187041340505	0.00260460213758051 \\
0.880228906915535	0.00260345987044275 \\
0.880706157108953	0.00259049539454281 \\
0.881313907623959	0.00257406360469759 \\
0.88357711634013	0.00251370761543512 \\
0.88575628070969	0.00245680613443255 \\
0.885860150516388	0.00245412206277251 \\
0.885973756819001	0.00245119072496891 \\
0.886578379643548	0.00243564415723085 \\
0.88662976023068	0.00243432633578777 \\
0.887201064437371	0.00241972599178553 \\
0.88721221948908	0.00241944240406156 \\
0.889489342121342	0.00236203987151384 \\
0.889610563902722	0.00235901842825115 \\
0.891226627532086	0.00231908494606614 \\
0.891601333881282	0.00230991560965776 \\
0.894395711960869	0.0022425651550293 \\
0.894941283658575	0.00222962722182274 \\
0.895002834569702	0.00222817156463861 \\
0.895316145165804	0.00222077663056552 \\
0.89572498422586	0.00221116188913584 \\
0.900281097465164	0.00210658088326454 \\
0.901776757070804	0.00207325676456094 \\
0.902902077475763	0.00204850733280182 \\
0.903042041253622	0.00204544817097485 \\
0.903417484940991	0.00203726394101977 \\
0.905006674071476	0.00200295215472579 \\
0.905428107857581	0.00199394416995347 \\
0.90623158203924	0.00197687558829784 \\
0.909863735264334	0.0019013942219317 \\
0.914192347433306	0.00181495735887438 \\
0.914860066850241	0.00180195458233356 \\
0.916826528028795	0.0017641712911427 \\
0.918089828672593	0.00174029322806746 \\
0.918205626409754	0.00173811998683959 \\
0.918453743387651	0.00173347047530115 \\
0.918779069418784	0.00172739336267114 \\
0.919195929572793	0.00171963556203991 \\
0.920730394218805	0.00169136119075119 \\
0.920881992386684	0.00168859190307558 \\
0.921716310403637	0.00167342368513346 \\
0.921720421578646	0.00167335045989603 \\
0.923498857953819	0.0016414518468082 \\
0.923553028604257	0.00164048979058862 \\
0.924006080008399	0.00163246039301157 \\
0.92420639372191	0.00162892322987318 \\
0.924470669810756	0.00162426591850817 \\
0.924724469351036	0.00161980465054512 \\
0.926231702262142	0.00159355357754976 \\
0.926990949332903	0.00158048293087631 \\
0.933531060787661	0.00147203006781638 \\
0.934434631069947	0.00145761528983712 \\
0.934570551019964	0.00145545846316963 \\
0.935534942058006	0.00144024298060685 \\
0.936268298594132	0.00142877397593111 \\
0.936269242169234	0.00142875977326185 \\
0.938329910665083	0.00139699724968523 \\
0.94030185596857	0.00136723555624485 \\
0.940807389024333	0.00135970476549119 \\
0.941016354925932	0.00135660241357982 \\
0.942120305962364	0.00134032918140292 \\
0.944292950605489	0.00130884442478418 \\
0.944561491811014	0.0013050043489784 \\
0.944673077956372	0.00130341062322259 \\
0.945708241295843	0.00128872005734593 \\
0.947179815483775	0.00126810825895518 \\
0.947693224914561	0.00126099167391658 \\
0.950107910712063	0.00122803379781544 \\
0.953012070755755	0.00118949252646416 \\
0.953264426846652	0.00118619855493307 \\
0.953432175112961	0.00118401425424963 \\
0.954059232722306	0.00117588194552809 \\
0.954894864067499	0.0011651300592348 \\
0.955427443299792	0.00115832593291998 \\
0.958107244064232	0.00112467200960964 \\
0.95825981412274	0.00112278386950493 \\
0.961834544341432	0.00107942661270499 \\
0.962013308437074	0.00107730191666633 \\
0.962557230525089	0.00107086077332497 \\
0.964107437971182	0.00105270661879331 \\
0.96546229210349	0.00103708484675735 \\
0.966544858040119	0.00102476368192583 \\
0.966587760026518	0.00102427823003381 \\
0.966792601928706	0.00102196400985122 \\
0.968404850237238	0.00100392627064139 \\
0.968813010049556	0.000999408308416605 \\
0.969045613941432	0.000996842514723539 \\
0.969871084100612	0.00098778901156038 \\
0.970530837594512	0.00098061200696975 \\
0.971272888982341	0.000972597626969218 \\
0.971527190360033	0.00096986599965021 \\
0.977559419065177	0.000907217268832028 \\
0.978837143824273	0.000894461118150502 \\
0.979500241452307	0.000887911068275571 \\
0.979698366054342	0.000885962625034153 \\
0.981925106253984	0.000864350993651897 \\
0.98193841332347	0.000864224217366427 \\
0.985466980517735	0.000831031997222453 \\
0.985624420362111	0.000829580414574593 \\
0.985660230866112	0.000829250377137214 \\
0.985903288996567	0.000827014911919832 \\
0.986891736095765	0.000817984808236361 \\
0.988177000983844	0.000806387572083622 \\
0.988188066807458	0.000806288328021765 \\
0.988339255736614	0.000804935116320848 \\
0.988532835659511	0.000803205592092127 \\
0.988627149659547	0.000802364142145962 \\
0.991317832181511	0.000778718502260745 \\
0.991792763797732	0.000774616491980851 \\
0.99271508471038	0.000766710145398974 \\
0.992808132462744	0.000765917589887977 \\
0.99282017193431	0.000765814911574125 \\
0.994749345227338	0.000749549944885075 \\
0.99866338972896	0.000717584975063801 \\
};
\addplot [thick, color1]
table [row sep=\\]{%
0.0011059794751318	0.00297616911120713 \\
0.00144600828615515	0.00297854142263532 \\
0.00151620394328711	0.00297903222963214 \\
0.00161381928831428	0.0029797141905874 \\
0.00193644961641948	0.00298197078518569 \\
0.0024363370221504	0.00298546953126788 \\
0.00384167794354173	0.00299532478675246 \\
0.00421063900202368	0.00299791758880019 \\
0.00632372037002726	0.0030128110665828 \\
0.00947686783375368	0.00303517119027674 \\
0.0099356428556221	0.00303843524307013 \\
0.0101317191976503	0.00303983339108527 \\
0.0102519961026396	0.00304069160483778 \\
0.0116572936797861	0.00305072590708733 \\
0.0122763055741271	0.0030551569070667 \\
0.0128807282915644	0.00305948755703866 \\
0.0131006826676461	0.00306106731295586 \\
0.0132105085232722	0.00306185660883784 \\
0.0150477352877949	0.00307507393881679 \\
0.0151511241047927	0.0030758180655539 \\
0.0168553495563415	0.00308813154697418 \\
0.0191895991753259	0.00310507253743708 \\
0.0192724193499449	0.00310567812994123 \\
0.0202131188025382	0.00311253475956619 \\
0.0251886934986562	0.00314904260449111 \\
0.0255409812646521	0.00315164285711944 \\
0.0256196546755231	0.00315222563222051 \\
0.0263800727822662	0.00315784732811153 \\
0.026825697968402	0.00316114956513047 \\
0.0272388337061176	0.00316420895978808 \\
0.0276781997826389	0.00316746812313795 \\
0.0277969734855988	0.00316835241392255 \\
0.027888291941232	0.00316903041675687 \\
0.032801074197002	0.00320572010241449 \\
0.0337126623182875	0.00321257463656366 \\
0.0349109578825388	0.00322160613723099 \\
0.0367757145561887	0.00323571055196226 \\
0.0368887253476963	0.00323656527325511 \\
0.0376301111814882	0.00324219488538802 \\
0.0383020767002249	0.00324730039574206 \\
0.0385765124363792	0.00324938911944628 \\
0.0386281055032889	0.00324978143908083 \\
0.0397134557815922	0.00325805577449501 \\
0.0400530872540832	0.0032606478780508 \\
0.0408224228009035	0.00326652871444821 \\
0.042416614098386	0.00327874766662717 \\
0.042542997860721	0.00327971857041121 \\
0.0425634328976269	0.00327987689524889 \\
0.0426938593122377	0.00328087690286338 \\
0.0436353215690566	0.00328811886720359 \\
0.0476491609572692	0.00331917311996222 \\
0.0476764809468383	0.00331938616000116 \\
0.0480632979275277	0.00332239386625588 \\
0.0489335324060199	0.00332916923798621 \\
0.0494953541501341	0.00333355367183685 \\
0.0498656975684827	0.00333644542843103 \\
0.0507299142819697	0.00334320310503244 \\
0.0520149216066462	0.00335327605716884 \\
0.053491737186121	0.00336489151231945 \\
0.0545456118282759	0.00337320170365274 \\
0.0549368100196436	0.0033762922976166 \\
0.0549667939602374	0.0033765307161957 \\
0.0555159955251522	0.00338087324053049 \\
0.0574839546417889	0.0033964856993407 \\
0.0576927502423773	0.00339814345352352 \\
0.0580609228746025	0.00340107409283519 \\
0.0610070220733733	0.00342460698448122 \\
0.0613718310828713	0.00342753296718001 \\
0.0624819644863142	0.00343644898384809 \\
0.0632024977130776	0.00344224786385894 \\
0.0641707318537266	0.00345005746930838 \\
0.064803290117067	0.00345516833476722 \\
0.0661546388560771	0.00346610951237381 \\
0.0675945492445934	0.00347780715674162 \\
0.0701208744758773	0.00349842547439039 \\
0.0704495440664726	0.00350111466832459 \\
0.0707913481046274	0.00350391631945968 \\
0.0709503961374458	0.00350521923974156 \\
0.0723079949898875	0.00351636880077422 \\
0.0729582139062805	0.00352172181010246 \\
0.0736427592129116	0.00352736609056592 \\
0.0738431652021745	0.00352901872247458 \\
0.074094042941424	0.00353109114803374 \\
0.0741988699821797	0.00353195494972169 \\
0.0745344402647278	0.00353472935967147 \\
0.0761651718345726	0.00354823656380177 \\
0.0762565104011357	0.00354899349622428 \\
0.0764247618188199	0.00355039071291685 \\
0.0780473992355428	0.00356388953514397 \\
0.0784939963588281	0.00356761459261179 \\
0.0787288834363844	0.003569575259462 \\
0.0796528614743472	0.00357729662209749 \\
0.0821291956626848	0.00359806953929365 \\
0.0836539892117674	0.00361091806553304 \\
0.0837976822096664	0.00361213204450905 \\
0.0840863722935196	0.0036145702470094 \\
0.0847664004367794	0.00362032186239958 \\
0.0848642352310873	0.00362115050666034 \\
0.0852150851090396	0.00362412096001208 \\
0.085891869345856	0.00362985953688622 \\
0.087068315817913	0.00363985565491021 \\
0.0885849846877879	0.00365278194658458 \\
0.0903431131516136	0.00366782234050333 \\
0.0909278592293694	0.00367283704690635 \\
0.0914296081009306	0.00367714837193489 \\
0.0922298300096035	0.00368403294123709 \\
0.0926517333242025	0.00368766556493938 \\
0.0930086769471365	0.00369074172340333 \\
0.0936368650628759	0.00369616458192468 \\
0.0939832817677504	0.00369915715418756 \\
0.0963578674192209	0.00371973565779626 \\
0.0969604240071604	0.00372497923672199 \\
0.0974174263218397	0.00372895714826882 \\
0.0997464409700664	0.00374929909594357 \\
0.100249930959155	0.00375371123664081 \\
0.100820744388044	0.00375871919095516 \\
0.101229398304142	0.0037623094394803 \\
0.102271393895604	0.00377147551625967 \\
0.102325882258744	0.00377195770852268 \\
0.103474916599901	0.00378209142945707 \\
0.104369461810087	0.00379000278189778 \\
0.104849386693897	0.00379425077699125 \\
0.105274572558957	0.00379802053794265 \\
0.105437385176067	0.00379946478642523 \\
0.105718642159513	0.00380196189507842 \\
0.106451679001465	0.00380847440101206 \\
0.106754108203929	0.00381116708740592 \\
0.106889477031479	0.00381236872635782 \\
0.108308586301563	0.00382502051070333 \\
0.108355506097126	0.00382544100284576 \\
0.110864090628752	0.0038479114882648 \\
0.112062699234041	0.00385869084857404 \\
0.112686863552251	0.00386431673541665 \\
0.113787419472266	0.00387425464577973 \\
0.114114908535732	0.00387721764855087 \\
0.114926076298882	0.0038845653180033 \\
0.115710925433917	0.00389168714173138 \\
0.116193807044869	0.00389607413671911 \\
0.116352662276253	0.00389751954935491 \\
0.116979956596317	0.00390322739258409 \\
0.117247317403657	0.00390566419810057 \\
0.118793828877862	0.00391978397965431 \\
0.119352906137166	0.00392489973455667 \\
0.120388240876593	0.00393439037725329 \\
0.122357154089364	0.0039525032043457 \\
0.125712799722249	0.00398355675861239 \\
0.126988316690772	0.00399542506784201 \\
0.130289689058528	0.00402629701420665 \\
0.130830575131366	0.00403137737885118 \\
0.13178708265069	0.00404037442058325 \\
0.132217070962463	0.00404442474246025 \\
0.132641072460891	0.0040484257042408 \\
0.133694684584262	0.00405837874859571 \\
0.135586813424106	0.00407631183043122 \\
0.136970534315695	0.00408947700634599 \\
0.137167897059254	0.00409135920926929 \\
0.13731957841644	0.0040928041562438 \\
0.138770949832641	0.00410666735842824 \\
0.141579671494468	0.0041336240246892 \\
0.142337772861435	0.00414093118160963 \\
0.142852169410084	0.00414589419960976 \\
0.14313843453823	0.00414865836501122 \\
0.143709892765619	0.00415418343618512 \\
0.143996587198751	0.00415695691481233 \\
0.144338718134467	0.0041602710261941 \\
0.145474509131844	0.00417128810659051 \\
0.146058924442183	0.00417696870863438 \\
0.147496370200373	0.00419096881523728 \\
0.147934500473159	0.00419524777680635 \\
0.148464696403164	0.00420042732730508 \\
0.149007159183894	0.00420573400333524 \\
0.151290695331351	0.00422814674675465 \\
0.153507740604538	0.00425001513212919 \\
0.153927498261582	0.00425416836515069 \\
0.153953160827713	0.00425442308187485 \\
0.154179436777791	0.00425666337832808 \\
0.155107736993068	0.004265864379704 \\
0.15519005879533	0.00426668347790837 \\
0.155418402417591	0.00426894938573241 \\
0.156078407887877	0.004275509621948 \\
0.158100170599977	0.00429566204547882 \\
0.158283533549941	0.00429749395698309 \\
0.158599777942073	0.00430065486580133 \\
0.159258263288539	0.00430724630132318 \\
0.159678910282026	0.0043114610016346 \\
0.162540881855109	0.00434024492278695 \\
0.164077384668961	0.00435577519237995 \\
0.164957348092599	0.00436469167470932 \\
0.165030574098427	0.00436543487012386 \\
0.165280388241905	0.00436796946451068 \\
0.167273692684132	0.00438824901357293 \\
0.168043854724866	0.0043961089104414 \\
0.168088628486329	0.00439656525850296 \\
0.168662857989897	0.00440243631601334 \\
0.169151147716595	0.00440743379294872 \\
0.169673214248392	0.00441278237849474 \\
0.170135609058708	0.00441752327606082 \\
0.171767974673073	0.00443430291488767 \\
0.1728518498484	0.00444547925144434 \\
0.172984599941167	0.00444684829562902 \\
0.176370551959836	0.00448194285854697 \\
0.176588273575192	0.0044842092320323 \\
0.176656553576852	0.00448491936549544 \\
0.17794804219768	0.00449838163331151 \\
0.178429717033185	0.00450341356918216 \\
0.179458814233081	0.00451418105512857 \\
0.180595104426831	0.00452609779313207 \\
0.18064479669491	0.00452661793678999 \\
0.182294707450728	0.00454397685825825 \\
0.182819109355123	0.00454950612038374 \\
0.189141614514094	0.00461668474599719 \\
0.190723527970293	0.00463363854214549 \\
0.192508826701743	0.00465284008532763 \\
0.200712508340217	0.00474204961210489 \\
0.202130410523186	0.0047576311044395 \\
0.202705868873484	0.00476396689191461 \\
0.202975660615516	0.00476694107055664 \\
0.2033001668374	0.00477051828056574 \\
0.204138047670012	0.00477977003902197 \\
0.204584283026216	0.00478470651432872 \\
0.205975712225296	0.00480012316256762 \\
0.208693969096644	0.00483037251979113 \\
0.209097012308874	0.00483487546443939 \\
0.210129299396579	0.00484641920775175 \\
0.210470524326312	0.00485023949295282 \\
0.210726758187617	0.00485311076045036 \\
0.210879765452585	0.00485482485964894 \\
0.213404555935633	0.00488321436569095 \\
0.214779680955682	0.00489874137565494 \\
0.215976334819965	0.00491228885948658 \\
0.21605980843853	0.00491323322057724 \\
0.21626681121775	0.00491558201611042 \\
0.217333435243771	0.00492769433185458 \\
0.217860184045886	0.0049336850643158 \\
0.218810300034316	0.00494451262056828 \\
0.219036830922259	0.00494709750637412 \\
0.219186045259484	0.00494879903271794 \\
0.219509161852637	0.00495248753577471 \\
0.219517283100912	0.00495258113369346 \\
0.220007941128277	0.00495818816125393 \\
0.221124031401732	0.0049709640443325 \\
0.221655686766872	0.00497705908492208 \\
0.222787328400317	0.00499006127938628 \\
0.222812240344664	0.00499034486711025 \\
0.224413114839207	0.00500878971070051 \\
0.225712117798195	0.00502379983663559 \\
0.226064824403414	0.00502788089215755 \\
0.22778297811216	0.0050478158518672 \\
0.227786137279108	0.00504785357043147 \\
0.228255816092263	0.0050533153116703 \\
0.22886379454373	0.00506039103493094 \\
0.229286896531361	0.00506532425060868 \\
0.231570483736477	0.00509200897067785 \\
0.232839821990075	0.00510689802467823 \\
0.234699792231533	0.00512878689914942 \\
0.236598748956898	0.00515122199431062 \\
0.236914864918852	0.00515496218577027 \\
0.238193170258512	0.0051701245829463 \\
0.238267801131912	0.0051710125990212 \\
0.241280255564163	0.00520690344274044 \\
0.241532477911657	0.00520992139354348 \\
0.242724887440966	0.00522419949993491 \\
0.243487528482621	0.00523334788158536 \\
0.243851755033328	0.00523772090673447 \\
0.243931127736832	0.00523867830634117 \\
0.245492487219521	0.00525747099891305 \\
0.245961870318215	0.00526313157752156 \\
0.247220233087027	0.00527833262458444 \\
0.248029547774932	0.00528813432902098 \\
0.248195183805212	0.00529014365747571 \\
0.252597476547261	0.00534374313428998 \\
0.254049623190913	0.00536152720451355 \\
0.254469169693275	0.00536667555570602 \\
0.254777550536903	0.00537046231329441 \\
0.254877558399671	0.0053716916590929 \\
0.257253649731868	0.0054009547457099 \\
0.257671137054733	0.00540610915049911 \\
0.26187840669548	0.00545830698683858 \\
0.264182169008257	0.0054870736785233 \\
0.265462287195374	0.00550311151891947 \\
0.267638564660905	0.00553046958521008 \\
0.268216851612881	0.00553776416927576 \\
0.269980014659659	0.00556003861129284 \\
0.270837940398052	0.00557090574875474 \\
0.271133018969332	0.00557464780285954 \\
0.271522794293612	0.00557959405705333 \\
0.272730212426688	0.00559493945911527 \\
0.27325151773824	0.00560157559812069 \\
0.273458501731612	0.00560421030968428 \\
0.275787496734588	0.00563394790515304 \\
0.275790346826632	0.00563398702070117 \\
0.275963502468307	0.00563620263710618 \\
0.276441724078765	0.00564232980832458 \\
0.277788622824976	0.00565960863605142 \\
0.278249554181787	0.00566553184762597 \\
0.278902433397602	0.00567393423989415 \\
0.280228184178203	0.00569101981818676 \\
0.282170737525181	0.00571613712236285 \\
0.284775883532131	0.00574996275827289 \\
0.285319633849948	0.0057570431381464 \\
0.285945143849347	0.00576519407331944 \\
0.287293498153222	0.0057828058488667 \\
0.288098109983407	0.00579333398491144 \\
0.288474372887691	0.00579826114699244 \\
0.288733693686359	0.00580166187137365 \\
0.291563977198166	0.00583886262029409 \\
0.293976605993451	0.00587072176858783 \\
0.294014528290392	0.00587122421711683 \\
0.294718175832172	0.00588054256513715 \\
0.294831463459233	0.00588204385712743 \\
0.294887681144015	0.00588278798386455 \\
0.299064488228799	0.00593836326152086 \\
0.299223468133874	0.00594048434868455 \\
0.299853491506122	0.00594890536740422 \\
0.301624334890634	0.00597262196242809 \\
0.302100714632492	0.00597901688888669 \\
0.302855829384862	0.0059891608543694 \\
0.303468685820656	0.00599740352481604 \\
0.304335545384868	0.0060090753249824 \\
0.305081848845274	0.00601913779973984 \\
0.305263786725066	0.00602159276604652 \\
0.305511643802704	0.00602493993937969 \\
0.30635774546767	0.00603637285530567 \\
0.307937429467832	0.00605776440352201 \\
0.308578187230734	0.00606645783409476 \\
0.313991998160918	0.00614026747643948 \\
0.315973848719028	0.00616745417937636 \\
0.317413649622303	0.00618725595995784 \\
0.319657848506035	0.00621821638196707 \\
0.319781920771261	0.00621993001550436 \\
0.320496004308145	0.0062298052944243 \\
0.320601310305949	0.00623126374557614 \\
0.322246867517597	0.00625406485050917 \\
0.323306124649751	0.00626877602189779 \\
0.323664358184811	0.00627375580370426 \\
0.323805398711091	0.00627571484073997 \\
0.324816846999054	0.00628979410976171 \\
0.325263683488062	0.00629601860418916 \\
0.327533460856979	0.0063277124427259 \\
0.328385869951453	0.00633964035660028 \\
0.329844054701458	0.00636008428409696 \\
0.33074904970515	0.00637279031798244 \\
0.332974695502457	0.00640411861240864 \\
0.333600133199531	0.00641293730586767 \\
0.336001356191777	0.00644687749445438 \\
0.338057278352338	0.00647602463141084 \\
0.33940563378102	0.00649518612772226 \\
0.339522610836627	0.00649684621021152 \\
0.341295751839904	0.00652209902182221 \\
0.343025583329651	0.00654679397121072 \\
0.344156227439929	0.00656296266242862 \\
0.344485478041551	0.00656767655164003 \\
0.346631556569637	0.00659844465553761 \\
0.348247004908186	0.00662165600806475 \\
0.348257869791207	0.00662181153893471 \\
0.350134953366802	0.00664884503930807 \\
0.352271571140202	0.00667968858033419 \\
0.352403536611892	0.00668159406632185 \\
0.356607161458692	0.00674250023439527 \\
0.357473355861859	0.00675508892163634 \\
0.359032837138255	0.00677777640521526 \\
0.360085447455855	0.00679311295971274 \\
0.360587038984981	0.00680042384192348 \\
0.3618814720936	0.0068193101324141 \\
0.362541325151431	0.0068289483897388 \\
0.363971114014582	0.0068498533219099 \\
0.36446377318254	0.00685706688091159 \\
0.364734654416428	0.00686103152111173 \\
0.364782079325344	0.00686172395944595 \\
0.36626622804169	0.00688347220420837 \\
0.36745676602671	0.00690093543380499 \\
0.367555756327517	0.0069023915566504 \\
0.367766921022286	0.00690549053251743 \\
0.368015883464081	0.00690914317965508 \\
0.368183587302196	0.00691160792484879 \\
0.368646745255496	0.00691841449588537 \\
0.368824340831231	0.0069210221990943 \\
0.36908666014638	0.00692487927153707 \\
0.370227788452905	0.00694165984168649 \\
0.371079364739364	0.00695418752729893 \\
0.371457393668236	0.00695975963026285 \\
0.372521553247257	0.00697543565183878 \\
0.372816229090882	0.00697978213429451 \\
0.373696444358129	0.00699275778606534 \\
0.375236418301489	0.00701549556106329 \\
0.375355278624935	0.00701724924147129 \\
0.375592850161798	0.00702075567096472 \\
0.375621310337888	0.00702117802575231 \\
0.37623340048021	0.00703021790832281 \\
0.376649481158299	0.00703637069091201 \\
0.37682934901581	0.00703903054818511 \\
0.377022374188538	0.00704188458621502 \\
0.378777395062222	0.00706785172224045 \\
0.379219181502585	0.00707439286634326 \\
0.380639994297166	0.00709544261917472 \\
0.381635433931239	0.00711020128801465 \\
0.382416367624347	0.00712178274989128 \\
0.383276747323035	0.0071345460601151 \\
0.383441774868681	0.00713700009509921 \\
0.384589480637987	0.00715404003858566 \\
0.38762147240542	0.00719910115003586 \\
0.387652450706726	0.00719956588000059 \\
0.38818745089578	0.00720752123743296 \\
0.388531576179462	0.00721264071762562 \\
0.391108514236782	0.00725099351257086 \\
0.391741157295481	0.00726041756570339 \\
0.396723908263121	0.0073346677236259 \\
0.39803082540264	0.00735415797680616 \\
0.398171926406937	0.00735626090317965 \\
0.398666017137866	0.00736362766474485 \\
0.398819673578751	0.00736592151224613 \\
0.4012576168302	0.00740227522328496 \\
0.402752737045505	0.00742457574233413 \\
0.403290070599278	0.00743258884176612 \\
0.404563584871068	0.00745157757773995 \\
0.406532702382757	0.00748093286529183 \\
0.406578018270242	0.00748160993680358 \\
0.413273296435021	0.00758130708709359 \\
0.413592377461711	0.00758605031296611 \\
0.41401191876112	0.00759229296818376 \\
0.414776267969876	0.00760365510359406 \\
0.414785799829368	0.00760379480198026 \\
0.415238430177184	0.00761052034795284 \\
0.416465974154779	0.00762875471264124 \\
0.416689013292801	0.00763206649571657 \\
0.418297500020148	0.00765593349933624 \\
0.422720878338663	0.00772142969071865 \\
0.424343600824592	0.00774539075791836 \\
0.425257001696199	0.00775887118652463 \\
0.42710227803989	0.00778604950755835 \\
0.427962517509508	0.00779870385304093 \\
0.429309169026115	0.0078184874728322 \\
0.429443446115563	0.00782046280801296 \\
0.429705476003957	0.00782430358231068 \\
0.431003390699289	0.0078433295711875 \\
0.432539042653788	0.00786580517888069 \\
0.433068569337496	0.0078735388815403 \\
0.433545956268458	0.00788050889968872 \\
0.433750136573834	0.00788349099457264 \\
0.433778478107147	0.00788390915840864 \\
0.434834104829426	0.00789929553866386 \\
0.435898870411909	0.00791479647159576 \\
0.43750740037395	0.00793816428631544 \\
0.437737100851006	0.00794149376451969 \\
0.439197385289126	0.00796264223754406 \\
0.440921894924801	0.00798755139112473 \\
0.442219378180834	0.00800623092800379 \\
0.446052885401706	0.00806114729493856 \\
0.447759455363191	0.00808543991297483 \\
0.448771556188603	0.0080998046323657 \\
0.448918332177756	0.00810188706964254 \\
0.449861491210787	0.00811523664742708 \\
0.449976789412666	0.00811686273664236 \\
0.450129821782654	0.00811902713030577 \\
0.453058505166632	0.00816023722290993 \\
0.455515296612047	0.00819454435259104 \\
0.461612686527044	0.00827857293188572 \\
0.462183229417738	0.00828634761273861 \\
0.46315332653893	0.00829952955245972 \\
0.463567773958142	0.00830514542758465 \\
0.464569005654222	0.00831867754459381 \\
0.46546966357775	0.0083308033645153 \\
0.465627783083443	0.00833292864263058 \\
0.466166033904132	0.00834015477448702 \\
0.466888847202291	0.0083498228341341 \\
0.468332203161028	0.00836905837059021 \\
0.469094139229819	0.00837916694581509 \\
0.469464236056835	0.00838405918329954 \\
0.469806075154756	0.00838857144117355 \\
0.470776441432414	0.00840135663747787 \\
0.471475988022769	0.00841052923351526 \\
0.473199954060133	0.00843301694840193 \\
0.474832247453521	0.00845413748174906 \\
0.475230428162318	0.00845926441252232 \\
0.476638306671274	0.00847730319947004 \\
0.478988899075304	0.00850711297243834 \\
0.479018256118887	0.00850748643279076 \\
0.479279919995254	0.00851077772676945 \\
0.479532435695615	0.00851395353674889 \\
0.479763011578347	0.00851684808731079 \\
0.480041882522935	0.0085203405469656 \\
0.480923824850651	0.0085313618183136 \\
0.481629462985493	0.00854012928903103 \\
0.481771508198592	0.00854189228266478 \\
0.482503736852496	0.00855095311999321 \\
0.483064995688331	0.00855787005275488 \\
0.48384062770339	0.00856738071888685 \\
0.484085120492796	0.00857037119567394 \\
0.484352733912412	0.00857363920658827 \\
0.484550939846654	0.00857605133205652 \\
0.485022749278845	0.00858179107308388 \\
0.485322068953861	0.0085854222998023 \\
0.48610300375999	0.00859486311674118 \\
0.486253849387163	0.00859667919576168 \\
0.489154935058063	0.00863126758486032 \\
0.489792628987292	0.00863876938819885 \\
0.490458183784177	0.00864656455814838 \\
0.492282643359321	0.00866773445159197 \\
0.493928358105999	0.00868656672537327 \\
0.495701335790147	0.00870656594634056 \\
0.496106252426634	0.00871109031140804 \\
0.496164487189195	0.0087117413058877 \\
0.498739819509518	0.00874012336134911 \\
0.500164108812496	0.00875553023070097 \\
0.500739446514191	0.00876169372349977 \\
0.500830786926564	0.00876266974955797 \\
0.502541719930356	0.00878077186644077 \\
0.503498564874392	0.00879075564444065 \\
0.504534081188611	0.00880144909024239 \\
0.504637823699037	0.0088025163859129 \\
0.506860898199194	0.00882502645254135 \\
0.507491092240963	0.00883130542933941 \\
0.508120202534381	0.00883752666413784 \\
0.510856303638406	0.00886402558535337 \\
0.512327455201249	0.00887789111584425 \\
0.51318641739938	0.00888585671782494 \\
0.513457954653542	0.00888835359364748 \\
0.515863112336536	0.00891006551682949 \\
0.516616766484905	0.00891671050339937 \\
0.517831355926389	0.00892726052552462 \\
0.518931921097757	0.00893663708120584 \\
0.521152065672689	0.00895503535866737 \\
0.524519468497769	0.00898158177733421 \\
0.524576690058109	0.00898201297968626 \\
0.525883589956303	0.00899184122681618 \\
0.526973717032038	0.0089998422190547 \\
0.528680408273136	0.00901198294013739 \\
0.530600365407266	0.00902508106082678 \\
0.530809662396022	0.00902647618204355 \\
0.53166475254574	0.00903208926320076 \\
0.532158485575632	0.0090352687984705 \\
0.535204926183725	0.00905400421470404 \\
0.535356594523575	0.00905489269644022 \\
0.537004451507049	0.00906431209295988 \\
0.537089668720381	0.00906478799879551 \\
0.537190935482585	0.00906534492969513 \\
0.538117362047647	0.00907039828598499 \\
0.538453506118723	0.00907219387590885 \\
0.539359907716354	0.00907693151384592 \\
0.539560791003075	0.00907795783132315 \\
0.541180326502456	0.00908598769456148 \\
0.547271589683928	0.00911169778555632 \\
0.548822201570025	0.00911707244813442 \\
0.549624872437239	0.00911966059356928 \\
0.549819206435189	0.00912026967853308 \\
0.550184928091146	0.00912138726562262 \\
0.552139674398207	0.00912691280245781 \\
0.552982438656419	0.00912904553115368 \\
0.5537594353206	0.00913087278604507 \\
0.553976139314498	0.00913136824965477 \\
0.554051563436848	0.0091315321624279 \\
0.55426017856816	0.00913198851048946 \\
0.555396555661942	0.00913430470973253 \\
0.555501425327998	0.00913450960069895 \\
0.556360039382527	0.00913605373352766 \\
0.557883446271028	0.00913839973509312 \\
0.558670348456701	0.00913940742611885 \\
0.559194391744592	0.00914000254124403 \\
0.559473603788983	0.00914029311388731 \\
0.560198442614008	0.00914096366614103 \\
0.560853933269278	0.00914146844297647 \\
0.561152915237902	0.00914166681468487 \\
0.561401591015197	0.00914181675761938 \\
0.562479951006079	0.00914229545742273 \\
0.563005908106818	0.00914243049919605 \\
0.563050000205171	0.0091424360871315 \\
0.563426857102662	0.00914248917251825 \\
0.563560604448239	0.00914249755442142 \\
0.564062038149409	0.00914249662309885 \\
0.565336310962797	0.00914223957806826 \\
0.565855772748177	0.0091420179232955 \\
0.566858287901599	0.00914141070097685 \\
0.566881879040954	0.00914139673113823 \\
0.568231608554438	0.00914018508046865 \\
0.569223717564581	0.00913900975137949 \\
0.569387586449175	0.00913879461586475 \\
0.569573989003313	0.00913853943347931 \\
0.569720721150053	0.0091383308172226 \\
0.572004618116245	0.00913441181182861 \\
0.572005154945678	0.00913441181182861 \\
0.572151528516064	0.00913411937654018 \\
0.57219889974288	0.00913401506841183 \\
0.572780347928331	0.00913278292864561 \\
0.573600339080632	0.00913088954985142 \\
0.576593284261927	0.0091225216165185 \\
0.576676798911885	0.00912225525826216 \\
0.576819659872325	0.00912179425358772 \\
0.577653861143233	0.009118995629251 \\
0.577917937736817	0.00911806710064411 \\
0.57793430586259	0.00911801401525736 \\
0.581759386894494	0.00910251867026091 \\
0.582475032298807	0.00909918639808893 \\
0.582896249817215	0.00909715425223112 \\
0.589571735768374	0.00905843731015921 \\
0.592388904473667	0.00903832633048296 \\
0.593611787705259	0.00902888923883438 \\
0.594151694891475	0.00902457907795906 \\
0.595008948927732	0.00901756901293993 \\
0.596105779635777	0.00900828558951616 \\
0.597490139381406	0.00899606291204691 \\
0.598951438337811	0.00898255128413439 \\
0.599201054717089	0.00898017827421427 \\
0.600228747033593	0.00897022243589163 \\
0.600693928524281	0.00896560866385698 \\
0.601589358474053	0.00895655248314142 \\
0.602529748251813	0.00894678104668856 \\
0.602916831603936	0.00894268229603767 \\
0.603866403426713	0.00893243402242661 \\
0.606129689323002	0.00890691392123699 \\
0.607808363354841	0.00888697989284992 \\
0.60811641039762	0.00888322852551937 \\
0.608424173673929	0.00887945294380188 \\
0.610203299896366	0.00885705463588238 \\
0.61052401101773	0.00885291211307049 \\
0.611007548691541	0.00884660985320807 \\
0.61345268097321	0.00881364475935698 \\
0.614300601551049	0.00880178064107895 \\
0.616415105925723	0.00877123698592186 \\
0.617595625529809	0.00875358283519745 \\
0.619695383178102	0.00872112158685923 \\
0.620626131680421	0.00870629213750362 \\
0.620884093030541	0.00870213657617569 \\
0.622959094549987	0.00866795890033245 \\
0.624820249131202	0.00863616727292538 \\
0.627396892203872	0.00859039556235075 \\
0.629679734717881	0.00854814425110817 \\
0.63108091324942	0.00852141343057156 \\
0.631412061821856	0.00851501151919365 \\
0.632209138475077	0.00849946308881044 \\
0.632608841524959	0.00849159341305494 \\
0.635174872896091	0.00843990966677666 \\
0.63680487509045	0.00840605329722166 \\
0.638872902814914	0.00836194679141045 \\
0.639214862746441	0.00835453532636166 \\
0.640549343973947	0.00832526665180922 \\
0.6407383188268	0.00832107849419117 \\
0.642765742598199	0.00827548932284117 \\
0.642772962045468	0.00827533006668091 \\
0.644393148200581	0.00823803246021271 \\
0.646791193304363	0.00818144716322422 \\
0.647126395490269	0.00817341078072786 \\
0.647747012654458	0.00815843511372805 \\
0.648487744871255	0.00814042426645756 \\
0.65164263639371	0.00806198734790087 \\
0.651778799711512	0.00805854052305222 \\
0.652422089544704	0.00804218463599682 \\
0.652778066297422	0.00803308468312025 \\
0.653037870316759	0.0080264238640666 \\
0.653946378314801	0.00800298247486353 \\
0.656181358570071	0.00794436782598495 \\
0.657515298781176	0.00790875218808651 \\
0.657916355325804	0.00789794884622097 \\
0.65821939898974	0.00788976065814495 \\
0.660347901626978	0.00783158373087645 \\
0.663462409580861	0.00774439051747322 \\
0.665002009135555	0.00770040508359671 \\
0.665125585501828	0.00769684743136168 \\
0.665840946327609	0.007676194421947 \\
0.669598702136104	0.00756573025137186 \\
0.669671107753933	0.00756356678903103 \\
0.670317902411108	0.00754421576857567 \\
0.67055566811489	0.00753707718104124 \\
0.670890712943537	0.00752700166776776 \\
0.671190867447461	0.00751795060932636 \\
0.672572121540986	0.00747604435309768 \\
0.674425718831603	0.00741916662082076 \\
0.67545414417829	0.00738729489967227 \\
0.67596832728122	0.00737127754837275 \\
0.676862814397186	0.0073432857170701 \\
0.677008756097011	0.00733870128169656 \\
0.677760880297266	0.00731501821428537 \\
0.680496299332769	0.00722795445472002 \\
0.6805627271546	0.00722581706941128 \\
0.682518497342478	0.00716267293319106 \\
0.682667061493577	0.00715784728527069 \\
0.683323391450582	0.00713648740202188 \\
0.683455311069743	0.00713218189775944 \\
0.686099541255968	0.00704529043287039 \\
0.686972247003404	0.00701635191217065 \\
0.687287921742946	0.00700585683807731 \\
0.688979470872378	0.00694933021441102 \\
0.689844777380402	0.00692024361342192 \\
0.69066753173656	0.00689248088747263 \\
0.690924484351316	0.00688379164785147 \\
0.691684985973077	0.00685801217332482 \\
0.694075688265434	0.00677643809467554 \\
0.694253786667502	0.0067703309468925 \\
0.696056771003981	0.00670825876295567 \\
0.696659969942023	0.00668740086257458 \\
0.69712902782289	0.00667115114629269 \\
0.697436633040946	0.00666047865524888 \\
0.697470314030711	0.00665930798277259 \\
0.698295232645012	0.00663063069805503 \\
0.698478216988194	0.00662425812333822 \\
0.699563853841568	0.00658637331798673 \\
0.700079752305739	0.00656832521781325 \\
0.700291212685059	0.00656092027202249 \\
0.701162742599685	0.00653034076094627 \\
0.702327231437584	0.00648936489596963 \\
0.702639966218633	0.00647833850234747 \\
0.703077231783761	0.00646289996802807 \\
0.703198302674555	0.00645862426608801 \\
0.703418311750763	0.00645084865391254 \\
0.704913184821873	0.00639789458364248 \\
0.705985343895602	0.00635979138314724 \\
0.70676366586065	0.0063320747576654 \\
0.70953480835097	0.00623300112783909 \\
0.709597763991885	0.00623074313625693 \\
0.710916839332999	0.00618338631466031 \\
0.710971707887323	0.00618141563609242 \\
0.711527088935361	0.00616143876686692 \\
0.711625741549748	0.00615788903087378 \\
0.712030078150365	0.00614333711564541 \\
0.713838731517111	0.00607811333611608 \\
0.71495171630821	0.00603789649903774 \\
0.715877253359645	0.00600441126152873 \\
0.717907954554698	0.00593081209808588 \\
0.718267373832691	0.0059177721850574 \\
0.719482266109768	0.00587366335093975 \\
0.720072506203388	0.00585221825167537 \\
0.72063303964824	0.00583184463903308 \\
0.721414113641739	0.00580344581976533 \\
0.721834558021562	0.00578815024346113 \\
0.721858215436245	0.0057872929610312 \\
0.722606600061919	0.00576006341725588 \\
0.723634632256085	0.00572264660149813 \\
0.724295951475884	0.00569857284426689 \\
0.724780388322819	0.00568093545734882 \\
0.725967182906895	0.00563772534951568 \\
0.728562998007155	0.0055432147346437 \\
0.729378939538106	0.00551351765170693 \\
0.729672766087416	0.00550282699987292 \\
0.730156271089799	0.00548523711040616 \\
0.730418624607748	0.00547569058835506 \\
0.731079272213317	0.00545166525989771 \\
0.73140958111192	0.00543965818360448 \\
0.734582373950397	0.00532443495467305 \\
0.735036686489948	0.00530795892700553 \\
0.736495162798249	0.00525512266904116 \\
0.737637169481189	0.00521380687132478 \\
0.738132551177952	0.00519590405747294 \\
0.738772404713377	0.00517279282212257 \\
0.73919080617973	0.00515769189223647 \\
0.739634766317876	0.00514168152585626 \\
0.739671405613807	0.00514035858213902 \\
0.740318229014976	0.00511704664677382 \\
0.741017797721675	0.00509185856208205 \\
0.741923938399663	0.00505927624180913 \\
0.742870439596604	0.00502529507502913 \\
0.743500813750507	0.00500269001349807 \\
0.745235425982657	0.00494063226506114 \\
0.745794924415534	0.00492065539583564 \\
0.746651135832285	0.00489013409242034 \\
0.748242417565033	0.00483355578035116 \\
0.749080152481028	0.0048038475215435 \\
0.750738597109713	0.00474521098658442 \\
0.754059260293618	0.00462853442877531 \\
0.754427277039572	0.00461566867306828 \\
0.755971311279052	0.00456183170899749 \\
0.7564660147832	0.0045446315780282 \\
0.756964107658398	0.00452734297141433 \\
0.757301980310619	0.00451562833040953 \\
0.757561417579673	0.00450664106756449 \\
0.759460973532791	0.00444106664508581 \\
0.759943823499425	0.00442445836961269 \\
0.76007481161266	0.00441995868459344 \\
0.76520949758525	0.00424513220787048 \\
0.769369251223668	0.00410590693354607 \\
0.77094746139779	0.00405368255451322 \\
0.771797041083953	0.00402571028098464 \\
0.772164013593626	0.00401366176083684 \\
0.772721858147557	0.00399537710472941 \\
0.774553209660584	0.00393566396087408 \\
0.774686774792601	0.00393132958561182 \\
0.774763255366	0.00392884993925691 \\
0.776676196809448	0.00386705179698765 \\
0.781826328102722	0.00370339886285365 \\
0.783936140597273	0.00363753852434456 \\
0.784010287514813	0.00363523396663368 \\
0.784889735175701	0.00360800023190677 \\
0.785018616770813	0.00360401929356158 \\
0.78602927274531	0.00357289495877922 \\
0.786818186718577	0.00354871433228254 \\
0.787974469502378	0.00351345562376082 \\
0.791139177359944	0.00341808004304767 \\
0.791586708117001	0.0034047260414809 \\
0.791796196079504	0.00339848594740033 \\
0.79182158322754	0.00339773111045361 \\
0.792072748806228	0.00339026027359068 \\
0.792682852079258	0.00337216281332076 \\
0.792705401732446	0.00337149342522025 \\
0.794569839509874	0.00331657775677741 \\
0.795023193386171	0.00330331269651651 \\
0.795367478837485	0.00329326256178319 \\
0.795527938976809	0.00328858732245862 \\
0.795634872431087	0.00328547228127718 \\
0.79745955888423	0.00323263299651444 \\
0.798224122257701	0.00321066123433411 \\
0.799881016007139	0.00316339777782559 \\
0.801765379467602	0.00311022391542792 \\
0.80362962295172	0.00305823259986937 \\
0.803814553638313	0.00305310846306384 \\
0.806469107208753	0.00298021733760834 \\
0.806697504962046	0.00297400425188243 \\
0.807337366299355	0.00295664602890611 \\
0.808888043138527	0.00291488203220069 \\
0.809190014741277	0.00290679722093046 \\
0.809307640373056	0.00290365377441049 \\
0.810501764417227	0.00287187355570495 \\
0.811889729396757	0.00283525534905493 \\
0.811942350832027	0.00283387233503163 \\
0.814196326909986	0.00277515756897628 \\
0.815151250895423	0.00275055691599846 \\
0.815544311745005	0.00274047860875726 \\
0.816375455024602	0.00271925865672529 \\
0.818069790025891	0.0026763800997287 \\
0.820698196177955	0.00261088227853179 \\
0.821263012964588	0.00259696785360575 \\
0.823441468430582	0.00254383776336908 \\
0.824103550036923	0.00252785882912576 \\
0.825253713607837	0.00250028260052204 \\
0.827170035611177	0.00245486386120319 \\
0.827687372625058	0.0024427124299109 \\
0.828681002627947	0.0024195103906095 \\
0.831688136352904	0.00235035014338791 \\
0.832015103492291	0.00234292773529887 \\
0.832618539764885	0.00232927640900016 \\
0.833979270360146	0.00229872879572213 \\
0.835071309810646	0.00227444525808096 \\
0.835699577265585	0.00226056994870305 \\
0.836498037763016	0.00224303477443755 \\
0.836836957998254	0.00223562493920326 \\
0.838046127292981	0.00220935046672821 \\
0.838525323246066	0.00219900975935161 \\
0.839239765753767	0.00218366226181388 \\
0.839402685897079	0.00218017632141709 \\
0.839999299682593	0.00216744467616081 \\
0.840281951751549	0.00216143508441746 \\
0.840483203711437	0.00215716543607414 \\
0.840691788412589	0.00215274444781244 \\
0.842685722208791	0.00211087753996253 \\
0.842993851211431	0.00210446817800403 \\
0.844759374999568	0.00206805299967527 \\
0.844835827371022	0.00206648954190314 \\
0.844992041004939	0.00206329533830285 \\
0.84621071404422	0.00203851517289877 \\
0.84647818555566	0.00203310954384506 \\
0.847551016932997	0.00201155012473464 \\
0.848091115090324	0.00200076983310282 \\
0.84863665263528	0.0019899292383343 \\
0.85175414414149	0.00192893005441874 \\
0.852055625215632	0.00192311522550881 \\
0.853183518340707	0.00190149643458426 \\
0.855762763379044	0.00185283413156867 \\
0.855845256518328	0.00185129547026008 \\
0.856101518575613	0.00184652395546436 \\
0.856353280691239	0.00184184568934143 \\
0.857431392977156	0.00182192586362362 \\
0.858507268631162	0.0018022321164608 \\
0.858727032795181	0.00179823278449476 \\
0.859611972462117	0.00178220355883241 \\
0.860598063349272	0.00176448782440275 \\
0.861479766548793	0.0017487775767222 \\
0.863021652496581	0.00172159413341433 \\
0.863191163941701	0.00171862868592143 \\
0.863243196088782	0.00171771924942732 \\
0.863264684077014	0.00171734415926039 \\
0.86433033865504	0.00169881258625537 \\
0.865386426798682	0.00168062071315944 \\
0.865574805896198	0.00167739484459162 \\
0.867009964664233	0.00165298709180206 \\
0.868402477420643	0.00162960309535265 \\
0.868462536448662	0.00162860041018575 \\
0.8687736544668	0.00162342004477978 \\
0.870769922748437	0.00159051758237183 \\
0.87137568194098	0.001580651383847 \\
0.871475162483056	0.00157903565559536 \\
0.872823623463892	0.00155728566460311 \\
0.873231322955811	0.00155076128430665 \\
0.87330913025223	0.00154951948206872 \\
0.874554061243274	0.00152976473327726 \\
0.874752616263773	0.00152663455810398 \\
0.875835647990412	0.00150966178625822 \\
0.878375805455995	0.00147051305975765 \\
0.879016462457315	0.00146078434772789 \\
0.879915986267341	0.00144722103141248 \\
0.880187041340505	0.0014431569725275 \\
0.880228906915535	0.00144252949394286 \\
0.880706157108953	0.00143540150020272 \\
0.881313907623959	0.00142636813689023 \\
0.88357711634013	0.00139318080618978 \\
0.88575628070969	0.00136188743636012 \\
0.885860150516388	0.00136041129007936 \\
0.885973756819001	0.00135879870504141 \\
0.886578379643548	0.00135024741757661 \\
0.88662976023068	0.00134952261578292 \\
0.887201064437371	0.00134149135556072 \\
0.88721221948908	0.00134133547544479 \\
0.889489342121342	0.00130975502543151 \\
0.889610563902722	0.00130809331312776 \\
0.891226627532086	0.00128611922264099 \\
0.891601333881282	0.0012810732005164 \\
0.894395711960869	0.00124400609638542 \\
0.894941283658575	0.00123688403982669 \\
0.895002834569702	0.0012360829859972 \\
0.895316145165804	0.00123201287351549 \\
0.89572498422586	0.00122671970166266 \\
0.900281097465164	0.00116913567762822 \\
0.901776757070804	0.00115078210365027 \\
0.902902077475763	0.00113714940380305 \\
0.903042041253622	0.00113546417560428 \\
0.903417484940991	0.00113095552660525 \\
0.905006674071476	0.00111205270513892 \\
0.905428107857581	0.00110708980355412 \\
0.90623158203924	0.0010976844932884 \\
0.909863735264334	0.00105608743615448 \\
0.914192347433306	0.00100843515247107 \\
0.914860066850241	0.0010012648999691 \\
0.916826528028795	0.000980428420007229 \\
0.918089828672593	0.000967258820310235 \\
0.918205626409754	0.000966059742495418 \\
0.918453743387651	0.000963494763709605 \\
0.918779069418784	0.000960142642725259 \\
0.919195929572793	0.000955863448325545 \\
0.920730394218805	0.000940265017561615 \\
0.920881992386684	0.000938737124670297 \\
0.921716310403637	0.000930368492845446 \\
0.921720421578646	0.000930328271351755 \\
0.923498857953819	0.000912726449314505 \\
0.923553028604257	0.000912195304408669 \\
0.924006080008399	0.000907764362636954 \\
0.92420639372191	0.000905812135897577 \\
0.924470669810756	0.000903242093045264 \\
0.924724469351036	0.000900780141819268 \\
0.926231702262142	0.000886290799826384 \\
0.926990949332903	0.000879076134879142 \\
0.933531060787661	0.000819192500784993 \\
0.934434631069947	0.000811230565886945 \\
0.934570551019964	0.000810039462521672 \\
0.935534942058006	0.000801634334493428 \\
0.936268298594132	0.000795298488810658 \\
0.936269242169234	0.000795290572568774 \\
0.938329910665083	0.000777741835918278 \\
0.94030185596857	0.000761295901611447 \\
0.940807389024333	0.000757133704610169 \\
0.941016354925932	0.000755419547203928 \\
0.942120305962364	0.000746425066608936 \\
0.944292950605489	0.000729020917788148 \\
0.944561491811014	0.000726897967979312 \\
0.944673077956372	0.000726016820408404 \\
0.945708241295843	0.0007178945816122 \\
0.947179815483775	0.000706498103681952 \\
0.947693224914561	0.000702562800142914 \\
0.950107910712063	0.000684335711412132 \\
0.953012070755755	0.000663015816826373 \\
0.953264426846652	0.000661193393170834 \\
0.953432175112961	0.000659984594676644 \\
0.954059232722306	0.000655485666356981 \\
0.954894864067499	0.000649536435957998 \\
0.955427443299792	0.00064577127341181 \\
0.958107244064232	0.000627146568149328 \\
0.95825981412274	0.000626101449597627 \\
0.961834544341432	0.000602100626565516 \\
0.962013308437074	0.000600924016907811 \\
0.962557230525089	0.000597357633523643 \\
0.964107437971182	0.000587305519729853 \\
0.96546229210349	0.000578654406126589 \\
0.966544858040119	0.000571830314584076 \\
0.966587760026518	0.000571561336982995 \\
0.966792601928706	0.00057027954608202 \\
0.968404850237238	0.00056028802646324 \\
0.968813010049556	0.000557785213459283 \\
0.969045613941432	0.000556363724172115 \\
0.969871084100612	0.00055134785361588 \\
0.970530837594512	0.000547371688298881 \\
0.971272888982341	0.000542931200470775 \\
0.971527190360033	0.000541417684871703 \\
0.977559419065177	0.000506694486830384 \\
0.978837143824273	0.000499622372444719 \\
0.979500241452307	0.000495990621857345 \\
0.979698366054342	0.000494910287670791 \\
0.981925106253984	0.000482925679534674 \\
0.98193841332347	0.000482855306472629 \\
0.985466980517735	0.000464444688986987 \\
0.985624420362111	0.000463639211375266 \\
0.985660230866112	0.000463456322904676 \\
0.985903288996567	0.000462216179585084 \\
0.986891736095765	0.000457206362625584 \\
0.988177000983844	0.000450771447503939 \\
0.988188066807458	0.000450716412160546 \\
0.988339255736614	0.000449965562438592 \\
0.988532835659511	0.000449005805421621 \\
0.988627149659547	0.000448539067292586 \\
0.991317832181511	0.000435416557593271 \\
0.991792763797732	0.000433139648521319 \\
0.99271508471038	0.00042875116923824 \\
0.992808132462744	0.000428311061114073 \\
0.99282017193431	0.000428254046710208 \\
0.994749345227338	0.000419224583311006 \\
0.99866338972896	0.000401475350372493 \\
};
\addplot [thick, color2]
table [row sep=\\]{%
0.0011059794751318	0.00105685205198824 \\
0.00144600828615515	0.00105765520129353 \\
0.00151620394328711	0.00105782132595778 \\
0.00161381928831428	0.00105805182829499 \\
0.00193644961641948	0.00105881586205214 \\
0.0024363370221504	0.00105999899096787 \\
0.00384167794354173	0.00106333335861564 \\
0.00421063900202368	0.00106421113014221 \\
0.00632372037002726	0.00106924877036363 \\
0.00947686783375368	0.00107680889777839 \\
0.0099356428556221	0.00107791239861399 \\
0.0101317191976503	0.00107838481198996 \\
0.0102519961026396	0.00107867480255663 \\
0.0116572936797861	0.00108206679578871 \\
0.0122763055741271	0.00108356459531933 \\
0.0128807282915644	0.0010850285179913 \\
0.0131006826676461	0.00108556193299592 \\
0.0132105085232722	0.00108582933899015 \\
0.0150477352877949	0.00109029468148947 \\
0.0151511241047927	0.0010905466042459 \\
0.0168553495563415	0.0010947062401101 \\
0.0191895991753259	0.0011004286352545 \\
0.0192724193499449	0.00110063224565238 \\
0.0202131188025382	0.00110294797923416 \\
0.0251886934986562	0.00111527252011001 \\
0.0255409812646521	0.00111614994239062 \\
0.0256196546755231	0.00111634656786919 \\
0.0263800727822662	0.00111824320629239 \\
0.026825697968402	0.00111935695167631 \\
0.0272388337061176	0.00112038967199624 \\
0.0276781997826389	0.00112148956395686 \\
0.0277969734855988	0.00112178770359606 \\
0.027888291941232	0.00112201599404216 \\
0.032801074197002	0.00113439105916768 \\
0.0337126623182875	0.00113670330028981 \\
0.0349109578825388	0.00113974779378623 \\
0.0367757145561887	0.00114450207911432 \\
0.0368887253476963	0.00114479009062052 \\
0.0376301111814882	0.00114668719470501 \\
0.0383020767002249	0.00114840746391565 \\
0.0385765124363792	0.00114911166019738 \\
0.0386281055032889	0.00114924379158765 \\
0.0397134557815922	0.0011520313564688 \\
0.0400530872540832	0.00115290505345911 \\
0.0408224228009035	0.00115488655865192 \\
0.042416614098386	0.00115900265518576 \\
0.042542997860721	0.00115932978224009 \\
0.0425634328976269	0.00115938298404217 \\
0.0426938593122377	0.00115971988998353 \\
0.0436353215690566	0.00116215890739113 \\
0.0476491609572692	0.00117261533159763 \\
0.0476764809468383	0.00117268715985119 \\
0.0480632979275277	0.00117369939107448 \\
0.0489335324060199	0.00117598020005971 \\
0.0494953541501341	0.0011774554150179 \\
0.0498656975684827	0.00117842899635434 \\
0.0507299142819697	0.00118070316966623 \\
0.0520149216066462	0.00118409295100719 \\
0.053491737186121	0.00118800019845366 \\
0.0545456118282759	0.00119079614523798 \\
0.0549368100196436	0.00119183561764657 \\
0.0549667939602374	0.00119191512931138 \\
0.0555159955251522	0.00119337625801563 \\
0.0574839546417889	0.00119862658903003 \\
0.0576927502423773	0.00119918468408287 \\
0.0580609228746025	0.00120016955770552 \\
0.0610070220733733	0.00120808102656156 \\
0.0613718310828713	0.00120906415395439 \\
0.0624819644863142	0.00121206138283014 \\
0.0632024977130776	0.00121400994248688 \\
0.0641707318537266	0.00121663371101022 \\
0.064803290117067	0.00121835118625313 \\
0.0661546388560771	0.00122202676720917 \\
0.0675945492445934	0.00122595636639744 \\
0.0701208744758773	0.00123287981841713 \\
0.0704495440664726	0.00123378401622176 \\
0.0707913481046274	0.00123472430277616 \\
0.0709503961374458	0.00123516155872494 \\
0.0723079949898875	0.00123890500981361 \\
0.0729582139062805	0.00124070164747536 \\
0.0736427592129116	0.00124259595759213 \\
0.0738431652021745	0.00124315090943128 \\
0.074094042941424	0.00124384637456387 \\
0.0741988699821797	0.00124413648154587 \\
0.0745344402647278	0.00124506803695112 \\
0.0761651718345726	0.00124960066750646 \\
0.0762565104011357	0.00124985468573868 \\
0.0764247618188199	0.00125032337382436 \\
0.0780473992355428	0.00125485274475068 \\
0.0784939963588281	0.00125610223039985 \\
0.0787288834363844	0.00125675951130688 \\
0.0796528614743472	0.00125934951938689 \\
0.0821291956626848	0.0012663162779063 \\
0.0836539892117674	0.00127062504179776 \\
0.0837976822096664	0.00127103214617819 \\
0.0840863722935196	0.00127184938173741 \\
0.0847664004367794	0.00127377721946687 \\
0.0848642352310873	0.00127405487000942 \\
0.0852150851090396	0.00127505080308765 \\
0.085891869345856	0.00127697410061955 \\
0.087068315817913	0.00128032476641238 \\
0.0885849846877879	0.00128465658053756 \\
0.0903431131516136	0.00128969585057348 \\
0.0909278592293694	0.00129137595649809 \\
0.0914296081009306	0.00129282032139599 \\
0.0922298300096035	0.00129512522835284 \\
0.0926517333242025	0.00129634293261915 \\
0.0930086769471365	0.001297373091802 \\
0.0936368650628759	0.00129918882157654 \\
0.0939832817677504	0.00130019115749747 \\
0.0963578674192209	0.00130708189681172 \\
0.0969604240071604	0.00130883697420359 \\
0.0974174263218397	0.00131016911473125 \\
0.0997464409700664	0.00131697778124362 \\
0.100249930959155	0.00131845462601632 \\
0.100820744388044	0.00132013054098934 \\
0.101229398304142	0.00132133101578802 \\
0.102271393895604	0.00132439867593348 \\
0.102325882258744	0.00132455909624696 \\
0.103474916599901	0.00132794980891049 \\
0.104369461810087	0.00133059569634497 \\
0.104849386693897	0.00133201701100916 \\
0.105274572558957	0.00133327767252922 \\
0.105437385176067	0.00133376044686884 \\
0.105718642159513	0.00133459561038762 \\
0.106451679001465	0.00133677339181304 \\
0.106754108203929	0.00133767316583544 \\
0.106889477031479	0.00133807584643364 \\
0.108308586301563	0.00134230661205947 \\
0.108355506097126	0.0013424470089376 \\
0.110864090628752	0.00134995835833251 \\
0.112062699234041	0.00135356141254306 \\
0.112686863552251	0.00135544105432928 \\
0.113787419472266	0.00135876226704568 \\
0.114114908535732	0.00135975214652717 \\
0.114926076298882	0.00136220688000321 \\
0.115710925433917	0.00136458582710475 \\
0.116193807044869	0.00136605114676058 \\
0.116352662276253	0.00136653380468488 \\
0.116979956596317	0.00136844126973301 \\
0.117247317403657	0.0013692551292479 \\
0.118793828877862	0.00137396983336657 \\
0.119352906137166	0.00137567787896842 \\
0.120388240876593	0.00137884786818177 \\
0.122357154089364	0.0013848936650902 \\
0.125712799722249	0.00139525812119246 \\
0.126988316690772	0.00139921752270311 \\
0.130289689058528	0.0014095155056566 \\
0.130830575131366	0.00141120981425047 \\
0.13178708265069	0.00141420948784798 \\
0.132217070962463	0.00141556106973439 \\
0.132641072460891	0.00141689472366124 \\
0.133694684584262	0.00142021279316396 \\
0.135586813424106	0.00142619130201638 \\
0.136970534315695	0.00143057783134282 \\
0.137167897059254	0.00143120542634279 \\
0.13731957841644	0.00143168703652918 \\
0.138770949832641	0.00143630709499121 \\
0.141579671494468	0.00144528620876372 \\
0.142337772861435	0.00144771952182055 \\
0.142852169410084	0.00144937261939049 \\
0.14313843453823	0.00145029334817082 \\
0.143709892765619	0.00145213340874761 \\
0.143996587198751	0.00145305646583438 \\
0.144338718134467	0.0014541600830853 \\
0.145474509131844	0.00145782809704542 \\
0.146058924442183	0.0014597192639485 \\
0.147496370200373	0.00146437960211188 \\
0.147934500473159	0.00146580336149782 \\
0.148464696403164	0.00146752770524472 \\
0.149007159183894	0.00146929419133812 \\
0.151290695331351	0.00147675152402371 \\
0.153507740604538	0.0014840264339 \\
0.153927498261582	0.00148540723603219 \\
0.153953160827713	0.00148549198638648 \\
0.154179436777791	0.00148623716086149 \\
0.155107736993068	0.00148929795250297 \\
0.15519005879533	0.00148956966586411 \\
0.155418402417591	0.00149032368790358 \\
0.156078407887877	0.00149250496178865 \\
0.158100170599977	0.00149920454714447 \\
0.158283533549941	0.00149981398135424 \\
0.158599777942073	0.00150086497887969 \\
0.159258263288539	0.00150305626448244 \\
0.159678910282026	0.00150445743929595 \\
0.162540881855109	0.00151402258779854 \\
0.164077384668961	0.00151918199844658 \\
0.164957348092599	0.00152214395347983 \\
0.165030574098427	0.00152239156886935 \\
0.165280388241905	0.0015232334844768 \\
0.167273692684132	0.00152996811084449 \\
0.168043854724866	0.00153257849160582 \\
0.168088628486329	0.0015327304136008 \\
0.168662857989897	0.00153467967174947 \\
0.169151147716595	0.00153633847367018 \\
0.169673214248392	0.00153811404015869 \\
0.170135609058708	0.00153968785889447 \\
0.171767974673073	0.00154525868128985 \\
0.1728518498484	0.00154896709136665 \\
0.172984599941167	0.00154942204244435 \\
0.176370551959836	0.00156106636859477 \\
0.176588273575192	0.00156181841157377 \\
0.176656553576852	0.00156205368693918 \\
0.17794804219768	0.00156652007717639 \\
0.178429717033185	0.00156818877439946 \\
0.179458814233081	0.00157175899948925 \\
0.180595104426831	0.00157571060117334 \\
0.18064479669491	0.00157588324509561 \\
0.182294707450728	0.00158163858577609 \\
0.182819109355123	0.00158347166143358 \\
0.189141614514094	0.00160573213361204 \\
0.190723527970293	0.00161134777590632 \\
0.192508826701743	0.00161770638078451 \\
0.200712508340217	0.00164722977206111 \\
0.202130410523186	0.00165238371118903 \\
0.202705868873484	0.00165447941981256 \\
0.202975660615516	0.00165546336211264 \\
0.2033001668374	0.00165664649102837 \\
0.204138047670012	0.00165970576927066 \\
0.204584283026216	0.00166133814491332 \\
0.205975712225296	0.00166643445845693 \\
0.208693969096644	0.00167643383610994 \\
0.209097012308874	0.00167792057618499 \\
0.210129299396579	0.00168173597194254 \\
0.210470524326312	0.00168299826327711 \\
0.210726758187617	0.00168394716456532 \\
0.210879765452585	0.00168451329227537 \\
0.213404555935633	0.00169389101210982 \\
0.214779680955682	0.00169901933986694 \\
0.215976334819965	0.0017034929478541 \\
0.21605980843853	0.00170380587223917 \\
0.21626681121775	0.00170458061620593 \\
0.217333435243771	0.00170857959892601 \\
0.217860184045886	0.00171055807732046 \\
0.218810300034316	0.00171413132920861 \\
0.219036830922259	0.00171498500276357 \\
0.219186045259484	0.0017155462410301 \\
0.219509161852637	0.00171676347963512 \\
0.219517283100912	0.00171679514460266 \\
0.220007941128277	0.00171864533331245 \\
0.221124031401732	0.00172286189626902 \\
0.221655686766872	0.00172487320378423 \\
0.222787328400317	0.00172916240990162 \\
0.222812240344664	0.00172925658989698 \\
0.224413114839207	0.00173534196801484 \\
0.225712117798195	0.0017402924131602 \\
0.226064824403414	0.00174163887277246 \\
0.22778297811216	0.00174821226391941 \\
0.227786137279108	0.00174822460394353 \\
0.228255816092263	0.00175002519972622 \\
0.22886379454373	0.00175235792994499 \\
0.229286896531361	0.00175398425199091 \\
0.231570483736477	0.00176278059370816 \\
0.232839821990075	0.00176768703386188 \\
0.234699792231533	0.00177489977795631 \\
0.236598748956898	0.00178228970617056 \\
0.236914864918852	0.00178352172952145 \\
0.238193170258512	0.0017885152483359 \\
0.238267801131912	0.00178880780003965 \\
0.241280255564163	0.0018006251193583 \\
0.241532477911657	0.00180161790922284 \\
0.242724887440966	0.00180631689727306 \\
0.243487528482621	0.00180932797957212 \\
0.243851755033328	0.00181076745502651 \\
0.243931127736832	0.00181108200922608 \\
0.245492487219521	0.00181726610753685 \\
0.245961870318215	0.00181912817060947 \\
0.247220233087027	0.00182412914000452 \\
0.248029547774932	0.00182735349517316 \\
0.248195183805212	0.00182801380287856 \\
0.252597476547261	0.00184563815128058 \\
0.254049623190913	0.00185148394666612 \\
0.254469169693275	0.00185317604336888 \\
0.254777550536903	0.0018544199410826 \\
0.254877558399671	0.00185482483357191 \\
0.257253649731868	0.00186444027349353 \\
0.257671137054733	0.00186613423284143 \\
0.26187840669548	0.0018832755740732 \\
0.264182169008257	0.00189271988347173 \\
0.265462287195374	0.0018979839514941 \\
0.267638564660905	0.00190696236677468 \\
0.268216851612881	0.00190935470163822 \\
0.269980014659659	0.00191666255705059 \\
0.270837940398052	0.001920226495713 \\
0.271133018969332	0.00192145386245102 \\
0.271522794293612	0.00192307541146874 \\
0.272730212426688	0.001928107929416 \\
0.27325151773824	0.00193028338253498 \\
0.273458501731612	0.00193114811554551 \\
0.275787496734588	0.00194089638534933 \\
0.275790346826632	0.00194090895820409 \\
0.275963502468307	0.00194163527339697 \\
0.276441724078765	0.00194364308845252 \\
0.277788622824976	0.00194930518046021 \\
0.278249554181787	0.00195124675519764 \\
0.278902433397602	0.00195399881340563 \\
0.280228184178203	0.00195959606207907 \\
0.282170737525181	0.00196782336570323 \\
0.284775883532131	0.00197889772243798 \\
0.285319633849948	0.00198121648281813 \\
0.285945143849347	0.00198388448916376 \\
0.287293498153222	0.00198964891023934 \\
0.288098109983407	0.0019930936396122 \\
0.288474372887691	0.00199470692314208 \\
0.288733693686359	0.00199581868946552 \\
0.291563977198166	0.00200798874720931 \\
0.293976605993451	0.00201840815134346 \\
0.294014528290392	0.0020185720641166 \\
0.294718175832172	0.00202161865308881 \\
0.294831463459233	0.00202210992574692 \\
0.294887681144015	0.00202235416509211 \\
0.299064488228799	0.00204051751643419 \\
0.299223468133874	0.00204121018759906 \\
0.299853491506122	0.00204396131448448 \\
0.301624334890634	0.0020517089869827 \\
0.302100714632492	0.00205379817634821 \\
0.302855829384862	0.00205711089074612 \\
0.303468685820656	0.00205980241298676 \\
0.304335545384868	0.00206361408345401 \\
0.305081848845274	0.00206689932383597 \\
0.305263786725066	0.00206770119257271 \\
0.305511643802704	0.00206879316829145 \\
0.30635774546767	0.0020725263748318 \\
0.307937429467832	0.00207950780168176 \\
0.308578187230734	0.00208234507590532 \\
0.313991998160918	0.00210642185993493 \\
0.315973848719028	0.00211528432555497 \\
0.317413649622303	0.00212173978798091 \\
0.319657848506035	0.00213182810693979 \\
0.319781920771261	0.00213238620199263 \\
0.320496004308145	0.00213560345582664 \\
0.320601310305949	0.00213607912883162 \\
0.322246867517597	0.00214350502938032 \\
0.323306124649751	0.00214829598553479 \\
0.323664358184811	0.00214991741813719 \\
0.323805398711091	0.00215055607259274 \\
0.324816846999054	0.00215514004230499 \\
0.325263683488062	0.002157166833058 \\
0.327533460856979	0.00216748239472508 \\
0.328385869951453	0.00217136414721608 \\
0.329844054701458	0.0021780151873827 \\
0.33074904970515	0.00218214886263013 \\
0.332974695502457	0.00219233683310449 \\
0.333600133199531	0.00219520507380366 \\
0.336001356191777	0.00220623798668385 \\
0.338057278352338	0.00221570977009833 \\
0.33940563378102	0.0022219349630177 \\
0.339522610836627	0.0022224741987884 \\
0.341295751839904	0.00223067682236433 \\
0.343025583329651	0.00223869481123984 \\
0.344156227439929	0.00224394327960908 \\
0.344485478041551	0.0022454746067524 \\
0.346631556569637	0.00225545861758292 \\
0.348247004908186	0.00226299022324383 \\
0.348257869791207	0.00226304004900157 \\
0.350134953366802	0.00227180775254965 \\
0.352271571140202	0.00228180876001716 \\
0.352403536611892	0.00228242599405348 \\
0.356607161458692	0.00230216304771602 \\
0.357473355861859	0.00230624037794769 \\
0.359032837138255	0.00231358781456947 \\
0.360085447455855	0.00231855432502925 \\
0.360587038984981	0.00232091988436878 \\
0.3618814720936	0.00232703401707113 \\
0.362541325151431	0.00233015418052673 \\
0.363971114014582	0.00233691837638617 \\
0.36446377318254	0.00233925110660493 \\
0.364734654416428	0.00234053330495954 \\
0.364782079325344	0.00234075821936131 \\
0.36626622804169	0.00234779203310609 \\
0.36745676602671	0.00235344003885984 \\
0.367555756327517	0.00235391058959067 \\
0.367766921022286	0.00235491245985031 \\
0.368015883464081	0.00235609407536685 \\
0.368183587302196	0.00235689058899879 \\
0.368646745255496	0.00235909014008939 \\
0.368824340831231	0.00235993228852749 \\
0.36908666014638	0.00236117979511619 \\
0.370227788452905	0.00236660335212946 \\
0.371079364739364	0.00237065320834517 \\
0.371457393668236	0.00237245159223676 \\
0.372521553247257	0.00237751635722816 \\
0.372816229090882	0.00237892102450132 \\
0.373696444358129	0.00238311267457902 \\
0.375236418301489	0.00239045359194279 \\
0.375355278624935	0.00239102030172944 \\
0.375592850161798	0.0023921534884721 \\
0.375621310337888	0.002392289461568 \\
0.37623340048021	0.00239520845934749 \\
0.376649481158299	0.00239719380624592 \\
0.37682934901581	0.00239805201999843 \\
0.377022374188538	0.00239897286519408 \\
0.378777395062222	0.00240735313855112 \\
0.379219181502585	0.00240946374833584 \\
0.380639994297166	0.00241625402122736 \\
0.381635433931239	0.00242101307958364 \\
0.382416367624347	0.00242474745027721 \\
0.383276747323035	0.00242886343039572 \\
0.383441774868681	0.00242965389043093 \\
0.384589480637987	0.00243514706380665 \\
0.38762147240542	0.00244966498576105 \\
0.387652450706726	0.00244981376454234 \\
0.38818745089578	0.00245237722992897 \\
0.388531576179462	0.0024540254380554 \\
0.391108514236782	0.00246637337841094 \\
0.391741157295481	0.00246940576471388 \\
0.396723908263121	0.00249328999780118 \\
0.39803082540264	0.00249955290928483 \\
0.398171926406937	0.00250022951513529 \\
0.398666017137866	0.00250259600579739 \\
0.398819673578751	0.00250333268195391 \\
0.4012576168302	0.00251501216553152 \\
0.402752737045505	0.00252217170782387 \\
0.403290070599278	0.00252474448643625 \\
0.404563584871068	0.00253083859570324 \\
0.406532702382757	0.0025402563624084 \\
0.406578018270242	0.002540472894907 \\
0.413273296435021	0.00257242564111948 \\
0.413592377461711	0.00257394532673061 \\
0.41401191876112	0.00257594231516123 \\
0.414776267969876	0.00257958029396832 \\
0.414785799829368	0.00257962569594383 \\
0.415238430177184	0.00258177821524441 \\
0.416465974154779	0.00258761341683567 \\
0.416689013292801	0.00258867349475622 \\
0.418297500020148	0.00259630708023906 \\
0.422720878338663	0.0026172399520874 \\
0.424343600824592	0.00262489262968302 \\
0.425257001696199	0.0026291951071471 \\
0.42710227803989	0.00263786781579256 \\
0.427962517509508	0.00264190416783094 \\
0.429309169026115	0.00264821364544332 \\
0.429443446115563	0.00264884065836668 \\
0.429705476003957	0.00265006581321359 \\
0.431003390699289	0.00265613035298884 \\
0.432539042653788	0.00266328919678926 \\
0.433068569337496	0.00266575277782977 \\
0.433545956268458	0.00266797258518636 \\
0.433750136573834	0.00266892113722861 \\
0.433778478107147	0.00266905361786485 \\
0.434834104829426	0.00267395144328475 \\
0.435898870411909	0.00267888349480927 \\
0.43750740037395	0.00268631731159985 \\
0.437737100851006	0.00268737482838333 \\
0.439197385289126	0.00269409827888012 \\
0.440921894924801	0.0027020110283047 \\
0.442219378180834	0.00270794099196792 \\
0.446052885401706	0.00272536231204867 \\
0.447759455363191	0.00273306155577302 \\
0.448771556188603	0.00273761060088873 \\
0.448918332177756	0.00273826997727156 \\
0.449861491210787	0.00274249538779259 \\
0.449976789412666	0.00274301110766828 \\
0.450129821782654	0.00274369516409934 \\
0.453058505166632	0.00275673158466816 \\
0.455515296612047	0.00276757054962218 \\
0.461612686527044	0.00279406784102321 \\
0.462183229417738	0.00279651675373316 \\
0.46315332653893	0.00280066393315792 \\
0.463567773958142	0.00280243181623518 \\
0.464569005654222	0.00280668772757053 \\
0.46546966357775	0.00281050032936037 \\
0.465627783083443	0.00281116762198508 \\
0.466166033904132	0.00281343748793006 \\
0.466888847202291	0.00281647616066039 \\
0.468332203161028	0.00282251345925033 \\
0.469094139229819	0.00282568414695561 \\
0.469464236056835	0.002827218035236 \\
0.469806075154756	0.00282863434404135 \\
0.470776441432414	0.00283264182507992 \\
0.471475988022769	0.00283551681786776 \\
0.473199954060133	0.00284255715087056 \\
0.474832247453521	0.00284916302189231 \\
0.475230428162318	0.00285076419822872 \\
0.476638306671274	0.00285639939829707 \\
0.478988899075304	0.00286570028401911 \\
0.479018256118887	0.0028658164665103 \\
0.479279919995254	0.00286684301681817 \\
0.479532435695615	0.00286783231422305 \\
0.479763011578347	0.00286873313598335 \\
0.480041882522935	0.00286982348188758 \\
0.480923824850651	0.00287325470708311 \\
0.481629462985493	0.00287598418071866 \\
0.481771508198592	0.00287653226405382 \\
0.482503736852496	0.00287934998050332 \\
0.483064995688331	0.0028815008699894 \\
0.48384062770339	0.00288445688784122 \\
0.484085120492796	0.00288538518361747 \\
0.484352733912412	0.00288640148937702 \\
0.484550939846654	0.00288715027272701 \\
0.485022749278845	0.00288893119432032 \\
0.485322068953861	0.00289005832746625 \\
0.48610300375999	0.00289298663847148 \\
0.486253849387163	0.0028935510199517 \\
0.489154935058063	0.00290426076389849 \\
0.489792628987292	0.0029065830167383 \\
0.490458183784177	0.00290899071842432 \\
0.492282643359321	0.00291552604176104 \\
0.493928358105999	0.0029213298112154 \\
0.495701335790147	0.00292748329229653 \\
0.496106252426634	0.00292887166142464 \\
0.496164487189195	0.00292907259427011 \\
0.498739819509518	0.00293778022751212 \\
0.500164108812496	0.0029424971435219 \\
0.500739446514191	0.00294438167475164 \\
0.500830786926564	0.00294467876665294 \\
0.502541719930356	0.00295020639896393 \\
0.503498564874392	0.00295324786566198 \\
0.504534081188611	0.00295650237239897 \\
0.504637823699037	0.00295682693831623 \\
0.506860898199194	0.00296366168186069 \\
0.507491092240963	0.00296556367538869 \\
0.508120202534381	0.00296744517982006 \\
0.510856303638406	0.00297544617205858 \\
0.512327455201249	0.00297961407341063 \\
0.51318641739938	0.00298200617544353 \\
0.513457954653542	0.00298275449313223 \\
0.515863112336536	0.00298924813978374 \\
0.516616766484905	0.00299122766591609 \\
0.517831355926389	0.00299436552450061 \\
0.518931921097757	0.00299714645370841 \\
0.521152065672689	0.00300258048810065 \\
0.524519468497769	0.00301036168821156 \\
0.524576690058109	0.00301048718392849 \\
0.525883589956303	0.00301334774121642 \\
0.526973717032038	0.00301566487178206 \\
0.528680408273136	0.00301916431635618 \\
0.530600365407266	0.00302291405387223 \\
0.530809662396022	0.00302331056445837 \\
0.53166475254574	0.00302490475587547 \\
0.532158485575632	0.00302580394782126 \\
0.535204926183725	0.00303105916827917 \\
0.535356594523575	0.0030313057359308 \\
0.537004451507049	0.00303390785120428 \\
0.537089668720381	0.00303403823636472 \\
0.537190935482585	0.00303419143892825 \\
0.538117362047647	0.00303557026199996 \\
0.538453506118723	0.00303605967201293 \\
0.539359907716354	0.00303734093904495 \\
0.539560791003075	0.0030376180075109 \\
0.541180326502456	0.00303976098075509 \\
0.547271589683928	0.00304631376639009 \\
0.548822201570025	0.00304758897982538 \\
0.549624872437239	0.00304818502627313 \\
0.549819206435189	0.00304832169786096 \\
0.550184928091146	0.0030485731549561 \\
0.552139674398207	0.00304976268671453 \\
0.552982438656419	0.00305019016377628 \\
0.5537594353206	0.00305053894408047 \\
0.553976139314498	0.003050631377846 \\
0.554051563436848	0.00305066001601517 \\
0.55426017856816	0.00305074290372431 \\
0.555396555661942	0.00305113545618951 \\
0.555501425327998	0.00305116758681834 \\
0.556360039382527	0.00305139389820397 \\
0.557883446271028	0.00305166514590383 \\
0.558670348456701	0.00305173662491143 \\
0.559194391744592	0.0030517578125 \\
0.559473603788983	0.00305176014080644 \\
0.560198442614008	0.0030517412815243 \\
0.560853933269278	0.00305168866179883 \\
0.561152915237902	0.00305165420286357 \\
0.561401591015197	0.00305161997675896 \\
0.562479951006079	0.00305141694843769 \\
0.563005908106818	0.00305128493346274 \\
0.563050000205171	0.00305127236060798 \\
0.563426857102662	0.00305116176605225 \\
0.563560604448239	0.00305112102068961 \\
0.564062038149409	0.00305095221847296 \\
0.565336310962797	0.00305043696425855 \\
0.565855772748177	0.00305018853396177 \\
0.566858287901599	0.00304964696988463 \\
0.566881879040954	0.00304963509552181 \\
0.568231608554438	0.00304877571761608 \\
0.569223717564581	0.00304805091582239 \\
0.569387586449175	0.00304792262613773 \\
0.569573989003313	0.00304777501150966 \\
0.569720721150053	0.00304765626788139 \\
0.572004618116245	0.00304558221250772 \\
0.572005154945678	0.00304558058269322 \\
0.572151528516064	0.00304543366655707 \\
0.57219889974288	0.00304538453929126 \\
0.572780347928331	0.00304477754980326 \\
0.573600339080632	0.00304387230426073 \\
0.576593284261927	0.00304007809609175 \\
0.576676798911885	0.00303996028378606 \\
0.576819659872325	0.00303975865244865 \\
0.577653861143233	0.00303854653611779 \\
0.577917937736817	0.00303814886137843 \\
0.57793430586259	0.00303812511265278 \\
0.581759386894494	0.00303168222308159 \\
0.582475032298807	0.00303033250384033 \\
0.582896249817215	0.00302951596677303 \\
0.589571735768374	0.00301440083421767 \\
0.592388904473667	0.00300677353516221 \\
0.593611787705259	0.00300322752445936 \\
0.594151694891475	0.00300161470659077 \\
0.595008948927732	0.0029989997856319 \\
0.596105779635777	0.00299554993398488 \\
0.597490139381406	0.00299102906137705 \\
0.598951438337811	0.00298605393618345 \\
0.599201054717089	0.00298518245108426 \\
0.600228747033593	0.00298153469339013 \\
0.600693928524281	0.00297984899953008 \\
0.601589358474053	0.00297654373571277 \\
0.602529748251813	0.00297298864461482 \\
0.602916831603936	0.00297149899415672 \\
0.603866403426713	0.00296778394840658 \\
0.606129689323002	0.00295856385491788 \\
0.607808363354841	0.00295139686204493 \\
0.60811641039762	0.00295005110092461 \\
0.608424173673929	0.00294869602657855 \\
0.610203299896366	0.00294068013317883 \\
0.61052401101773	0.00293920235708356 \\
0.611007548691541	0.00293695321306586 \\
0.61345268097321	0.00292521854862571 \\
0.614300601551049	0.00292100757360458 \\
0.616415105925723	0.00291019282303751 \\
0.617595625529809	0.00290395575575531 \\
0.619695383178102	0.00289251655340195 \\
0.620626131680421	0.00288730184547603 \\
0.620884093030541	0.00288584106601775 \\
0.622959094549987	0.00287384795956314 \\
0.624820249131202	0.00286272051744163 \\
0.627396892203872	0.00284673878923059 \\
0.629679734717881	0.00283202272839844 \\
0.63108091324942	0.00282273069024086 \\
0.631412061821856	0.0028205078560859 \\
0.632209138475077	0.00281510944478214 \\
0.632608841524959	0.00281237810850143 \\
0.635174872896091	0.00279447017237544 \\
0.63680487509045	0.00278275879099965 \\
0.638872902814914	0.00276752514764667 \\
0.639214862746441	0.00276496703736484 \\
0.640549343973947	0.00275487592443824 \\
0.6407383188268	0.00275343214161694 \\
0.642765742598199	0.00273773446679115 \\
0.642772962045468	0.00273767928592861 \\
0.644393148200581	0.002724853111431 \\
0.646791193304363	0.0027054192032665 \\
0.647126395490269	0.002702662255615 \\
0.647747012654458	0.00269752671010792 \\
0.648487744871255	0.00269135204143822 \\
0.65164263639371	0.00266449083574116 \\
0.651778799711512	0.00266331201419234 \\
0.652422089544704	0.00265771779231727 \\
0.652778066297422	0.00265460670925677 \\
0.653037870316759	0.00265232799574733 \\
0.653946378314801	0.00264431699179113 \\
0.656181358570071	0.00262430170550942 \\
0.657515298781176	0.00261215167120099 \\
0.657916355325804	0.0026084694545716 \\
0.65821939898974	0.00260567781515419 \\
0.660347901626978	0.00258585740812123 \\
0.663462409580861	0.00255618756636977 \\
0.665002009135555	0.00254123797640204 \\
0.665125585501828	0.00254002865403891 \\
0.665840946327609	0.00253301369957626 \\
0.669598702136104	0.00249552680179477 \\
0.669671107753933	0.00249479315243661 \\
0.670317902411108	0.00248823361471295 \\
0.67055566811489	0.00248581380583346 \\
0.670890712943537	0.0024823984131217 \\
0.671190867447461	0.00247933063656092 \\
0.672572121540986	0.00246513541787863 \\
0.674425718831603	0.00244587939232588 \\
0.67545414417829	0.00243509584106505 \\
0.67596832728122	0.00242967810481787 \\
0.676862814397186	0.00242021284066141 \\
0.677008756097011	0.00241866358555853 \\
0.677760880297266	0.00241065723821521 \\
0.680496299332769	0.00238124653697014 \\
0.6805627271546	0.0023805252276361 \\
0.682518497342478	0.00235921307466924 \\
0.682667061493577	0.00235758512280881 \\
0.683323391450582	0.00235037854872644 \\
0.683455311069743	0.00234892801381648 \\
0.686099541255968	0.00231963279657066 \\
0.686972247003404	0.00230988231487572 \\
0.687287921742946	0.00230634585022926 \\
0.688979470872378	0.00228731194511056 \\
0.689844777380402	0.00227751978673041 \\
0.69066753173656	0.0022681774571538 \\
0.690924484351316	0.00226525333710015 \\
0.691684985973077	0.00225658062845469 \\
0.694075688265434	0.00222915131598711 \\
0.694253786667502	0.00222709774971008 \\
0.696056771003981	0.00220624031499028 \\
0.696659969942023	0.00219923374243081 \\
0.69712902782289	0.00219377665780485 \\
0.697436633040946	0.00219019292853773 \\
0.697470314030711	0.00218979944474995 \\
0.698295232645012	0.00218017096631229 \\
0.698478216988194	0.00217803241685033 \\
0.699563853841568	0.00216531660407782 \\
0.700079752305739	0.00215925951488316 \\
0.700291212685059	0.00215677381493151 \\
0.701162742599685	0.00214651692658663 \\
0.702327231437584	0.00213277246803045 \\
0.702639966218633	0.00212907535023987 \\
0.703077231783761	0.0021238992922008 \\
0.703198302674555	0.00212246598675847 \\
0.703418311750763	0.00211985898204148 \\
0.704913184821873	0.00210211169905961 \\
0.705985343895602	0.0020893441978842 \\
0.70676366586065	0.00208005914464593 \\
0.70953480835097	0.00204688799567521 \\
0.709597763991885	0.00204613199457526 \\
0.710916839332999	0.00203028600662947 \\
0.710971707887323	0.002029626397416 \\
0.711527088935361	0.00202294276095927 \\
0.711625741549748	0.00202175485901535 \\
0.712030078150365	0.00201688753440976 \\
0.713838731517111	0.00199507595971227 \\
0.71495171630821	0.00198163231834769 \\
0.715877253359645	0.00197044014930725 \\
0.717907954554698	0.00194585171993822 \\
0.718267373832691	0.00194149662274867 \\
0.719482266109768	0.0019267670577392 \\
0.720072506203388	0.00191960693337023 \\
0.72063303964824	0.00191280548460782 \\
0.721414113641739	0.00190332753118128 \\
0.721834558021562	0.00189822260290384 \\
0.721858215436245	0.00189793587196618 \\
0.722606600061919	0.00188885000534356 \\
0.723634632256085	0.00187636737246066 \\
0.724295951475884	0.00186833832412958 \\
0.724780388322819	0.0018624565564096 \\
0.725967182906895	0.00184804701711982 \\
0.728562998007155	0.00181654584594071 \\
0.729378939538106	0.0018066520569846 \\
0.729672766087416	0.00180309033021331 \\
0.730156271089799	0.00179723091423512 \\
0.730418624607748	0.00179405161179602 \\
0.731079272213317	0.00178604887332767 \\
0.73140958111192	0.00178204965777695 \\
0.734582373950397	0.00174369174055755 \\
0.735036686489948	0.00173820927739143 \\
0.736495162798249	0.00172062928322703 \\
0.737637169481189	0.00170688680373132 \\
0.738132551177952	0.00170093262568116 \\
0.738772404713377	0.00169324735179543 \\
0.73919080617973	0.00168822694104165 \\
0.739634766317876	0.00168290291912854 \\
0.739671405613807	0.00168246345128864 \\
0.740318229014976	0.00167471379972994 \\
0.741017797721675	0.00166634214110672 \\
0.741923938399663	0.00165551353711635 \\
0.742870439596604	0.0016442216001451 \\
0.743500813750507	0.00163671234622598 \\
0.745235425982657	0.00161609926726669 \\
0.745794924415534	0.00160946580581367 \\
0.746651135832285	0.00159933185204864 \\
0.748242417565033	0.00158054963685572 \\
0.749080152481028	0.00157068972475827 \\
0.750738597109713	0.0015512335812673 \\
0.754059260293618	0.00151253759395331 \\
0.754427277039572	0.00150827190373093 \\
0.755971311279052	0.00149042520206422 \\
0.7564660147832	0.00148472480941564 \\
0.756964107658398	0.00147899566218257 \\
0.757301980310619	0.001475112978369 \\
0.757561417579673	0.00147213519085199 \\
0.759460973532791	0.00145041022915393 \\
0.759943823499425	0.00144490960519761 \\
0.76007481161266	0.00144341879058629 \\
0.76520949758525	0.00138554093427956 \\
0.769369251223668	0.00133948505390435 \\
0.77094746139779	0.00132221751846373 \\
0.771797041083953	0.00131297030020505 \\
0.772164013593626	0.00130898761563003 \\
0.772721858147557	0.00130294426344335 \\
0.774553209660584	0.0012832114007324 \\
0.774686774792601	0.0012817793758586 \\
0.774763255366	0.00128095981199294 \\
0.776676196809448	0.00126054487191141 \\
0.781826328102722	0.00120651291217655 \\
0.783936140597273	0.00118478026706725 \\
0.784010287514813	0.00118402065709233 \\
0.784889735175701	0.0011750360717997 \\
0.785018616770813	0.00117372279055417 \\
0.78602927274531	0.00116345728747547 \\
0.786818186718577	0.00115548237226903 \\
0.787974469502378	0.00114385574124753 \\
0.791139177359944	0.00111241592094302 \\
0.791586708117001	0.00110801542177796 \\
0.791796196079504	0.00110595941077918 \\
0.79182158322754	0.00110571016557515 \\
0.792072748806228	0.00110324844717979 \\
0.792682852079258	0.00109728530514985 \\
0.792705401732446	0.00109706481453031 \\
0.794569839509874	0.00107897340785712 \\
0.795023193386171	0.00107460410799831 \\
0.795367478837485	0.00107129430398345 \\
0.795527938976809	0.00106975412927568 \\
0.795634872431087	0.00106872851029038 \\
0.79745955888423	0.00105132895987481 \\
0.798224122257701	0.00104409514460713 \\
0.799881016007139	0.00102853681892157 \\
0.801765379467602	0.00101103784982115 \\
0.80362962295172	0.00099393236450851 \\
0.803814553638313	0.000992246670648456 \\
0.806469107208753	0.000968273845501244 \\
0.806697504962046	0.000966230756603181 \\
0.807337366299355	0.000960523495450616 \\
0.808888043138527	0.000946793472394347 \\
0.809190014741277	0.000944136059843004 \\
0.809307640373056	0.000943102757446468 \\
0.810501764417227	0.000932657858356833 \\
0.811889729396757	0.000920624530408531 \\
0.811942350832027	0.000920170219615102 \\
0.814196326909986	0.000900881073903292 \\
0.815151250895423	0.000892800919245929 \\
0.815544311745005	0.000889490998815745 \\
0.816375455024602	0.000882522319443524 \\
0.818069790025891	0.000868443981744349 \\
0.820698196177955	0.000846945389639586 \\
0.821263012964588	0.000842379056848586 \\
0.823441468430582	0.000824946968350559 \\
0.824103550036923	0.000819704844616354 \\
0.825253713607837	0.000810660072602332 \\
0.827170035611177	0.000795765896327794 \\
0.827687372625058	0.000791781640145928 \\
0.828681002627947	0.000784175237640738 \\
0.831688136352904	0.000761507311835885 \\
0.832015103492291	0.000759074930101633 \\
0.832618539764885	0.000754601729568094 \\
0.833979270360146	0.000744593213312328 \\
0.835071309810646	0.000736638903617859 \\
0.835699577265585	0.00073209434049204 \\
0.836498037763016	0.000726351456250995 \\
0.836836957998254	0.000723924662452191 \\
0.838046127292981	0.00071532151196152 \\
0.838525323246066	0.000711935805156827 \\
0.839239765753767	0.000706910970620811 \\
0.839402685897079	0.000705769751220942 \\
0.839999299682593	0.000701602082699537 \\
0.840281951751549	0.000699635071214288 \\
0.840483203711437	0.000698237330652773 \\
0.840691788412589	0.000696790404617786 \\
0.842685722208791	0.000683088670484722 \\
0.842993851211431	0.00068099150666967 \\
0.844759374999568	0.000669077620841563 \\
0.844835827371022	0.000668565975502133 \\
0.844992041004939	0.000667520798742771 \\
0.84621071404422	0.000659415265545249 \\
0.84647818555566	0.000657647557090968 \\
0.847551016932997	0.000650596455670893 \\
0.848091115090324	0.000647071225102991 \\
0.84863665263528	0.000643526553176343 \\
0.85175414414149	0.00062358524883166 \\
0.852055625215632	0.000621684943325818 \\
0.853183518340707	0.000614619464613497 \\
0.855762763379044	0.000598720158450305 \\
0.855845256518328	0.000598217651713639 \\
0.856101518575613	0.000596658675931394 \\
0.856353280691239	0.000595130492001772 \\
0.857431392977156	0.000588623690418899 \\
0.858507268631162	0.000582192500587553 \\
0.858727032795181	0.000580886146053672 \\
0.859611972462117	0.000575652171391994 \\
0.860598063349272	0.000569867785088718 \\
0.861479766548793	0.00056473899167031 \\
0.863021652496581	0.000555865932255983 \\
0.863191163941701	0.000554898113477975 \\
0.863243196088782	0.000554601312614977 \\
0.863264684077014	0.000554478901904076 \\
0.86433033865504	0.000548430951312184 \\
0.865386426798682	0.000542494934052229 \\
0.865574805896198	0.00054144230671227 \\
0.867009964664233	0.000533479324076325 \\
0.868402477420643	0.000525851442944258 \\
0.868462536448662	0.000525524606928229 \\
0.8687736544668	0.000523835071362555 \\
0.870769922748437	0.000513105187565088 \\
0.87137568194098	0.000509888050146401 \\
0.871475162483056	0.000509361270815134 \\
0.872823623463892	0.000502270762808621 \\
0.873231322955811	0.000500143854878843 \\
0.87330913025223	0.000499738962389529 \\
0.874554061243274	0.000493299856316298 \\
0.874752616263773	0.000492279650643468 \\
0.875835647990412	0.000486748322146013 \\
0.878375805455995	0.000473993248306215 \\
0.879016462457315	0.000470823957584798 \\
0.879915986267341	0.000466406083432958 \\
0.880187041340505	0.000465082615846768 \\
0.880228906915535	0.000464878161437809 \\
0.880706157108953	0.000462556548882276 \\
0.881313907623959	0.000459614850115031 \\
0.88357711634013	0.00044880885980092 \\
0.88575628070969	0.000438622169895098 \\
0.885860150516388	0.000438141723861918 \\
0.885973756819001	0.000437616952694952 \\
0.886578379643548	0.000434833928011358 \\
0.88662976023068	0.000434598012361676 \\
0.887201064437371	0.000431984430179 \\
0.88721221948908	0.000431933614891022 \\
0.889489342121342	0.000421658391132951 \\
0.889610563902722	0.000421117583755404 \\
0.891226627532086	0.000413969799410552 \\
0.891601333881282	0.000412328488891944 \\
0.894395711960869	0.000400274526327848 \\
0.894941283658575	0.000397958938265219 \\
0.895002834569702	0.000397698400774971 \\
0.895316145165804	0.000396375282434747 \\
0.89572498422586	0.000394654431147501 \\
0.900281097465164	0.000375939591322094 \\
0.901776757070804	0.00036997688584961 \\
0.902902077475763	0.000365548417903483 \\
0.903042041253622	0.000365001149475574 \\
0.903417484940991	0.000363536761142313 \\
0.905006674071476	0.000357398035703227 \\
0.905428107857581	0.000355786411091685 \\
0.90623158203924	0.00035273254616186 \\
0.909863735264334	0.000339229562086985 \\
0.914192347433306	0.000323768297675997 \\
0.914860066850241	0.000321442726999521 \\
0.916826528028795	0.000314684875775129 \\
0.918089828672593	0.000310414674459025 \\
0.918205626409754	0.000310025847284123 \\
0.918453743387651	0.000309194467263296 \\
0.918779069418784	0.000308107555611059 \\
0.919195929572793	0.000306720030494034 \\
0.920730394218805	0.000301663676509634 \\
0.920881992386684	0.000301168620353565 \\
0.921716310403637	0.000298456172458827 \\
0.921720421578646	0.000298443104838952 \\
0.923498857953819	0.000292739190626889 \\
0.923553028604257	0.000292567041469738 \\
0.924006080008399	0.000291131465928629 \\
0.92420639372191	0.000290498981485143 \\
0.924470669810756	0.000289666088065132 \\
0.924724469351036	0.000288868672214448 \\
0.926231702262142	0.000284174719126895 \\
0.926990949332903	0.000281837943475693 \\
0.933531060787661	0.00026244911714457 \\
0.934434631069947	0.000259872525930405 \\
0.934570551019964	0.000259487016592175 \\
0.935534942058006	0.000256767234532163 \\
0.936268298594132	0.000254717306233943 \\
0.936269242169234	0.000254714541370049 \\
0.938329910665083	0.000249037548201159 \\
0.94030185596857	0.000243718212004751 \\
0.940807389024333	0.000242372421780601 \\
0.941016354925932	0.000241818037466146 \\
0.942120305962364	0.000238909764448181 \\
0.944292950605489	0.000233283251873218 \\
0.944561491811014	0.000232596910791472 \\
0.944673077956372	0.000232312071602792 \\
0.945708241295843	0.000229686935199425 \\
0.947179815483775	0.000226004063733853 \\
0.947693224914561	0.00022473229910247 \\
0.950107910712063	0.000218843561015092 \\
0.953012070755755	0.000211957667488605 \\
0.953264426846652	0.000211369319004007 \\
0.953432175112961	0.000210979051189497 \\
0.954059232722306	0.000209526173421182 \\
0.954894864067499	0.000207605306059122 \\
0.955427443299792	0.000206389973754995 \\
0.958107244064232	0.000200378053705208 \\
0.95825981412274	0.000200040769414045 \\
0.961834544341432	0.000192296516615897 \\
0.962013308437074	0.000191916988114826 \\
0.962557230525089	0.000190766615560278 \\
0.964107437971182	0.000187524274224415 \\
0.96546229210349	0.000184734526555985 \\
0.966544858040119	0.000182534000487067 \\
0.966587760026518	0.000182447300176136 \\
0.966792601928706	0.000182034127647057 \\
0.968404850237238	0.000178812944795936 \\
0.968813010049556	0.000178006099304184 \\
0.969045613941432	0.000177547975908965 \\
0.969871084100612	0.0001759313599905 \\
0.970530837594512	0.000174649772816338 \\
0.971272888982341	0.00017321873747278 \\
0.971527190360033	0.00017273091361858 \\
0.977559419065177	0.000161545598530211 \\
0.978837143824273	0.000159268296556547 \\
0.979500241452307	0.000158099050167948 \\
0.979698366054342	0.000157751215738244 \\
0.981925106253984	0.000153893299284391 \\
0.98193841332347	0.000153870714711957 \\
0.985466980517735	0.000147946120705456 \\
0.985624420362111	0.000147687082062475 \\
0.985660230866112	0.00014762811770197 \\
0.985903288996567	0.00014722922060173 \\
0.986891736095765	0.000145617479574867 \\
0.988177000983844	0.000143547731568106 \\
0.988188066807458	0.000143529992783442 \\
0.988339255736614	0.000143288430990651 \\
0.988532835659511	0.000142979813972488 \\
0.988627149659547	0.000142829609103501 \\
0.991317832181511	0.000138609859277494 \\
0.991792763797732	0.000137877796078101 \\
0.99271508471038	0.00013646692968905 \\
0.992808132462744	0.000136325470521115 \\
0.99282017193431	0.000136307207867503 \\
0.994749345227338	0.00013340481382329 \\
0.99866338972896	0.000127701496239752 \\
};
\addplot [thick, black]
table [row sep=\\]{%
0.0011059794751318	9.8438307759352e-05 \\
0.00144600828615515	9.88940519164316e-05 \\
0.00151620394328711	9.89884356386028e-05 \\
0.00161381928831428	9.91198830888607e-05 \\
0.00193644961641948	9.95553709799424e-05 \\
0.0024363370221504	0.000100233904959168 \\
0.00384167794354173	0.000102166326541919 \\
0.00421063900202368	0.000102679907286074 \\
0.00632372037002726	0.000105670835182536 \\
0.00947686783375368	0.000110296670754906 \\
0.0099356428556221	0.000110986213258002 \\
0.0101317191976503	0.000111282344732899 \\
0.0102519961026396	0.000111464381916448 \\
0.0116572936797861	0.0001136131468229 \\
0.0122763055741271	0.000114572809252422 \\
0.0128807282915644	0.000115517599624582 \\
0.0131006826676461	0.000115863404062111 \\
0.0132105085232722	0.000116036491817795 \\
0.0150477352877949	0.000118969473987818 \\
0.0151511241047927	0.000119136748253368 \\
0.0168553495563415	0.000121927616419271 \\
0.0191895991753259	0.000125856458907947 \\
0.0192724193499449	0.000125998281873763 \\
0.0202131188025382	0.000127619015984237 \\
0.0251886934986562	0.000136544564156793 \\
0.0255409812646521	0.000137199647724628 \\
0.0256196546755231	0.000137346374685876 \\
0.0263800727822662	0.000138772593345493 \\
0.026825697968402	0.000139615527587011 \\
0.0272388337061176	0.000140401272801682 \\
0.0276781997826389	0.000141241849632934 \\
0.0277969734855988	0.000141470023663715 \\
0.027888291941232	0.00014164557796903 \\
0.032801074197002	0.000151422369526699 \\
0.0337126623182875	0.000153309549205005 \\
0.0349109578825388	0.000155825677211396 \\
0.0367757145561887	0.000159823830472305 \\
0.0368887253476963	0.000160069248522632 \\
0.0376301111814882	0.000161689764354378 \\
0.0383020767002249	0.000163172517204657 \\
0.0385765124363792	0.000163781965966336 \\
0.0386281055032889	0.000163896766025573 \\
0.0397134557815922	0.000166331490618177 \\
0.0400530872540832	0.000167100602993742 \\
0.0408224228009035	0.000168856204254553 \\
0.042416614098386	0.000172553074662574 \\
0.042542997860721	0.000172849613591097 \\
0.0425634328976269	0.000172897707670927 \\
0.0426938593122377	0.000173204185557552 \\
0.0436353215690566	0.000175433655385859 \\
0.0476491609572692	0.000185265045729466 \\
0.0476764809468383	0.00018533383263275 \\
0.0480632979275277	0.000186310397111811 \\
0.0489335324060199	0.00018852592620533 \\
0.0494953541501341	0.000189970232895575 \\
0.0498656975684827	0.000190928330994211 \\
0.0507299142819697	0.000193183121155016 \\
0.0520149216066462	0.000196584776858799 \\
0.053491737186121	0.000200568450964056 \\
0.0545456118282759	0.000203460338525474 \\
0.0549368100196436	0.000204544252483174 \\
0.0549667939602374	0.000204627562197857 \\
0.0555159955251522	0.000206160068046302 \\
0.0574839546417889	0.000211745369597338 \\
0.0576927502423773	0.000212346712942235 \\
0.0580609228746025	0.000213411301956512 \\
0.0610070220733733	0.00022212477051653 \\
0.0613718310828713	0.000223228198592551 \\
0.0624819644863142	0.000226619900786318 \\
0.0632024977130776	0.000228848846745677 \\
0.0641707318537266	0.000231878409977071 \\
0.064803290117067	0.000233879356528632 \\
0.0661546388560771	0.000238211752730422 \\
0.0675945492445934	0.000242916634306312 \\
0.0701208744758773	0.000251396559178829 \\
0.0704495440664726	0.00025252130581066 \\
0.0707913481046274	0.000253696402069181 \\
0.0709503961374458	0.00025424495106563 \\
0.0723079949898875	0.000258976564509794 \\
0.0729582139062805	0.000261273846263066 \\
0.0736427592129116	0.000263714377069846 \\
0.0738431652021745	0.000264433067059144 \\
0.074094042941424	0.000265335664153099 \\
0.0741988699821797	0.00026571360649541 \\
0.0745344402647278	0.00026692749815993 \\
0.0761651718345726	0.000272905337624252 \\
0.0762565104011357	0.000273244164418429 \\
0.0764247618188199	0.000273869169177487 \\
0.0780473992355428	0.000279971747659147 \\
0.0784939963588281	0.000281675165751949 \\
0.0787288834363844	0.000282575027085841 \\
0.0796528614743472	0.000286143476841971 \\
0.0821291956626848	0.000295930309221148 \\
0.0836539892117674	0.000302122556604445 \\
0.0837976822096664	0.000302712782286108 \\
0.0840863722935196	0.000303901761071756 \\
0.0847664004367794	0.000306721456581727 \\
0.0848642352310873	0.000307129230350256 \\
0.0852150851090396	0.000308596150716767 \\
0.085891869345856	0.000311445532133803 \\
0.087068315817913	0.000316461344482377 \\
0.0885849846877879	0.000323046988341957 \\
0.0903431131516136	0.000330852868501097 \\
0.0909278592293694	0.000333490432240069 \\
0.0914296081009306	0.000335770600941032 \\
0.0922298300096035	0.000339439342496917 \\
0.0926517333242025	0.000341389648383483 \\
0.0930086769471365	0.000343048304785043 \\
0.0936368650628759	0.000345987267792225 \\
0.0939832817677504	0.000347618653904647 \\
0.0963578674192209	0.0003590103588067 \\
0.0969604240071604	0.00036196009023115 \\
0.0974174263218397	0.000364213308785111 \\
0.0997464409700664	0.000375915551558137 \\
0.100249930959155	0.00037849455839023 \\
0.100820744388044	0.000381439662305638 \\
0.101229398304142	0.000383561942726374 \\
0.102271393895604	0.000389027467463166 \\
0.102325882258744	0.000389315508073196 \\
0.103474916599901	0.000395437178667635 \\
0.104369461810087	0.000400269724195823 \\
0.104849386693897	0.000402886391384527 \\
0.105274572558957	0.000405219150707126 \\
0.105437385176067	0.000406115781515837 \\
0.105718642159513	0.000407669867854565 \\
0.106451679001465	0.000411747518228367 \\
0.106754108203929	0.00041344208875671 \\
0.106889477031479	0.000414202426327392 \\
0.108308586301563	0.000422260694904253 \\
0.108355506097126	0.000422529905335978 \\
0.110864090628752	0.000437169248471037 \\
0.112062699234041	0.000444341683760285 \\
0.112686863552251	0.00044812323176302 \\
0.113787419472266	0.000454869266832247 \\
0.114114908535732	0.00045689643593505 \\
0.114926076298882	0.000461955962236971 \\
0.115710925433917	0.000466904923086986 \\
0.116193807044869	0.000469975493615493 \\
0.116352662276253	0.00047099040239118 \\
0.116979956596317	0.000475018663564697 \\
0.117247317403657	0.000476745917694643 \\
0.118793828877862	0.00048686150694266 \\
0.119352906137166	0.000490570557303727 \\
0.120388240876593	0.000497514614835382 \\
0.122357154089364	0.000510992133058608 \\
0.125712799722249	0.000534807855729014 \\
0.126988316690772	0.000544148671906441 \\
0.130289689058528	0.000569088791962713 \\
0.130830575131366	0.000573282537516207 \\
0.13178708265069	0.000580774154514074 \\
0.132217070962463	0.000584173831157386 \\
0.132641072460891	0.000587545917369425 \\
0.133694684584262	0.000596008496358991 \\
0.135586813424106	0.00061151385307312 \\
0.136970534315695	0.00062310736393556 \\
0.137167897059254	0.000624778855126351 \\
0.13731957841644	0.000626066525001079 \\
0.138770949832641	0.000638521858491004 \\
0.141579671494468	0.00066333229187876 \\
0.142337772861435	0.00067019258858636 \\
0.142852169410084	0.000674887851346284 \\
0.14313843453823	0.000677515054121614 \\
0.143709892765619	0.000682789890561253 \\
0.143996587198751	0.000685451610479504 \\
0.144338718134467	0.000688641623128206 \\
0.145474509131844	0.000699338736012578 \\
0.146058924442183	0.000704907812178135 \\
0.147496370200373	0.000718793540727347 \\
0.147934500473159	0.000723080360330641 \\
0.148464696403164	0.000728301587514579 \\
0.149007159183894	0.00073368294397369 \\
0.151290695331351	0.000756774505134672 \\
0.153507740604538	0.000779887544922531 \\
0.153927498261582	0.000784342875704169 \\
0.153953160827713	0.000784615869633853 \\
0.154179436777791	0.000787028460763395 \\
0.155107736993068	0.000797005195636302 \\
0.15519005879533	0.000797896063886583 \\
0.155418402417591	0.000800371868535876 \\
0.156078407887877	0.000807572272606194 \\
0.158100170599977	0.000830032862722874 \\
0.158283533549941	0.00083210039883852 \\
0.158599777942073	0.000835678190924227 \\
0.159258263288539	0.000843178364448249 \\
0.159678910282026	0.000848004769068211 \\
0.162540881855109	0.000881580810528249 \\
0.164077384668961	0.000900151673704386 \\
0.164957348092599	0.000910962524358183 \\
0.165030574098427	0.00091186782810837 \\
0.165280388241905	0.000914963893592358 \\
0.167273692684132	0.000940043770242482 \\
0.168043854724866	0.00094991730293259 \\
0.168088628486329	0.000950494431890547 \\
0.168662857989897	0.000957927899435163 \\
0.169151147716595	0.000964294245932251 \\
0.169673214248392	0.000971148256212473 \\
0.170135609058708	0.000977259129285812 \\
0.171767974673073	0.000999140786007047 \\
0.1728518498484	0.00101393938530236 \\
0.172984599941167	0.00101576664019376 \\
0.176370551959836	0.00106350739952177 \\
0.176588273575192	0.0010666532907635 \\
0.176656553576852	0.00106764154043049 \\
0.17794804219768	0.00108650862239301 \\
0.178429717033185	0.00109363056253642 \\
0.179458814233081	0.00110900204163045 \\
0.180595104426831	0.00112622557207942 \\
0.18064479669491	0.00112698459997773 \\
0.182294707450728	0.001152487937361 \\
0.182819109355123	0.00116071326192468 \\
0.189141614514094	0.00126462592743337 \\
0.190723527970293	0.00129204627592117 \\
0.192508826701743	0.00132370472420007 \\
0.200712508340217	0.00147943280171603 \\
0.202130410523186	0.00150814559310675 \\
0.202705868873484	0.00151995604392141 \\
0.202975660615516	0.00152552512008697 \\
0.2033001668374	0.00153225054964423 \\
0.204138047670012	0.00154975289478898 \\
0.204584283026216	0.00155915506184101 \\
0.205975712225296	0.00158884050324559 \\
0.208693969096644	0.00164847052656114 \\
0.209097012308874	0.0016575010959059 \\
0.210129299396579	0.00168085342738777 \\
0.210470524326312	0.00168864382430911 \\
0.210726758187617	0.00169451837427914 \\
0.210879765452585	0.00169803597964346 \\
0.213404555935633	0.0017571416683495 \\
0.214779680955682	0.00179019232746214 \\
0.215976334819965	0.00181945809163153 \\
0.21605980843853	0.00182151736225933 \\
0.21626681121775	0.00182663451414555 \\
0.217333435243771	0.00185322586912662 \\
0.217860184045886	0.00186650012619793 \\
0.218810300034316	0.00189068540930748 \\
0.219036830922259	0.00189649779349566 \\
0.219186045259484	0.00190033519174904 \\
0.219509161852637	0.00190867308992893 \\
0.219517283100912	0.00190888298675418 \\
0.220007941128277	0.00192161579616368 \\
0.221124031401732	0.00195089413318783 \\
0.221655686766872	0.00196499703451991 \\
0.222787328400317	0.00199535535648465 \\
0.222812240344664	0.00199602753855288 \\
0.224413114839207	0.00203979015350342 \\
0.225712117798195	0.00207600276917219 \\
0.226064824403414	0.00208594440482557 \\
0.22778297811216	0.00213506119325757 \\
0.227786137279108	0.0021351536270231 \\
0.228255816092263	0.00214878050610423 \\
0.22886379454373	0.00216654734686017 \\
0.229286896531361	0.002179000293836 \\
0.231570483736477	0.00224744668230414 \\
0.232839821990075	0.00228641624562442 \\
0.234699792231533	0.00234473845921457 \\
0.236598748956898	0.00240581342950463 \\
0.236914864918852	0.00241613294929266 \\
0.238193170258512	0.00245831743814051 \\
0.238267801131912	0.00246080383658409 \\
0.241280255564163	0.00256324443034828 \\
0.241532477911657	0.00257201166823506 \\
0.242724887440966	0.00261386809870601 \\
0.243487528482621	0.00264099333435297 \\
0.243851755033328	0.002654047915712 \\
0.243931127736832	0.00265690102241933 \\
0.245492487219521	0.00271365256048739 \\
0.245961870318215	0.00273094815202057 \\
0.247220233087027	0.00277786096557975 \\
0.248029547774932	0.00280845607630908 \\
0.248195183805212	0.00281475950032473 \\
0.252597476547261	0.0029875454492867 \\
0.254049623190913	0.00304682948626578 \\
0.254469169693275	0.00306417304091156 \\
0.254777550536903	0.00307698477990925 \\
0.254877558399671	0.00308115291409194 \\
0.257253649731868	0.00318181118927896 \\
0.257671137054733	0.00319983367808163 \\
0.26187840669548	0.00338721717707813 \\
0.264182169008257	0.00349441659636796 \\
0.265462287195374	0.00355543312616646 \\
0.267638564660905	0.0036616139113903 \\
0.268216851612881	0.00369035708718002 \\
0.269980014659659	0.00377938384190202 \\
0.270837940398052	0.00382347265258431 \\
0.271133018969332	0.00383875425904989 \\
0.271522794293612	0.00385903380811214 \\
0.272730212426688	0.00392253650352359 \\
0.27325151773824	0.00395027315244079 \\
0.273458501731612	0.00396134192124009 \\
0.275787496734588	0.00408801576122642 \\
0.275790346826632	0.00408817268908024 \\
0.275963502468307	0.00409775087609887 \\
0.276441724078765	0.0041243196465075 \\
0.277788622824976	0.0042000743560493 \\
0.278249554181787	0.00422631530091166 \\
0.278902433397602	0.00426376424729824 \\
0.280228184178203	0.00434082560241222 \\
0.282170737525181	0.00445625185966492 \\
0.284775883532131	0.00461585586890578 \\
0.285319633849948	0.0046498803421855 \\
0.285945143849347	0.00468932697549462 \\
0.287293498153222	0.00477549666538835 \\
0.288098109983407	0.00482766563072801 \\
0.288474372887691	0.0048522581346333 \\
0.288733693686359	0.00486927619203925 \\
0.291563977198166	0.0050589432939887 \\
0.293976605993451	0.00522640766575933 \\
0.294014528290392	0.0052290833555162 \\
0.294718175832172	0.00527898129075766 \\
0.294831463459233	0.00528705818578601 \\
0.294887681144015	0.00529107172042131 \\
0.299064488228799	0.00559784984216094 \\
0.299223468133874	0.00560986902564764 \\
0.299853491506122	0.00565775576978922 \\
0.301624334890634	0.00579453399404883 \\
0.302100714632492	0.00583188934251666 \\
0.302855829384862	0.00589158851653337 \\
0.303468685820656	0.00594048760831356 \\
0.304335545384868	0.00601034192368388 \\
0.305081848845274	0.00607113586738706 \\
0.305263786725066	0.00608604913577437 \\
0.305511643802704	0.00610642181709409 \\
0.30635774546767	0.00617648661136627 \\
0.307937429467832	0.00630944315344095 \\
0.308578187230734	0.0063641807064414 \\
0.313991998160918	0.00684587238356471 \\
0.315973848719028	0.00703111197799444 \\
0.317413649622303	0.00716880196705461 \\
0.319657848506035	0.00738876685500145 \\
0.319781920771261	0.00740112224593759 \\
0.320496004308145	0.00747262267395854 \\
0.320601310305949	0.0074832271784544 \\
0.322246867517597	0.00765085499733686 \\
0.323306124649751	0.00776072638109326 \\
0.323664358184811	0.00779823819175363 \\
0.323805398711091	0.00781305506825447 \\
0.324816846999054	0.00792013294994831 \\
0.325263683488062	0.00796790141612291 \\
0.327533460856979	0.00821499712765217 \\
0.328385869951453	0.00830973871052265 \\
0.329844054701458	0.00847433134913445 \\
0.33074904970515	0.00857809837907553 \\
0.332974695502457	0.00883869081735611 \\
0.333600133199531	0.00891332048922777 \\
0.336001356191777	0.00920569151639938 \\
0.338057278352338	0.00946354493498802 \\
0.33940563378102	0.00963652972131968 \\
0.339522610836627	0.0096516776829958 \\
0.341295751839904	0.00988428667187691 \\
0.343025583329651	0.0101165659725666 \\
0.344156227439929	0.0102712884545326 \\
0.344485478041551	0.010316789150238 \\
0.346631556569637	0.0106182470917702 \\
0.348247004908186	0.010850896127522 \\
0.348257869791207	0.010852481238544 \\
0.350134953366802	0.0111291892826557 \\
0.352271571140202	0.0114526487886906 \\
0.352403536611892	0.011472930200398 \\
0.356607161458692	0.012137814424932 \\
0.357473355861859	0.0122794732451439 \\
0.359032837138255	0.0125386426225305 \\
0.360085447455855	0.0127166174352169 \\
0.360587038984981	0.0128022944554687 \\
0.3618814720936	0.013026051223278 \\
0.362541325151431	0.0131415938958526 \\
0.363971114014582	0.0133954342454672 \\
0.36446377318254	0.0134840225800872 \\
0.364734654416428	0.0135329645127058 \\
0.364782079325344	0.0135415475815535 \\
0.36626622804169	0.0138130458071828 \\
0.36745676602671	0.014034696854651 \\
0.367555756327517	0.014053288847208 \\
0.367766921022286	0.0140930116176605 \\
0.368015883464081	0.0141399959102273 \\
0.368183587302196	0.0141717344522476 \\
0.368646745255496	0.0142597435042262 \\
0.368824340831231	0.0142936259508133 \\
0.36908666014638	0.0143438335508108 \\
0.370227788452905	0.014564260840416 \\
0.371079364739364	0.0147309126332402 \\
0.371457393668236	0.0148054985329509 \\
0.372521553247257	0.0150174489244819 \\
0.372816229090882	0.015076675452292 \\
0.373696444358129	0.0152549324557185 \\
0.375236418301489	0.015571809373796 \\
0.375355278624935	0.0155965266749263 \\
0.375592850161798	0.0156460665166378 \\
0.375621310337888	0.0156520083546638 \\
0.37623340048021	0.0157803911715746 \\
0.376649481158299	0.0158682484179735 \\
0.37682934901581	0.0159063879400492 \\
0.377022374188538	0.0159474071115255 \\
0.378777395062222	0.0163251720368862 \\
0.379219181502585	0.0164216458797455 \\
0.380639994297166	0.0167357213795185 \\
0.381635433931239	0.0169592723250389 \\
0.382416367624347	0.0171366985887289 \\
0.383276747323035	0.0173342861235142 \\
0.383441774868681	0.0173724424093962 \\
0.384589480637987	0.0176400803029537 \\
0.38762147240542	0.0183667354285717 \\
0.387652450706726	0.0183743052184582 \\
0.38818745089578	0.018505584448576 \\
0.388531576179462	0.0185905173420906 \\
0.391108514236782	0.0192387793213129 \\
0.391741157295481	0.0194012764841318 \\
0.396723908263121	0.0207290146499872 \\
0.39803082540264	0.021091740578413 \\
0.398171926406937	0.0211312752217054 \\
0.398666017137866	0.0212702602148056 \\
0.398819673578751	0.021313676610589 \\
0.4012576168302	0.0220140684396029 \\
0.402752737045505	0.0224546752870083 \\
0.403290070599278	0.0226151291280985 \\
0.404563584871068	0.0229998584836721 \\
0.406532702382757	0.0236073676496744 \\
0.406578018270242	0.023621529340744 \\
0.413273296435021	0.0258076433092356 \\
0.413592377461711	0.0259166043251753 \\
0.41401191876112	0.0260605569928885 \\
0.414776267969876	0.0263248104602098 \\
0.414785799829368	0.02632812038064 \\
0.415238430177184	0.0264858473092318 \\
0.416465974154779	0.0269182417541742 \\
0.416689013292801	0.0269975550472736 \\
0.418297500020148	0.0275761876255274 \\
0.422720878338663	0.0292302034795284 \\
0.424343600824592	0.0298607591539621 \\
0.425257001696199	0.0302214901894331 \\
0.42710227803989	0.0309631004929543 \\
0.427962517509508	0.0313148312270641 \\
0.429309169026115	0.0318732112646103 \\
0.429443446115563	0.0319294147193432 \\
0.429705476003957	0.0320393443107605 \\
0.431003390699289	0.0325893647968769 \\
0.432539042653788	0.033251915127039 \\
0.433068569337496	0.0334833674132824 \\
0.433545956268458	0.033693365752697 \\
0.433750136573834	0.0337835587561131 \\
0.433778478107147	0.0337961167097092 \\
0.434834104829426	0.034266360104084 \\
0.435898870411909	0.0347470827400684 \\
0.43750740037395	0.03548564016819 \\
0.437737100851006	0.0355923250317574 \\
0.439197385289126	0.0362778566777706 \\
0.440921894924801	0.0371037870645523 \\
0.442219378180834	0.0377370491623878 \\
0.446052885401706	0.0396693162620068 \\
0.447759455363191	0.0405596606433392 \\
0.448771556188603	0.0410967022180557 \\
0.448918332177756	0.0411751568317413 \\
0.449861491210787	0.0416826456785202 \\
0.449976789412666	0.0417450927197933 \\
0.450129821782654	0.0418281219899654 \\
0.453058505166632	0.0434477403759956 \\
0.455515296612047	0.0448523312807083 \\
0.461612686527044	0.048527467995882 \\
0.462183229417738	0.048885602504015 \\
0.46315332653893	0.0495003238320351 \\
0.463567773958142	0.0497651435434818 \\
0.464569005654222	0.0504105053842068 \\
0.46546966357775	0.0509978197515011 \\
0.465627783083443	0.0511015877127647 \\
0.466166033904132	0.0514563210308552 \\
0.466888847202291	0.0519363358616829 \\
0.468332203161028	0.0529076233506203 \\
0.469094139229819	0.05342723056674 \\
0.469464236056835	0.0536813400685787 \\
0.469806075154756	0.0539170317351818 \\
0.470776441432414	0.054591566324234 \\
0.471475988022769	0.055082693696022 \\
0.473199954060133	0.0563109032809734 \\
0.474832247453521	0.0574974790215492 \\
0.475230428162318	0.0577904470264912 \\
0.476638306671274	0.0588376969099045 \\
0.478988899075304	0.0606257803738117 \\
0.479018256118887	0.0606484822928905 \\
0.479279919995254	0.0608506798744202 \\
0.479532435695615	0.0610464550554752 \\
0.479763011578347	0.0612257048487663 \\
0.480041882522935	0.0614431798458099 \\
0.480923824850651	0.0621357150375843 \\
0.481629462985493	0.0626949965953827 \\
0.481771508198592	0.0628081560134888 \\
0.482503736852496	0.0633945092558861 \\
0.483064995688331	0.0638474300503731 \\
0.48384062770339	0.0644782185554504 \\
0.484085120492796	0.0646782442927361 \\
0.484352733912412	0.0648978874087334 \\
0.484550939846654	0.0650609880685806 \\
0.485022749278845	0.0654507875442505 \\
0.485322068953861	0.0656991824507713 \\
0.48610300375999	0.066351443529129 \\
0.486253849387163	0.0664781183004379 \\
0.489154935058063	0.068958468735218 \\
0.489792628987292	0.0695150271058083 \\
0.490458183784177	0.0701003596186638 \\
0.492282643359321	0.0717282667756081 \\
0.493928358105999	0.0732265412807465 \\
0.495701335790147	0.0748728439211845 \\
0.496106252426634	0.0752535536885262 \\
0.496164487189195	0.0753084719181061 \\
0.498739819509518	0.0777733474969864 \\
0.500164108812496	0.0791681110858917 \\
0.500739446514191	0.0797379836440086 \\
0.500830786926564	0.0798287391662598 \\
0.502541719930356	0.081547349691391 \\
0.503498564874392	0.0825230479240417 \\
0.504534081188611	0.0835909619927406 \\
0.504637823699037	0.0836986973881721 \\
0.506860898199194	0.0860364735126495 \\
0.507491092240963	0.0867098793387413 \\
0.508120202534381	0.0873867869377136 \\
0.510856303638406	0.0903870314359665 \\
0.512327455201249	0.0920381098985672 \\
0.51318641739938	0.0930146500468254 \\
0.513457954653542	0.0933252945542336 \\
0.515863112336536	0.0961175560951233 \\
0.516616766484905	0.0970077440142632 \\
0.517831355926389	0.0984578877687454 \\
0.518931921097757	0.0997883826494217 \\
0.521152065672689	0.102521173655987 \\
0.524519468497769	0.106792517006397 \\
0.524576690058109	0.106866404414177 \\
0.525883589956303	0.108566947281361 \\
0.526973717032038	0.110003530979156 \\
0.528680408273136	0.112285912036896 \\
0.530600365407266	0.114902883768082 \\
0.530809662396022	0.115191332995892 \\
0.53166475254574	0.116376392543316 \\
0.532158485575632	0.117065422236919 \\
0.535204926183725	0.12139567732811 \\
0.535356594523575	0.121614776551723 \\
0.537004451507049	0.124017745256424 \\
0.537089668720381	0.124143071472645 \\
0.537190935482585	0.124292209744453 \\
0.538117362047647	0.125663504004478 \\
0.538453506118723	0.126164376735687 \\
0.539359907716354	0.127523168921471 \\
0.539560791003075	0.127825945615768 \\
0.541180326502456	0.130289852619171 \\
0.547271589683928	0.13992004096508 \\
0.548822201570025	0.14246466755867 \\
0.549624872437239	0.143796920776367 \\
0.549819206435189	0.144121125340462 \\
0.550184928091146	0.144732594490051 \\
0.552139674398207	0.148038282990456 \\
0.552982438656419	0.149482503533363 \\
0.5537594353206	0.150824323296547 \\
0.553976139314498	0.15120030939579 \\
0.554051563436848	0.151331335306168 \\
0.55426017856816	0.151694312691689 \\
0.555396555661942	0.153683811426163 \\
0.555501425327998	0.153868451714516 \\
0.556360039382527	0.155387356877327 \\
0.557883446271028	0.158112198114395 \\
0.558670348456701	0.159534841775894 \\
0.559194391744592	0.16048800945282 \\
0.559473603788983	0.160997614264488 \\
0.560198442614008	0.162326961755753 \\
0.560853933269278	0.16353677213192 \\
0.561152915237902	0.164091005921364 \\
0.561401591015197	0.16455303132534 \\
0.562479951006079	0.166568920016289 \\
0.563005908106818	0.167559295892715 \\
0.563050000205171	0.167642444372177 \\
0.563426857102662	0.168355241417885 \\
0.563560604448239	0.168608784675598 \\
0.564062038149409	0.16956202685833 \\
0.565336310962797	0.172003731131554 \\
0.565855772748177	0.17300720512867 \\
0.566858287901599	0.174956575036049 \\
0.566881879040954	0.175002694129944 \\
0.568231608554438	0.177654951810837 \\
0.569223717564581	0.179624542593956 \\
0.569387586449175	0.179951652884483 \\
0.569573989003313	0.180324047803879 \\
0.569720721150053	0.180617690086365 \\
0.572004618116245	0.185236111283302 \\
0.572005154945678	0.185237169265747 \\
0.572151528516064	0.185536295175552 \\
0.57219889974288	0.18563312292099 \\
0.572780347928331	0.186825349926949 \\
0.573600339080632	0.188516616821289 \\
0.576593284261927	0.194789275527 \\
0.576676798911885	0.194966554641724 \\
0.576819659872325	0.195270121097565 \\
0.577653861143233	0.19704969227314 \\
0.577917937736817	0.197615429759026 \\
0.57793430586259	0.197650671005249 \\
0.581759386894494	0.205985382199287 \\
0.582475032298807	0.207573354244232 \\
0.582896249817215	0.208512157201767 \\
0.589571735768374	0.223807021975517 \\
0.592388904473667	0.230496346950531 \\
0.593611787705259	0.233443021774292 \\
0.594151694891475	0.234752357006073 \\
0.595008948927732	0.236841797828674 \\
0.596105779635777	0.239533543586731 \\
0.597490139381406	0.242960780858994 \\
0.598951438337811	0.246614515781403 \\
0.599201054717089	0.247242197394371 \\
0.600228747033593	0.249838039278984 \\
0.600693928524281	0.251019060611725 \\
0.601589358474053	0.253302812576294 \\
0.602529748251813	0.255715757608414 \\
0.602916831603936	0.256713300943375 \\
0.603866403426713	0.259171396493912 \\
0.606129689323002	0.265091300010681 \\
0.607808363354841	0.269537299871445 \\
0.60811641039762	0.270358204841614 \\
0.608424173673929	0.271180063486099 \\
0.610203299896366	0.2759610414505 \\
0.61052401101773	0.276828348636627 \\
0.611007548691541	0.278139412403107 \\
0.61345268097321	0.284826129674911 \\
0.614300601551049	0.287167280912399 \\
0.616415105925723	0.293055444955826 \\
0.617595625529809	0.296373158693314 \\
0.619695383178102	0.302327662706375 \\
0.620626131680421	0.304988503456116 \\
0.620884093030541	0.305728316307068 \\
0.622959094549987	0.311715334653854 \\
0.624820249131202	0.317139118909836 \\
0.627396892203872	0.324730545282364 \\
0.629679734717881	0.331534177064896 \\
0.63108091324942	0.335745245218277 \\
0.631412061821856	0.336744427680969 \\
0.632209138475077	0.33915501832962 \\
0.632608841524959	0.34036710858345 \\
0.635174872896091	0.348197460174561 \\
0.63680487509045	0.353214651346207 \\
0.638872902814914	0.359626412391663 \\
0.639214862746441	0.360691577196121 \\
0.640549343973947	0.364861369132996 \\
0.6407383188268	0.36545330286026 \\
0.642765742598199	0.371831029653549 \\
0.642772962045468	0.371853798627853 \\
0.644393148200581	0.376982897520065 \\
0.646791193304363	0.384624719619751 \\
0.647126395490269	0.385697424411774 \\
0.647747012654458	0.38768681883812 \\
0.648487744871255	0.390065729618073 \\
0.65164263639371	0.400255501270294 \\
0.651778799711512	0.400697410106659 \\
0.652422089544704	0.402786701917648 \\
0.652778066297422	0.40394452214241 \\
0.653037870316759	0.404790043830872 \\
0.653946378314801	0.407751560211182 \\
0.656181358570071	0.415064513683319 \\
0.657515298781176	0.41944745182991 \\
0.657916355325804	0.420767694711685 \\
0.65821939898974	0.421765923500061 \\
0.660347901626978	0.428795427083969 \\
0.663462409580861	0.439132004976273 \\
0.665002009135555	0.444261521100998 \\
0.665125585501828	0.444673955440521 \\
0.665840946327609	0.447061777114868 \\
0.669598702136104	0.459642678499222 \\
0.669671107753933	0.459885656833649 \\
0.670317902411108	0.462056845426559 \\
0.67055566811489	0.462855249643326 \\
0.670890712943537	0.463980793952942 \\
0.671190867447461	0.464989602565765 \\
0.672572121540986	0.469634592533112 \\
0.674425718831603	0.47587588429451 \\
0.67545414417829	0.479342013597488 \\
0.67596832728122	0.481075823307037 \\
0.676862814397186	0.484092891216278 \\
0.677008756097011	0.484585136175156 \\
0.677760880297266	0.487123131752014 \\
0.680496299332769	0.496357589960098 \\
0.6805627271546	0.496581941843033 \\
0.682518497342478	0.503185987472534 \\
0.682667061493577	0.503687620162964 \\
0.683323391450582	0.505903661251068 \\
0.683455311069743	0.50634902715683 \\
0.686099541255968	0.515272974967957 \\
0.686972247003404	0.518216550350189 \\
0.687287921742946	0.519280850887299 \\
0.688979470872378	0.524980902671814 \\
0.689844777380402	0.527894258499146 \\
0.69066753173656	0.530662536621094 \\
0.690924484351316	0.531526744365692 \\
0.691684985973077	0.534083247184753 \\
0.694075688265434	0.542107582092285 \\
0.694253786667502	0.542704522609711 \\
0.696056771003981	0.548740327358246 \\
0.696659969942023	0.550756514072418 \\
0.69712902782289	0.55232310295105 \\
0.697436633040946	0.55334997177124 \\
0.697470314030711	0.553462445735931 \\
0.698295232645012	0.556213557720184 \\
0.698478216988194	0.556823551654816 \\
0.699563853841568	0.560437917709351 \\
0.700079752305739	0.562153339385986 \\
0.700291212685059	0.562855958938599 \\
0.701162742599685	0.565749287605286 \\
0.702327231437584	0.569608092308044 \\
0.702639966218633	0.570642650127411 \\
0.703077231783761	0.572088897228241 \\
0.703198302674555	0.572489023208618 \\
0.703418311750763	0.573215782642365 \\
0.704913184821873	0.57814610004425 \\
0.705985343895602	0.581672549247742 \\
0.70676366586065	0.584227204322815 \\
0.70953480835097	0.593285381793976 \\
0.709597763991885	0.593490302562714 \\
0.710916839332999	0.597779452800751 \\
0.710971707887323	0.597957491874695 \\
0.711527088935361	0.599758684635162 \\
0.711625741549748	0.600078284740448 \\
0.712030078150365	0.601387441158295 \\
0.713838731517111	0.607225894927979 \\
0.71495171630821	0.610803306102753 \\
0.715877253359645	0.613769292831421 \\
0.717907954554698	0.620246767997742 \\
0.718267373832691	0.621388852596283 \\
0.719482266109768	0.625239014625549 \\
0.720072506203388	0.627103984355927 \\
0.72063303964824	0.628871440887451 \\
0.721414113641739	0.631328761577606 \\
0.721834558021562	0.632648587226868 \\
0.721858215436245	0.632722854614258 \\
0.722606600061919	0.635067045688629 \\
0.723634632256085	0.638276815414429 \\
0.724295951475884	0.640334844589233 \\
0.724780388322819	0.641839265823364 \\
0.725967182906895	0.64551305770874 \\
0.728562998007155	0.653488039970398 \\
0.729378939538106	0.655977129936218 \\
0.729672766087416	0.656871557235718 \\
0.730156271089799	0.658340573310852 \\
0.730418624607748	0.659136533737183 \\
0.731079272213317	0.661136984825134 \\
0.73140958111192	0.662134885787964 \\
0.734582373950397	0.671645641326904 \\
0.735036686489948	0.672996163368225 \\
0.736495162798249	0.677312314510345 \\
0.737637169481189	0.680671095848083 \\
0.738132551177952	0.682122349739075 \\
0.738772404713377	0.683991312980652 \\
0.73919080617973	0.685210466384888 \\
0.739634766317876	0.68650096654892 \\
0.739671405613807	0.686607480049133 \\
0.740318229014976	0.688482344150543 \\
0.741017797721675	0.690503180027008 \\
0.741923938399663	0.693109691143036 \\
0.742870439596604	0.695819318294525 \\
0.743500813750507	0.697616398334503 \\
0.745235425982657	0.702530324459076 \\
0.745794924415534	0.704105317592621 \\
0.746651135832285	0.706506133079529 \\
0.748242417565033	0.710937798023224 \\
0.749080152481028	0.713254749774933 \\
0.750738597109713	0.717808425426483 \\
0.754059260293618	0.726792573928833 \\
0.754427277039572	0.727777242660522 \\
0.755971311279052	0.731883823871613 \\
0.7564660147832	0.733191192150116 \\
0.756964107658398	0.734503388404846 \\
0.757301980310619	0.735391199588776 \\
0.757561417579673	0.736071527004242 \\
0.759460973532791	0.741018652915955 \\
0.759943823499425	0.742266476154327 \\
0.76007481161266	0.74260425567627 \\
0.76520949758525	0.755617022514343 \\
0.769369251223668	0.765826523303986 \\
0.77094746139779	0.76962149143219 \\
0.771797041083953	0.771646440029144 \\
0.772164013593626	0.772517085075378 \\
0.772721858147557	0.773836433887482 \\
0.774553209660584	0.778128683567047 \\
0.774686774792601	0.778439402580261 \\
0.774763255366	0.778617084026337 \\
0.776676196809448	0.783030986785889 \\
0.781826328102722	0.794595539569855 \\
0.783936140597273	0.799198806285858 \\
0.784010287514813	0.799359261989594 \\
0.784889735175701	0.801253855228424 \\
0.785018616770813	0.801530480384827 \\
0.78602927274531	0.803688704967499 \\
0.786818186718577	0.805361211299896 \\
0.787974469502378	0.807792782783508 \\
0.791139177359944	0.814329147338867 \\
0.791586708117001	0.815239310264587 \\
0.791796196079504	0.815664291381836 \\
0.79182158322754	0.815715670585632 \\
0.792072748806228	0.816224157810211 \\
0.792682852079258	0.817454218864441 \\
0.792705401732446	0.817499697208405 \\
0.794569839509874	0.82121878862381 \\
0.795023193386171	0.822114169597626 \\
0.795367478837485	0.822791755199432 \\
0.795527938976809	0.82310676574707 \\
0.795634872431087	0.82331657409668 \\
0.79745955888423	0.826865315437317 \\
0.798224122257701	0.828335464000702 \\
0.799881016007139	0.831487357616425 \\
0.801765379467602	0.835015296936035 \\
0.80362962295172	0.838447034358978 \\
0.803814553638313	0.838784217834473 \\
0.806469107208753	0.843562662601471 \\
0.806697504962046	0.843968331813812 \\
0.807337366299355	0.845100343227386 \\
0.808888043138527	0.847815692424774 \\
0.809190014741277	0.848339915275574 \\
0.809307640373056	0.848543703556061 \\
0.810501764417227	0.850599765777588 \\
0.811889729396757	0.852960407733917 \\
0.811942350832027	0.853049397468567 \\
0.814196326909986	0.856815338134766 \\
0.815151250895423	0.858386337757111 \\
0.815544311745005	0.85902863740921 \\
0.816375455024602	0.860379040241241 \\
0.818069790025891	0.863098323345184 \\
0.820698196177955	0.867228269577026 \\
0.821263012964588	0.868101835250854 \\
0.823441468430582	0.871425330638885 \\
0.824103550036923	0.872421205043793 \\
0.825253713607837	0.874135494232178 \\
0.827170035611177	0.876947700977325 \\
0.827687372625058	0.877697587013245 \\
0.828681002627947	0.879126727581024 \\
0.831688136352904	0.88336443901062 \\
0.832015103492291	0.883817195892334 \\
0.832618539764885	0.884649097919464 \\
0.833979270360146	0.886505722999573 \\
0.835071309810646	0.887976765632629 \\
0.835699577265585	0.888815522193909 \\
0.836498037763016	0.889873623847961 \\
0.836836957998254	0.89031994342804 \\
0.838046127292981	0.891899764537811 \\
0.838525323246066	0.892520189285278 \\
0.839239765753767	0.893439471721649 \\
0.839402685897079	0.893648147583008 \\
0.839999299682593	0.8944091796875 \\
0.840281951751549	0.894768059253693 \\
0.840483203711437	0.895022869110107 \\
0.840691788412589	0.895286500453949 \\
0.842685722208791	0.897776484489441 \\
0.842993851211431	0.898156583309174 \\
0.844759374999568	0.900310218334198 \\
0.844835827371022	0.90040248632431 \\
0.844992041004939	0.900590896606445 \\
0.84621071404422	0.902050018310547 \\
0.84647818555566	0.9023677110672 \\
0.847551016932997	0.903632640838623 \\
0.848091115090324	0.904263973236084 \\
0.84863665263528	0.904897689819336 \\
0.85175414414149	0.908448100090027 \\
0.852055625215632	0.908784985542297 \\
0.853183518340707	0.910035669803619 \\
0.855762763379044	0.912837505340576 \\
0.855845256518328	0.912925839424133 \\
0.856101518575613	0.913199543952942 \\
0.856353280691239	0.913467824459076 \\
0.857431392977156	0.91460782289505 \\
0.858507268631162	0.915731906890869 \\
0.858727032795181	0.915959775447845 \\
0.859611972462117	0.916872024536133 \\
0.860598063349272	0.917877793312073 \\
0.861479766548793	0.918767690658569 \\
0.863021652496581	0.920302927494049 \\
0.863191163941701	0.92046993970871 \\
0.863243196088782	0.920521259307861 \\
0.863264684077014	0.920542359352112 \\
0.86433033865504	0.921584963798523 \\
0.865386426798682	0.922605633735657 \\
0.865574805896198	0.922786355018616 \\
0.867009964664233	0.924151062965393 \\
0.868402477420643	0.925454080104828 \\
0.868462536448662	0.925509810447693 \\
0.8687736544668	0.925797820091248 \\
0.870769922748437	0.927621781826019 \\
0.87137568194098	0.928167104721069 \\
0.871475162483056	0.928256213665009 \\
0.872823623463892	0.929454982280731 \\
0.873231322955811	0.929813802242279 \\
0.87330913025223	0.929882049560547 \\
0.874554061243274	0.930966138839722 \\
0.874752616263773	0.931137681007385 \\
0.875835647990412	0.932066023349762 \\
0.878375805455995	0.934197723865509 \\
0.879016462457315	0.934725522994995 \\
0.879915986267341	0.935459852218628 \\
0.880187041340505	0.935679614543915 \\
0.880228906915535	0.935713529586792 \\
0.880706157108953	0.936098515987396 \\
0.881313907623959	0.93658572435379 \\
0.88357711634013	0.938369750976562 \\
0.88575628070969	0.940042972564697 \\
0.885860150516388	0.940121710300446 \\
0.885973756819001	0.940207660198212 \\
0.886578379643548	0.940663039684296 \\
0.88662976023068	0.940701603889465 \\
0.887201064437371	0.941128730773926 \\
0.88721221948908	0.941136956214905 \\
0.889489342121342	0.942810654640198 \\
0.889610563902722	0.942898511886597 \\
0.891226627532086	0.944057285785675 \\
0.891601333881282	0.944322764873505 \\
0.894395711960869	0.946265518665314 \\
0.894941283658575	0.946637332439423 \\
0.895002834569702	0.94667911529541 \\
0.895316145165804	0.946891367435455 \\
0.89572498422586	0.947167098522186 \\
0.900281097465164	0.950149536132812 \\
0.901776757070804	0.951093196868896 \\
0.902902077475763	0.951792180538177 \\
0.903042041253622	0.951878368854523 \\
0.903417484940991	0.952109038829803 \\
0.905006674071476	0.953073740005493 \\
0.905428107857581	0.953326404094696 \\
0.90623158203924	0.953804552555084 \\
0.909863735264334	0.955908417701721 \\
0.914192347433306	0.958296239376068 \\
0.914860066850241	0.958653330802917 \\
0.916826528028795	0.959688127040863 \\
0.918089828672593	0.960339605808258 \\
0.918205626409754	0.96039891242981 \\
0.918453743387651	0.960525572299957 \\
0.918779069418784	0.960690915584564 \\
0.919195929572793	0.960901856422424 \\
0.920730394218805	0.961669147014618 \\
0.920881992386684	0.961744070053101 \\
0.921716310403637	0.962154448032379 \\
0.921720421578646	0.962156414985657 \\
0.923498857953819	0.963016867637634 \\
0.923553028604257	0.963042736053467 \\
0.924006080008399	0.963258743286133 \\
0.92420639372191	0.963353872299194 \\
0.924470669810756	0.963478982448578 \\
0.924724469351036	0.963598847389221 \\
0.926231702262142	0.964302361011505 \\
0.926990949332903	0.964651763439178 \\
0.933531060787661	0.967527985572815 \\
0.934434631069947	0.967907130718231 \\
0.934570551019964	0.967963755130768 \\
0.935534942058006	0.968362987041473 \\
0.936268298594132	0.968663215637207 \\
0.936269242169234	0.968663692474365 \\
0.938329910665083	0.96949291229248 \\
0.94030185596857	0.970266401767731 \\
0.940807389024333	0.97046160697937 \\
0.941016354925932	0.970541894435883 \\
0.942120305962364	0.970962822437286 \\
0.944292950605489	0.971774041652679 \\
0.944561491811014	0.971872746944427 \\
0.944673077956372	0.971913635730743 \\
0.945708241295843	0.972290575504303 \\
0.947179815483775	0.972817897796631 \\
0.947693224914561	0.972999572753906 \\
0.950107910712063	0.973838269710541 \\
0.953012070755755	0.974813342094421 \\
0.953264426846652	0.974896371364594 \\
0.953432175112961	0.974951446056366 \\
0.954059232722306	0.975156128406525 \\
0.954894864067499	0.975426435470581 \\
0.955427443299792	0.975597202777863 \\
0.958107244064232	0.976438999176025 \\
0.95825981412274	0.976486086845398 \\
0.961834544341432	0.977562665939331 \\
0.962013308437074	0.977615296840668 \\
0.962557230525089	0.977774500846863 \\
0.964107437971182	0.978222012519836 \\
0.96546229210349	0.978605926036835 \\
0.966544858040119	0.978908002376556 \\
0.966587760026518	0.978919863700867 \\
0.966792601928706	0.978976547718048 \\
0.968404850237238	0.979417145252228 \\
0.968813010049556	0.979527294635773 \\
0.969045613941432	0.979589760303497 \\
0.969871084100612	0.979809999465942 \\
0.970530837594512	0.97998434305191 \\
0.971272888982341	0.980178654193878 \\
0.971527190360033	0.980244874954224 \\
0.977559419065177	0.981752395629883 \\
0.978837143824273	0.982056856155396 \\
0.979500241452307	0.982212901115417 \\
0.979698366054342	0.982259273529053 \\
0.981925106253984	0.982772290706635 \\
0.98193841332347	0.982775270938873 \\
0.985466980517735	0.983558237552643 \\
0.985624420362111	0.983592391014099 \\
0.985660230866112	0.98360013961792 \\
0.985903288996567	0.983652591705322 \\
0.986891736095765	0.983864426612854 \\
0.988177000983844	0.984135687351227 \\
0.988188066807458	0.984137952327728 \\
0.988339255736614	0.984169602394104 \\
0.988532835659511	0.984210014343262 \\
0.988627149659547	0.984229624271393 \\
0.991317832181511	0.984779894351959 \\
0.991792763797732	0.984875023365021 \\
0.99271508471038	0.985058128833771 \\
0.992808132462744	0.985076487064362 \\
0.99282017193431	0.985078811645508 \\
0.994749345227338	0.985454201698303 \\
0.99866338972896	0.98618745803833 \\
};
\addplot [thick, green!50.0!black, dashed]
table [row sep=\\]{%
0.33333	0 \\
0.33333	1 \\
};
\addplot [thick, blue, dashed]
table [row sep=\\]{%
0.66666	0 \\
0.66666	1 \\
};
\addplot [thick, blue]
table [row sep=\\]{%
0.679128589815018	0 \\
0.679128589815018	1 \\
};
\node at (axis cs:0.1,0.95)[
  scale=0.7,
  anchor=north west,
  text=black,
  rotate=0.0
]{ $\boldsymbol{q}_{0,1}$};
\node at (axis cs:0.5,0.95)[
  scale=0.7,
  anchor=north west,
  text=black,
  rotate=0.0
]{ $\boldsymbol{q}_{0,2}$};
\node at (axis cs:0.9,0.95)[
  scale=0.7,
  anchor=north west,
  text=black,
  rotate=0.0
]{ $\boldsymbol{q}_{0,3}$};
\end{axis}

\end{tikzpicture}
        \vspace{-0.5cm}
        \caption{Quantization $\hat{\q}^{(s)}_1(\gls{x}_1)$ resulting from the thresholding (\ref{eq:ht}) at iterations $t = 5$ and $m_{\text{max}} = 3$.}
    \end{subfigure}%
    
    \begin{subfigure}[t]{\textwidth}
        \centering
        \input{figures/chapitre4/True_simulated_data/feature_0_iteration_300.tex}
        \vspace{-0.5cm}
        \caption{Quantizations $\hat{\q}^{(s)}_1(\gls{x}_1)$ resulting from the thresholding (\ref{eq:ht}) at iterations $t = 300$ and $m_{\text{max}} = 3$.}
    \end{subfigure}
    
    \caption{\label{fig:MAP} Quantizations $\hat{\q}^{(s)}_1(\gls{x}_1)$ of experiment (a) resulting from the thresholding (\ref{eq:ht}).}
\end{figure}

\subsection{Benchmark data} \label{subsec:exp_benchmark}

To test further the effectiveness of \textit{glmdisc} in a predictive setting, we gathered 6 datasets from the UCI library: the Adult dataset ($n=48,842$, $d=14$), the Australian dataset ($n=690$, $d=14$), the Bands dataset ($n=512$, $d=39$), the Credit-screening dataset ($n=690$, $d=15$), the German dataset ($n=1,000$, $d=20$) and the Heart dataset ($n=270$, $d=13$). Each of these datasets have mixed (continuous and categorical) features and a binary response to predict. To get more information about these datasets, their respective features, and the predictive task associated with them, readers may refer to the UCI website\footnote{\cite{Dua:2017} : http://archive.ics.uci.edu/ml}.

Now that the proposed approach was shown empirically consistent, \textit{i.e.}\ it is able to find the true quantization in a well-specified setting, it is desirable to verify the previous claim that embedding the learning of a good quantization in the predictive task \textit{via glmdisc} is better than other methods that rely on \textit{ad hoc} criteria. As we were primarily interested in \gls{lr}, I will compare the proposed approach to a na\"{\i}ve linear \gls{lr} (hereafter ALLR), \textit{i.e.}\ on non-quantized data, a \gls{lr} on continuous discretized data using the now standard MDLP algorithm from~\cite{fayyad1993multi} and categorical grouped data using $\chi^2$ tests of independence between each pair of factor levels and the target in the same fashion as the ChiMerge discretization algorithm proposed by~\cite{kerber1992chimerge} (hereafter MDLP/$\chi^2$). In this section and the next, Gini indices are reported on a random 30~\% test set and CIs are given following a method found in~\cite{sun2014fast}. Table~\ref{tab:banchmark} shows our approach yields significantly better results on these rather small datasets where the added flexibility of quantization might help the predictive task.

\begin{table}
    \centering
        \caption{Gini indices (the greater the value, the better the performance) of our proposed quantization algorithm \textit{glmdisc} and two baselines: ALLR and MDLP / $\chi^2$ tests obtained on several benchmark datasets from the UCI library.}
    \label{tab:banchmark}
\begin{small}
\begin{tabular}{lllll}
Dataset & ALLR & \textit{ad hoc} methods & \makecell{Our proposal:\\ \textit{glmdisc}-NN} & \makecell{Our proposal:\\ \textit{glmdisc}-SEM} \\
\hline
Adult & 81.4 (1.0) & \textbf{85.3} (0.9) & 80.4 (1.0) & 81.5 (1.0) \\
Australian & 72.1 (10.4) & 84.1 (7.5) & 92.5 (4.5) & \textbf{100} (0) \\
Bands & 48.3 (17.8) & 47.3 (17.6) & 58.5 (12.0) & \textbf{58.7} (12.0) \\
Credit & 81.3 (9.6) & 88.7 (6.4) & \textbf{92.0} (4.7) & 87.7 (6.4) \\
German & 52.0 (11.3) & 54.6 (11.2) & \textbf{69.2} (9.1) & 54.5 (10) \\
Heart & 80.3 (12.1) & 78.7 (13.1) & \textbf{86.3} (10.6) & 82.2 (11.2) 
\end{tabular}
\end{small}
\end{table}

\subsection{\textit{Credit Scoring} data} \label{subsec:exp_real}


Discretization, grouping and interaction screening are preprocessing steps relatively ``manually'' performed in the field of \textit{Credit Scoring}, using $\chi^2$ tests for each feature or so-called Weights of Evidence (\cite{zeng2014necessary}). This back and forth process takes a lot of time and effort and provides no particular statistical guarantee.

Table~\ref{tab:real_data} shows Gini coefficients of several portfolios for which there are $n=50,000$, $n=30,000$, $n=50,000$, $n=100,000$, $n=235,000$ and $n=7,500$ clients respectively and $d=25$, $d=16$, $d=15$, $d=14$, $d=14$ and $d=16$ features respectively. Approximately half of these features were categorical, with a number of factor levels ranging from $2$ to $100$. 

We compare the rather manual, in-house approach that yields the current performance, the na\"{\i}ve linear \gls{lr} and \textit{ad hoc} methods introduced in the previous section and finally our \textit{glmdisc} proposal. Beside the classification performance, interpretability is maintained and unsurprisingly, the learned representation comes often close to the ``manual'' approach: for example, the complicated in-house coding of job types is roughly grouped by \textit{glmdisc} into \textit{e.g.}\ ``worker'', ``technician'', \textit{etc.} Notice that even if the ``na\"{\i}ve'' \gls{lr} reaches some very decent predictive results, its poor interpretability skill (no quantization at all) excludes it from standard use in the company.

Our approach shows approximately similar results than MDLP/$\chi^2$, potentially due to the fact that contrary to the two previous experiments with simulated or UCI data, the classes are imbalanced ($< 3 \%$ defaulting loans), which would require special treatment while back-propagating the gradients~\cite{anand1993improved}. Note however that it is never significantly worse; for the Electronics dataset and as was the case for most UCI datasets, \textit{glmdisc} is significantly superior, which in the \textit{Credit Scoring} business might end up saving millions to the financial institution.

Table~\ref{tab:real_data_cont} is somewhat similar but is an earlier work: no CI is reported, only continuous features are considered so that pure discretization methods can be compared, namely MDLP and ChiMerge. Three portfolios are used with approx.\ 10 features and $n = 180{,}000$, $n = 30{,}000$, and $n = 100{,}000$ respectively. The proposed algorithm \textit{glmdisc}-SEM performs best, but is rather similar to the achieved performance of MDLP. ChiMerge does poorly since its parameter $\alpha$ (the rejection zone of the $\chi^2$ tests) is not optimized which is blatant on Portfolio 3 where approx.\ $2{,}000$ intervals are created, so that predictions are very ``noisy''.

The usefulness of discretization and grouping is clear on \textit{Credit Scoring} data and although \textit{glmdisc} does not always perform significantly better than the manual approach, it allows practitioners to focus on other tasks by saving a lot of time, as was already stressed out. As a rule of thumb, a month is generally allocated to data pre-processing for a single data scientist working on a single scorecard. On Google Collaboratory, and relying on Keras (\cite{chollet2015keras}) and Tensorflow (\cite{tensorflow2015-whitepaper}) as a backend, it took less than an hour to perform discretization and grouping for all datasets. As for the \textit{glmdisc}-SEM method, quantization of datasets of approx.\ $n = 10{,}000$ observations and approx.\ $d = 10$ take about 2 hours on a laptop within a single CPU core. On such a small rig, $n = 100{,}000$ observations and trying to perform interaction screening becomes however prohibitive (approx.\ 3 days). However, using higher computing power aside, there is still room for improvement, \textit{e.g.}\ parallel computing, replacing bottleneck functions with C++ code, etc. Moreover, the ChiMerge and MDLP methods implemented in the $\textsf{R}$ package \rinline{discretization} are not much faster while showing inferior performance and being capable of only discretization on non-missing values.



\begin{table}
    \centering
        \caption{Gini indices (the greater the value, the better the performance) of our proposed quantization algorithm \textit{glmdisc}, the two baselines of Table~\ref{tab:banchmark} and the current scorecard (manual / expert representation) obtained on several portfolios of Cr\'edit Agricole Consumer Finance.}
    \label{tab:real_data}
\begin{footnotesize}
%\begin{tabular}{lp{0.1\linewidth}p{0.138\linewidth}p{0.13\linewidth}p{0.15\linewidth}p{0.15\linewidth}l}
\begin{tabular}{llllll}
Portfolio & ALLR & \makecell{Current\\performance} & \makecell{\textit{ad hoc}\\methods} & \makecell{Our proposal:\\ \textit{glmdisc}-NN} & \makecell{Our proposal:\\ \textit{glmdisc}-SEM} \\
\hline
Automobile & \bf{59.3} (3.1) & 55.6 (3.4) & \bf{59.3} (3.0) & 58.9 (2.6) & 57.8 (2.9) \\
Renovation & 52.3 (5.5) & 50.9 (5.6) & 54.0 (5.1) & \bf{56.7} (4.8) & 55.5 (5.2) \\
Standard & 39.7 (3.3) & 37.1 (3.8) & \bf{45.3} (3.1) & 43.8 (3.2) & 36.7 (3.7) \\
Revolving & 62.7 (2.8) & 58.5 (3.2) & \bf{63.2} (2.8) & 62.3 (2.8) & 60.7 (2.8) \\
Mass retail & 52.8 (5.3) & 48.7 (6.0) & 61.4 (4.7) & \bf{61.8} (4.6) & 61.0 (4.7) \\
Electronics & 52.9 (11.9) & 55.8 (10.8) & 56.3 (10.2)  & \bf{72.6} (7.4) & 62.0 (9.5)
\end{tabular}
\end{footnotesize}
\end{table}


\begin{table}
    \centering
        \caption{Gini indices for three other portfolios of Cr\'edit Agricole Consumer Finance involving only continuous features and following three methods: ChiMerge, MDLP and \textit{glmdisc}-SEM compared to the current performance.}
    \label{tab:real_data_cont}
%\begin{footnotesize}
%\begin{tabular}{lp{0.1\linewidth}p{0.138\linewidth}p{0.13\linewidth}p{0.15\linewidth}p{0.15\linewidth}l}
\begin{tabular}{lllll}
Portfolio & \makecell{Current\\performance} & ChiMerge & MDLP & \makecell{Our proposal:\\ \textit{glmdisc}-SEM} \\
\hline
1 & 57.5 & 16.5 & \textbf{58.0} & \textbf{58.0} \\
2 & 27.0 & 26.7 & 29.2 & \textbf{30.0} \\
3 & 70.0 & 0 & \textbf{71.3} & \textbf{71.3}
\end{tabular}
%\end{footnotesize}
\end{table}

\section{Concluding remarks}

\subsection{Handling missing data}

For categorical features, handling missing data is straightforward: the level ``missing'' is simply considered a separate level, that can eventually be merged in the proposed algorithm with any other level. If it is \gls{mnar} (\textit{e.g.}\ co-borrower information missing because there is none) and such clients are significantly different from other clients in terms of creditworthiness, then such a treatment makes sense. If it is \gls{mar} and \textit{e.g.}\ highly correlated with some of the feature's levels (for example, the feature ``number of children'' could be either $0$ or missing to mean the borrower has no child), the proposed algorithm is highly likely to group these levels.

For continuous features, the same strategy can be employed: they can be encoded as ``missing'' and considered a separate level. However, this prevents this level to be merged with another one by having \textit{e.g.}\ a level ``$[0;200] \text{ or missing}$''.

\subsection{Integrating constraints on the cut-points}

Another problem that \gls{cacf} faces is to have interpretable cutpoints, \text{i.e.}\ having discretization intervals of the form $[0;200]$ and not $[0.389 ; 211.2]$ which are arguably less interpretable. But it is also highly subjective, and it would require the addition of an hyperparameter, namely the set of admissible discretization and / or the rounding to perform for each feature $j$ such that we did not pursue this problem. For the record, it is interesting to note that a straightforward rounding might not work: in the optimization community, it is well known that integer problems require special algorithmic treatment (dubbed integer programming). As an undergraduate, I applied some of these techniques to financial data in~\cite{projet_recherche} where I give a counterexample. Additionally, forcing estimated cutpoints to fall into a constrained set might drastically change predictive performance if levels collapse as on Figure~\ref{fig:constraint}.

\begin{figure}[!ht]
\begin{subfigure}[t]{0.5\textwidth}
\begin{tikzpicture}[scale=0.8]
      \draw[->] (-1,0) -- (5,0) node[right] {Amont of rent};
      \draw[->] (0,-1) -- (0,3) node[above] {Level};

		\node[scale=0.5,red,circle, fill] (x1) at  (0.5,0) {};
		\node[scale=0.5,red,circle, fill] (x2) at  (1.35,0) {};
		\node[scale=0.5,red,circle, fill] (x3) at  (2.2,0) {};
		\node[scale=0.5,red,circle, fill] (x4) at  (3.05,0) {};
		\node[scale=0.5,red,circle, fill] (x5) at  (3.9,0) {};

		\node[scale=0.7] (x1) at  (0.5,-0.5) {$0$};
		\node[scale=0.7] (x2) at  (1.35,-0.5) {$100$};
		\node[scale=0.7] (x3) at  (2.2,-0.5) {$200$};
		\node[scale=0.7] (x4) at  (3.05,-0.5) {$300$};
		\node[scale=0.7] (x5) at  (3.9,-0.5) {\ldots};


		\node[scale=0.5,red,circle, fill] (x1) at  (0,0.8) {};
		\node[scale=0.5,red,circle, fill] (x2) at  (0,1.6) {};
		\node[scale=0.5,red,circle, fill] (x3) at  (0,2.4) {};

		\node[scale=0.7] (x1) at  (-0.4,0.8) {$1$};
		\node[scale=0.7] (x2) at  (-0.4,1.6) {$2$};
		\node[scale=0.7] (x3) at  (-0.4,2.4) {$3$};

      \draw[-] (0.5,0.8) -- (1.7,0.8);
      \draw[-] (1.7,1.6) -- (2.6,1.6);
      \draw[-] (2.6,2.4) -- (4.5,2.4);

      \draw[dashed] (1.7,1.6) -- (1.7,0);
      \draw[dashed] (2.6,2.4) -- (2.6,0);

      \draw[<-] (1.35,0.4) -- (1.7,0.4);
      \draw[->] (1.7,0.4) -- (2.2,0.4) node[right] {?};

\end{tikzpicture}
\caption{If the admissible quantizations are those displayed in red, and the estimated cutpoints are the dashed lines, which ``rounding'' shall be chosen?}
\label{fig:constraint1}
\end{subfigure}
\begin{subfigure}[t]{0.5\textwidth}
\begin{tikzpicture}[scale=0.8]
      \draw[->] (-1,0) -- (5,0) node[right] {Amont of rent};
      \draw[->] (0,-1) -- (0,3) node[above] {Level};

		\node[scale=0.5,red,circle, fill] (x1) at  (0.5,0) {};
		\node[scale=0.5,red,circle, fill] (x2) at  (1.35,0) {};
		\node[scale=0.5,red,circle, fill] (x3) at  (2.2,0) {};
		\node[scale=0.5,red,circle, fill] (x4) at  (3.05,0) {};
		\node[scale=0.5,red,circle, fill] (x5) at  (3.9,0) {};

		\node[scale=0.7] (x1) at  (0.5,-0.5) {$0$};
		\node[scale=0.7] (x2) at  (1.35,-0.5) {$100$};
		\node[scale=0.7] (x3) at  (2.2,-0.5) {$200$};
		\node[scale=0.7] (x4) at  (3.05,-0.5) {$300$};
		\node[scale=0.7] (x5) at  (3.9,-0.5) {\ldots};


		\node[scale=0.5,red,circle, fill] (x1) at  (0,0.8) {};
		\node[scale=0.5,red,circle, fill] (x2) at  (0,1.6) {};
		\node[scale=0.5,red,circle, fill] (x3) at  (0,2.4) {};

		\node[scale=0.7] (x1) at  (-0.4,0.8) {$1$};
		\node[scale=0.7] (x2) at  (-0.4,1.6) {$2$};
		\node[scale=0.7] (x3) at  (-0.4,2.4) {$3$};

      \draw[-] (0.5,0.8) -- (1.7,0.8);
      \draw[-] (1.7,1.6) -- (2.6,1.6);
      \draw[-] (2.6,2.4) -- (4.5,2.4);

      \draw[dashed] (1.7,1.6) -- (1.7,0);
      \draw[dashed] (2.6,2.4) -- (2.6,0);

      \draw[<-] (1.35,0.4) -- (1.7,0.4);
      \draw[<-] (2.2,0.4) -- (2.6,0.4);

\end{tikzpicture}
\caption{In this setting, rounding has no side effect.}
\label{fig:constraint2}
\end{subfigure}
\begin{subfigure}[t]{0.5\textwidth}
\begin{tikzpicture}[scale=0.8]
      \draw[->] (-1,0) -- (5,0) node[right] {Amont of rent};
      \draw[->] (0,-1) -- (0,3) node[above] {Level};

		\node[scale=0.5,red,circle, fill] (x1) at  (0.5,0) {};
		\node[scale=0.5,red,circle, fill] (x2) at  (1.35,0) {};
		\node[scale=0.5,red,circle, fill] (x3) at  (2.2,0) {};
		\node[scale=0.5,red,circle, fill] (x4) at  (3.05,0) {};
		\node[scale=0.5,red,circle, fill] (x5) at  (3.9,0) {};

		\node[scale=0.7] (x1) at  (0.5,-0.5) {$0$};
		\node[scale=0.7] (x2) at  (1.35,-0.5) {$100$};
		\node[scale=0.7] (x3) at  (2.2,-0.5) {$200$};
		\node[scale=0.7] (x4) at  (3.05,-0.5) {$300$};
		\node[scale=0.7] (x5) at  (3.9,-0.5) {\ldots};


		\node[scale=0.5,red,circle, fill] (x1) at  (0,0.8) {};
		\node[scale=0.5,red,circle, fill] (x2) at  (0,1.6) {};
		\node[scale=0.5,red,circle, fill] (x3) at  (0,2.4) {};

		\node[scale=0.7] (x1) at  (-0.4,0.8) {$1$};
		\node[scale=0.7] (x2) at  (-0.4,1.6) {$2$};
		\node[scale=0.7] (x3) at  (-0.4,2.4) {$3$};

      \draw[-] (0.5,0.8) -- (0.9,0.8);
      \draw[-] (0.9,1.6) -- (2.6,1.6);
      \draw[-] (2.6,2.4) -- (4.5,2.4);

      \draw[dashed] (0.9,1.6) -- (0.9,0);
      \draw[dashed] (2.6,2.4) -- (2.6,0);

      \draw[<-] (0.5,0.4) -- (0.9,0.4);
      \draw[<-] (2.2,0.4) -- (2.6,0.4);

\end{tikzpicture}
\caption{The first level is ``collapsed'' and disappears, with possible consequences to the predictive task.}
\label{fig:constraint3}
\end{subfigure}
\begin{subfigure}[t]{0.5\textwidth}
\begin{tikzpicture}[scale=0.8]
      \draw[->] (-1,0) -- (5,0) node[right] {Amont of rent};
      \draw[->] (0,-1) -- (0,3) node[above] {Level};

		\node[scale=0.5,red,circle, fill] (x1) at  (0.5,0) {};
		\node[scale=0.5,red,circle, fill] (x2) at  (1.35,0) {};
		\node[scale=0.5,red,circle, fill] (x3) at  (2.2,0) {};
		\node[scale=0.5,red,circle, fill] (x4) at  (3.05,0) {};
		\node[scale=0.5,red,circle, fill] (x5) at  (3.9,0) {};

		\node[scale=0.7] (x1) at  (0.5,-0.5) {$0$};
		\node[scale=0.7] (x2) at  (1.35,-0.5) {$100$};
		\node[scale=0.7] (x3) at  (2.2,-0.5) {$200$};
		\node[scale=0.7] (x4) at  (3.05,-0.5) {$300$};
		\node[scale=0.7] (x5) at  (3.9,-0.5) {\ldots};


		\node[scale=0.5,red,circle, fill] (x1) at  (0,0.8) {};
		\node[scale=0.5,red,circle, fill] (x2) at  (0,1.6) {};
		\node[scale=0.5,red,circle, fill] (x3) at  (0,2.4) {};

		\node[scale=0.7] (x1) at  (-0.4,0.8) {$1$};
		\node[scale=0.7] (x2) at  (-0.4,1.6) {$2$};
		\node[scale=0.7] (x3) at  (-0.4,2.4) {$3$};

      \draw[-] (0.5,0.8) -- (2,0.8);
      \draw[-] (2,1.6) -- (2.6,1.6);
      \draw[-] (2.6,2.4) -- (4.5,2.4);

      \draw[dashed] (2,1.6) -- (2,0);
      \draw[dashed] (2.6,2.4) -- (2.6,0);

      \draw[<-] (2.2,0.4) -- (2,0.4);
      \draw[<-] (2.2,0.4) -- (2.6,0.4);

\end{tikzpicture}
\caption{The narrow middle level is collapsed, also with possible predictive consequences, since \textit{e.g.}\ rent amounts are often very concentrated around their mean.}
\label{fig:constraint4}
\end{subfigure}
\caption{Different settings of estimated quantizations and the consequences of constraints on the set of admissible cutpoints.}
\label{fig:constraint}
\end{figure}

\subsection{Wrapping up}

Feature quantization (discretization for continuous features, grouping of factor levels for categorical ones) in a supervised multivariate classification setting is a recurring problem in many industrial contexts. This setting was formalized as a highly combinatorial representation learning problem and a new algorithmic approach, named \textit{glmdisc}, has been proposed as a sensible approximation of a classical statistical information criterion.

The first proposed implementation relies on the use of a neural network of particular architecture and specifically a softmax approximation of each discretized or grouped feature. The second proposed implementation relies on an SEM algorithm and a polytomic multiclass \gls{lr} approximation in the same flavor as the softmax. These proposals can alternatively be replaced by any other univariate multiclass predictive model, which make them flexible and adaptable to other problems. Prediction of the target feature, given quantized features, was exemplified with \gls{lr}, although here as well, it can be swapped with any other supervised classification model.

The experiments showed that, as was sensed empirically by statisticians in the field of \textit{Credit Scoring}, discretization and grouping can indeed provide better models than standard \gls{lr}. This novel approach allows practitioners to have a fully automated and statistically well-grounded tool that achieves better performance than \textit{ad hoc} industrial practices at the price of decent computing time but much less of the practitioner's valuable time.

As described in the introduction, \gls{lr} is additive in its inputs which does not allow to take into account conditional dependency, as stated by~\cite{berry2010testing}. This problem is often dealt with by sparsely introducing ``interactions'', \textit{i.e.}\ products of two features. This leads again to a model selection challenge on a highly combinatorial discrete space that could be solved with a similar approach. In a broader context with no restriction on the predictive model, \cite{tsang2018detecting} already made use of neural networks to estimate the presence or absence of statistical interactions. We take another approach in the subsequent chapter where we tackle the parsimonious addition of pairwise interactions among quantized features, that might influence the quantization process introduced in this chapter.

\printbibliography[heading=subbibliography, title=References of Chapter 3]
%
% Cinquième chapitre
\chapter{Interaction discovery for logistic regression} \label{chap5}

\minitoc


\textcolor{red}{needs quote}

\textit{Nota Bene :} Ce chapitre s'inspire fortement ... \textcolor{red}{à adapter au moment de l'envoi du manuscrit}

\selectlanguage{english}

Continuing my pursuit of interpretable representation learning algorithms for logistic regression, I tackle in this chapter a common problem in \textit{Credit Scoring} and other application contexts relying either on logistic regression or additive models of the form $g(y) = \sum_{j=1}^d \glssymbol{bth}_j' \q_j(x_j)$. To further reduce the model bias discussed in Section~\ref{chap1:sec3} and thus obtain better predictive performance while maintaining interpretability, \textit{Credit Scoring} practitioners are used to introducing pairwise interactions.


\section{Motivation: XOR function}

As described in the introduction, logistic regression is linear in its inputs which does not allow to take into account conditional dependency, see~\cite{berry2010testing}. This problem is often dealt with by sparsely introducing ``interactions'', \textit{i.e.}\ products of two features. Unfortunately, this leads again to a model selection challenge as the number of pairs of features is $\dfrac{d(d-1)}{2}$. We denote by $\bdelta$ the upper triangular matrix with $\delta_{k,\ell} = 1$ if $k < \ell$ and features $k$ and $\ell$ ``interact'' in the logistic regression in the sense of~\cite{berry2010testing}. The logistic regression with interactions $\bdelta$ is thus:
\begin{equation} \label{eq:reglog_sans}
\text{logit}[p_{\glssymbol{bth}}(1|\bm{f}(\bm{x}),\bdelta)] = \theta_{0} + \sum_{j=1}^d \theta_{j}^{f_j(x^j)} + \sum_{1 \leq k < \ell \leq d} \delta_{k,\ell} \theta_{k,\ell}^{f_p(x^p),f_q(x^q)},
\end{equation}
where $\glssymbol{bth} = (\theta_{0},\theta_{1}^1,\dots,\theta_{1}^{m_1},\dots,\theta_{d}^1,\dots,\theta_{d}^{m_d},\theta_{1,2}^{1,1},\dots,\theta_{1,2}^{m_1,m_2},\dots,\theta_{d-1,d}^{1,1},\dots,\theta_{d-1,d}^{m_{d-1},m_d})$ and for all features $j$, $f_j(x_j)=m_j$ is set as the ``reference'' value and consequently for all $j$, $\theta_{j}^{m_j}=0$ and for all $1 \leq k < \ell \leq d$, $\theta_{k,\ell}^{m_k,m_{\ell}}=0$.


\section{Pairwise interaction screening as a feature selection problem}

\textcolor{red}{eq:criterion undefined = c'est le BIC du chapitre précédent à harmoniser}

\textcolor{red}{sec:discwithout undefined = à reprendre}

Criterion~\eqref{eq:criterion} developed in the context of discretization and grouping can be adapted to take into account interactions
\begin{equation}\label{eq:criterion_inter}
\bm{f}^{\text{opt}},\bdelta^{\text{opt}} = \argmin_{\bm{f},\bdelta} \text{BIC}[p_{\glssymbol{bth}}(\bm{y}|\bm{f}(\bm{x}),\bdelta)].
\end{equation}
The combinatorics involved in this problem are much higher than those of criterion~\eqref{eq:criterion}, which already lead to an untractable greedy approach. For each feasible discretization scheme of Section~\ref{sec:discwithout}, there is now $2^{\frac{d(d-1)}{2}}$ models to test! In the following section, we will first consider the discretization fixed and develop a stochastic approach similar to the one proposed for discretization and grouping of factor levels.

With a fixed discretization scheme $\bm{e} = \bm{f}(\glssymbol{bx})$, criterion~\ref{eq:criterion_inter} amounts to $\bdelta^{\text{opt}} = \argmin_{\bdelta} \text{BIC}[p_{\glssymbol{bth}}(y|\bm{e},\bdelta)]$ which optimization through a greedy approach is untractable with more than a few features ($d > 10$).

\section{A novel model selection approach}


%\section{Discretization and levels merging \textit{with} interactions} \label{sec:discwith}



%\subsection{With a fixed discretization}


Again, $\bdelta$ can be seen as an observation of a latent random matrix and $\bdelta^{\text{opt}}$ can be found using a stochastic approach. The BIC criterion has a desirable property relating it to the posterior probability of the model given the data, \textit{i.e.}\ $p(\bdelta | \be, y) \propto p(y | \be, \bdelta) p(\bdelta) \approx \exp(-\text{BIC}[p_{\glssymbol{bth}}(y|\bm{e},\bdelta)]/2)  p(\bdelta)$. Consequently, one can design an MCMC algorithm like Metropolis-Hastings~\cite{hastings1970monte} which should draw ``good'' interaction matrices $\bdelta$ (\textit{i.e.}\ close to $\bdelta^{\text{opt}}$) from the target distribution $p(\bdelta | \be, y)$.

\section{Interaction screening with a fixed quantization}

Metropolis-Hastings only requires a proposal of a transition probability between two states of the Markov chain $q: ({\{0,1\}}^{\frac{d(d-1)}{2}},{\{0,1\}}^{\frac{d(d-1)}{2}}) \mapsto \mathbb{R}$, which would require to compute $2^{d(d-1)}$ probabilities (\textit{i.e.}\ one per unique couple of matrices ($\bdelta^{(1)},\bdelta^{(2)}$)). It is thus desirable to reduce this combinatorics by making further assumptions. In what follows, we restrict possible transitions to matrices that are on a one unit $L^1$ distance to the current interaction matrix, s.t.\ $q(\bdelta^{(1)},\bdelta^{(2)}) = 0$ if $\sum_{k=1}^d \sum_{\ell=1}^d |\delta^{(1)}_{k,\ell} - \delta^{(2)}_{k,\ell}| \neq 1$.

Only $\frac{d(d-1)}{2}$ coefficients are now needed, which can be reinterpreted as the probability to switch on (resp. off) an entry of $\bdelta^{(1)}$ which is currently off (resp. on). We claim that a good intuition about whether two features interact is the relative gain (or loss) in BIC between their bivariate model \textit{with} their interaction and this model \textit{without} their interaction. The rational behind such a procedure is the following: $\forall \: 1 \leq k < \ell \leq d, \; p(\delta_{k,\ell}=1 | e_k, e_{\ell}, y) \propto p_{\glssymbol{bth}}(y | e_k, e_{\ell}, \delta_{k,\ell}=1) p(\delta_{k,\ell}=1) \approx \exp(-\text{BIC}[p_{\glssymbol{bth}}(y | e_k, e_{\ell}, \delta_{k,\ell}=1)]/2) p(\delta_{k,\ell}=1)$. Setting a uniform prior $p(\delta_{k,\ell}=1) =\begin{cases} 0 \text{ if } k \geq \ell \\ \frac{1}{2} \text{ otherwise.} \end{cases}$ and writing the same expression for $\delta_{k,\ell} = 0$ yields: $p_{k,\ell} = p(\delta_{k,\ell}=1 | e_k, e_{\ell}, y) \approx \exp \left( \frac{\text{BIC}[p_{\glssymbol{bth}}(y | e_k, e_{\ell}, \delta_{k,\ell}=0)]-\text{BIC}[p_{\glssymbol{bth}}(y | e_k, e_{\ell}, \delta_{k,\ell}=1)]}{2} \right)$. We normalize $p_{k,\ell}$ s.t.\ $\sum_{1 \leq k < \ell \leq d} p_{k,\ell} = 1$.

We claim that if $p_{k,\ell}$ is close to $1$ (resp. $0$), then there is a strong chance that $\delta_{k,\ell}^\star = 1$ (resp. $\delta_{k,\ell}^\star = 0$) even in the full multivariate model, which is in particular true if features $e_k$ and $e_{\ell}$ are independent to other features $e_r$, $r \neq k$, $r \neq \ell$. Consequently, if at step $(s)$ of the Markov chain, $\delta_{k,\ell}^{(s)} = 1$ (resp. $0$) and $p_{k,\ell}$ is close to $0$ (resp. $1$), a good candidate for $\bdelta^{(s+1)}$ should be to change $\delta_{k,\ell}$ to $\delta_{k,\ell}^{(s+1)} = 0$ (resp. $\delta_{k,\ell}^{(s+1)} = 1$). Our proposal is thus to calculate the difference between the current interaction matrix and $(p_{k,\ell})_{1 \leq k,\ell, \leq d}$ which we denote by $q_{k,\ell}^{(s)} = |\delta_{k,\ell}^{(s)} - p_{k,\ell}|$ and normalize. This defines a proper transition probability between two interaction matrices: $q(\bdelta^{(s)},\bdelta') = \begin{cases} 0 \text{ if } \sum_{k=1}^d \sum_{\ell=1}^d |\delta^{(s)}_{k,\ell} - \delta_{k,\ell}'| \neq 1, \\ q_{k,\ell}^{(s)} \text{ for the unique couple } (k,\ell) \text{ s.t.} \: \delta^{(s)}_{k,\ell} \neq \delta_{k,\ell}'. \end{cases}$

Now, a Metropolis-Hastings step can be conducted by drawing $\bdelta' \sim q(\bdelta^{(s)},\cdot)$. The acceptance probability of this candidate is given by $\alpha = \min \left( 1, \frac{p(\bdelta' | \be,y)}{p(\bdelta^{(s)} | \be, y)} \frac{1-q(\bdelta^{(s)},\bdelta')}{q(\bdelta^{(s)},\bdelta')} \right) \approx \min \left( 1, \exp \left( \frac{\text{BIC}[p_{\glssymbol{bth}}(y | \be, \bdelta^{(s)})] - \text{BIC}[p_{\glssymbol{bth}}(y | \be, \bdelta')]}{2} \right) \frac{1-q(\bdelta^{(s)},\bdelta')}{q(\bdelta^{(s)},\bdelta')} \right)$. If $\alpha \geq 1$, the candidate is accepted and $\bdelta^{(s+1)} = \bdelta'$; otherwise, the candidate is accepted with probability $\alpha$ s.t.\ $\bdelta^{(s+1)} = \begin{cases} \bdelta' \text{ with probability } \alpha, \\ \bdelta^{(s)} \text{ with probability } 1-\alpha. \end{cases}$

The existence of the stationary distribution $p(\bdelta | \be,y)$ is guaranteed by construction of the Metropolis-Hastings algorithm as the generated Markov chain fulfills the detailed balance condition. The uniqueness of the stationary distribution is given by the ergodicity of the Markov chain: as $\forall \: 1 \leq  k < \ell \leq d, \: q_{k,\ell} > 0$ and a transition changes only one entry $\delta_{k,\ell}$ of the interaction matrix, every state can be reached in at most $\frac{d(d-1)}{2}$ steps.

In practice with a fixed discretization scheme, this stochastic approach is probably outperformed in computing time by Lasso-based methods or correlation-based methods like~\cite{simon}, which might obtain a suboptimal model in a fixed computing time, contrary to our approach which might take lots of steps to converge in distribution. Its double benefit however lies in the ability of the practitioner to define before-hand how many steps shall be performed and the natural integration to the discretization and grouping of factor levels algorithm proposed in the previous section, which we develop in the next one.

\section{Interaction screening and quantization}

Quand on connaît $\delta$, on tire $E$ comme avant.

Quand on connaît $E$, on tire $\delta$ comme montré ci-dessus.

L'algorithme de Gibbs fonctionne comme ça.



\section{Numerical experiments}


\subsection{Experiments on simulated data \textit{with} interactions}

\subsubsection{With a fixed discretization}





\subsubsection{While discretizing and grouping}

\begin{figure}
\centering
\resizebox{\linewidth}{6cm}{%
\input{figures/chapitre5/plot3_3}
}
\caption{\label{fig:simulated_interaction}Distribution of the kind of interactions chosen by \textit{glmdisc} on 100 simulations.}
\end{figure}




\begin{figure}
\centering
\begin{tikzpicture}
\tikzset{vertex/.style = {shape=circle,draw,minimum size=1.5em}}
\tikzset{edge/.style = {->,> = latex'}}
% vertices
\node[vertex] (x1) at  (0,1.5) {$\glssymbol{X}^1$};
\node[vertex] (xj) at  (0,0) {$\glssymbol{X}^j$};
\node[vertex] (xd) at  (0,-1.5) {$\glssymbol{X}^{d}$};

\node[vertex] (delta) at  (2.5,3) {$\delta$};

\node[vertex] (q1) at  (2.5,1.5) {$Q^1$};
\node[vertex] (qj) at  (2.5,0) {$Q^j$};
\node[vertex] (qd) at  (2.5,-1.5) {$Q^{d}$};

\node[vertex] (y) at (5,0) {$\glssymbol{Y}$};

%edges
\draw[edge] (x1) to (delta);
\draw[edge] (xj) to (delta);
\draw[edge] (xd) to (delta);

\draw[edge] (delta) to (y);

\draw[edge] (x1) to (q1);
\draw[edge] (xj) to (qj);
\draw[edge] (xd) to (qd);
\draw[edge] (q1) to (y);
\draw[edge] (qj) to (y);
\draw[edge] (qd) to (y);

\draw[dashed] (x1) to (xj);
\draw[dashed] (xj) to (xd);

\draw[dashed] (q1) to (qj);
\draw[dashed] (qj) to (qd);
\end{tikzpicture}
\caption{\label{fig:dep}Dépendance entre $\glssymbol{X}^j$,$\E^j$ et $\glssymbol{Y}$} 
\end{figure}




\subsection{Benchmark of \textit{glmdisc} against other approaches} \label{sec:exp}

\subsection{Simulated data from a misspecifed model}

\subsection{Real data from Crédit Agricole Consumer Finance}

\section{Conclusion} \label{sec:ccl}

The essentially industrial problem of introducing pairwise interactions in a supervised multivariate classification setting was formalized and a new approach, relying on a Metropolis-Hastings algorithm has been proposed. This algorithm relies on the use of logistic regression, although other predictive models can be plugged in place of $p_{\glssymbol{bth}}$.

The true underlying motivation was to perform intetraction screening while quantizing data using the approach developed in the preceding Chapter: \textit{glmdisc}.
The experiments showed that, as was sensed empirically by statisticians in the field of \textit{Credit Scoring}, interactions between quantized features can indeed provide better models than without interactions, or standard logistic regression. This novel approach allows practitioners to have a fully automized and statistically well-grounded tool that achieves better performance than both \textit{ad hoc} industrial practices and academic discretization heuristics at the price of decent computing time but much less of the practitioner's valuable time.

The code used for numerical experiments is available as packages, see Appendix~\ref{app2}.














\printbibliography[heading=subbibliography, title=References of Chapter 4]

%
% Sixième chapitre
\chapter{Tree-structure segmentation for logistic regression} \label{chap6}

\minitoc

\textcolor{red}{needs quote}

\textit{Nota Bene :} Ce chapitre s'inspire fortement ... \textcolor{red}{à adapter au moment de l'envoi du manuscrit}

\selectlanguage{english}




\section{Introduction}

\subsection{Context}

En matières de crédit à la consommation, les instituts financiers cherchent à automatiser la d´ecision de financement tout en ne s´electionnant que les clients susceptibles
de rembourser ledit cr´edit. Depuis une quarantaine d’ann´ees, le Credit Scoring consiste `a construire des mod`eles
de classification supervis´ee $p_{\glssymbol{bth}}$ `a partir des donn´ees demand´ees au clients x = (xj )
d
1
et de l’observation du
remboursement des clients pass´es $y \in \{0, 1\}$. Historiquement, des scores diff´erents sont d´evelopp´es sur des
march´es (e.g. grande distribution, ´electrom´enager, . . . ) et/ou des produits (e.g. renouvelable, amortissable, . . . )
et/ou des partenaires et/ou des profils clients diff´erents dans l’esprit de la figure 1. Ce d´ecoupage est historique
et rel`eve d’un a priori. On cherche ici `a rationnaliser cette pratique en consid´erant le cluster d’appartenance du
client comme un param`etre `a optimiser. Si l’on note K le nombre de scores `a construire (inconnu) et c = 1..K
chaque score, correspondant `a un cluster de clients, le m´elange de r´egressions logistiques s’´ecrit :
$p(y|x) = \sum_{c=1}^K p_{\glssymbol{bth}_c}(y|x, c)p(c|x)$,
o`u l’on restreint p(c|x) `a prendre la forme de la figure 1, de telle sorte que le m´elange n’est pas “flou” comme
pour un mod`ele de m´elange classique o`u la contribution de chaque classe est pond´er´ee par sa probabilit´e. La
difficult´e d’une approche directe r´eside dans cette contrainte discr`ete.


\tikzstyle{level 1}=[level distance=1.5cm, sibling distance=7cm]
\tikzstyle{level 2}=[level distance=1.5cm, sibling distance=4cm]
\tikzstyle{level 3}=[level distance=2cm, sibling distance=2cm]

\begin{figure}
\resizebox{\textwidth}{!}{
\centering
\begin{tikzpicture}
  [
    sibling distance        = 15em,
    level distance          = 5em,
    edge from parent/.style = {draw, -latex},
    every node/.style       = {font=\footnotesize},
    sloped
  ]
  \node [root] {\textcolor{black}{Clientèle}}
    child { node [dummy] {}
      child { node [dummy] {}
        child { node [env] {\textcolor{black}{$p_{\theta_1}(y|x)$}}
          edge from parent node [below] {Retraités} }
        child { node [env] {\textcolor{black}{$p_{\theta_2}(y|x)$}}
          edge from parent node [above] {Salariés} }
        child { node [env] {\textcolor{black}{$p_{\theta_3}(y|x)$}}
                edge from parent node [above] {Autres} }
        edge from parent node [above] {Crédit renouvelable} }
      child { node [env] {\textcolor{black}{$p_{\theta_4}(y|x)$}}
              edge from parent node [above, align=center]
                {Amortissable} }
              edge from parent node [above] {Electroménager} }
    child { node [dummy] {}
      child { node [dummy] {}
        child { node [env] {\textcolor{black}{$p_{\theta_5}(y|x)$}}
          edge from parent node [above] {Location} }
        child { node [env] {\textcolor{black}{$p_{\theta_6}(y|x)$}}
                edge from parent node [above] {Amortissable} }
        edge from parent node [above] {Fiat} }
      child { node [env] {\textcolor{black}{$p_{\theta_7}(y|x)$}}
              edge from parent node [above, align=center]
                {Kawasaki} }
              edge from parent node [above] {Automobile} };
\end{tikzpicture}
}
\caption{Cartographie simplifiée de la chaîne de construction des scores.} 
\label{fig:arbre}
\end{figure}




\subsection{In-house \textit{ad hoc} practice}





\section{Litterature review}


\subsection{Clustering methods}


\subsection{Direct approaches: logistic regression trees}



\section{Logistic regression trees as a combinatorial model selection problem}


\subsection{Cardinality example}


\subsection{Logistic regression tree selection}



\section{A mixture and latent feature-based relaxation}




\section{A stochastic estimation strategy}


\section{Numerical experiments}


\subsection{Empirical consistency on simulated data}

\subsection{Benchmark on \textit{Credit Scoring} data}




\bigskip

Ce chapitre.


\printbibliography[heading=subbibliography, title=References of Chapter 5]


%
% Septième chapitre
\chapter{High dimensional data in \textit{Credit Scoring}} \label{chap7}

\selectlanguage{english}

\epigraph{All [problems due to high dimension] may be subsumed under the heading “the curse of dimensionality”. Since this is a curse, [...], there is no need to feel discouraged about the possibility of obtaining significant results despite it.}{R. Bellman, ``Dynamic programming'', 1957}

\minitoc

Various sources estimate the growth of created data to be exponential. However, the difficulty of processing these data has superseded the difficulty of storing them: ``data is the new oil'' is the catch-phrase often repeated in industry. While this oil has been extensively extracted and stored in a lot of application contexts, including \textit{Credit Scoring}, there is not always a motor capable of burning it. \textit{Scalability} refers to the problem of applying an existing method to increasingly more data. It turns out that, either by lack of computing power and / or by statistical properties or assumptions not met, not all methods are scalable.
Consequently, the statistics and machine learning communities have already tackled lots of problems stemming specifically from large $n$ and / or large $d$ settings.
These problems form a vast literature and are out of the scope of the present work.
The aim of this Chapter is to give a concise context of high-dimensional data w.r.t.\ the \textit{Credit Scoring} industry, what problems does it give rise to, and some simple existing solutions from an eluded literature review.

\section{Motivation}

This first Section aims at presenting this well-known paradigm in the context of \textit{Credit Scoring} and the two sub-problems that were identified and tackled in this Chapter.

\subsection{Industrial context}

Being a CIFRE PhD, .

\subsubsection{Traditional longitudinal data}


\subsubsection{Transactional data}

\paragraph{Payment data}

Once a loan has been , monthly payments due by clients are most of the time debited from their main bank account. These debit might be accepted or refused by their bank depending on their balance. This is exactly the basis of the approximate . Such data are presented on Table~\ref{tab:payment_data}.

\begin{table}[ht]
    \centering
    \caption{Payment data.}
    \label{tab:payment_data}
    \begin{small}
\begin{tabular}{lllllll}
Client & Date & Should pay & Has paid & Type & Outstanding & Status \\
 \hline
1 & 05/01/2019:10:00:00 & 50 & 0 & Automatic debit & 5{,}000 & Refused \\
1 & 08/01/2019:10:00:00 & 50 & 50 & Automatic debit & 4{,}950 & Accepted \\
1 & 05/02/2019:10:00:00 & 50 & 0 & Automatic debit & 5{,}000 & Refused \\
1 & 08/02/2019:10:00:00 & 50 & 0 & Automatic debit & 4{,}950 & Refused
\end{tabular}
    \end{small}
\end{table}


\paragraph{Recovery data}

In the case of Client 1 from the previous example in Table~\ref{tab:payment_data}, once the automatic debit is refused, it enters a recovery process that can be long and complex and is way out of the scope of the present manuscript. It creates however tremendous amounts of data, that could be used in the context of \textit{Credit Scoring}, \textit{e.g.}\ for better assessment of the class good / bad borrower or as predictive features for a known client that would apply for another loan.

\begin{table}[ht]
    \centering
    \caption{Monthly per-client recovery data.}
    \label{tab:recovery_data}
    \begin{small}
\begin{tabular}{lllllll}
Client & Date & Should pay & Fees & Has paid & Type & Status \\
 \hline
1 & 09/02/2019:11:24:12 & 50 & 10 & 0 & 4{,}960 & Manual recovery by phone \\
1 & 09/02/2019:11:26:09 & 60 & 0 & 60 & 4{,}900 & Credit card payment \\
\end{tabular}
    \end{small}
\end{table}

\paragraph{Credit card data}

Transactions from credit card holders are recorded and can easily be retrieved. They are well-structured but contain lots of text fields, as exemplified on Table~\ref{tab:credit_card_data}.

\begin{table}[ht]
    \centering
    \caption{Daily per-client credit card data.}
    \label{tab:credit_card_data}
    \begin{tiny}
\begin{tabular}{lllllll}
Client & Date & Amount & Company & Location & Category & \dots \\
 \hline
1 & 01/01/2019:09:05:18 & 10.9 & Amazon & Online & Online retail & \dots \\
1 & 01/01/2019:12:50:25 & 14.5 & Les 3 Brasseurs & 22 Place de la Gare, 59800 LILLE & Restaurant & \dots \\
1 & 02/01/2019:19:10:20 & 78.9 & Carrefour & 1 Avenue Willy Brandt, 59000 LILLE & Retail consumer goods & \dots 
\end{tabular}
    \end{tiny}
\end{table}


\paragraph{Log data}

In the same fashion as online retailers adjust the layout of their products given the information gathered on the potential customer through its visitation pattern, .

\begin{table}[ht]
    \centering
    \caption{Log data.}
    \label{tab:log_data}
    \begin{small}
\begin{tabular}{lllllll}
Client & Platform & Device & Date & URL \\
 \hline
1 & Leboncoin & MAC OS & 10/01/2019:22:33:50 &  \\
1 & Main site & MAC OS & 10/01/2019:22:34:10 &  \\
1 & Main site & MAC OS & 10/01/2019:22:34:30 &  \\
1 & Main site & MAC OS & 10/01/2019:22:34:10 &  \\
\end{tabular}
    \end{small}
\end{table}





\paragraph{Marketing data}

Finally, clients often apply to loans after having been exposed to diverse forms of adverts, some of which can be properly recorded and affected to a client, \textit{e.g.}\ mailing or e-mailing campaigns, Google AdWords, etc. An example of such data is visible on Table~\ref{tab:marketing_data}. These data can be very informative of the target good / bad borrower of each client: a prospective client coming from AdWords in the middle of the night might be riskier than a targeted prospect via an opened email on a week-end afternoon for example.

\begin{table}[ht]
    \centering
    \caption{Marketing data.}
    \label{tab:marketing_data}
    \begin{tiny}
\begin{tabular}{lllllll}
Client & Marketing lever & Date & Device & Opened & Visited & URL \\
 \hline
1 & email & 02/03/2019:15:02:54 & Android & Yes & No & /media/new\_credit\_ad/car\_loan\&id=1\&\dots \\
1 & mail & 02/04/2019:10:00:00 & NA & NA & NA & NA \\
1 & Google Adword & 15/04/2019:12:10:10 & Windows & NA & Yes & /adword/personal\_credit\&id=1\&\dots \\
\end{tabular}
    \end{tiny}
\end{table}

All these kinds of data are not directly used by \gls{cacf} in its \textit{Credit Scoring} practices, although by simply looking at the exemplary Tables, one is able to draw simple intuitions of signals of low / high risk of default. In the subsequent Section, two problems pertaining the usage of these data, justifying in a sense why they were not used to this day, are identified and formalized.


\subsection{Two identified sub-problems}

A very simple way of dealing with all examples of additional data of the preceding Section is to add them as columns of our ``traditional'' longitudinal data. Taking Table~\ref{tab:credit_card_data} as an example, each credit card transaction can be reshaped so as to fit in separate columns relative to payment \#1, payment \#2, etc. This would yield Table~\ref{tab:example_longitudinal}.


\begin{table}[ht]
    \centering
    \caption{Long data.}
    \label{tab:example_longitudinal}
    \begin{tiny}
\begin{tabular}{lllllll}
Client & Marketing lever & Date & Device & Opened & Visited & URL \\
 \hline
1 & email & 02/03/2019:15:02:54 & Android & Yes & No & /media/new\_credit\_ad/car\_loan\&id=1\&\dots \\
1 & mail & 02/04/2019:10:00:00 & NA & NA & NA & NA \\
1 & Google Adword & 15/04/2019:12:10:10 & Windows & NA & Yes & /adword/personal\_credit\&id=1\&\dots \\
\end{tabular}
    \end{tiny}
\end{table}

As a consequence, we are artificially back to a traditional longitudinal setting with a very high number of covariates $d$.

%\subsubsection{Using big $n$, big $d$ data}
%
%\subsubsection{Using unstructured and / or fine-grained data}



\section{Longitudinal data in high dimension}

This Section is an eluded literature review of problems that arise in high dimension for ``classical'' longitudinal data. It was first tackled by bio-statisticians working with omics data, such as DNA that can span over thousands of features for each patient, which yields a situation where more features than observations are available!

\subsection{The $d > n$ setting}

In the next Section, we review the statistical properties associated with the curse of dimensionality mentioned in the epigraph of this Chapter and attributed to .

\subsection{The curse of dimensionality}


\subsection{The blessings of dimensionality}


\subsection{Dimension reduction}

A straightforward way of avoiding the curse(s) of dimensionality is to get back to a small dimension $d'$ relative to $n$ by pre-processing the $d$ features. In Chapter~\ref{chap4}, and particularly Section~\ref{}, it was argued that quantization could be thought of dimensionality reduction, because information was compressed in intervals and ``meta''-groups for continuous and categorical features respectively without affecting predictive power (on the contrary!). Two way more classical ways of performing dimensionality reduction are presented here: combining original features in principal components, which was already discussed in Chapter~\ref{chap6} when building segments of clients, and feature selection, which are the subjects of the two subsequent Section respectively.

\subsubsection{By combining input features}



\subsubsection{By selecting input features}


\section{New data types in a supervised classification setting}


\bigskip

This last Chapter closes my journey of some shortcomings of current practices of \textit{Credit Scoring}. It opens .


\printbibliography[heading=subbibliography, title=References of Chapter 6]
%
% Chapitre  de conclusion (générale)
%%%%%%%%%%%%%%%%%%%%%%%%%%%%%%%%%%%%%%%%%%%%%%%%%%%%%%%%%%%%%%%%%%%%%%%%%%%%%%%
%\chapter*{Conclusion: High dimensional unstructured data in \textit{Credit Scoring}} \label{ccl}
\chapter*{Conclusion and prospects} \label{ccl}

\selectlanguage{english}

\epigraph{It's not complicated, it's just a lot of it.}{Richard Feynman, interview for \textit{The World from Another Point of View}, Yorkshire Television, 1972.}

\minitoc

Various sources estimate the growth of created data to be exponential. However, the difficulty of processing these data has superseded the difficulty of storing them: ``data is the new oil'' is the catch-phrase often repeated in industry. While this oil has been extensively extracted and stored in a lot of application contexts, including \textit{Credit Scoring}, there is not always a motor capable of burning it. \textit{Scalability} refers to the problem of applying an existing method to increasingly more data. It turns out that, either by lack of computing power and / or by statistical properties or assumptions not met, not all methods are scalable.
Consequently, the statistics and machine learning communities have already tackled lots of problems stemming specifically from large $n$ and / or large $d$ settings.
These problems form a vast literature and are out of the scope of the present work.
The aim of these concluding remarks is to give a concise context of high-dimensional data w.r.t.\ the \textit{Credit Scoring} industry, what problems does it give rise to, and some simple existing solutions from an eluded literature review.

\section{Motivation}

This first section aims at presenting the data currently collected but remaining unused in the context of \textit{Credit Scoring} and the two sub-problems that were identified and tackled in this chapter.

\subsection{Industrial context}

Technological advancements in big data storage and processing has sparked interest in exploiting these for \textit{Credit Scoring}, although most hereafter presented data sources were available for quite some time\dots

\paragraph{Payment data}

Once a loan has been granted, monthly payments due by clients are most of the time debited from their main bank account. These debit might be accepted or refused by their bank depending on their balance and several tries might be performed before going into a recovery process. Such data are presented on Table~\ref{tab:payment_data}.

\begin{table}[ht]
    \centering
    \caption{Payment data.}
    \label{tab:payment_data}
    \begin{small}
\begin{tabular}{lllllll}
Client & Date & Should pay & Has paid & Type & Outstanding & Status \\
 \hline
1 & 05/01/2019:10:00:00 & 50 & 0 & Automatic debit & 5{,}000 & Refused \\
1 & 08/01/2019:10:00:00 & 50 & 50 & Automatic debit & 4{,}950 & Accepted \\
1 & 05/02/2019:10:00:00 & 50 & 0 & Automatic debit & 4{,}950 & Refused \\
1 & 08/02/2019:10:00:00 & 50 & 0 & Automatic debit & 4{,}950 & Refused
\end{tabular}
    \end{small}
\end{table}


\paragraph{Recovery data}

In the case of Client 1 from the previous example in Table~\ref{tab:payment_data}, once the second automatic debit is refused, it enters a recovery process that can be long and complex and is way out of the scope of the present manuscript. It creates however tremendous amounts of data, that could be used in the context of \textit{Credit Scoring}, \textit{e.g.}\ for better assessment of the class to predict (good / bad borrower) or as predictive features for a known client that applies for another loan.

\begin{table}[ht]
    \centering
    \caption{Monthly per-client recovery data.}
    \label{tab:recovery_data}
    \begin{small}
\begin{tabular}{lllllll}
Client & Date & Should pay & Fees & Has paid & Outstanding & Status \\
 \hline
1 & 09/02/2019:11:24:12 & 50 & 10 & 0 & 4{,}960 & Manual recovery by phone \\
1 & 09/02/2019:11:26:09 & 60 & 0 & 60 & 4{,}900 & Debit card payment \\
\end{tabular}
    \end{small}
\end{table}

\paragraph{Credit card data}

Transactions from credit card holders are recorded and can easily be retrieved. They are well-structured but contain lots of text fields, as exemplified on Table~\ref{tab:credit_card_data}.

\begin{table}[ht]
    \centering
    \caption{Daily per-client credit card data.}
    \label{tab:credit_card_data}
\begin{tiny}
\resizebox{\textwidth}{!}{\begin{tabular}{lllllll}
Client & Date & Amount & Company & Location & Category & \dots \\
 \hline
1 & 01/01/2019:09:05:18 & 10.9 & Amazon & Online & Online retail & \dots \\
1 & 01/01/2019:12:50:25 & 14.5 & Les 3 Brasseurs & 22 Place de la Gare, 59800 LILLE & Restaurant & \dots \\
1 & 02/01/2019:19:10:20 & 78.9 & Carrefour & 1 Avenue Willy Brandt, 59000 LILLE & Retail consumer goods & \dots 
\end{tabular}
}
\end{tiny}
\end{table}


\paragraph{Log data}

In the same fashion as social media users are targeted with personalized ads thanks to their visitation pattern~\cite{Yan:2017:YAY:3109859.3109923}, connexion logs can be used to personalize the loan offer in terms of amount, rate, \dots An example is given in Table~\ref{tab:log_data}.

\begin{table}[ht]
    \centering
    \caption{Log data.}
    \label{tab:log_data}
    \begin{small}
\begin{tabular}{lllllll}
Client & Platform & Device & Date & URL \\
 \hline
1 & Leboncoin & MAC OS & 10/12/2018:22:33:50 & /leboncoin/Nord/Electromenager/ \\
1 & Main site & MAC OS & 10/12/2018:22:34:10 & /sofinco/home \\
1 & Main site & MAC OS & 10/12/2018:22:34:30 & /sofinco/perso/electromenager \\
1 & Main site & MAC OS & 10/12/2018:22:35:12 & /sofinco/simulation \\
\end{tabular}
    \end{small}
\end{table}





\paragraph{Marketing data}

Finally, clients often apply to loans after having been exposed to diverse forms of adverts, some of which can be properly recorded and affected to a client, \textit{e.g.}\ mailing or e-mailing campaigns, Google AdWords, etc. An example of such data is visible on Table~\ref{tab:marketing_data}. These data can be very informative of the class (good / bad borrower) of each client: a prospective client coming from AdWords in the middle of the night might be riskier than a targeted prospect via an email on a week-end afternoon for example.

\begin{table}[ht]
    \centering
    \caption{Marketing data.}
    \label{tab:marketing_data}
\begin{tiny}
\resizebox{\textwidth}{!}{\begin{tabular}{lllllll}
Client & Marketing lever & Date & Device & Opened & Visited & URL \\
 \hline
1 & email & 11/12/2018:15:02:54 & Android & Yes & No & /media/new\_credit\_ad/car\_loan\&id=1\&\dots \\
1 & mail & 12/12/2018:10:00:00 & NA & NA & NA & NA \\
1 & Google Adword & 13/12/2019:12:10:10 & Windows & NA & Yes & /adword/personal\_credit\&id=1\&\dots \\
\end{tabular}}
\end{tiny}
\end{table}

All these kinds of data are not directly used by \gls{cacf} in its \textit{Credit Scoring} practices, although by simply looking at the exemplary tables, one is able to draw simple intuitions of signals of low / high risk of default. In the subsequent section, two problems pertaining the usage of these data, justifying in a sense why they were not used to this day, are identified and formalized.


\subsection{Two identified sub-problems}

A very simple way of dealing with all examples of additional data of the preceding Section is to add them as columns of our ``traditional'' data (displayed in Figure~\ref{tab:design} for example). Taking Table~\ref{tab:marketing_data} as an example, each marketing contact with the client can be reshaped so as to fit in separate columns relative to contact \#1, contact \#2, etc. This would yield Table~\ref{tab:example_longitudinal} and we could easily imagine appending to it log, credit card, recovery and payment data in a similar way. As a consequence, we are artificially back to a traditional setting with a very high number of covariates $d$. This setting is the subject of the next section. 

A probably clever way to use these data is to exploit their temporal structure, just as Recurrent Neural Networks have been able, on \textit{e.g.}\ sentiment analysis problems by analysing raw text, to perform better than methods not making use of this structure and requiring manual pre-processing such as n-grams~\cite{manning1999foundations}. However, such methods are hardly interpretable~\cite{lou2012intelligible}, which forced practitioners to resort to manual, intractable feature engineering, such as counting the number of credit card transactions over various time periods, which serve as inputs to simple models such as \gls{lr}. To avoid this time-consuming task without harming the interpretability of the resulting model nor its performance, and similarly to the quantization approach developed in Chapter~\ref{chap4}, suitable structured representations of these data have to be automatically extracted. Some techniques are provided in the subsequent section.

\begin{table}[ht]
    \centering
    \caption{Long data.}
    \label{tab:example_longitudinal}
    \begin{tiny}
\begin{tabular}{llllllllll}
Client & Job & Children & Marketing lever 1 & Date 1 & Device 1 & Marketing lever 2 & Date 2 & Device 2 & \dots \\
 \hline
1 & Skilled worker & 1 & email & 02/03/2019:15:02:54 & Android & Google Adword & 04/03/2019:12:01:01 & Windows & \dots \\
2 & Technician & 3 & mail & 02/04/2019:10:00:00 & NA & NA & NA & \dots\\
3 & Executive & 0 & Google Adword & 15/04/2019:12:10:10 & Windows & mail & 01/05/2019:10:00:00 & NA & \dots \\
\end{tabular}
    \end{tiny}
\end{table}


\section{Longitudinal data in high dimension}

This section is an eluded literature review of problems that arise in high dimension for ``classical'' data. It was first tackled by bio-statisticians working with omics data, such as DNA that can span over thousands of features for each patient, which yields a situation where more features than observations are available!

\textbf{Remarque :} les parties suivantes sont incomplètes à ce jour.

\subsection{Remark on the $d > n$ setting}

A lot of work has been dedicated to this setting (see~\cite{buhlmann2011statistics} for a review) since a lot of classical statistical methods do not work out-of-the box, including \gls{lr}. Independently from their relative consistency properties on selecting the ``best'' features, penalization methods naturally select at most $d$ features (whatever the amount of penalization)~\cite{zhu2004classification} such that we can assume in what follows $d \leq n$.
% Recall from Section~\ref{subsec:gradient} that the \gls{lr} coefficient $\gls{bth}$ is iteratively calculated from $ $ which is not defined for $d > n$.

\setlength{\epigraphwidth}{0.8\textwidth}

In the next section, we review the statistical properties associated with the ``curse of dimensionality'', a term attributed to Bellman:
\epigraph{All [problems due to high dimension] may be subsumed under the heading “the curse of dimensionality”. Since this is a curse, [...], there is no need to feel discouraged about the possibility of obtaining significant results despite it.}{R. Bellman, ``Dynamic programming'', 1957}

\subsection{The curse of dimensionality}

The foundation of statistics is that by having enough observations of random variables, we can approximate well (possibly intractable) integrals of their (possibly unknown) \gls{pdf} by an empirical average. Thus, a major problem in high dimensions is their relative emptiness, which makes the use of averages obsolete. Two classical examples are usually given: first, suppose your data lives in $[0,1]^d$ and you want to cover a neighbourhood of the origin of volume $v < 1$, \textit{e.g.}\ to perform nearest-neighbours classification. You want to know which fraction $s$ of each dimension needs to be covered. This fraction is given by $s = v^{1/d}$, such that for example a volume $v=0.5$ on a square is covered by $s = \sqrt{2}/2 \approx 0.71$, on a cube by $s = \sqrt[3]{0.5} \approx 0.79$, and so on. In high dimensional spaces such that $d>>1$, this fraction $s$ is approximately $1$ for every $v > \epsilon$: hence, neighbourhoods are not local anymore. Second, and somewhat subsequently to this first remark, the expected squared euclidean distance between two independent variables drawn uniformly in $[0,1]^d$ is $d/6$ as illustrated on Figure~\ref{fig:distance}. Consequently, it is often concluded that high dimension spaces are ``empty'' since points are all far away from each others.

\begin{figure}[!ht]
\centering
\resizebox{\textwidth}{!}{% Created by tikzDevice version 0.12 on 2019-03-11 13:44:34
% !TEX encoding = UTF-8 Unicode
\begin{tikzpicture}[x=1pt,y=1pt]
\definecolor{fillColor}{RGB}{255,255,255}
\path[use as bounding box,fill=fillColor,fill opacity=0.00] (0,0) rectangle (578.16,289.08);
\begin{scope}
\path[clip] (  0.00,  0.00) rectangle (578.16,289.08);
\definecolor{drawColor}{RGB}{255,255,255}
\definecolor{fillColor}{RGB}{255,255,255}

\path[draw=drawColor,line width= 0.6pt,line join=round,line cap=round,fill=fillColor] (  0.00,  0.00) rectangle (578.16,289.08);
\end{scope}
\begin{scope}
\path[clip] ( 38.36, 30.72) rectangle (572.66,283.58);
\definecolor{fillColor}{RGB}{255,0,0}

\path[fill=fillColor,fill opacity=0.40] ( 63.38, 42.22) rectangle ( 64.84, 60.64);

\path[fill=fillColor,fill opacity=0.40] ( 64.84, 42.22) rectangle ( 66.30, 76.72);

\path[fill=fillColor,fill opacity=0.40] ( 66.30, 42.22) rectangle ( 67.76, 91.52);

\path[fill=fillColor,fill opacity=0.40] ( 67.76, 42.22) rectangle ( 69.23,104.41);

\path[fill=fillColor,fill opacity=0.40] ( 69.23, 42.22) rectangle ( 70.69,114.19);

\path[fill=fillColor,fill opacity=0.40] ( 70.69, 42.22) rectangle ( 72.15,123.13);

\path[fill=fillColor,fill opacity=0.40] ( 72.15, 42.22) rectangle ( 73.61,129.03);

\path[fill=fillColor,fill opacity=0.40] ( 73.61, 42.22) rectangle ( 75.08,133.45);

\path[fill=fillColor,fill opacity=0.40] ( 75.08, 42.22) rectangle ( 76.54,137.33);

\path[fill=fillColor,fill opacity=0.40] ( 76.54, 42.22) rectangle ( 78.00,138.66);

\path[fill=fillColor,fill opacity=0.40] ( 78.00, 42.22) rectangle ( 79.46,138.05);

\path[fill=fillColor,fill opacity=0.40] ( 79.46, 42.22) rectangle ( 80.93,137.39);

\path[fill=fillColor,fill opacity=0.40] ( 80.93, 42.22) rectangle ( 82.39,135.54);

\path[fill=fillColor,fill opacity=0.40] ( 82.39, 42.22) rectangle ( 83.85,131.75);

\path[fill=fillColor,fill opacity=0.40] ( 83.85, 42.22) rectangle ( 85.31,126.95);

\path[fill=fillColor,fill opacity=0.40] ( 85.31, 42.22) rectangle ( 86.77,120.10);

\path[fill=fillColor,fill opacity=0.40] ( 86.77, 42.22) rectangle ( 88.24,112.48);

\path[fill=fillColor,fill opacity=0.40] ( 88.24, 42.22) rectangle ( 89.70,103.49);

\path[fill=fillColor,fill opacity=0.40] ( 89.70, 42.22) rectangle ( 91.16, 94.31);

\path[fill=fillColor,fill opacity=0.40] ( 91.16, 42.22) rectangle ( 92.62, 84.84);

\path[fill=fillColor,fill opacity=0.40] ( 92.62, 42.22) rectangle ( 94.09, 73.51);

\path[fill=fillColor,fill opacity=0.40] ( 94.09, 42.22) rectangle ( 95.55, 62.03);

\path[fill=fillColor,fill opacity=0.40] ( 95.55, 42.22) rectangle ( 97.01, 54.55);

\path[fill=fillColor,fill opacity=0.40] ( 97.01, 42.22) rectangle ( 98.47, 49.92);

\path[fill=fillColor,fill opacity=0.40] ( 98.47, 42.22) rectangle ( 99.94, 46.82);

\path[fill=fillColor,fill opacity=0.40] ( 99.94, 42.22) rectangle (101.40, 44.75);

\path[fill=fillColor,fill opacity=0.40] (101.40, 42.22) rectangle (102.86, 43.56);

\path[fill=fillColor,fill opacity=0.40] (102.86, 42.22) rectangle (104.32, 42.75);

\path[fill=fillColor,fill opacity=0.40] (104.32, 42.22) rectangle (105.78, 42.40);

\path[fill=fillColor,fill opacity=0.40] (105.78, 42.22) rectangle (107.25, 42.23);

\path[fill=fillColor,fill opacity=0.40] (107.25, 42.22) rectangle (108.71, 42.22);

\path[fill=fillColor,fill opacity=0.40] (108.71, 42.22) rectangle (110.17, 42.22);

\path[fill=fillColor,fill opacity=0.40] (110.17, 42.22) rectangle (111.63, 42.22);

\path[fill=fillColor,fill opacity=0.40] (111.63, 42.22) rectangle (113.10, 42.22);

\path[fill=fillColor,fill opacity=0.40] (113.10, 42.22) rectangle (114.56, 42.22);

\path[fill=fillColor,fill opacity=0.40] (114.56, 42.22) rectangle (116.02, 42.22);

\path[fill=fillColor,fill opacity=0.40] (116.02, 42.22) rectangle (117.48, 42.22);

\path[fill=fillColor,fill opacity=0.40] (117.48, 42.22) rectangle (118.95, 42.22);

\path[fill=fillColor,fill opacity=0.40] (118.95, 42.22) rectangle (120.41, 42.22);

\path[fill=fillColor,fill opacity=0.40] (120.41, 42.22) rectangle (121.87, 42.22);

\path[fill=fillColor,fill opacity=0.40] (121.87, 42.22) rectangle (123.33, 42.22);

\path[fill=fillColor,fill opacity=0.40] (123.33, 42.22) rectangle (124.79, 42.22);

\path[fill=fillColor,fill opacity=0.40] (124.79, 42.22) rectangle (126.26, 42.22);

\path[fill=fillColor,fill opacity=0.40] (126.26, 42.22) rectangle (127.72, 42.22);

\path[fill=fillColor,fill opacity=0.40] (127.72, 42.22) rectangle (129.18, 42.22);

\path[fill=fillColor,fill opacity=0.40] (129.18, 42.22) rectangle (130.64, 42.22);

\path[fill=fillColor,fill opacity=0.40] (130.64, 42.22) rectangle (132.11, 42.22);

\path[fill=fillColor,fill opacity=0.40] (132.11, 42.22) rectangle (133.57, 42.22);

\path[fill=fillColor,fill opacity=0.40] (133.57, 42.22) rectangle (135.03, 42.22);

\path[fill=fillColor,fill opacity=0.40] (135.03, 42.22) rectangle (136.49, 42.22);

\path[fill=fillColor,fill opacity=0.40] (136.49, 42.22) rectangle (137.96, 42.22);

\path[fill=fillColor,fill opacity=0.40] (137.96, 42.22) rectangle (139.42, 42.22);

\path[fill=fillColor,fill opacity=0.40] (139.42, 42.22) rectangle (140.88, 42.22);

\path[fill=fillColor,fill opacity=0.40] (140.88, 42.22) rectangle (142.34, 42.22);

\path[fill=fillColor,fill opacity=0.40] (142.34, 42.22) rectangle (143.80, 42.22);

\path[fill=fillColor,fill opacity=0.40] (143.80, 42.22) rectangle (145.27, 42.22);

\path[fill=fillColor,fill opacity=0.40] (145.27, 42.22) rectangle (146.73, 42.22);

\path[fill=fillColor,fill opacity=0.40] (146.73, 42.22) rectangle (148.19, 42.22);

\path[fill=fillColor,fill opacity=0.40] (148.19, 42.22) rectangle (149.65, 42.22);

\path[fill=fillColor,fill opacity=0.40] (149.65, 42.22) rectangle (151.12, 42.22);

\path[fill=fillColor,fill opacity=0.40] (151.12, 42.22) rectangle (152.58, 42.22);

\path[fill=fillColor,fill opacity=0.40] (152.58, 42.22) rectangle (154.04, 42.22);

\path[fill=fillColor,fill opacity=0.40] (154.04, 42.22) rectangle (155.50, 42.22);

\path[fill=fillColor,fill opacity=0.40] (155.50, 42.22) rectangle (156.96, 42.22);

\path[fill=fillColor,fill opacity=0.40] (156.96, 42.22) rectangle (158.43, 42.22);

\path[fill=fillColor,fill opacity=0.40] (158.43, 42.22) rectangle (159.89, 42.22);

\path[fill=fillColor,fill opacity=0.40] (159.89, 42.22) rectangle (161.35, 42.22);

\path[fill=fillColor,fill opacity=0.40] (161.35, 42.22) rectangle (162.81, 42.22);

\path[fill=fillColor,fill opacity=0.40] (162.81, 42.22) rectangle (164.28, 42.22);

\path[fill=fillColor,fill opacity=0.40] (164.28, 42.22) rectangle (165.74, 42.22);

\path[fill=fillColor,fill opacity=0.40] (165.74, 42.22) rectangle (167.20, 42.22);

\path[fill=fillColor,fill opacity=0.40] (167.20, 42.22) rectangle (168.66, 42.22);

\path[fill=fillColor,fill opacity=0.40] (168.66, 42.22) rectangle (170.13, 42.22);

\path[fill=fillColor,fill opacity=0.40] (170.13, 42.22) rectangle (171.59, 42.22);

\path[fill=fillColor,fill opacity=0.40] (171.59, 42.22) rectangle (173.05, 42.22);

\path[fill=fillColor,fill opacity=0.40] (173.05, 42.22) rectangle (174.51, 42.22);

\path[fill=fillColor,fill opacity=0.40] (174.51, 42.22) rectangle (175.97, 42.22);

\path[fill=fillColor,fill opacity=0.40] (175.97, 42.22) rectangle (177.44, 42.22);

\path[fill=fillColor,fill opacity=0.40] (177.44, 42.22) rectangle (178.90, 42.22);

\path[fill=fillColor,fill opacity=0.40] (178.90, 42.22) rectangle (180.36, 42.22);

\path[fill=fillColor,fill opacity=0.40] (180.36, 42.22) rectangle (181.82, 42.22);

\path[fill=fillColor,fill opacity=0.40] (181.82, 42.22) rectangle (183.29, 42.22);

\path[fill=fillColor,fill opacity=0.40] (183.29, 42.22) rectangle (184.75, 42.22);

\path[fill=fillColor,fill opacity=0.40] (184.75, 42.22) rectangle (186.21, 42.22);

\path[fill=fillColor,fill opacity=0.40] (186.21, 42.22) rectangle (187.67, 42.22);

\path[fill=fillColor,fill opacity=0.40] (187.67, 42.22) rectangle (189.14, 42.22);

\path[fill=fillColor,fill opacity=0.40] (189.14, 42.22) rectangle (190.60, 42.22);

\path[fill=fillColor,fill opacity=0.40] (190.60, 42.22) rectangle (192.06, 42.22);

\path[fill=fillColor,fill opacity=0.40] (192.06, 42.22) rectangle (193.52, 42.22);

\path[fill=fillColor,fill opacity=0.40] (193.52, 42.22) rectangle (194.98, 42.22);

\path[fill=fillColor,fill opacity=0.40] (194.98, 42.22) rectangle (196.45, 42.22);

\path[fill=fillColor,fill opacity=0.40] (196.45, 42.22) rectangle (197.91, 42.22);

\path[fill=fillColor,fill opacity=0.40] (197.91, 42.22) rectangle (199.37, 42.22);

\path[fill=fillColor,fill opacity=0.40] (199.37, 42.22) rectangle (200.83, 42.22);

\path[fill=fillColor,fill opacity=0.40] (200.83, 42.22) rectangle (202.30, 42.22);

\path[fill=fillColor,fill opacity=0.40] (202.30, 42.22) rectangle (203.76, 42.22);

\path[fill=fillColor,fill opacity=0.40] (203.76, 42.22) rectangle (205.22, 42.22);

\path[fill=fillColor,fill opacity=0.40] (205.22, 42.22) rectangle (206.68, 42.22);

\path[fill=fillColor,fill opacity=0.40] (206.68, 42.22) rectangle (208.15, 42.22);

\path[fill=fillColor,fill opacity=0.40] (208.15, 42.22) rectangle (209.61, 42.22);

\path[fill=fillColor,fill opacity=0.40] (209.61, 42.22) rectangle (211.07, 42.22);

\path[fill=fillColor,fill opacity=0.40] (211.07, 42.22) rectangle (212.53, 42.22);

\path[fill=fillColor,fill opacity=0.40] (212.53, 42.22) rectangle (213.99, 42.22);

\path[fill=fillColor,fill opacity=0.40] (213.99, 42.22) rectangle (215.46, 42.22);

\path[fill=fillColor,fill opacity=0.40] (215.46, 42.22) rectangle (216.92, 42.22);

\path[fill=fillColor,fill opacity=0.40] (216.92, 42.22) rectangle (218.38, 42.22);

\path[fill=fillColor,fill opacity=0.40] (218.38, 42.22) rectangle (219.84, 42.22);

\path[fill=fillColor,fill opacity=0.40] (219.84, 42.22) rectangle (221.31, 42.22);

\path[fill=fillColor,fill opacity=0.40] (221.31, 42.22) rectangle (222.77, 42.22);

\path[fill=fillColor,fill opacity=0.40] (222.77, 42.22) rectangle (224.23, 42.22);

\path[fill=fillColor,fill opacity=0.40] (224.23, 42.22) rectangle (225.69, 42.22);

\path[fill=fillColor,fill opacity=0.40] (225.69, 42.22) rectangle (227.16, 42.22);

\path[fill=fillColor,fill opacity=0.40] (227.16, 42.22) rectangle (228.62, 42.22);

\path[fill=fillColor,fill opacity=0.40] (228.62, 42.22) rectangle (230.08, 42.22);

\path[fill=fillColor,fill opacity=0.40] (230.08, 42.22) rectangle (231.54, 42.22);

\path[fill=fillColor,fill opacity=0.40] (231.54, 42.22) rectangle (233.00, 42.22);

\path[fill=fillColor,fill opacity=0.40] (233.00, 42.22) rectangle (234.47, 42.22);

\path[fill=fillColor,fill opacity=0.40] (234.47, 42.22) rectangle (235.93, 42.22);

\path[fill=fillColor,fill opacity=0.40] (235.93, 42.22) rectangle (237.39, 42.22);

\path[fill=fillColor,fill opacity=0.40] (237.39, 42.22) rectangle (238.85, 42.22);

\path[fill=fillColor,fill opacity=0.40] (238.85, 42.22) rectangle (240.32, 42.22);

\path[fill=fillColor,fill opacity=0.40] (240.32, 42.22) rectangle (241.78, 42.22);

\path[fill=fillColor,fill opacity=0.40] (241.78, 42.22) rectangle (243.24, 42.22);

\path[fill=fillColor,fill opacity=0.40] (243.24, 42.22) rectangle (244.70, 42.22);

\path[fill=fillColor,fill opacity=0.40] (244.70, 42.22) rectangle (246.16, 42.22);

\path[fill=fillColor,fill opacity=0.40] (246.16, 42.22) rectangle (247.63, 42.22);

\path[fill=fillColor,fill opacity=0.40] (247.63, 42.22) rectangle (249.09, 42.22);

\path[fill=fillColor,fill opacity=0.40] (249.09, 42.22) rectangle (250.55, 42.22);

\path[fill=fillColor,fill opacity=0.40] (250.55, 42.22) rectangle (252.01, 42.22);

\path[fill=fillColor,fill opacity=0.40] (252.01, 42.22) rectangle (253.48, 42.22);

\path[fill=fillColor,fill opacity=0.40] (253.48, 42.22) rectangle (254.94, 42.22);

\path[fill=fillColor,fill opacity=0.40] (254.94, 42.22) rectangle (256.40, 42.22);

\path[fill=fillColor,fill opacity=0.40] (256.40, 42.22) rectangle (257.86, 42.22);

\path[fill=fillColor,fill opacity=0.40] (257.86, 42.22) rectangle (259.33, 42.22);

\path[fill=fillColor,fill opacity=0.40] (259.33, 42.22) rectangle (260.79, 42.22);

\path[fill=fillColor,fill opacity=0.40] (260.79, 42.22) rectangle (262.25, 42.22);

\path[fill=fillColor,fill opacity=0.40] (262.25, 42.22) rectangle (263.71, 42.22);

\path[fill=fillColor,fill opacity=0.40] (263.71, 42.22) rectangle (265.17, 42.22);

\path[fill=fillColor,fill opacity=0.40] (265.17, 42.22) rectangle (266.64, 42.22);

\path[fill=fillColor,fill opacity=0.40] (266.64, 42.22) rectangle (268.10, 42.22);

\path[fill=fillColor,fill opacity=0.40] (268.10, 42.22) rectangle (269.56, 42.22);

\path[fill=fillColor,fill opacity=0.40] (269.56, 42.22) rectangle (271.02, 42.22);

\path[fill=fillColor,fill opacity=0.40] (271.02, 42.22) rectangle (272.49, 42.22);

\path[fill=fillColor,fill opacity=0.40] (272.49, 42.22) rectangle (273.95, 42.22);

\path[fill=fillColor,fill opacity=0.40] (273.95, 42.22) rectangle (275.41, 42.22);

\path[fill=fillColor,fill opacity=0.40] (275.41, 42.22) rectangle (276.87, 42.22);

\path[fill=fillColor,fill opacity=0.40] (276.87, 42.22) rectangle (278.34, 42.22);

\path[fill=fillColor,fill opacity=0.40] (278.34, 42.22) rectangle (279.80, 42.22);

\path[fill=fillColor,fill opacity=0.40] (279.80, 42.22) rectangle (281.26, 42.22);

\path[fill=fillColor,fill opacity=0.40] (281.26, 42.22) rectangle (282.72, 42.22);

\path[fill=fillColor,fill opacity=0.40] (282.72, 42.22) rectangle (284.18, 42.22);

\path[fill=fillColor,fill opacity=0.40] (284.18, 42.22) rectangle (285.65, 42.22);

\path[fill=fillColor,fill opacity=0.40] (285.65, 42.22) rectangle (287.11, 42.22);

\path[fill=fillColor,fill opacity=0.40] (287.11, 42.22) rectangle (288.57, 42.22);

\path[fill=fillColor,fill opacity=0.40] (288.57, 42.22) rectangle (290.03, 42.22);

\path[fill=fillColor,fill opacity=0.40] (290.03, 42.22) rectangle (291.50, 42.22);

\path[fill=fillColor,fill opacity=0.40] (291.50, 42.22) rectangle (292.96, 42.22);

\path[fill=fillColor,fill opacity=0.40] (292.96, 42.22) rectangle (294.42, 42.22);

\path[fill=fillColor,fill opacity=0.40] (294.42, 42.22) rectangle (295.88, 42.22);

\path[fill=fillColor,fill opacity=0.40] (295.88, 42.22) rectangle (297.35, 42.22);

\path[fill=fillColor,fill opacity=0.40] (297.35, 42.22) rectangle (298.81, 42.22);

\path[fill=fillColor,fill opacity=0.40] (298.81, 42.22) rectangle (300.27, 42.22);

\path[fill=fillColor,fill opacity=0.40] (300.27, 42.22) rectangle (301.73, 42.22);

\path[fill=fillColor,fill opacity=0.40] (301.73, 42.22) rectangle (303.19, 42.22);

\path[fill=fillColor,fill opacity=0.40] (303.19, 42.22) rectangle (304.66, 42.22);

\path[fill=fillColor,fill opacity=0.40] (304.66, 42.22) rectangle (306.12, 42.22);

\path[fill=fillColor,fill opacity=0.40] (306.12, 42.22) rectangle (307.58, 42.22);

\path[fill=fillColor,fill opacity=0.40] (307.58, 42.22) rectangle (309.04, 42.22);

\path[fill=fillColor,fill opacity=0.40] (309.04, 42.22) rectangle (310.51, 42.22);

\path[fill=fillColor,fill opacity=0.40] (310.51, 42.22) rectangle (311.97, 42.22);

\path[fill=fillColor,fill opacity=0.40] (311.97, 42.22) rectangle (313.43, 42.22);

\path[fill=fillColor,fill opacity=0.40] (313.43, 42.22) rectangle (314.89, 42.22);

\path[fill=fillColor,fill opacity=0.40] (314.89, 42.22) rectangle (316.36, 42.22);

\path[fill=fillColor,fill opacity=0.40] (316.36, 42.22) rectangle (317.82, 42.22);

\path[fill=fillColor,fill opacity=0.40] (317.82, 42.22) rectangle (319.28, 42.22);

\path[fill=fillColor,fill opacity=0.40] (319.28, 42.22) rectangle (320.74, 42.22);

\path[fill=fillColor,fill opacity=0.40] (320.74, 42.22) rectangle (322.20, 42.22);

\path[fill=fillColor,fill opacity=0.40] (322.20, 42.22) rectangle (323.67, 42.22);

\path[fill=fillColor,fill opacity=0.40] (323.67, 42.22) rectangle (325.13, 42.22);

\path[fill=fillColor,fill opacity=0.40] (325.13, 42.22) rectangle (326.59, 42.22);

\path[fill=fillColor,fill opacity=0.40] (326.59, 42.22) rectangle (328.05, 42.22);

\path[fill=fillColor,fill opacity=0.40] (328.05, 42.22) rectangle (329.52, 42.22);

\path[fill=fillColor,fill opacity=0.40] (329.52, 42.22) rectangle (330.98, 42.22);

\path[fill=fillColor,fill opacity=0.40] (330.98, 42.22) rectangle (332.44, 42.22);

\path[fill=fillColor,fill opacity=0.40] (332.44, 42.22) rectangle (333.90, 42.22);

\path[fill=fillColor,fill opacity=0.40] (333.90, 42.22) rectangle (335.36, 42.22);

\path[fill=fillColor,fill opacity=0.40] (335.36, 42.22) rectangle (336.83, 42.22);

\path[fill=fillColor,fill opacity=0.40] (336.83, 42.22) rectangle (338.29, 42.22);

\path[fill=fillColor,fill opacity=0.40] (338.29, 42.22) rectangle (339.75, 42.22);

\path[fill=fillColor,fill opacity=0.40] (339.75, 42.22) rectangle (341.21, 42.22);

\path[fill=fillColor,fill opacity=0.40] (341.21, 42.22) rectangle (342.68, 42.22);

\path[fill=fillColor,fill opacity=0.40] (342.68, 42.22) rectangle (344.14, 42.22);

\path[fill=fillColor,fill opacity=0.40] (344.14, 42.22) rectangle (345.60, 42.22);

\path[fill=fillColor,fill opacity=0.40] (345.60, 42.22) rectangle (347.06, 42.22);

\path[fill=fillColor,fill opacity=0.40] (347.06, 42.22) rectangle (348.53, 42.22);

\path[fill=fillColor,fill opacity=0.40] (348.53, 42.22) rectangle (349.99, 42.22);

\path[fill=fillColor,fill opacity=0.40] (349.99, 42.22) rectangle (351.45, 42.22);

\path[fill=fillColor,fill opacity=0.40] (351.45, 42.22) rectangle (352.91, 42.22);

\path[fill=fillColor,fill opacity=0.40] (352.91, 42.22) rectangle (354.37, 42.22);

\path[fill=fillColor,fill opacity=0.40] (354.37, 42.22) rectangle (355.84, 42.22);

\path[fill=fillColor,fill opacity=0.40] (355.84, 42.22) rectangle (357.30, 42.22);

\path[fill=fillColor,fill opacity=0.40] (357.30, 42.22) rectangle (358.76, 42.22);

\path[fill=fillColor,fill opacity=0.40] (358.76, 42.22) rectangle (360.22, 42.22);

\path[fill=fillColor,fill opacity=0.40] (360.22, 42.22) rectangle (361.69, 42.22);

\path[fill=fillColor,fill opacity=0.40] (361.69, 42.22) rectangle (363.15, 42.22);

\path[fill=fillColor,fill opacity=0.40] (363.15, 42.22) rectangle (364.61, 42.22);

\path[fill=fillColor,fill opacity=0.40] (364.61, 42.22) rectangle (366.07, 42.22);

\path[fill=fillColor,fill opacity=0.40] (366.07, 42.22) rectangle (367.54, 42.22);

\path[fill=fillColor,fill opacity=0.40] (367.54, 42.22) rectangle (369.00, 42.22);

\path[fill=fillColor,fill opacity=0.40] (369.00, 42.22) rectangle (370.46, 42.22);

\path[fill=fillColor,fill opacity=0.40] (370.46, 42.22) rectangle (371.92, 42.22);

\path[fill=fillColor,fill opacity=0.40] (371.92, 42.22) rectangle (373.38, 42.22);

\path[fill=fillColor,fill opacity=0.40] (373.38, 42.22) rectangle (374.85, 42.22);

\path[fill=fillColor,fill opacity=0.40] (374.85, 42.22) rectangle (376.31, 42.22);

\path[fill=fillColor,fill opacity=0.40] (376.31, 42.22) rectangle (377.77, 42.22);

\path[fill=fillColor,fill opacity=0.40] (377.77, 42.22) rectangle (379.23, 42.22);

\path[fill=fillColor,fill opacity=0.40] (379.23, 42.22) rectangle (380.70, 42.22);

\path[fill=fillColor,fill opacity=0.40] (380.70, 42.22) rectangle (382.16, 42.22);

\path[fill=fillColor,fill opacity=0.40] (382.16, 42.22) rectangle (383.62, 42.22);

\path[fill=fillColor,fill opacity=0.40] (383.62, 42.22) rectangle (385.08, 42.22);

\path[fill=fillColor,fill opacity=0.40] (385.08, 42.22) rectangle (386.55, 42.22);

\path[fill=fillColor,fill opacity=0.40] (386.55, 42.22) rectangle (388.01, 42.22);

\path[fill=fillColor,fill opacity=0.40] (388.01, 42.22) rectangle (389.47, 42.22);

\path[fill=fillColor,fill opacity=0.40] (389.47, 42.22) rectangle (390.93, 42.22);

\path[fill=fillColor,fill opacity=0.40] (390.93, 42.22) rectangle (392.39, 42.22);

\path[fill=fillColor,fill opacity=0.40] (392.39, 42.22) rectangle (393.86, 42.22);

\path[fill=fillColor,fill opacity=0.40] (393.86, 42.22) rectangle (395.32, 42.22);

\path[fill=fillColor,fill opacity=0.40] (395.32, 42.22) rectangle (396.78, 42.22);

\path[fill=fillColor,fill opacity=0.40] (396.78, 42.22) rectangle (398.24, 42.22);

\path[fill=fillColor,fill opacity=0.40] (398.24, 42.22) rectangle (399.71, 42.22);

\path[fill=fillColor,fill opacity=0.40] (399.71, 42.22) rectangle (401.17, 42.22);

\path[fill=fillColor,fill opacity=0.40] (401.17, 42.22) rectangle (402.63, 42.22);

\path[fill=fillColor,fill opacity=0.40] (402.63, 42.22) rectangle (404.09, 42.22);

\path[fill=fillColor,fill opacity=0.40] (404.09, 42.22) rectangle (405.56, 42.22);

\path[fill=fillColor,fill opacity=0.40] (405.56, 42.22) rectangle (407.02, 42.22);

\path[fill=fillColor,fill opacity=0.40] (407.02, 42.22) rectangle (408.48, 42.22);

\path[fill=fillColor,fill opacity=0.40] (408.48, 42.22) rectangle (409.94, 42.22);

\path[fill=fillColor,fill opacity=0.40] (409.94, 42.22) rectangle (411.40, 42.22);

\path[fill=fillColor,fill opacity=0.40] (411.40, 42.22) rectangle (412.87, 42.22);

\path[fill=fillColor,fill opacity=0.40] (412.87, 42.22) rectangle (414.33, 42.22);

\path[fill=fillColor,fill opacity=0.40] (414.33, 42.22) rectangle (415.79, 42.22);

\path[fill=fillColor,fill opacity=0.40] (415.79, 42.22) rectangle (417.25, 42.22);

\path[fill=fillColor,fill opacity=0.40] (417.25, 42.22) rectangle (418.72, 42.22);

\path[fill=fillColor,fill opacity=0.40] (418.72, 42.22) rectangle (420.18, 42.22);

\path[fill=fillColor,fill opacity=0.40] (420.18, 42.22) rectangle (421.64, 42.22);

\path[fill=fillColor,fill opacity=0.40] (421.64, 42.22) rectangle (423.10, 42.22);

\path[fill=fillColor,fill opacity=0.40] (423.10, 42.22) rectangle (424.56, 42.22);

\path[fill=fillColor,fill opacity=0.40] (424.56, 42.22) rectangle (426.03, 42.22);

\path[fill=fillColor,fill opacity=0.40] (426.03, 42.22) rectangle (427.49, 42.22);

\path[fill=fillColor,fill opacity=0.40] (427.49, 42.22) rectangle (428.95, 42.22);

\path[fill=fillColor,fill opacity=0.40] (428.95, 42.22) rectangle (430.41, 42.22);

\path[fill=fillColor,fill opacity=0.40] (430.41, 42.22) rectangle (431.88, 42.22);

\path[fill=fillColor,fill opacity=0.40] (431.88, 42.22) rectangle (433.34, 42.22);

\path[fill=fillColor,fill opacity=0.40] (433.34, 42.22) rectangle (434.80, 42.22);

\path[fill=fillColor,fill opacity=0.40] (434.80, 42.22) rectangle (436.26, 42.22);

\path[fill=fillColor,fill opacity=0.40] (436.26, 42.22) rectangle (437.73, 42.22);

\path[fill=fillColor,fill opacity=0.40] (437.73, 42.22) rectangle (439.19, 42.22);

\path[fill=fillColor,fill opacity=0.40] (439.19, 42.22) rectangle (440.65, 42.22);

\path[fill=fillColor,fill opacity=0.40] (440.65, 42.22) rectangle (442.11, 42.22);

\path[fill=fillColor,fill opacity=0.40] (442.11, 42.22) rectangle (443.57, 42.22);

\path[fill=fillColor,fill opacity=0.40] (443.57, 42.22) rectangle (445.04, 42.22);

\path[fill=fillColor,fill opacity=0.40] (445.04, 42.22) rectangle (446.50, 42.22);

\path[fill=fillColor,fill opacity=0.40] (446.50, 42.22) rectangle (447.96, 42.22);

\path[fill=fillColor,fill opacity=0.40] (447.96, 42.22) rectangle (449.42, 42.22);

\path[fill=fillColor,fill opacity=0.40] (449.42, 42.22) rectangle (450.89, 42.22);

\path[fill=fillColor,fill opacity=0.40] (450.89, 42.22) rectangle (452.35, 42.22);

\path[fill=fillColor,fill opacity=0.40] (452.35, 42.22) rectangle (453.81, 42.22);

\path[fill=fillColor,fill opacity=0.40] (453.81, 42.22) rectangle (455.27, 42.22);

\path[fill=fillColor,fill opacity=0.40] (455.27, 42.22) rectangle (456.74, 42.22);

\path[fill=fillColor,fill opacity=0.40] (456.74, 42.22) rectangle (458.20, 42.22);

\path[fill=fillColor,fill opacity=0.40] (458.20, 42.22) rectangle (459.66, 42.22);

\path[fill=fillColor,fill opacity=0.40] (459.66, 42.22) rectangle (461.12, 42.22);

\path[fill=fillColor,fill opacity=0.40] (461.12, 42.22) rectangle (462.58, 42.22);

\path[fill=fillColor,fill opacity=0.40] (462.58, 42.22) rectangle (464.05, 42.22);

\path[fill=fillColor,fill opacity=0.40] (464.05, 42.22) rectangle (465.51, 42.22);

\path[fill=fillColor,fill opacity=0.40] (465.51, 42.22) rectangle (466.97, 42.22);

\path[fill=fillColor,fill opacity=0.40] (466.97, 42.22) rectangle (468.43, 42.22);

\path[fill=fillColor,fill opacity=0.40] (468.43, 42.22) rectangle (469.90, 42.22);

\path[fill=fillColor,fill opacity=0.40] (469.90, 42.22) rectangle (471.36, 42.22);

\path[fill=fillColor,fill opacity=0.40] (471.36, 42.22) rectangle (472.82, 42.22);

\path[fill=fillColor,fill opacity=0.40] (472.82, 42.22) rectangle (474.28, 42.22);

\path[fill=fillColor,fill opacity=0.40] (474.28, 42.22) rectangle (475.75, 42.22);

\path[fill=fillColor,fill opacity=0.40] (475.75, 42.22) rectangle (477.21, 42.22);

\path[fill=fillColor,fill opacity=0.40] (477.21, 42.22) rectangle (478.67, 42.22);

\path[fill=fillColor,fill opacity=0.40] (478.67, 42.22) rectangle (480.13, 42.22);

\path[fill=fillColor,fill opacity=0.40] (480.13, 42.22) rectangle (481.59, 42.22);

\path[fill=fillColor,fill opacity=0.40] (481.59, 42.22) rectangle (483.06, 42.22);

\path[fill=fillColor,fill opacity=0.40] (483.06, 42.22) rectangle (484.52, 42.22);

\path[fill=fillColor,fill opacity=0.40] (484.52, 42.22) rectangle (485.98, 42.22);

\path[fill=fillColor,fill opacity=0.40] (485.98, 42.22) rectangle (487.44, 42.22);

\path[fill=fillColor,fill opacity=0.40] (487.44, 42.22) rectangle (488.91, 42.22);

\path[fill=fillColor,fill opacity=0.40] (488.91, 42.22) rectangle (490.37, 42.22);

\path[fill=fillColor,fill opacity=0.40] (490.37, 42.22) rectangle (491.83, 42.22);

\path[fill=fillColor,fill opacity=0.40] (491.83, 42.22) rectangle (493.29, 42.22);

\path[fill=fillColor,fill opacity=0.40] (493.29, 42.22) rectangle (494.76, 42.22);

\path[fill=fillColor,fill opacity=0.40] (494.76, 42.22) rectangle (496.22, 42.22);

\path[fill=fillColor,fill opacity=0.40] (496.22, 42.22) rectangle (497.68, 42.22);

\path[fill=fillColor,fill opacity=0.40] (497.68, 42.22) rectangle (499.14, 42.22);

\path[fill=fillColor,fill opacity=0.40] (499.14, 42.22) rectangle (500.60, 42.22);

\path[fill=fillColor,fill opacity=0.40] (500.60, 42.22) rectangle (502.07, 42.22);

\path[fill=fillColor,fill opacity=0.40] (502.07, 42.22) rectangle (503.53, 42.22);

\path[fill=fillColor,fill opacity=0.40] (503.53, 42.22) rectangle (504.99, 42.22);

\path[fill=fillColor,fill opacity=0.40] (504.99, 42.22) rectangle (506.45, 42.22);

\path[fill=fillColor,fill opacity=0.40] (506.45, 42.22) rectangle (507.92, 42.22);

\path[fill=fillColor,fill opacity=0.40] (507.92, 42.22) rectangle (509.38, 42.22);

\path[fill=fillColor,fill opacity=0.40] (509.38, 42.22) rectangle (510.84, 42.22);

\path[fill=fillColor,fill opacity=0.40] (510.84, 42.22) rectangle (512.30, 42.22);

\path[fill=fillColor,fill opacity=0.40] (512.30, 42.22) rectangle (513.76, 42.22);

\path[fill=fillColor,fill opacity=0.40] (513.76, 42.22) rectangle (515.23, 42.22);

\path[fill=fillColor,fill opacity=0.40] (515.23, 42.22) rectangle (516.69, 42.22);

\path[fill=fillColor,fill opacity=0.40] (516.69, 42.22) rectangle (518.15, 42.22);

\path[fill=fillColor,fill opacity=0.40] (518.15, 42.22) rectangle (519.61, 42.22);

\path[fill=fillColor,fill opacity=0.40] (519.61, 42.22) rectangle (521.08, 42.22);

\path[fill=fillColor,fill opacity=0.40] (521.08, 42.22) rectangle (522.54, 42.22);

\path[fill=fillColor,fill opacity=0.40] (522.54, 42.22) rectangle (524.00, 42.22);

\path[fill=fillColor,fill opacity=0.40] (524.00, 42.22) rectangle (525.46, 42.22);

\path[fill=fillColor,fill opacity=0.40] (525.46, 42.22) rectangle (526.93, 42.22);

\path[fill=fillColor,fill opacity=0.40] (526.93, 42.22) rectangle (528.39, 42.22);

\path[fill=fillColor,fill opacity=0.40] (528.39, 42.22) rectangle (529.85, 42.22);

\path[fill=fillColor,fill opacity=0.40] (529.85, 42.22) rectangle (531.31, 42.22);

\path[fill=fillColor,fill opacity=0.40] (531.31, 42.22) rectangle (532.77, 42.22);

\path[fill=fillColor,fill opacity=0.40] (532.77, 42.22) rectangle (534.24, 42.22);

\path[fill=fillColor,fill opacity=0.40] (534.24, 42.22) rectangle (535.70, 42.22);

\path[fill=fillColor,fill opacity=0.40] (535.70, 42.22) rectangle (537.16, 42.22);

\path[fill=fillColor,fill opacity=0.40] (537.16, 42.22) rectangle (538.62, 42.22);

\path[fill=fillColor,fill opacity=0.40] (538.62, 42.22) rectangle (540.09, 42.22);

\path[fill=fillColor,fill opacity=0.40] (540.09, 42.22) rectangle (541.55, 42.22);

\path[fill=fillColor,fill opacity=0.40] (541.55, 42.22) rectangle (543.01, 42.22);

\path[fill=fillColor,fill opacity=0.40] (543.01, 42.22) rectangle (544.47, 42.22);

\path[fill=fillColor,fill opacity=0.40] (544.47, 42.22) rectangle (545.94, 42.22);

\path[fill=fillColor,fill opacity=0.40] (545.94, 42.22) rectangle (547.40, 42.22);
\definecolor{fillColor}{RGB}{0,0,255}

\path[fill=fillColor,fill opacity=0.60] ( 63.62, 42.22) rectangle ( 65.56, 42.25);

\path[fill=fillColor,fill opacity=0.60] ( 65.56, 42.22) rectangle ( 67.51, 42.69);

\path[fill=fillColor,fill opacity=0.60] ( 67.51, 42.22) rectangle ( 69.45, 43.96);

\path[fill=fillColor,fill opacity=0.60] ( 69.45, 42.22) rectangle ( 71.39, 47.07);

\path[fill=fillColor,fill opacity=0.60] ( 71.39, 42.22) rectangle ( 73.34, 52.70);

\path[fill=fillColor,fill opacity=0.60] ( 73.34, 42.22) rectangle ( 75.28, 61.08);

\path[fill=fillColor,fill opacity=0.60] ( 75.28, 42.22) rectangle ( 77.22, 72.06);

\path[fill=fillColor,fill opacity=0.60] ( 77.22, 42.22) rectangle ( 79.17, 86.95);

\path[fill=fillColor,fill opacity=0.60] ( 79.17, 42.22) rectangle ( 81.11,103.04);

\path[fill=fillColor,fill opacity=0.60] ( 81.11, 42.22) rectangle ( 83.05,119.22);

\path[fill=fillColor,fill opacity=0.60] ( 83.05, 42.22) rectangle ( 85.00,137.44);

\path[fill=fillColor,fill opacity=0.60] ( 85.00, 42.22) rectangle ( 86.94,155.14);

\path[fill=fillColor,fill opacity=0.60] ( 86.94, 42.22) rectangle ( 88.88,168.20);

\path[fill=fillColor,fill opacity=0.60] ( 88.88, 42.22) rectangle ( 90.83,177.50);

\path[fill=fillColor,fill opacity=0.60] ( 90.83, 42.22) rectangle ( 92.77,183.87);

\path[fill=fillColor,fill opacity=0.60] ( 92.77, 42.22) rectangle ( 94.72,182.87);

\path[fill=fillColor,fill opacity=0.60] ( 94.72, 42.22) rectangle ( 96.66,171.82);

\path[fill=fillColor,fill opacity=0.60] ( 96.66, 42.22) rectangle ( 98.60,155.45);

\path[fill=fillColor,fill opacity=0.60] ( 98.60, 42.22) rectangle (100.55,133.79);

\path[fill=fillColor,fill opacity=0.60] (100.55, 42.22) rectangle (102.49,112.34);

\path[fill=fillColor,fill opacity=0.60] (102.49, 42.22) rectangle (104.43, 92.47);

\path[fill=fillColor,fill opacity=0.60] (104.43, 42.22) rectangle (106.38, 76.30);

\path[fill=fillColor,fill opacity=0.60] (106.38, 42.22) rectangle (108.32, 62.80);

\path[fill=fillColor,fill opacity=0.60] (108.32, 42.22) rectangle (110.26, 54.20);

\path[fill=fillColor,fill opacity=0.60] (110.26, 42.22) rectangle (112.21, 48.03);

\path[fill=fillColor,fill opacity=0.60] (112.21, 42.22) rectangle (114.15, 45.05);

\path[fill=fillColor,fill opacity=0.60] (114.15, 42.22) rectangle (116.09, 43.40);

\path[fill=fillColor,fill opacity=0.60] (116.09, 42.22) rectangle (118.04, 42.67);

\path[fill=fillColor,fill opacity=0.60] (118.04, 42.22) rectangle (119.98, 42.36);

\path[fill=fillColor,fill opacity=0.60] (119.98, 42.22) rectangle (121.92, 42.25);

\path[fill=fillColor,fill opacity=0.60] (121.92, 42.22) rectangle (123.87, 42.23);

\path[fill=fillColor,fill opacity=0.60] (123.87, 42.22) rectangle (125.81, 42.22);

\path[fill=fillColor,fill opacity=0.60] (125.81, 42.22) rectangle (127.76, 42.22);

\path[fill=fillColor,fill opacity=0.60] (127.76, 42.22) rectangle (129.70, 42.22);

\path[fill=fillColor,fill opacity=0.60] (129.70, 42.22) rectangle (131.64, 42.22);

\path[fill=fillColor,fill opacity=0.60] (131.64, 42.22) rectangle (133.59, 42.22);

\path[fill=fillColor,fill opacity=0.60] (133.59, 42.22) rectangle (135.53, 42.22);

\path[fill=fillColor,fill opacity=0.60] (135.53, 42.22) rectangle (137.47, 42.22);

\path[fill=fillColor,fill opacity=0.60] (137.47, 42.22) rectangle (139.42, 42.22);

\path[fill=fillColor,fill opacity=0.60] (139.42, 42.22) rectangle (141.36, 42.22);

\path[fill=fillColor,fill opacity=0.60] (141.36, 42.22) rectangle (143.30, 42.22);

\path[fill=fillColor,fill opacity=0.60] (143.30, 42.22) rectangle (145.25, 42.22);

\path[fill=fillColor,fill opacity=0.60] (145.25, 42.22) rectangle (147.19, 42.22);

\path[fill=fillColor,fill opacity=0.60] (147.19, 42.22) rectangle (149.13, 42.22);

\path[fill=fillColor,fill opacity=0.60] (149.13, 42.22) rectangle (151.08, 42.22);

\path[fill=fillColor,fill opacity=0.60] (151.08, 42.22) rectangle (153.02, 42.22);

\path[fill=fillColor,fill opacity=0.60] (153.02, 42.22) rectangle (154.96, 42.22);

\path[fill=fillColor,fill opacity=0.60] (154.96, 42.22) rectangle (156.91, 42.22);

\path[fill=fillColor,fill opacity=0.60] (156.91, 42.22) rectangle (158.85, 42.22);

\path[fill=fillColor,fill opacity=0.60] (158.85, 42.22) rectangle (160.80, 42.22);

\path[fill=fillColor,fill opacity=0.60] (160.80, 42.22) rectangle (162.74, 42.22);

\path[fill=fillColor,fill opacity=0.60] (162.74, 42.22) rectangle (164.68, 42.22);

\path[fill=fillColor,fill opacity=0.60] (164.68, 42.22) rectangle (166.63, 42.22);

\path[fill=fillColor,fill opacity=0.60] (166.63, 42.22) rectangle (168.57, 42.22);

\path[fill=fillColor,fill opacity=0.60] (168.57, 42.22) rectangle (170.51, 42.22);

\path[fill=fillColor,fill opacity=0.60] (170.51, 42.22) rectangle (172.46, 42.22);

\path[fill=fillColor,fill opacity=0.60] (172.46, 42.22) rectangle (174.40, 42.22);

\path[fill=fillColor,fill opacity=0.60] (174.40, 42.22) rectangle (176.34, 42.22);

\path[fill=fillColor,fill opacity=0.60] (176.34, 42.22) rectangle (178.29, 42.22);

\path[fill=fillColor,fill opacity=0.60] (178.29, 42.22) rectangle (180.23, 42.22);

\path[fill=fillColor,fill opacity=0.60] (180.23, 42.22) rectangle (182.17, 42.22);

\path[fill=fillColor,fill opacity=0.60] (182.17, 42.22) rectangle (184.12, 42.22);

\path[fill=fillColor,fill opacity=0.60] (184.12, 42.22) rectangle (186.06, 42.22);

\path[fill=fillColor,fill opacity=0.60] (186.06, 42.22) rectangle (188.00, 42.22);

\path[fill=fillColor,fill opacity=0.60] (188.00, 42.22) rectangle (189.95, 42.22);

\path[fill=fillColor,fill opacity=0.60] (189.95, 42.22) rectangle (191.89, 42.22);

\path[fill=fillColor,fill opacity=0.60] (191.89, 42.22) rectangle (193.84, 42.22);

\path[fill=fillColor,fill opacity=0.60] (193.84, 42.22) rectangle (195.78, 42.22);

\path[fill=fillColor,fill opacity=0.60] (195.78, 42.22) rectangle (197.72, 42.22);

\path[fill=fillColor,fill opacity=0.60] (197.72, 42.22) rectangle (199.67, 42.22);

\path[fill=fillColor,fill opacity=0.60] (199.67, 42.22) rectangle (201.61, 42.22);

\path[fill=fillColor,fill opacity=0.60] (201.61, 42.22) rectangle (203.55, 42.22);

\path[fill=fillColor,fill opacity=0.60] (203.55, 42.22) rectangle (205.50, 42.22);

\path[fill=fillColor,fill opacity=0.60] (205.50, 42.22) rectangle (207.44, 42.22);

\path[fill=fillColor,fill opacity=0.60] (207.44, 42.22) rectangle (209.38, 42.22);

\path[fill=fillColor,fill opacity=0.60] (209.38, 42.22) rectangle (211.33, 42.22);

\path[fill=fillColor,fill opacity=0.60] (211.33, 42.22) rectangle (213.27, 42.22);

\path[fill=fillColor,fill opacity=0.60] (213.27, 42.22) rectangle (215.21, 42.22);

\path[fill=fillColor,fill opacity=0.60] (215.21, 42.22) rectangle (217.16, 42.22);

\path[fill=fillColor,fill opacity=0.60] (217.16, 42.22) rectangle (219.10, 42.22);

\path[fill=fillColor,fill opacity=0.60] (219.10, 42.22) rectangle (221.04, 42.22);

\path[fill=fillColor,fill opacity=0.60] (221.04, 42.22) rectangle (222.99, 42.22);

\path[fill=fillColor,fill opacity=0.60] (222.99, 42.22) rectangle (224.93, 42.22);

\path[fill=fillColor,fill opacity=0.60] (224.93, 42.22) rectangle (226.88, 42.22);

\path[fill=fillColor,fill opacity=0.60] (226.88, 42.22) rectangle (228.82, 42.22);

\path[fill=fillColor,fill opacity=0.60] (228.82, 42.22) rectangle (230.76, 42.22);

\path[fill=fillColor,fill opacity=0.60] (230.76, 42.22) rectangle (232.71, 42.22);

\path[fill=fillColor,fill opacity=0.60] (232.71, 42.22) rectangle (234.65, 42.22);

\path[fill=fillColor,fill opacity=0.60] (234.65, 42.22) rectangle (236.59, 42.22);

\path[fill=fillColor,fill opacity=0.60] (236.59, 42.22) rectangle (238.54, 42.22);

\path[fill=fillColor,fill opacity=0.60] (238.54, 42.22) rectangle (240.48, 42.22);

\path[fill=fillColor,fill opacity=0.60] (240.48, 42.22) rectangle (242.42, 42.22);

\path[fill=fillColor,fill opacity=0.60] (242.42, 42.22) rectangle (244.37, 42.22);

\path[fill=fillColor,fill opacity=0.60] (244.37, 42.22) rectangle (246.31, 42.22);

\path[fill=fillColor,fill opacity=0.60] (246.31, 42.22) rectangle (248.25, 42.22);

\path[fill=fillColor,fill opacity=0.60] (248.25, 42.22) rectangle (250.20, 42.22);

\path[fill=fillColor,fill opacity=0.60] (250.20, 42.22) rectangle (252.14, 42.22);

\path[fill=fillColor,fill opacity=0.60] (252.14, 42.22) rectangle (254.09, 42.22);

\path[fill=fillColor,fill opacity=0.60] (254.09, 42.22) rectangle (256.03, 42.22);

\path[fill=fillColor,fill opacity=0.60] (256.03, 42.22) rectangle (257.97, 42.22);

\path[fill=fillColor,fill opacity=0.60] (257.97, 42.22) rectangle (259.92, 42.22);

\path[fill=fillColor,fill opacity=0.60] (259.92, 42.22) rectangle (261.86, 42.22);

\path[fill=fillColor,fill opacity=0.60] (261.86, 42.22) rectangle (263.80, 42.22);

\path[fill=fillColor,fill opacity=0.60] (263.80, 42.22) rectangle (265.75, 42.22);

\path[fill=fillColor,fill opacity=0.60] (265.75, 42.22) rectangle (267.69, 42.22);

\path[fill=fillColor,fill opacity=0.60] (267.69, 42.22) rectangle (269.63, 42.22);

\path[fill=fillColor,fill opacity=0.60] (269.63, 42.22) rectangle (271.58, 42.22);

\path[fill=fillColor,fill opacity=0.60] (271.58, 42.22) rectangle (273.52, 42.22);

\path[fill=fillColor,fill opacity=0.60] (273.52, 42.22) rectangle (275.46, 42.22);

\path[fill=fillColor,fill opacity=0.60] (275.46, 42.22) rectangle (277.41, 42.22);

\path[fill=fillColor,fill opacity=0.60] (277.41, 42.22) rectangle (279.35, 42.22);

\path[fill=fillColor,fill opacity=0.60] (279.35, 42.22) rectangle (281.29, 42.22);

\path[fill=fillColor,fill opacity=0.60] (281.29, 42.22) rectangle (283.24, 42.22);

\path[fill=fillColor,fill opacity=0.60] (283.24, 42.22) rectangle (285.18, 42.22);

\path[fill=fillColor,fill opacity=0.60] (285.18, 42.22) rectangle (287.13, 42.22);

\path[fill=fillColor,fill opacity=0.60] (287.13, 42.22) rectangle (289.07, 42.22);

\path[fill=fillColor,fill opacity=0.60] (289.07, 42.22) rectangle (291.01, 42.22);

\path[fill=fillColor,fill opacity=0.60] (291.01, 42.22) rectangle (292.96, 42.22);

\path[fill=fillColor,fill opacity=0.60] (292.96, 42.22) rectangle (294.90, 42.22);

\path[fill=fillColor,fill opacity=0.60] (294.90, 42.22) rectangle (296.84, 42.22);

\path[fill=fillColor,fill opacity=0.60] (296.84, 42.22) rectangle (298.79, 42.22);

\path[fill=fillColor,fill opacity=0.60] (298.79, 42.22) rectangle (300.73, 42.22);

\path[fill=fillColor,fill opacity=0.60] (300.73, 42.22) rectangle (302.67, 42.22);

\path[fill=fillColor,fill opacity=0.60] (302.67, 42.22) rectangle (304.62, 42.22);

\path[fill=fillColor,fill opacity=0.60] (304.62, 42.22) rectangle (306.56, 42.22);

\path[fill=fillColor,fill opacity=0.60] (306.56, 42.22) rectangle (308.50, 42.22);

\path[fill=fillColor,fill opacity=0.60] (308.50, 42.22) rectangle (310.45, 42.22);

\path[fill=fillColor,fill opacity=0.60] (310.45, 42.22) rectangle (312.39, 42.22);

\path[fill=fillColor,fill opacity=0.60] (312.39, 42.22) rectangle (314.33, 42.22);

\path[fill=fillColor,fill opacity=0.60] (314.33, 42.22) rectangle (316.28, 42.22);

\path[fill=fillColor,fill opacity=0.60] (316.28, 42.22) rectangle (318.22, 42.22);

\path[fill=fillColor,fill opacity=0.60] (318.22, 42.22) rectangle (320.17, 42.22);

\path[fill=fillColor,fill opacity=0.60] (320.17, 42.22) rectangle (322.11, 42.22);

\path[fill=fillColor,fill opacity=0.60] (322.11, 42.22) rectangle (324.05, 42.22);

\path[fill=fillColor,fill opacity=0.60] (324.05, 42.22) rectangle (326.00, 42.22);

\path[fill=fillColor,fill opacity=0.60] (326.00, 42.22) rectangle (327.94, 42.22);

\path[fill=fillColor,fill opacity=0.60] (327.94, 42.22) rectangle (329.88, 42.22);

\path[fill=fillColor,fill opacity=0.60] (329.88, 42.22) rectangle (331.83, 42.22);

\path[fill=fillColor,fill opacity=0.60] (331.83, 42.22) rectangle (333.77, 42.22);

\path[fill=fillColor,fill opacity=0.60] (333.77, 42.22) rectangle (335.71, 42.22);

\path[fill=fillColor,fill opacity=0.60] (335.71, 42.22) rectangle (337.66, 42.22);

\path[fill=fillColor,fill opacity=0.60] (337.66, 42.22) rectangle (339.60, 42.22);

\path[fill=fillColor,fill opacity=0.60] (339.60, 42.22) rectangle (341.54, 42.22);

\path[fill=fillColor,fill opacity=0.60] (341.54, 42.22) rectangle (343.49, 42.22);

\path[fill=fillColor,fill opacity=0.60] (343.49, 42.22) rectangle (345.43, 42.22);

\path[fill=fillColor,fill opacity=0.60] (345.43, 42.22) rectangle (347.37, 42.22);

\path[fill=fillColor,fill opacity=0.60] (347.37, 42.22) rectangle (349.32, 42.22);

\path[fill=fillColor,fill opacity=0.60] (349.32, 42.22) rectangle (351.26, 42.22);

\path[fill=fillColor,fill opacity=0.60] (351.26, 42.22) rectangle (353.21, 42.22);

\path[fill=fillColor,fill opacity=0.60] (353.21, 42.22) rectangle (355.15, 42.22);

\path[fill=fillColor,fill opacity=0.60] (355.15, 42.22) rectangle (357.09, 42.22);

\path[fill=fillColor,fill opacity=0.60] (357.09, 42.22) rectangle (359.04, 42.22);

\path[fill=fillColor,fill opacity=0.60] (359.04, 42.22) rectangle (360.98, 42.22);

\path[fill=fillColor,fill opacity=0.60] (360.98, 42.22) rectangle (362.92, 42.22);

\path[fill=fillColor,fill opacity=0.60] (362.92, 42.22) rectangle (364.87, 42.22);

\path[fill=fillColor,fill opacity=0.60] (364.87, 42.22) rectangle (366.81, 42.22);

\path[fill=fillColor,fill opacity=0.60] (366.81, 42.22) rectangle (368.75, 42.22);

\path[fill=fillColor,fill opacity=0.60] (368.75, 42.22) rectangle (370.70, 42.22);

\path[fill=fillColor,fill opacity=0.60] (370.70, 42.22) rectangle (372.64, 42.22);

\path[fill=fillColor,fill opacity=0.60] (372.64, 42.22) rectangle (374.58, 42.22);

\path[fill=fillColor,fill opacity=0.60] (374.58, 42.22) rectangle (376.53, 42.22);

\path[fill=fillColor,fill opacity=0.60] (376.53, 42.22) rectangle (378.47, 42.22);

\path[fill=fillColor,fill opacity=0.60] (378.47, 42.22) rectangle (380.41, 42.22);

\path[fill=fillColor,fill opacity=0.60] (380.41, 42.22) rectangle (382.36, 42.22);

\path[fill=fillColor,fill opacity=0.60] (382.36, 42.22) rectangle (384.30, 42.22);

\path[fill=fillColor,fill opacity=0.60] (384.30, 42.22) rectangle (386.25, 42.22);

\path[fill=fillColor,fill opacity=0.60] (386.25, 42.22) rectangle (388.19, 42.22);

\path[fill=fillColor,fill opacity=0.60] (388.19, 42.22) rectangle (390.13, 42.22);

\path[fill=fillColor,fill opacity=0.60] (390.13, 42.22) rectangle (392.08, 42.22);

\path[fill=fillColor,fill opacity=0.60] (392.08, 42.22) rectangle (394.02, 42.22);

\path[fill=fillColor,fill opacity=0.60] (394.02, 42.22) rectangle (395.96, 42.22);

\path[fill=fillColor,fill opacity=0.60] (395.96, 42.22) rectangle (397.91, 42.22);

\path[fill=fillColor,fill opacity=0.60] (397.91, 42.22) rectangle (399.85, 42.22);

\path[fill=fillColor,fill opacity=0.60] (399.85, 42.22) rectangle (401.79, 42.22);

\path[fill=fillColor,fill opacity=0.60] (401.79, 42.22) rectangle (403.74, 42.22);

\path[fill=fillColor,fill opacity=0.60] (403.74, 42.22) rectangle (405.68, 42.22);

\path[fill=fillColor,fill opacity=0.60] (405.68, 42.22) rectangle (407.62, 42.22);

\path[fill=fillColor,fill opacity=0.60] (407.62, 42.22) rectangle (409.57, 42.22);

\path[fill=fillColor,fill opacity=0.60] (409.57, 42.22) rectangle (411.51, 42.22);

\path[fill=fillColor,fill opacity=0.60] (411.51, 42.22) rectangle (413.45, 42.22);

\path[fill=fillColor,fill opacity=0.60] (413.45, 42.22) rectangle (415.40, 42.22);

\path[fill=fillColor,fill opacity=0.60] (415.40, 42.22) rectangle (417.34, 42.22);

\path[fill=fillColor,fill opacity=0.60] (417.34, 42.22) rectangle (419.29, 42.22);

\path[fill=fillColor,fill opacity=0.60] (419.29, 42.22) rectangle (421.23, 42.22);

\path[fill=fillColor,fill opacity=0.60] (421.23, 42.22) rectangle (423.17, 42.22);

\path[fill=fillColor,fill opacity=0.60] (423.17, 42.22) rectangle (425.12, 42.22);

\path[fill=fillColor,fill opacity=0.60] (425.12, 42.22) rectangle (427.06, 42.22);

\path[fill=fillColor,fill opacity=0.60] (427.06, 42.22) rectangle (429.00, 42.22);

\path[fill=fillColor,fill opacity=0.60] (429.00, 42.22) rectangle (430.95, 42.22);

\path[fill=fillColor,fill opacity=0.60] (430.95, 42.22) rectangle (432.89, 42.22);

\path[fill=fillColor,fill opacity=0.60] (432.89, 42.22) rectangle (434.83, 42.22);

\path[fill=fillColor,fill opacity=0.60] (434.83, 42.22) rectangle (436.78, 42.22);

\path[fill=fillColor,fill opacity=0.60] (436.78, 42.22) rectangle (438.72, 42.22);

\path[fill=fillColor,fill opacity=0.60] (438.72, 42.22) rectangle (440.66, 42.22);

\path[fill=fillColor,fill opacity=0.60] (440.66, 42.22) rectangle (442.61, 42.22);

\path[fill=fillColor,fill opacity=0.60] (442.61, 42.22) rectangle (444.55, 42.22);

\path[fill=fillColor,fill opacity=0.60] (444.55, 42.22) rectangle (446.49, 42.22);

\path[fill=fillColor,fill opacity=0.60] (446.49, 42.22) rectangle (448.44, 42.22);

\path[fill=fillColor,fill opacity=0.60] (448.44, 42.22) rectangle (450.38, 42.22);

\path[fill=fillColor,fill opacity=0.60] (450.38, 42.22) rectangle (452.33, 42.22);

\path[fill=fillColor,fill opacity=0.60] (452.33, 42.22) rectangle (454.27, 42.22);

\path[fill=fillColor,fill opacity=0.60] (454.27, 42.22) rectangle (456.21, 42.22);

\path[fill=fillColor,fill opacity=0.60] (456.21, 42.22) rectangle (458.16, 42.22);

\path[fill=fillColor,fill opacity=0.60] (458.16, 42.22) rectangle (460.10, 42.22);

\path[fill=fillColor,fill opacity=0.60] (460.10, 42.22) rectangle (462.04, 42.22);

\path[fill=fillColor,fill opacity=0.60] (462.04, 42.22) rectangle (463.99, 42.22);

\path[fill=fillColor,fill opacity=0.60] (463.99, 42.22) rectangle (465.93, 42.22);

\path[fill=fillColor,fill opacity=0.60] (465.93, 42.22) rectangle (467.87, 42.22);

\path[fill=fillColor,fill opacity=0.60] (467.87, 42.22) rectangle (469.82, 42.22);

\path[fill=fillColor,fill opacity=0.60] (469.82, 42.22) rectangle (471.76, 42.22);

\path[fill=fillColor,fill opacity=0.60] (471.76, 42.22) rectangle (473.70, 42.22);

\path[fill=fillColor,fill opacity=0.60] (473.70, 42.22) rectangle (475.65, 42.22);

\path[fill=fillColor,fill opacity=0.60] (475.65, 42.22) rectangle (477.59, 42.22);

\path[fill=fillColor,fill opacity=0.60] (477.59, 42.22) rectangle (479.54, 42.22);

\path[fill=fillColor,fill opacity=0.60] (479.54, 42.22) rectangle (481.48, 42.22);

\path[fill=fillColor,fill opacity=0.60] (481.48, 42.22) rectangle (483.42, 42.22);

\path[fill=fillColor,fill opacity=0.60] (483.42, 42.22) rectangle (485.37, 42.22);

\path[fill=fillColor,fill opacity=0.60] (485.37, 42.22) rectangle (487.31, 42.22);

\path[fill=fillColor,fill opacity=0.60] (487.31, 42.22) rectangle (489.25, 42.22);

\path[fill=fillColor,fill opacity=0.60] (489.25, 42.22) rectangle (491.20, 42.22);

\path[fill=fillColor,fill opacity=0.60] (491.20, 42.22) rectangle (493.14, 42.22);

\path[fill=fillColor,fill opacity=0.60] (493.14, 42.22) rectangle (495.08, 42.22);

\path[fill=fillColor,fill opacity=0.60] (495.08, 42.22) rectangle (497.03, 42.22);

\path[fill=fillColor,fill opacity=0.60] (497.03, 42.22) rectangle (498.97, 42.22);

\path[fill=fillColor,fill opacity=0.60] (498.97, 42.22) rectangle (500.91, 42.22);

\path[fill=fillColor,fill opacity=0.60] (500.91, 42.22) rectangle (502.86, 42.22);

\path[fill=fillColor,fill opacity=0.60] (502.86, 42.22) rectangle (504.80, 42.22);

\path[fill=fillColor,fill opacity=0.60] (504.80, 42.22) rectangle (506.74, 42.22);

\path[fill=fillColor,fill opacity=0.60] (506.74, 42.22) rectangle (508.69, 42.22);

\path[fill=fillColor,fill opacity=0.60] (508.69, 42.22) rectangle (510.63, 42.22);

\path[fill=fillColor,fill opacity=0.60] (510.63, 42.22) rectangle (512.58, 42.22);

\path[fill=fillColor,fill opacity=0.60] (512.58, 42.22) rectangle (514.52, 42.22);

\path[fill=fillColor,fill opacity=0.60] (514.52, 42.22) rectangle (516.46, 42.22);

\path[fill=fillColor,fill opacity=0.60] (516.46, 42.22) rectangle (518.41, 42.22);

\path[fill=fillColor,fill opacity=0.60] (518.41, 42.22) rectangle (520.35, 42.22);

\path[fill=fillColor,fill opacity=0.60] (520.35, 42.22) rectangle (522.29, 42.22);

\path[fill=fillColor,fill opacity=0.60] (522.29, 42.22) rectangle (524.24, 42.22);

\path[fill=fillColor,fill opacity=0.60] (524.24, 42.22) rectangle (526.18, 42.22);

\path[fill=fillColor,fill opacity=0.60] (526.18, 42.22) rectangle (528.12, 42.22);

\path[fill=fillColor,fill opacity=0.60] (528.12, 42.22) rectangle (530.07, 42.22);

\path[fill=fillColor,fill opacity=0.60] (530.07, 42.22) rectangle (532.01, 42.22);

\path[fill=fillColor,fill opacity=0.60] (532.01, 42.22) rectangle (533.95, 42.22);

\path[fill=fillColor,fill opacity=0.60] (533.95, 42.22) rectangle (535.90, 42.22);

\path[fill=fillColor,fill opacity=0.60] (535.90, 42.22) rectangle (537.84, 42.22);

\path[fill=fillColor,fill opacity=0.60] (537.84, 42.22) rectangle (539.78, 42.22);

\path[fill=fillColor,fill opacity=0.60] (539.78, 42.22) rectangle (541.73, 42.22);

\path[fill=fillColor,fill opacity=0.60] (541.73, 42.22) rectangle (543.67, 42.22);

\path[fill=fillColor,fill opacity=0.60] (543.67, 42.22) rectangle (545.62, 42.22);

\path[fill=fillColor,fill opacity=0.60] (545.62, 42.22) rectangle (547.56, 42.22);
\definecolor{fillColor}{RGB}{0,255,0}

\path[fill=fillColor,fill opacity=0.60] ( 63.73, 42.22) rectangle ( 65.90, 42.22);

\path[fill=fillColor,fill opacity=0.60] ( 65.90, 42.22) rectangle ( 68.07, 42.22);

\path[fill=fillColor,fill opacity=0.60] ( 68.07, 42.22) rectangle ( 70.23, 42.22);

\path[fill=fillColor,fill opacity=0.60] ( 70.23, 42.22) rectangle ( 72.40, 42.23);

\path[fill=fillColor,fill opacity=0.60] ( 72.40, 42.22) rectangle ( 74.57, 42.29);

\path[fill=fillColor,fill opacity=0.60] ( 74.57, 42.22) rectangle ( 76.74, 42.47);

\path[fill=fillColor,fill opacity=0.60] ( 76.74, 42.22) rectangle ( 78.91, 42.98);

\path[fill=fillColor,fill opacity=0.60] ( 78.91, 42.22) rectangle ( 81.07, 44.27);

\path[fill=fillColor,fill opacity=0.60] ( 81.07, 42.22) rectangle ( 83.24, 47.30);

\path[fill=fillColor,fill opacity=0.60] ( 83.24, 42.22) rectangle ( 85.41, 52.08);

\path[fill=fillColor,fill opacity=0.60] ( 85.41, 42.22) rectangle ( 87.58, 60.60);

\path[fill=fillColor,fill opacity=0.60] ( 87.58, 42.22) rectangle ( 89.75, 72.39);

\path[fill=fillColor,fill opacity=0.60] ( 89.75, 42.22) rectangle ( 91.91, 89.36);

\path[fill=fillColor,fill opacity=0.60] ( 91.91, 42.22) rectangle ( 94.08,110.95);

\path[fill=fillColor,fill opacity=0.60] ( 94.08, 42.22) rectangle ( 96.25,135.00);

\path[fill=fillColor,fill opacity=0.60] ( 96.25, 42.22) rectangle ( 98.42,160.82);

\path[fill=fillColor,fill opacity=0.60] ( 98.42, 42.22) rectangle (100.59,181.15);

\path[fill=fillColor,fill opacity=0.60] (100.59, 42.22) rectangle (102.75,197.13);

\path[fill=fillColor,fill opacity=0.60] (102.75, 42.22) rectangle (104.92,204.57);

\path[fill=fillColor,fill opacity=0.60] (104.92, 42.22) rectangle (107.09,200.85);

\path[fill=fillColor,fill opacity=0.60] (107.09, 42.22) rectangle (109.26,186.36);

\path[fill=fillColor,fill opacity=0.60] (109.26, 42.22) rectangle (111.43,162.14);

\path[fill=fillColor,fill opacity=0.60] (111.43, 42.22) rectangle (113.59,135.79);

\path[fill=fillColor,fill opacity=0.60] (113.59, 42.22) rectangle (115.76,110.08);

\path[fill=fillColor,fill opacity=0.60] (115.76, 42.22) rectangle (117.93, 86.13);

\path[fill=fillColor,fill opacity=0.60] (117.93, 42.22) rectangle (120.10, 69.34);

\path[fill=fillColor,fill opacity=0.60] (120.10, 42.22) rectangle (122.27, 57.01);

\path[fill=fillColor,fill opacity=0.60] (122.27, 42.22) rectangle (124.43, 49.44);

\path[fill=fillColor,fill opacity=0.60] (124.43, 42.22) rectangle (126.60, 45.53);

\path[fill=fillColor,fill opacity=0.60] (126.60, 42.22) rectangle (128.77, 43.39);

\path[fill=fillColor,fill opacity=0.60] (128.77, 42.22) rectangle (130.94, 42.68);

\path[fill=fillColor,fill opacity=0.60] (130.94, 42.22) rectangle (133.11, 42.39);

\path[fill=fillColor,fill opacity=0.60] (133.11, 42.22) rectangle (135.27, 42.25);

\path[fill=fillColor,fill opacity=0.60] (135.27, 42.22) rectangle (137.44, 42.24);

\path[fill=fillColor,fill opacity=0.60] (137.44, 42.22) rectangle (139.61, 42.22);

\path[fill=fillColor,fill opacity=0.60] (139.61, 42.22) rectangle (141.78, 42.22);

\path[fill=fillColor,fill opacity=0.60] (141.78, 42.22) rectangle (143.95, 42.22);

\path[fill=fillColor,fill opacity=0.60] (143.95, 42.22) rectangle (146.11, 42.22);

\path[fill=fillColor,fill opacity=0.60] (146.11, 42.22) rectangle (148.28, 42.22);

\path[fill=fillColor,fill opacity=0.60] (148.28, 42.22) rectangle (150.45, 42.22);

\path[fill=fillColor,fill opacity=0.60] (150.45, 42.22) rectangle (152.62, 42.22);

\path[fill=fillColor,fill opacity=0.60] (152.62, 42.22) rectangle (154.79, 42.22);

\path[fill=fillColor,fill opacity=0.60] (154.79, 42.22) rectangle (156.95, 42.22);

\path[fill=fillColor,fill opacity=0.60] (156.95, 42.22) rectangle (159.12, 42.22);

\path[fill=fillColor,fill opacity=0.60] (159.12, 42.22) rectangle (161.29, 42.22);

\path[fill=fillColor,fill opacity=0.60] (161.29, 42.22) rectangle (163.46, 42.22);

\path[fill=fillColor,fill opacity=0.60] (163.46, 42.22) rectangle (165.63, 42.22);

\path[fill=fillColor,fill opacity=0.60] (165.63, 42.22) rectangle (167.79, 42.22);

\path[fill=fillColor,fill opacity=0.60] (167.79, 42.22) rectangle (169.96, 42.22);

\path[fill=fillColor,fill opacity=0.60] (169.96, 42.22) rectangle (172.13, 42.22);

\path[fill=fillColor,fill opacity=0.60] (172.13, 42.22) rectangle (174.30, 42.22);

\path[fill=fillColor,fill opacity=0.60] (174.30, 42.22) rectangle (176.47, 42.22);

\path[fill=fillColor,fill opacity=0.60] (176.47, 42.22) rectangle (178.63, 42.22);

\path[fill=fillColor,fill opacity=0.60] (178.63, 42.22) rectangle (180.80, 42.22);

\path[fill=fillColor,fill opacity=0.60] (180.80, 42.22) rectangle (182.97, 42.22);

\path[fill=fillColor,fill opacity=0.60] (182.97, 42.22) rectangle (185.14, 42.22);

\path[fill=fillColor,fill opacity=0.60] (185.14, 42.22) rectangle (187.31, 42.22);

\path[fill=fillColor,fill opacity=0.60] (187.31, 42.22) rectangle (189.47, 42.22);

\path[fill=fillColor,fill opacity=0.60] (189.47, 42.22) rectangle (191.64, 42.22);

\path[fill=fillColor,fill opacity=0.60] (191.64, 42.22) rectangle (193.81, 42.22);

\path[fill=fillColor,fill opacity=0.60] (193.81, 42.22) rectangle (195.98, 42.22);

\path[fill=fillColor,fill opacity=0.60] (195.98, 42.22) rectangle (198.15, 42.22);

\path[fill=fillColor,fill opacity=0.60] (198.15, 42.22) rectangle (200.31, 42.22);

\path[fill=fillColor,fill opacity=0.60] (200.31, 42.22) rectangle (202.48, 42.22);

\path[fill=fillColor,fill opacity=0.60] (202.48, 42.22) rectangle (204.65, 42.22);

\path[fill=fillColor,fill opacity=0.60] (204.65, 42.22) rectangle (206.82, 42.22);

\path[fill=fillColor,fill opacity=0.60] (206.82, 42.22) rectangle (208.99, 42.22);

\path[fill=fillColor,fill opacity=0.60] (208.99, 42.22) rectangle (211.15, 42.22);

\path[fill=fillColor,fill opacity=0.60] (211.15, 42.22) rectangle (213.32, 42.22);

\path[fill=fillColor,fill opacity=0.60] (213.32, 42.22) rectangle (215.49, 42.22);

\path[fill=fillColor,fill opacity=0.60] (215.49, 42.22) rectangle (217.66, 42.22);

\path[fill=fillColor,fill opacity=0.60] (217.66, 42.22) rectangle (219.83, 42.22);

\path[fill=fillColor,fill opacity=0.60] (219.83, 42.22) rectangle (221.99, 42.22);

\path[fill=fillColor,fill opacity=0.60] (221.99, 42.22) rectangle (224.16, 42.22);

\path[fill=fillColor,fill opacity=0.60] (224.16, 42.22) rectangle (226.33, 42.22);

\path[fill=fillColor,fill opacity=0.60] (226.33, 42.22) rectangle (228.50, 42.22);

\path[fill=fillColor,fill opacity=0.60] (228.50, 42.22) rectangle (230.67, 42.22);

\path[fill=fillColor,fill opacity=0.60] (230.67, 42.22) rectangle (232.83, 42.22);

\path[fill=fillColor,fill opacity=0.60] (232.83, 42.22) rectangle (235.00, 42.22);

\path[fill=fillColor,fill opacity=0.60] (235.00, 42.22) rectangle (237.17, 42.22);

\path[fill=fillColor,fill opacity=0.60] (237.17, 42.22) rectangle (239.34, 42.22);

\path[fill=fillColor,fill opacity=0.60] (239.34, 42.22) rectangle (241.51, 42.22);

\path[fill=fillColor,fill opacity=0.60] (241.51, 42.22) rectangle (243.67, 42.22);

\path[fill=fillColor,fill opacity=0.60] (243.67, 42.22) rectangle (245.84, 42.22);

\path[fill=fillColor,fill opacity=0.60] (245.84, 42.22) rectangle (248.01, 42.22);

\path[fill=fillColor,fill opacity=0.60] (248.01, 42.22) rectangle (250.18, 42.22);

\path[fill=fillColor,fill opacity=0.60] (250.18, 42.22) rectangle (252.35, 42.22);

\path[fill=fillColor,fill opacity=0.60] (252.35, 42.22) rectangle (254.51, 42.22);

\path[fill=fillColor,fill opacity=0.60] (254.51, 42.22) rectangle (256.68, 42.22);

\path[fill=fillColor,fill opacity=0.60] (256.68, 42.22) rectangle (258.85, 42.22);

\path[fill=fillColor,fill opacity=0.60] (258.85, 42.22) rectangle (261.02, 42.22);

\path[fill=fillColor,fill opacity=0.60] (261.02, 42.22) rectangle (263.19, 42.22);

\path[fill=fillColor,fill opacity=0.60] (263.19, 42.22) rectangle (265.35, 42.22);

\path[fill=fillColor,fill opacity=0.60] (265.35, 42.22) rectangle (267.52, 42.22);

\path[fill=fillColor,fill opacity=0.60] (267.52, 42.22) rectangle (269.69, 42.22);

\path[fill=fillColor,fill opacity=0.60] (269.69, 42.22) rectangle (271.86, 42.22);

\path[fill=fillColor,fill opacity=0.60] (271.86, 42.22) rectangle (274.03, 42.22);

\path[fill=fillColor,fill opacity=0.60] (274.03, 42.22) rectangle (276.19, 42.22);

\path[fill=fillColor,fill opacity=0.60] (276.19, 42.22) rectangle (278.36, 42.22);

\path[fill=fillColor,fill opacity=0.60] (278.36, 42.22) rectangle (280.53, 42.22);

\path[fill=fillColor,fill opacity=0.60] (280.53, 42.22) rectangle (282.70, 42.22);

\path[fill=fillColor,fill opacity=0.60] (282.70, 42.22) rectangle (284.87, 42.22);

\path[fill=fillColor,fill opacity=0.60] (284.87, 42.22) rectangle (287.03, 42.22);

\path[fill=fillColor,fill opacity=0.60] (287.03, 42.22) rectangle (289.20, 42.22);

\path[fill=fillColor,fill opacity=0.60] (289.20, 42.22) rectangle (291.37, 42.22);

\path[fill=fillColor,fill opacity=0.60] (291.37, 42.22) rectangle (293.54, 42.22);

\path[fill=fillColor,fill opacity=0.60] (293.54, 42.22) rectangle (295.71, 42.22);

\path[fill=fillColor,fill opacity=0.60] (295.71, 42.22) rectangle (297.87, 42.22);

\path[fill=fillColor,fill opacity=0.60] (297.87, 42.22) rectangle (300.04, 42.22);

\path[fill=fillColor,fill opacity=0.60] (300.04, 42.22) rectangle (302.21, 42.22);

\path[fill=fillColor,fill opacity=0.60] (302.21, 42.22) rectangle (304.38, 42.22);

\path[fill=fillColor,fill opacity=0.60] (304.38, 42.22) rectangle (306.55, 42.22);

\path[fill=fillColor,fill opacity=0.60] (306.55, 42.22) rectangle (308.71, 42.22);

\path[fill=fillColor,fill opacity=0.60] (308.71, 42.22) rectangle (310.88, 42.22);

\path[fill=fillColor,fill opacity=0.60] (310.88, 42.22) rectangle (313.05, 42.22);

\path[fill=fillColor,fill opacity=0.60] (313.05, 42.22) rectangle (315.22, 42.22);

\path[fill=fillColor,fill opacity=0.60] (315.22, 42.22) rectangle (317.39, 42.22);

\path[fill=fillColor,fill opacity=0.60] (317.39, 42.22) rectangle (319.55, 42.22);

\path[fill=fillColor,fill opacity=0.60] (319.55, 42.22) rectangle (321.72, 42.22);

\path[fill=fillColor,fill opacity=0.60] (321.72, 42.22) rectangle (323.89, 42.22);

\path[fill=fillColor,fill opacity=0.60] (323.89, 42.22) rectangle (326.06, 42.22);

\path[fill=fillColor,fill opacity=0.60] (326.06, 42.22) rectangle (328.23, 42.22);

\path[fill=fillColor,fill opacity=0.60] (328.23, 42.22) rectangle (330.39, 42.22);

\path[fill=fillColor,fill opacity=0.60] (330.39, 42.22) rectangle (332.56, 42.22);

\path[fill=fillColor,fill opacity=0.60] (332.56, 42.22) rectangle (334.73, 42.22);

\path[fill=fillColor,fill opacity=0.60] (334.73, 42.22) rectangle (336.90, 42.22);

\path[fill=fillColor,fill opacity=0.60] (336.90, 42.22) rectangle (339.07, 42.22);

\path[fill=fillColor,fill opacity=0.60] (339.07, 42.22) rectangle (341.24, 42.22);

\path[fill=fillColor,fill opacity=0.60] (341.24, 42.22) rectangle (343.40, 42.22);

\path[fill=fillColor,fill opacity=0.60] (343.40, 42.22) rectangle (345.57, 42.22);

\path[fill=fillColor,fill opacity=0.60] (345.57, 42.22) rectangle (347.74, 42.22);

\path[fill=fillColor,fill opacity=0.60] (347.74, 42.22) rectangle (349.91, 42.22);

\path[fill=fillColor,fill opacity=0.60] (349.91, 42.22) rectangle (352.08, 42.22);

\path[fill=fillColor,fill opacity=0.60] (352.08, 42.22) rectangle (354.24, 42.22);

\path[fill=fillColor,fill opacity=0.60] (354.24, 42.22) rectangle (356.41, 42.22);

\path[fill=fillColor,fill opacity=0.60] (356.41, 42.22) rectangle (358.58, 42.22);

\path[fill=fillColor,fill opacity=0.60] (358.58, 42.22) rectangle (360.75, 42.22);

\path[fill=fillColor,fill opacity=0.60] (360.75, 42.22) rectangle (362.92, 42.22);

\path[fill=fillColor,fill opacity=0.60] (362.92, 42.22) rectangle (365.08, 42.22);

\path[fill=fillColor,fill opacity=0.60] (365.08, 42.22) rectangle (367.25, 42.22);

\path[fill=fillColor,fill opacity=0.60] (367.25, 42.22) rectangle (369.42, 42.22);

\path[fill=fillColor,fill opacity=0.60] (369.42, 42.22) rectangle (371.59, 42.22);

\path[fill=fillColor,fill opacity=0.60] (371.59, 42.22) rectangle (373.76, 42.22);

\path[fill=fillColor,fill opacity=0.60] (373.76, 42.22) rectangle (375.92, 42.22);

\path[fill=fillColor,fill opacity=0.60] (375.92, 42.22) rectangle (378.09, 42.22);

\path[fill=fillColor,fill opacity=0.60] (378.09, 42.22) rectangle (380.26, 42.22);

\path[fill=fillColor,fill opacity=0.60] (380.26, 42.22) rectangle (382.43, 42.22);

\path[fill=fillColor,fill opacity=0.60] (382.43, 42.22) rectangle (384.60, 42.22);

\path[fill=fillColor,fill opacity=0.60] (384.60, 42.22) rectangle (386.76, 42.22);

\path[fill=fillColor,fill opacity=0.60] (386.76, 42.22) rectangle (388.93, 42.22);

\path[fill=fillColor,fill opacity=0.60] (388.93, 42.22) rectangle (391.10, 42.22);

\path[fill=fillColor,fill opacity=0.60] (391.10, 42.22) rectangle (393.27, 42.22);

\path[fill=fillColor,fill opacity=0.60] (393.27, 42.22) rectangle (395.44, 42.22);

\path[fill=fillColor,fill opacity=0.60] (395.44, 42.22) rectangle (397.60, 42.22);

\path[fill=fillColor,fill opacity=0.60] (397.60, 42.22) rectangle (399.77, 42.22);

\path[fill=fillColor,fill opacity=0.60] (399.77, 42.22) rectangle (401.94, 42.22);

\path[fill=fillColor,fill opacity=0.60] (401.94, 42.22) rectangle (404.11, 42.22);

\path[fill=fillColor,fill opacity=0.60] (404.11, 42.22) rectangle (406.28, 42.22);

\path[fill=fillColor,fill opacity=0.60] (406.28, 42.22) rectangle (408.44, 42.22);

\path[fill=fillColor,fill opacity=0.60] (408.44, 42.22) rectangle (410.61, 42.22);

\path[fill=fillColor,fill opacity=0.60] (410.61, 42.22) rectangle (412.78, 42.22);

\path[fill=fillColor,fill opacity=0.60] (412.78, 42.22) rectangle (414.95, 42.22);

\path[fill=fillColor,fill opacity=0.60] (414.95, 42.22) rectangle (417.12, 42.22);

\path[fill=fillColor,fill opacity=0.60] (417.12, 42.22) rectangle (419.28, 42.22);

\path[fill=fillColor,fill opacity=0.60] (419.28, 42.22) rectangle (421.45, 42.22);

\path[fill=fillColor,fill opacity=0.60] (421.45, 42.22) rectangle (423.62, 42.22);

\path[fill=fillColor,fill opacity=0.60] (423.62, 42.22) rectangle (425.79, 42.22);

\path[fill=fillColor,fill opacity=0.60] (425.79, 42.22) rectangle (427.96, 42.22);

\path[fill=fillColor,fill opacity=0.60] (427.96, 42.22) rectangle (430.12, 42.22);

\path[fill=fillColor,fill opacity=0.60] (430.12, 42.22) rectangle (432.29, 42.22);

\path[fill=fillColor,fill opacity=0.60] (432.29, 42.22) rectangle (434.46, 42.22);

\path[fill=fillColor,fill opacity=0.60] (434.46, 42.22) rectangle (436.63, 42.22);

\path[fill=fillColor,fill opacity=0.60] (436.63, 42.22) rectangle (438.80, 42.22);

\path[fill=fillColor,fill opacity=0.60] (438.80, 42.22) rectangle (440.96, 42.22);

\path[fill=fillColor,fill opacity=0.60] (440.96, 42.22) rectangle (443.13, 42.22);

\path[fill=fillColor,fill opacity=0.60] (443.13, 42.22) rectangle (445.30, 42.22);

\path[fill=fillColor,fill opacity=0.60] (445.30, 42.22) rectangle (447.47, 42.22);

\path[fill=fillColor,fill opacity=0.60] (447.47, 42.22) rectangle (449.64, 42.22);

\path[fill=fillColor,fill opacity=0.60] (449.64, 42.22) rectangle (451.80, 42.22);

\path[fill=fillColor,fill opacity=0.60] (451.80, 42.22) rectangle (453.97, 42.22);

\path[fill=fillColor,fill opacity=0.60] (453.97, 42.22) rectangle (456.14, 42.22);

\path[fill=fillColor,fill opacity=0.60] (456.14, 42.22) rectangle (458.31, 42.22);

\path[fill=fillColor,fill opacity=0.60] (458.31, 42.22) rectangle (460.48, 42.22);

\path[fill=fillColor,fill opacity=0.60] (460.48, 42.22) rectangle (462.64, 42.22);

\path[fill=fillColor,fill opacity=0.60] (462.64, 42.22) rectangle (464.81, 42.22);

\path[fill=fillColor,fill opacity=0.60] (464.81, 42.22) rectangle (466.98, 42.22);

\path[fill=fillColor,fill opacity=0.60] (466.98, 42.22) rectangle (469.15, 42.22);

\path[fill=fillColor,fill opacity=0.60] (469.15, 42.22) rectangle (471.32, 42.22);

\path[fill=fillColor,fill opacity=0.60] (471.32, 42.22) rectangle (473.48, 42.22);

\path[fill=fillColor,fill opacity=0.60] (473.48, 42.22) rectangle (475.65, 42.22);

\path[fill=fillColor,fill opacity=0.60] (475.65, 42.22) rectangle (477.82, 42.22);

\path[fill=fillColor,fill opacity=0.60] (477.82, 42.22) rectangle (479.99, 42.22);

\path[fill=fillColor,fill opacity=0.60] (479.99, 42.22) rectangle (482.16, 42.22);

\path[fill=fillColor,fill opacity=0.60] (482.16, 42.22) rectangle (484.32, 42.22);

\path[fill=fillColor,fill opacity=0.60] (484.32, 42.22) rectangle (486.49, 42.22);

\path[fill=fillColor,fill opacity=0.60] (486.49, 42.22) rectangle (488.66, 42.22);

\path[fill=fillColor,fill opacity=0.60] (488.66, 42.22) rectangle (490.83, 42.22);

\path[fill=fillColor,fill opacity=0.60] (490.83, 42.22) rectangle (493.00, 42.22);

\path[fill=fillColor,fill opacity=0.60] (493.00, 42.22) rectangle (495.16, 42.22);

\path[fill=fillColor,fill opacity=0.60] (495.16, 42.22) rectangle (497.33, 42.22);

\path[fill=fillColor,fill opacity=0.60] (497.33, 42.22) rectangle (499.50, 42.22);

\path[fill=fillColor,fill opacity=0.60] (499.50, 42.22) rectangle (501.67, 42.22);

\path[fill=fillColor,fill opacity=0.60] (501.67, 42.22) rectangle (503.84, 42.22);

\path[fill=fillColor,fill opacity=0.60] (503.84, 42.22) rectangle (506.00, 42.22);

\path[fill=fillColor,fill opacity=0.60] (506.00, 42.22) rectangle (508.17, 42.22);

\path[fill=fillColor,fill opacity=0.60] (508.17, 42.22) rectangle (510.34, 42.22);

\path[fill=fillColor,fill opacity=0.60] (510.34, 42.22) rectangle (512.51, 42.22);

\path[fill=fillColor,fill opacity=0.60] (512.51, 42.22) rectangle (514.68, 42.22);

\path[fill=fillColor,fill opacity=0.60] (514.68, 42.22) rectangle (516.84, 42.22);

\path[fill=fillColor,fill opacity=0.60] (516.84, 42.22) rectangle (519.01, 42.22);

\path[fill=fillColor,fill opacity=0.60] (519.01, 42.22) rectangle (521.18, 42.22);

\path[fill=fillColor,fill opacity=0.60] (521.18, 42.22) rectangle (523.35, 42.22);

\path[fill=fillColor,fill opacity=0.60] (523.35, 42.22) rectangle (525.52, 42.22);

\path[fill=fillColor,fill opacity=0.60] (525.52, 42.22) rectangle (527.68, 42.22);

\path[fill=fillColor,fill opacity=0.60] (527.68, 42.22) rectangle (529.85, 42.22);

\path[fill=fillColor,fill opacity=0.60] (529.85, 42.22) rectangle (532.02, 42.22);

\path[fill=fillColor,fill opacity=0.60] (532.02, 42.22) rectangle (534.19, 42.22);

\path[fill=fillColor,fill opacity=0.60] (534.19, 42.22) rectangle (536.36, 42.22);

\path[fill=fillColor,fill opacity=0.60] (536.36, 42.22) rectangle (538.52, 42.22);

\path[fill=fillColor,fill opacity=0.60] (538.52, 42.22) rectangle (540.69, 42.22);

\path[fill=fillColor,fill opacity=0.60] (540.69, 42.22) rectangle (542.86, 42.22);

\path[fill=fillColor,fill opacity=0.60] (542.86, 42.22) rectangle (545.03, 42.22);

\path[fill=fillColor,fill opacity=0.60] (545.03, 42.22) rectangle (547.20, 42.22);
\definecolor{fillColor}{RGB}{255,165,0}

\path[fill=fillColor,fill opacity=0.60] ( 63.82, 42.22) rectangle ( 66.17, 42.22);

\path[fill=fillColor,fill opacity=0.60] ( 66.17, 42.22) rectangle ( 68.51, 42.22);

\path[fill=fillColor,fill opacity=0.60] ( 68.51, 42.22) rectangle ( 70.86, 42.22);

\path[fill=fillColor,fill opacity=0.60] ( 70.86, 42.22) rectangle ( 73.20, 42.22);

\path[fill=fillColor,fill opacity=0.60] ( 73.20, 42.22) rectangle ( 75.55, 42.22);

\path[fill=fillColor,fill opacity=0.60] ( 75.55, 42.22) rectangle ( 77.89, 42.22);

\path[fill=fillColor,fill opacity=0.60] ( 77.89, 42.22) rectangle ( 80.24, 42.22);

\path[fill=fillColor,fill opacity=0.60] ( 80.24, 42.22) rectangle ( 82.58, 42.22);

\path[fill=fillColor,fill opacity=0.60] ( 82.58, 42.22) rectangle ( 84.93, 42.22);

\path[fill=fillColor,fill opacity=0.60] ( 84.93, 42.22) rectangle ( 87.28, 42.23);

\path[fill=fillColor,fill opacity=0.60] ( 87.28, 42.22) rectangle ( 89.62, 42.26);

\path[fill=fillColor,fill opacity=0.60] ( 89.62, 42.22) rectangle ( 91.97, 42.33);

\path[fill=fillColor,fill opacity=0.60] ( 91.97, 42.22) rectangle ( 94.31, 42.56);

\path[fill=fillColor,fill opacity=0.60] ( 94.31, 42.22) rectangle ( 96.66, 43.29);

\path[fill=fillColor,fill opacity=0.60] ( 96.66, 42.22) rectangle ( 99.00, 44.69);

\path[fill=fillColor,fill opacity=0.60] ( 99.00, 42.22) rectangle (101.35, 47.84);

\path[fill=fillColor,fill opacity=0.60] (101.35, 42.22) rectangle (103.69, 53.41);

\path[fill=fillColor,fill opacity=0.60] (103.69, 42.22) rectangle (106.04, 63.77);

\path[fill=fillColor,fill opacity=0.60] (106.04, 42.22) rectangle (108.39, 79.38);

\path[fill=fillColor,fill opacity=0.60] (108.39, 42.22) rectangle (110.73,101.72);

\path[fill=fillColor,fill opacity=0.60] (110.73, 42.22) rectangle (113.08,128.16);

\path[fill=fillColor,fill opacity=0.60] (113.08, 42.22) rectangle (115.42,158.51);

\path[fill=fillColor,fill opacity=0.60] (115.42, 42.22) rectangle (117.77,189.00);

\path[fill=fillColor,fill opacity=0.60] (117.77, 42.22) rectangle (120.11,210.21);

\path[fill=fillColor,fill opacity=0.60] (120.11, 42.22) rectangle (122.46,221.35);

\path[fill=fillColor,fill opacity=0.60] (122.46, 42.22) rectangle (124.80,214.92);

\path[fill=fillColor,fill opacity=0.60] (124.80, 42.22) rectangle (127.15,196.42);

\path[fill=fillColor,fill opacity=0.60] (127.15, 42.22) rectangle (129.50,170.24);

\path[fill=fillColor,fill opacity=0.60] (129.50, 42.22) rectangle (131.84,136.60);

\path[fill=fillColor,fill opacity=0.60] (131.84, 42.22) rectangle (134.19,106.28);

\path[fill=fillColor,fill opacity=0.60] (134.19, 42.22) rectangle (136.53, 82.62);

\path[fill=fillColor,fill opacity=0.60] (136.53, 42.22) rectangle (138.88, 65.21);

\path[fill=fillColor,fill opacity=0.60] (138.88, 42.22) rectangle (141.22, 54.00);

\path[fill=fillColor,fill opacity=0.60] (141.22, 42.22) rectangle (143.57, 47.52);

\path[fill=fillColor,fill opacity=0.60] (143.57, 42.22) rectangle (145.91, 44.44);

\path[fill=fillColor,fill opacity=0.60] (145.91, 42.22) rectangle (148.26, 42.98);

\path[fill=fillColor,fill opacity=0.60] (148.26, 42.22) rectangle (150.61, 42.51);

\path[fill=fillColor,fill opacity=0.60] (150.61, 42.22) rectangle (152.95, 42.29);

\path[fill=fillColor,fill opacity=0.60] (152.95, 42.22) rectangle (155.30, 42.26);

\path[fill=fillColor,fill opacity=0.60] (155.30, 42.22) rectangle (157.64, 42.23);

\path[fill=fillColor,fill opacity=0.60] (157.64, 42.22) rectangle (159.99, 42.22);

\path[fill=fillColor,fill opacity=0.60] (159.99, 42.22) rectangle (162.33, 42.22);

\path[fill=fillColor,fill opacity=0.60] (162.33, 42.22) rectangle (164.68, 42.22);

\path[fill=fillColor,fill opacity=0.60] (164.68, 42.22) rectangle (167.02, 42.22);

\path[fill=fillColor,fill opacity=0.60] (167.02, 42.22) rectangle (169.37, 42.22);

\path[fill=fillColor,fill opacity=0.60] (169.37, 42.22) rectangle (171.72, 42.22);

\path[fill=fillColor,fill opacity=0.60] (171.72, 42.22) rectangle (174.06, 42.22);

\path[fill=fillColor,fill opacity=0.60] (174.06, 42.22) rectangle (176.41, 42.22);

\path[fill=fillColor,fill opacity=0.60] (176.41, 42.22) rectangle (178.75, 42.22);

\path[fill=fillColor,fill opacity=0.60] (178.75, 42.22) rectangle (181.10, 42.22);

\path[fill=fillColor,fill opacity=0.60] (181.10, 42.22) rectangle (183.44, 42.22);

\path[fill=fillColor,fill opacity=0.60] (183.44, 42.22) rectangle (185.79, 42.22);

\path[fill=fillColor,fill opacity=0.60] (185.79, 42.22) rectangle (188.13, 42.22);

\path[fill=fillColor,fill opacity=0.60] (188.13, 42.22) rectangle (190.48, 42.22);

\path[fill=fillColor,fill opacity=0.60] (190.48, 42.22) rectangle (192.83, 42.22);

\path[fill=fillColor,fill opacity=0.60] (192.83, 42.22) rectangle (195.17, 42.22);

\path[fill=fillColor,fill opacity=0.60] (195.17, 42.22) rectangle (197.52, 42.22);

\path[fill=fillColor,fill opacity=0.60] (197.52, 42.22) rectangle (199.86, 42.22);

\path[fill=fillColor,fill opacity=0.60] (199.86, 42.22) rectangle (202.21, 42.22);

\path[fill=fillColor,fill opacity=0.60] (202.21, 42.22) rectangle (204.55, 42.22);

\path[fill=fillColor,fill opacity=0.60] (204.55, 42.22) rectangle (206.90, 42.22);

\path[fill=fillColor,fill opacity=0.60] (206.90, 42.22) rectangle (209.24, 42.22);

\path[fill=fillColor,fill opacity=0.60] (209.24, 42.22) rectangle (211.59, 42.22);

\path[fill=fillColor,fill opacity=0.60] (211.59, 42.22) rectangle (213.94, 42.22);

\path[fill=fillColor,fill opacity=0.60] (213.94, 42.22) rectangle (216.28, 42.22);

\path[fill=fillColor,fill opacity=0.60] (216.28, 42.22) rectangle (218.63, 42.22);

\path[fill=fillColor,fill opacity=0.60] (218.63, 42.22) rectangle (220.97, 42.22);

\path[fill=fillColor,fill opacity=0.60] (220.97, 42.22) rectangle (223.32, 42.22);

\path[fill=fillColor,fill opacity=0.60] (223.32, 42.22) rectangle (225.66, 42.22);

\path[fill=fillColor,fill opacity=0.60] (225.66, 42.22) rectangle (228.01, 42.22);

\path[fill=fillColor,fill opacity=0.60] (228.01, 42.22) rectangle (230.36, 42.22);

\path[fill=fillColor,fill opacity=0.60] (230.36, 42.22) rectangle (232.70, 42.22);

\path[fill=fillColor,fill opacity=0.60] (232.70, 42.22) rectangle (235.05, 42.22);

\path[fill=fillColor,fill opacity=0.60] (235.05, 42.22) rectangle (237.39, 42.22);

\path[fill=fillColor,fill opacity=0.60] (237.39, 42.22) rectangle (239.74, 42.22);

\path[fill=fillColor,fill opacity=0.60] (239.74, 42.22) rectangle (242.08, 42.22);

\path[fill=fillColor,fill opacity=0.60] (242.08, 42.22) rectangle (244.43, 42.22);

\path[fill=fillColor,fill opacity=0.60] (244.43, 42.22) rectangle (246.77, 42.22);

\path[fill=fillColor,fill opacity=0.60] (246.77, 42.22) rectangle (249.12, 42.22);

\path[fill=fillColor,fill opacity=0.60] (249.12, 42.22) rectangle (251.47, 42.22);

\path[fill=fillColor,fill opacity=0.60] (251.47, 42.22) rectangle (253.81, 42.22);

\path[fill=fillColor,fill opacity=0.60] (253.81, 42.22) rectangle (256.16, 42.22);

\path[fill=fillColor,fill opacity=0.60] (256.16, 42.22) rectangle (258.50, 42.22);

\path[fill=fillColor,fill opacity=0.60] (258.50, 42.22) rectangle (260.85, 42.22);

\path[fill=fillColor,fill opacity=0.60] (260.85, 42.22) rectangle (263.19, 42.22);

\path[fill=fillColor,fill opacity=0.60] (263.19, 42.22) rectangle (265.54, 42.22);

\path[fill=fillColor,fill opacity=0.60] (265.54, 42.22) rectangle (267.88, 42.22);

\path[fill=fillColor,fill opacity=0.60] (267.88, 42.22) rectangle (270.23, 42.22);

\path[fill=fillColor,fill opacity=0.60] (270.23, 42.22) rectangle (272.58, 42.22);

\path[fill=fillColor,fill opacity=0.60] (272.58, 42.22) rectangle (274.92, 42.22);

\path[fill=fillColor,fill opacity=0.60] (274.92, 42.22) rectangle (277.27, 42.22);

\path[fill=fillColor,fill opacity=0.60] (277.27, 42.22) rectangle (279.61, 42.22);

\path[fill=fillColor,fill opacity=0.60] (279.61, 42.22) rectangle (281.96, 42.22);

\path[fill=fillColor,fill opacity=0.60] (281.96, 42.22) rectangle (284.30, 42.22);

\path[fill=fillColor,fill opacity=0.60] (284.30, 42.22) rectangle (286.65, 42.22);

\path[fill=fillColor,fill opacity=0.60] (286.65, 42.22) rectangle (288.99, 42.22);

\path[fill=fillColor,fill opacity=0.60] (288.99, 42.22) rectangle (291.34, 42.22);

\path[fill=fillColor,fill opacity=0.60] (291.34, 42.22) rectangle (293.69, 42.22);

\path[fill=fillColor,fill opacity=0.60] (293.69, 42.22) rectangle (296.03, 42.22);

\path[fill=fillColor,fill opacity=0.60] (296.03, 42.22) rectangle (298.38, 42.22);

\path[fill=fillColor,fill opacity=0.60] (298.38, 42.22) rectangle (300.72, 42.22);

\path[fill=fillColor,fill opacity=0.60] (300.72, 42.22) rectangle (303.07, 42.22);

\path[fill=fillColor,fill opacity=0.60] (303.07, 42.22) rectangle (305.41, 42.22);

\path[fill=fillColor,fill opacity=0.60] (305.41, 42.22) rectangle (307.76, 42.22);

\path[fill=fillColor,fill opacity=0.60] (307.76, 42.22) rectangle (310.10, 42.22);

\path[fill=fillColor,fill opacity=0.60] (310.10, 42.22) rectangle (312.45, 42.22);

\path[fill=fillColor,fill opacity=0.60] (312.45, 42.22) rectangle (314.80, 42.22);

\path[fill=fillColor,fill opacity=0.60] (314.80, 42.22) rectangle (317.14, 42.22);

\path[fill=fillColor,fill opacity=0.60] (317.14, 42.22) rectangle (319.49, 42.22);

\path[fill=fillColor,fill opacity=0.60] (319.49, 42.22) rectangle (321.83, 42.22);

\path[fill=fillColor,fill opacity=0.60] (321.83, 42.22) rectangle (324.18, 42.22);

\path[fill=fillColor,fill opacity=0.60] (324.18, 42.22) rectangle (326.52, 42.22);

\path[fill=fillColor,fill opacity=0.60] (326.52, 42.22) rectangle (328.87, 42.22);

\path[fill=fillColor,fill opacity=0.60] (328.87, 42.22) rectangle (331.21, 42.22);

\path[fill=fillColor,fill opacity=0.60] (331.21, 42.22) rectangle (333.56, 42.22);

\path[fill=fillColor,fill opacity=0.60] (333.56, 42.22) rectangle (335.91, 42.22);

\path[fill=fillColor,fill opacity=0.60] (335.91, 42.22) rectangle (338.25, 42.22);

\path[fill=fillColor,fill opacity=0.60] (338.25, 42.22) rectangle (340.60, 42.22);

\path[fill=fillColor,fill opacity=0.60] (340.60, 42.22) rectangle (342.94, 42.22);

\path[fill=fillColor,fill opacity=0.60] (342.94, 42.22) rectangle (345.29, 42.22);

\path[fill=fillColor,fill opacity=0.60] (345.29, 42.22) rectangle (347.63, 42.22);

\path[fill=fillColor,fill opacity=0.60] (347.63, 42.22) rectangle (349.98, 42.22);

\path[fill=fillColor,fill opacity=0.60] (349.98, 42.22) rectangle (352.32, 42.22);

\path[fill=fillColor,fill opacity=0.60] (352.32, 42.22) rectangle (354.67, 42.22);

\path[fill=fillColor,fill opacity=0.60] (354.67, 42.22) rectangle (357.02, 42.22);

\path[fill=fillColor,fill opacity=0.60] (357.02, 42.22) rectangle (359.36, 42.22);

\path[fill=fillColor,fill opacity=0.60] (359.36, 42.22) rectangle (361.71, 42.22);

\path[fill=fillColor,fill opacity=0.60] (361.71, 42.22) rectangle (364.05, 42.22);

\path[fill=fillColor,fill opacity=0.60] (364.05, 42.22) rectangle (366.40, 42.22);

\path[fill=fillColor,fill opacity=0.60] (366.40, 42.22) rectangle (368.74, 42.22);

\path[fill=fillColor,fill opacity=0.60] (368.74, 42.22) rectangle (371.09, 42.22);

\path[fill=fillColor,fill opacity=0.60] (371.09, 42.22) rectangle (373.43, 42.22);

\path[fill=fillColor,fill opacity=0.60] (373.43, 42.22) rectangle (375.78, 42.22);

\path[fill=fillColor,fill opacity=0.60] (375.78, 42.22) rectangle (378.13, 42.22);

\path[fill=fillColor,fill opacity=0.60] (378.13, 42.22) rectangle (380.47, 42.22);

\path[fill=fillColor,fill opacity=0.60] (380.47, 42.22) rectangle (382.82, 42.22);

\path[fill=fillColor,fill opacity=0.60] (382.82, 42.22) rectangle (385.16, 42.22);

\path[fill=fillColor,fill opacity=0.60] (385.16, 42.22) rectangle (387.51, 42.22);

\path[fill=fillColor,fill opacity=0.60] (387.51, 42.22) rectangle (389.85, 42.22);

\path[fill=fillColor,fill opacity=0.60] (389.85, 42.22) rectangle (392.20, 42.22);

\path[fill=fillColor,fill opacity=0.60] (392.20, 42.22) rectangle (394.54, 42.22);

\path[fill=fillColor,fill opacity=0.60] (394.54, 42.22) rectangle (396.89, 42.22);

\path[fill=fillColor,fill opacity=0.60] (396.89, 42.22) rectangle (399.24, 42.22);

\path[fill=fillColor,fill opacity=0.60] (399.24, 42.22) rectangle (401.58, 42.22);

\path[fill=fillColor,fill opacity=0.60] (401.58, 42.22) rectangle (403.93, 42.22);

\path[fill=fillColor,fill opacity=0.60] (403.93, 42.22) rectangle (406.27, 42.22);

\path[fill=fillColor,fill opacity=0.60] (406.27, 42.22) rectangle (408.62, 42.22);

\path[fill=fillColor,fill opacity=0.60] (408.62, 42.22) rectangle (410.96, 42.22);

\path[fill=fillColor,fill opacity=0.60] (410.96, 42.22) rectangle (413.31, 42.22);

\path[fill=fillColor,fill opacity=0.60] (413.31, 42.22) rectangle (415.65, 42.22);

\path[fill=fillColor,fill opacity=0.60] (415.65, 42.22) rectangle (418.00, 42.22);

\path[fill=fillColor,fill opacity=0.60] (418.00, 42.22) rectangle (420.35, 42.22);

\path[fill=fillColor,fill opacity=0.60] (420.35, 42.22) rectangle (422.69, 42.22);

\path[fill=fillColor,fill opacity=0.60] (422.69, 42.22) rectangle (425.04, 42.22);

\path[fill=fillColor,fill opacity=0.60] (425.04, 42.22) rectangle (427.38, 42.22);

\path[fill=fillColor,fill opacity=0.60] (427.38, 42.22) rectangle (429.73, 42.22);

\path[fill=fillColor,fill opacity=0.60] (429.73, 42.22) rectangle (432.07, 42.22);

\path[fill=fillColor,fill opacity=0.60] (432.07, 42.22) rectangle (434.42, 42.22);

\path[fill=fillColor,fill opacity=0.60] (434.42, 42.22) rectangle (436.76, 42.22);

\path[fill=fillColor,fill opacity=0.60] (436.76, 42.22) rectangle (439.11, 42.22);

\path[fill=fillColor,fill opacity=0.60] (439.11, 42.22) rectangle (441.46, 42.22);

\path[fill=fillColor,fill opacity=0.60] (441.46, 42.22) rectangle (443.80, 42.22);

\path[fill=fillColor,fill opacity=0.60] (443.80, 42.22) rectangle (446.15, 42.22);

\path[fill=fillColor,fill opacity=0.60] (446.15, 42.22) rectangle (448.49, 42.22);

\path[fill=fillColor,fill opacity=0.60] (448.49, 42.22) rectangle (450.84, 42.22);

\path[fill=fillColor,fill opacity=0.60] (450.84, 42.22) rectangle (453.18, 42.22);

\path[fill=fillColor,fill opacity=0.60] (453.18, 42.22) rectangle (455.53, 42.22);

\path[fill=fillColor,fill opacity=0.60] (455.53, 42.22) rectangle (457.88, 42.22);

\path[fill=fillColor,fill opacity=0.60] (457.88, 42.22) rectangle (460.22, 42.22);

\path[fill=fillColor,fill opacity=0.60] (460.22, 42.22) rectangle (462.57, 42.22);

\path[fill=fillColor,fill opacity=0.60] (462.57, 42.22) rectangle (464.91, 42.22);

\path[fill=fillColor,fill opacity=0.60] (464.91, 42.22) rectangle (467.26, 42.22);

\path[fill=fillColor,fill opacity=0.60] (467.26, 42.22) rectangle (469.60, 42.22);

\path[fill=fillColor,fill opacity=0.60] (469.60, 42.22) rectangle (471.95, 42.22);

\path[fill=fillColor,fill opacity=0.60] (471.95, 42.22) rectangle (474.29, 42.22);

\path[fill=fillColor,fill opacity=0.60] (474.29, 42.22) rectangle (476.64, 42.22);

\path[fill=fillColor,fill opacity=0.60] (476.64, 42.22) rectangle (478.99, 42.22);

\path[fill=fillColor,fill opacity=0.60] (478.99, 42.22) rectangle (481.33, 42.22);

\path[fill=fillColor,fill opacity=0.60] (481.33, 42.22) rectangle (483.68, 42.22);

\path[fill=fillColor,fill opacity=0.60] (483.68, 42.22) rectangle (486.02, 42.22);

\path[fill=fillColor,fill opacity=0.60] (486.02, 42.22) rectangle (488.37, 42.22);

\path[fill=fillColor,fill opacity=0.60] (488.37, 42.22) rectangle (490.71, 42.22);

\path[fill=fillColor,fill opacity=0.60] (490.71, 42.22) rectangle (493.06, 42.22);

\path[fill=fillColor,fill opacity=0.60] (493.06, 42.22) rectangle (495.40, 42.22);

\path[fill=fillColor,fill opacity=0.60] (495.40, 42.22) rectangle (497.75, 42.22);

\path[fill=fillColor,fill opacity=0.60] (497.75, 42.22) rectangle (500.10, 42.22);

\path[fill=fillColor,fill opacity=0.60] (500.10, 42.22) rectangle (502.44, 42.22);

\path[fill=fillColor,fill opacity=0.60] (502.44, 42.22) rectangle (504.79, 42.22);

\path[fill=fillColor,fill opacity=0.60] (504.79, 42.22) rectangle (507.13, 42.22);

\path[fill=fillColor,fill opacity=0.60] (507.13, 42.22) rectangle (509.48, 42.22);

\path[fill=fillColor,fill opacity=0.60] (509.48, 42.22) rectangle (511.82, 42.22);

\path[fill=fillColor,fill opacity=0.60] (511.82, 42.22) rectangle (514.17, 42.22);

\path[fill=fillColor,fill opacity=0.60] (514.17, 42.22) rectangle (516.51, 42.22);

\path[fill=fillColor,fill opacity=0.60] (516.51, 42.22) rectangle (518.86, 42.22);

\path[fill=fillColor,fill opacity=0.60] (518.86, 42.22) rectangle (521.21, 42.22);

\path[fill=fillColor,fill opacity=0.60] (521.21, 42.22) rectangle (523.55, 42.22);

\path[fill=fillColor,fill opacity=0.60] (523.55, 42.22) rectangle (525.90, 42.22);

\path[fill=fillColor,fill opacity=0.60] (525.90, 42.22) rectangle (528.24, 42.22);

\path[fill=fillColor,fill opacity=0.60] (528.24, 42.22) rectangle (530.59, 42.22);

\path[fill=fillColor,fill opacity=0.60] (530.59, 42.22) rectangle (532.93, 42.22);

\path[fill=fillColor,fill opacity=0.60] (532.93, 42.22) rectangle (535.28, 42.22);

\path[fill=fillColor,fill opacity=0.60] (535.28, 42.22) rectangle (537.62, 42.22);

\path[fill=fillColor,fill opacity=0.60] (537.62, 42.22) rectangle (539.97, 42.22);

\path[fill=fillColor,fill opacity=0.60] (539.97, 42.22) rectangle (542.32, 42.22);

\path[fill=fillColor,fill opacity=0.60] (542.32, 42.22) rectangle (544.66, 42.22);

\path[fill=fillColor,fill opacity=0.60] (544.66, 42.22) rectangle (547.01, 42.22);
\definecolor{fillColor}{RGB}{238,130,238}

\path[fill=fillColor,fill opacity=0.60] ( 63.84, 42.22) rectangle ( 66.22, 42.22);

\path[fill=fillColor,fill opacity=0.60] ( 66.22, 42.22) rectangle ( 68.61, 42.22);

\path[fill=fillColor,fill opacity=0.60] ( 68.61, 42.22) rectangle ( 70.99, 42.22);

\path[fill=fillColor,fill opacity=0.60] ( 70.99, 42.22) rectangle ( 73.37, 42.22);

\path[fill=fillColor,fill opacity=0.60] ( 73.37, 42.22) rectangle ( 75.76, 42.22);

\path[fill=fillColor,fill opacity=0.60] ( 75.76, 42.22) rectangle ( 78.14, 42.22);

\path[fill=fillColor,fill opacity=0.60] ( 78.14, 42.22) rectangle ( 80.52, 42.22);

\path[fill=fillColor,fill opacity=0.60] ( 80.52, 42.22) rectangle ( 82.91, 42.22);

\path[fill=fillColor,fill opacity=0.60] ( 82.91, 42.22) rectangle ( 85.29, 42.22);

\path[fill=fillColor,fill opacity=0.60] ( 85.29, 42.22) rectangle ( 87.67, 42.22);

\path[fill=fillColor,fill opacity=0.60] ( 87.67, 42.22) rectangle ( 90.06, 42.22);

\path[fill=fillColor,fill opacity=0.60] ( 90.06, 42.22) rectangle ( 92.44, 42.22);

\path[fill=fillColor,fill opacity=0.60] ( 92.44, 42.22) rectangle ( 94.82, 42.22);

\path[fill=fillColor,fill opacity=0.60] ( 94.82, 42.22) rectangle ( 97.21, 42.22);

\path[fill=fillColor,fill opacity=0.60] ( 97.21, 42.22) rectangle ( 99.59, 42.22);

\path[fill=fillColor,fill opacity=0.60] ( 99.59, 42.22) rectangle (101.97, 42.22);

\path[fill=fillColor,fill opacity=0.60] (101.97, 42.22) rectangle (104.36, 42.22);

\path[fill=fillColor,fill opacity=0.60] (104.36, 42.22) rectangle (106.74, 42.22);

\path[fill=fillColor,fill opacity=0.60] (106.74, 42.22) rectangle (109.12, 42.22);

\path[fill=fillColor,fill opacity=0.60] (109.12, 42.22) rectangle (111.51, 42.22);

\path[fill=fillColor,fill opacity=0.60] (111.51, 42.22) rectangle (113.89, 42.22);

\path[fill=fillColor,fill opacity=0.60] (113.89, 42.22) rectangle (116.27, 42.22);

\path[fill=fillColor,fill opacity=0.60] (116.27, 42.22) rectangle (118.66, 42.22);

\path[fill=fillColor,fill opacity=0.60] (118.66, 42.22) rectangle (121.04, 42.22);

\path[fill=fillColor,fill opacity=0.60] (121.04, 42.22) rectangle (123.43, 42.27);

\path[fill=fillColor,fill opacity=0.60] (123.43, 42.22) rectangle (125.81, 42.29);

\path[fill=fillColor,fill opacity=0.60] (125.81, 42.22) rectangle (128.19, 42.52);

\path[fill=fillColor,fill opacity=0.60] (128.19, 42.22) rectangle (130.58, 43.10);

\path[fill=fillColor,fill opacity=0.60] (130.58, 42.22) rectangle (132.96, 44.42);

\path[fill=fillColor,fill opacity=0.60] (132.96, 42.22) rectangle (135.34, 46.88);

\path[fill=fillColor,fill opacity=0.60] (135.34, 42.22) rectangle (137.73, 52.64);

\path[fill=fillColor,fill opacity=0.60] (137.73, 42.22) rectangle (140.11, 61.79);

\path[fill=fillColor,fill opacity=0.60] (140.11, 42.22) rectangle (142.49, 77.14);

\path[fill=fillColor,fill opacity=0.60] (142.49, 42.22) rectangle (144.88, 99.75);

\path[fill=fillColor,fill opacity=0.60] (144.88, 42.22) rectangle (147.26,128.99);

\path[fill=fillColor,fill opacity=0.60] (147.26, 42.22) rectangle (149.64,161.81);

\path[fill=fillColor,fill opacity=0.60] (149.64, 42.22) rectangle (152.03,193.77);

\path[fill=fillColor,fill opacity=0.60] (152.03, 42.22) rectangle (154.41,216.68);

\path[fill=fillColor,fill opacity=0.60] (154.41, 42.22) rectangle (156.79,225.02);

\path[fill=fillColor,fill opacity=0.60] (156.79, 42.22) rectangle (159.18,220.81);

\path[fill=fillColor,fill opacity=0.60] (159.18, 42.22) rectangle (161.56,198.95);

\path[fill=fillColor,fill opacity=0.60] (161.56, 42.22) rectangle (163.94,166.65);

\path[fill=fillColor,fill opacity=0.60] (163.94, 42.22) rectangle (166.33,133.71);

\path[fill=fillColor,fill opacity=0.60] (166.33, 42.22) rectangle (168.71,103.30);

\path[fill=fillColor,fill opacity=0.60] (168.71, 42.22) rectangle (171.09, 79.68);

\path[fill=fillColor,fill opacity=0.60] (171.09, 42.22) rectangle (173.48, 62.39);

\path[fill=fillColor,fill opacity=0.60] (173.48, 42.22) rectangle (175.86, 51.96);

\path[fill=fillColor,fill opacity=0.60] (175.86, 42.22) rectangle (178.24, 46.41);

\path[fill=fillColor,fill opacity=0.60] (178.24, 42.22) rectangle (180.63, 44.04);

\path[fill=fillColor,fill opacity=0.60] (180.63, 42.22) rectangle (183.01, 42.86);

\path[fill=fillColor,fill opacity=0.60] (183.01, 42.22) rectangle (185.40, 42.42);

\path[fill=fillColor,fill opacity=0.60] (185.40, 42.22) rectangle (187.78, 42.29);

\path[fill=fillColor,fill opacity=0.60] (187.78, 42.22) rectangle (190.16, 42.22);

\path[fill=fillColor,fill opacity=0.60] (190.16, 42.22) rectangle (192.55, 42.22);

\path[fill=fillColor,fill opacity=0.60] (192.55, 42.22) rectangle (194.93, 42.22);

\path[fill=fillColor,fill opacity=0.60] (194.93, 42.22) rectangle (197.31, 42.22);

\path[fill=fillColor,fill opacity=0.60] (197.31, 42.22) rectangle (199.70, 42.22);

\path[fill=fillColor,fill opacity=0.60] (199.70, 42.22) rectangle (202.08, 42.22);

\path[fill=fillColor,fill opacity=0.60] (202.08, 42.22) rectangle (204.46, 42.22);

\path[fill=fillColor,fill opacity=0.60] (204.46, 42.22) rectangle (206.85, 42.22);

\path[fill=fillColor,fill opacity=0.60] (206.85, 42.22) rectangle (209.23, 42.22);

\path[fill=fillColor,fill opacity=0.60] (209.23, 42.22) rectangle (211.61, 42.22);

\path[fill=fillColor,fill opacity=0.60] (211.61, 42.22) rectangle (214.00, 42.22);

\path[fill=fillColor,fill opacity=0.60] (214.00, 42.22) rectangle (216.38, 42.22);

\path[fill=fillColor,fill opacity=0.60] (216.38, 42.22) rectangle (218.76, 42.22);

\path[fill=fillColor,fill opacity=0.60] (218.76, 42.22) rectangle (221.15, 42.22);

\path[fill=fillColor,fill opacity=0.60] (221.15, 42.22) rectangle (223.53, 42.22);

\path[fill=fillColor,fill opacity=0.60] (223.53, 42.22) rectangle (225.91, 42.22);

\path[fill=fillColor,fill opacity=0.60] (225.91, 42.22) rectangle (228.30, 42.22);

\path[fill=fillColor,fill opacity=0.60] (228.30, 42.22) rectangle (230.68, 42.22);

\path[fill=fillColor,fill opacity=0.60] (230.68, 42.22) rectangle (233.06, 42.22);

\path[fill=fillColor,fill opacity=0.60] (233.06, 42.22) rectangle (235.45, 42.22);

\path[fill=fillColor,fill opacity=0.60] (235.45, 42.22) rectangle (237.83, 42.22);

\path[fill=fillColor,fill opacity=0.60] (237.83, 42.22) rectangle (240.21, 42.22);

\path[fill=fillColor,fill opacity=0.60] (240.21, 42.22) rectangle (242.60, 42.22);

\path[fill=fillColor,fill opacity=0.60] (242.60, 42.22) rectangle (244.98, 42.22);

\path[fill=fillColor,fill opacity=0.60] (244.98, 42.22) rectangle (247.37, 42.22);

\path[fill=fillColor,fill opacity=0.60] (247.37, 42.22) rectangle (249.75, 42.22);

\path[fill=fillColor,fill opacity=0.60] (249.75, 42.22) rectangle (252.13, 42.22);

\path[fill=fillColor,fill opacity=0.60] (252.13, 42.22) rectangle (254.52, 42.22);

\path[fill=fillColor,fill opacity=0.60] (254.52, 42.22) rectangle (256.90, 42.22);

\path[fill=fillColor,fill opacity=0.60] (256.90, 42.22) rectangle (259.28, 42.22);

\path[fill=fillColor,fill opacity=0.60] (259.28, 42.22) rectangle (261.67, 42.22);

\path[fill=fillColor,fill opacity=0.60] (261.67, 42.22) rectangle (264.05, 42.22);

\path[fill=fillColor,fill opacity=0.60] (264.05, 42.22) rectangle (266.43, 42.22);

\path[fill=fillColor,fill opacity=0.60] (266.43, 42.22) rectangle (268.82, 42.22);

\path[fill=fillColor,fill opacity=0.60] (268.82, 42.22) rectangle (271.20, 42.22);

\path[fill=fillColor,fill opacity=0.60] (271.20, 42.22) rectangle (273.58, 42.22);

\path[fill=fillColor,fill opacity=0.60] (273.58, 42.22) rectangle (275.97, 42.22);

\path[fill=fillColor,fill opacity=0.60] (275.97, 42.22) rectangle (278.35, 42.22);

\path[fill=fillColor,fill opacity=0.60] (278.35, 42.22) rectangle (280.73, 42.22);

\path[fill=fillColor,fill opacity=0.60] (280.73, 42.22) rectangle (283.12, 42.22);

\path[fill=fillColor,fill opacity=0.60] (283.12, 42.22) rectangle (285.50, 42.22);

\path[fill=fillColor,fill opacity=0.60] (285.50, 42.22) rectangle (287.88, 42.22);

\path[fill=fillColor,fill opacity=0.60] (287.88, 42.22) rectangle (290.27, 42.22);

\path[fill=fillColor,fill opacity=0.60] (290.27, 42.22) rectangle (292.65, 42.22);

\path[fill=fillColor,fill opacity=0.60] (292.65, 42.22) rectangle (295.03, 42.22);

\path[fill=fillColor,fill opacity=0.60] (295.03, 42.22) rectangle (297.42, 42.22);

\path[fill=fillColor,fill opacity=0.60] (297.42, 42.22) rectangle (299.80, 42.22);

\path[fill=fillColor,fill opacity=0.60] (299.80, 42.22) rectangle (302.18, 42.22);

\path[fill=fillColor,fill opacity=0.60] (302.18, 42.22) rectangle (304.57, 42.22);

\path[fill=fillColor,fill opacity=0.60] (304.57, 42.22) rectangle (306.95, 42.22);

\path[fill=fillColor,fill opacity=0.60] (306.95, 42.22) rectangle (309.34, 42.22);

\path[fill=fillColor,fill opacity=0.60] (309.34, 42.22) rectangle (311.72, 42.22);

\path[fill=fillColor,fill opacity=0.60] (311.72, 42.22) rectangle (314.10, 42.22);

\path[fill=fillColor,fill opacity=0.60] (314.10, 42.22) rectangle (316.49, 42.22);

\path[fill=fillColor,fill opacity=0.60] (316.49, 42.22) rectangle (318.87, 42.22);

\path[fill=fillColor,fill opacity=0.60] (318.87, 42.22) rectangle (321.25, 42.22);

\path[fill=fillColor,fill opacity=0.60] (321.25, 42.22) rectangle (323.64, 42.22);

\path[fill=fillColor,fill opacity=0.60] (323.64, 42.22) rectangle (326.02, 42.22);

\path[fill=fillColor,fill opacity=0.60] (326.02, 42.22) rectangle (328.40, 42.22);

\path[fill=fillColor,fill opacity=0.60] (328.40, 42.22) rectangle (330.79, 42.22);

\path[fill=fillColor,fill opacity=0.60] (330.79, 42.22) rectangle (333.17, 42.22);

\path[fill=fillColor,fill opacity=0.60] (333.17, 42.22) rectangle (335.55, 42.22);

\path[fill=fillColor,fill opacity=0.60] (335.55, 42.22) rectangle (337.94, 42.22);

\path[fill=fillColor,fill opacity=0.60] (337.94, 42.22) rectangle (340.32, 42.22);

\path[fill=fillColor,fill opacity=0.60] (340.32, 42.22) rectangle (342.70, 42.22);

\path[fill=fillColor,fill opacity=0.60] (342.70, 42.22) rectangle (345.09, 42.22);

\path[fill=fillColor,fill opacity=0.60] (345.09, 42.22) rectangle (347.47, 42.22);

\path[fill=fillColor,fill opacity=0.60] (347.47, 42.22) rectangle (349.85, 42.22);

\path[fill=fillColor,fill opacity=0.60] (349.85, 42.22) rectangle (352.24, 42.22);

\path[fill=fillColor,fill opacity=0.60] (352.24, 42.22) rectangle (354.62, 42.22);

\path[fill=fillColor,fill opacity=0.60] (354.62, 42.22) rectangle (357.00, 42.22);

\path[fill=fillColor,fill opacity=0.60] (357.00, 42.22) rectangle (359.39, 42.22);

\path[fill=fillColor,fill opacity=0.60] (359.39, 42.22) rectangle (361.77, 42.22);

\path[fill=fillColor,fill opacity=0.60] (361.77, 42.22) rectangle (364.15, 42.22);

\path[fill=fillColor,fill opacity=0.60] (364.15, 42.22) rectangle (366.54, 42.22);

\path[fill=fillColor,fill opacity=0.60] (366.54, 42.22) rectangle (368.92, 42.22);

\path[fill=fillColor,fill opacity=0.60] (368.92, 42.22) rectangle (371.31, 42.22);

\path[fill=fillColor,fill opacity=0.60] (371.31, 42.22) rectangle (373.69, 42.22);

\path[fill=fillColor,fill opacity=0.60] (373.69, 42.22) rectangle (376.07, 42.22);

\path[fill=fillColor,fill opacity=0.60] (376.07, 42.22) rectangle (378.46, 42.22);

\path[fill=fillColor,fill opacity=0.60] (378.46, 42.22) rectangle (380.84, 42.22);

\path[fill=fillColor,fill opacity=0.60] (380.84, 42.22) rectangle (383.22, 42.22);

\path[fill=fillColor,fill opacity=0.60] (383.22, 42.22) rectangle (385.61, 42.22);

\path[fill=fillColor,fill opacity=0.60] (385.61, 42.22) rectangle (387.99, 42.22);

\path[fill=fillColor,fill opacity=0.60] (387.99, 42.22) rectangle (390.37, 42.22);

\path[fill=fillColor,fill opacity=0.60] (390.37, 42.22) rectangle (392.76, 42.22);

\path[fill=fillColor,fill opacity=0.60] (392.76, 42.22) rectangle (395.14, 42.22);

\path[fill=fillColor,fill opacity=0.60] (395.14, 42.22) rectangle (397.52, 42.22);

\path[fill=fillColor,fill opacity=0.60] (397.52, 42.22) rectangle (399.91, 42.22);

\path[fill=fillColor,fill opacity=0.60] (399.91, 42.22) rectangle (402.29, 42.22);

\path[fill=fillColor,fill opacity=0.60] (402.29, 42.22) rectangle (404.67, 42.22);

\path[fill=fillColor,fill opacity=0.60] (404.67, 42.22) rectangle (407.06, 42.22);

\path[fill=fillColor,fill opacity=0.60] (407.06, 42.22) rectangle (409.44, 42.22);

\path[fill=fillColor,fill opacity=0.60] (409.44, 42.22) rectangle (411.82, 42.22);

\path[fill=fillColor,fill opacity=0.60] (411.82, 42.22) rectangle (414.21, 42.22);

\path[fill=fillColor,fill opacity=0.60] (414.21, 42.22) rectangle (416.59, 42.22);

\path[fill=fillColor,fill opacity=0.60] (416.59, 42.22) rectangle (418.97, 42.22);

\path[fill=fillColor,fill opacity=0.60] (418.97, 42.22) rectangle (421.36, 42.22);

\path[fill=fillColor,fill opacity=0.60] (421.36, 42.22) rectangle (423.74, 42.22);

\path[fill=fillColor,fill opacity=0.60] (423.74, 42.22) rectangle (426.12, 42.22);

\path[fill=fillColor,fill opacity=0.60] (426.12, 42.22) rectangle (428.51, 42.22);

\path[fill=fillColor,fill opacity=0.60] (428.51, 42.22) rectangle (430.89, 42.22);

\path[fill=fillColor,fill opacity=0.60] (430.89, 42.22) rectangle (433.28, 42.22);

\path[fill=fillColor,fill opacity=0.60] (433.28, 42.22) rectangle (435.66, 42.22);

\path[fill=fillColor,fill opacity=0.60] (435.66, 42.22) rectangle (438.04, 42.22);

\path[fill=fillColor,fill opacity=0.60] (438.04, 42.22) rectangle (440.43, 42.22);

\path[fill=fillColor,fill opacity=0.60] (440.43, 42.22) rectangle (442.81, 42.22);

\path[fill=fillColor,fill opacity=0.60] (442.81, 42.22) rectangle (445.19, 42.22);

\path[fill=fillColor,fill opacity=0.60] (445.19, 42.22) rectangle (447.58, 42.22);

\path[fill=fillColor,fill opacity=0.60] (447.58, 42.22) rectangle (449.96, 42.22);

\path[fill=fillColor,fill opacity=0.60] (449.96, 42.22) rectangle (452.34, 42.22);

\path[fill=fillColor,fill opacity=0.60] (452.34, 42.22) rectangle (454.73, 42.22);

\path[fill=fillColor,fill opacity=0.60] (454.73, 42.22) rectangle (457.11, 42.22);

\path[fill=fillColor,fill opacity=0.60] (457.11, 42.22) rectangle (459.49, 42.22);

\path[fill=fillColor,fill opacity=0.60] (459.49, 42.22) rectangle (461.88, 42.22);

\path[fill=fillColor,fill opacity=0.60] (461.88, 42.22) rectangle (464.26, 42.22);

\path[fill=fillColor,fill opacity=0.60] (464.26, 42.22) rectangle (466.64, 42.22);

\path[fill=fillColor,fill opacity=0.60] (466.64, 42.22) rectangle (469.03, 42.22);

\path[fill=fillColor,fill opacity=0.60] (469.03, 42.22) rectangle (471.41, 42.22);

\path[fill=fillColor,fill opacity=0.60] (471.41, 42.22) rectangle (473.79, 42.22);

\path[fill=fillColor,fill opacity=0.60] (473.79, 42.22) rectangle (476.18, 42.22);

\path[fill=fillColor,fill opacity=0.60] (476.18, 42.22) rectangle (478.56, 42.22);

\path[fill=fillColor,fill opacity=0.60] (478.56, 42.22) rectangle (480.94, 42.22);

\path[fill=fillColor,fill opacity=0.60] (480.94, 42.22) rectangle (483.33, 42.22);

\path[fill=fillColor,fill opacity=0.60] (483.33, 42.22) rectangle (485.71, 42.22);

\path[fill=fillColor,fill opacity=0.60] (485.71, 42.22) rectangle (488.09, 42.22);

\path[fill=fillColor,fill opacity=0.60] (488.09, 42.22) rectangle (490.48, 42.22);

\path[fill=fillColor,fill opacity=0.60] (490.48, 42.22) rectangle (492.86, 42.22);

\path[fill=fillColor,fill opacity=0.60] (492.86, 42.22) rectangle (495.25, 42.22);

\path[fill=fillColor,fill opacity=0.60] (495.25, 42.22) rectangle (497.63, 42.22);

\path[fill=fillColor,fill opacity=0.60] (497.63, 42.22) rectangle (500.01, 42.22);

\path[fill=fillColor,fill opacity=0.60] (500.01, 42.22) rectangle (502.40, 42.22);

\path[fill=fillColor,fill opacity=0.60] (502.40, 42.22) rectangle (504.78, 42.22);

\path[fill=fillColor,fill opacity=0.60] (504.78, 42.22) rectangle (507.16, 42.22);

\path[fill=fillColor,fill opacity=0.60] (507.16, 42.22) rectangle (509.55, 42.22);

\path[fill=fillColor,fill opacity=0.60] (509.55, 42.22) rectangle (511.93, 42.22);

\path[fill=fillColor,fill opacity=0.60] (511.93, 42.22) rectangle (514.31, 42.22);

\path[fill=fillColor,fill opacity=0.60] (514.31, 42.22) rectangle (516.70, 42.22);

\path[fill=fillColor,fill opacity=0.60] (516.70, 42.22) rectangle (519.08, 42.22);

\path[fill=fillColor,fill opacity=0.60] (519.08, 42.22) rectangle (521.46, 42.22);

\path[fill=fillColor,fill opacity=0.60] (521.46, 42.22) rectangle (523.85, 42.22);

\path[fill=fillColor,fill opacity=0.60] (523.85, 42.22) rectangle (526.23, 42.22);

\path[fill=fillColor,fill opacity=0.60] (526.23, 42.22) rectangle (528.61, 42.22);

\path[fill=fillColor,fill opacity=0.60] (528.61, 42.22) rectangle (531.00, 42.22);

\path[fill=fillColor,fill opacity=0.60] (531.00, 42.22) rectangle (533.38, 42.22);

\path[fill=fillColor,fill opacity=0.60] (533.38, 42.22) rectangle (535.76, 42.22);

\path[fill=fillColor,fill opacity=0.60] (535.76, 42.22) rectangle (538.15, 42.22);

\path[fill=fillColor,fill opacity=0.60] (538.15, 42.22) rectangle (540.53, 42.22);

\path[fill=fillColor,fill opacity=0.60] (540.53, 42.22) rectangle (542.91, 42.22);

\path[fill=fillColor,fill opacity=0.60] (542.91, 42.22) rectangle (545.30, 42.22);

\path[fill=fillColor,fill opacity=0.60] (545.30, 42.22) rectangle (547.68, 42.22);
\definecolor{fillColor}{RGB}{230,159,0}

\path[fill=fillColor,fill opacity=0.60] ( 63.88, 42.22) rectangle ( 66.36, 42.22);

\path[fill=fillColor,fill opacity=0.60] ( 66.36, 42.22) rectangle ( 68.83, 42.22);

\path[fill=fillColor,fill opacity=0.60] ( 68.83, 42.22) rectangle ( 71.30, 42.22);

\path[fill=fillColor,fill opacity=0.60] ( 71.30, 42.22) rectangle ( 73.78, 42.22);

\path[fill=fillColor,fill opacity=0.60] ( 73.78, 42.22) rectangle ( 76.25, 42.22);

\path[fill=fillColor,fill opacity=0.60] ( 76.25, 42.22) rectangle ( 78.72, 42.22);

\path[fill=fillColor,fill opacity=0.60] ( 78.72, 42.22) rectangle ( 81.20, 42.22);

\path[fill=fillColor,fill opacity=0.60] ( 81.20, 42.22) rectangle ( 83.67, 42.22);

\path[fill=fillColor,fill opacity=0.60] ( 83.67, 42.22) rectangle ( 86.14, 42.22);

\path[fill=fillColor,fill opacity=0.60] ( 86.14, 42.22) rectangle ( 88.61, 42.22);

\path[fill=fillColor,fill opacity=0.60] ( 88.61, 42.22) rectangle ( 91.09, 42.22);

\path[fill=fillColor,fill opacity=0.60] ( 91.09, 42.22) rectangle ( 93.56, 42.22);

\path[fill=fillColor,fill opacity=0.60] ( 93.56, 42.22) rectangle ( 96.03, 42.22);

\path[fill=fillColor,fill opacity=0.60] ( 96.03, 42.22) rectangle ( 98.51, 42.22);

\path[fill=fillColor,fill opacity=0.60] ( 98.51, 42.22) rectangle (100.98, 42.22);

\path[fill=fillColor,fill opacity=0.60] (100.98, 42.22) rectangle (103.45, 42.22);

\path[fill=fillColor,fill opacity=0.60] (103.45, 42.22) rectangle (105.93, 42.22);

\path[fill=fillColor,fill opacity=0.60] (105.93, 42.22) rectangle (108.40, 42.22);

\path[fill=fillColor,fill opacity=0.60] (108.40, 42.22) rectangle (110.87, 42.22);

\path[fill=fillColor,fill opacity=0.60] (110.87, 42.22) rectangle (113.35, 42.22);

\path[fill=fillColor,fill opacity=0.60] (113.35, 42.22) rectangle (115.82, 42.22);

\path[fill=fillColor,fill opacity=0.60] (115.82, 42.22) rectangle (118.29, 42.22);

\path[fill=fillColor,fill opacity=0.60] (118.29, 42.22) rectangle (120.76, 42.22);

\path[fill=fillColor,fill opacity=0.60] (120.76, 42.22) rectangle (123.24, 42.22);

\path[fill=fillColor,fill opacity=0.60] (123.24, 42.22) rectangle (125.71, 42.22);

\path[fill=fillColor,fill opacity=0.60] (125.71, 42.22) rectangle (128.18, 42.22);

\path[fill=fillColor,fill opacity=0.60] (128.18, 42.22) rectangle (130.66, 42.22);

\path[fill=fillColor,fill opacity=0.60] (130.66, 42.22) rectangle (133.13, 42.22);

\path[fill=fillColor,fill opacity=0.60] (133.13, 42.22) rectangle (135.60, 42.22);

\path[fill=fillColor,fill opacity=0.60] (135.60, 42.22) rectangle (138.08, 42.22);

\path[fill=fillColor,fill opacity=0.60] (138.08, 42.22) rectangle (140.55, 42.22);

\path[fill=fillColor,fill opacity=0.60] (140.55, 42.22) rectangle (143.02, 42.22);

\path[fill=fillColor,fill opacity=0.60] (143.02, 42.22) rectangle (145.50, 42.22);

\path[fill=fillColor,fill opacity=0.60] (145.50, 42.22) rectangle (147.97, 42.22);

\path[fill=fillColor,fill opacity=0.60] (147.97, 42.22) rectangle (150.44, 42.22);

\path[fill=fillColor,fill opacity=0.60] (150.44, 42.22) rectangle (152.92, 42.22);

\path[fill=fillColor,fill opacity=0.60] (152.92, 42.22) rectangle (155.39, 42.22);

\path[fill=fillColor,fill opacity=0.60] (155.39, 42.22) rectangle (157.86, 42.22);

\path[fill=fillColor,fill opacity=0.60] (157.86, 42.22) rectangle (160.33, 42.23);

\path[fill=fillColor,fill opacity=0.60] (160.33, 42.22) rectangle (162.81, 42.23);

\path[fill=fillColor,fill opacity=0.60] (162.81, 42.22) rectangle (165.28, 42.31);

\path[fill=fillColor,fill opacity=0.60] (165.28, 42.22) rectangle (167.75, 42.60);

\path[fill=fillColor,fill opacity=0.60] (167.75, 42.22) rectangle (170.23, 43.13);

\path[fill=fillColor,fill opacity=0.60] (170.23, 42.22) rectangle (172.70, 44.81);

\path[fill=fillColor,fill opacity=0.60] (172.70, 42.22) rectangle (175.17, 48.15);

\path[fill=fillColor,fill opacity=0.60] (175.17, 42.22) rectangle (177.65, 54.76);

\path[fill=fillColor,fill opacity=0.60] (177.65, 42.22) rectangle (180.12, 67.10);

\path[fill=fillColor,fill opacity=0.60] (180.12, 42.22) rectangle (182.59, 86.99);

\path[fill=fillColor,fill opacity=0.60] (182.59, 42.22) rectangle (185.07,113.49);

\path[fill=fillColor,fill opacity=0.60] (185.07, 42.22) rectangle (187.54,147.26);

\path[fill=fillColor,fill opacity=0.60] (187.54, 42.22) rectangle (190.01,183.16);

\path[fill=fillColor,fill opacity=0.60] (190.01, 42.22) rectangle (192.49,215.04);

\path[fill=fillColor,fill opacity=0.60] (192.49, 42.22) rectangle (194.96,231.08);

\path[fill=fillColor,fill opacity=0.60] (194.96, 42.22) rectangle (197.43,232.77);

\path[fill=fillColor,fill opacity=0.60] (197.43, 42.22) rectangle (199.90,215.11);

\path[fill=fillColor,fill opacity=0.60] (199.90, 42.22) rectangle (202.38,181.69);

\path[fill=fillColor,fill opacity=0.60] (202.38, 42.22) rectangle (204.85,145.99);

\path[fill=fillColor,fill opacity=0.60] (204.85, 42.22) rectangle (207.32,111.48);

\path[fill=fillColor,fill opacity=0.60] (207.32, 42.22) rectangle (209.80, 84.54);

\path[fill=fillColor,fill opacity=0.60] (209.80, 42.22) rectangle (212.27, 65.85);

\path[fill=fillColor,fill opacity=0.60] (212.27, 42.22) rectangle (214.74, 53.68);

\path[fill=fillColor,fill opacity=0.60] (214.74, 42.22) rectangle (217.22, 47.18);

\path[fill=fillColor,fill opacity=0.60] (217.22, 42.22) rectangle (219.69, 44.22);

\path[fill=fillColor,fill opacity=0.60] (219.69, 42.22) rectangle (222.16, 42.92);

\path[fill=fillColor,fill opacity=0.60] (222.16, 42.22) rectangle (224.64, 42.47);

\path[fill=fillColor,fill opacity=0.60] (224.64, 42.22) rectangle (227.11, 42.29);

\path[fill=fillColor,fill opacity=0.60] (227.11, 42.22) rectangle (229.58, 42.24);

\path[fill=fillColor,fill opacity=0.60] (229.58, 42.22) rectangle (232.05, 42.22);

\path[fill=fillColor,fill opacity=0.60] (232.05, 42.22) rectangle (234.53, 42.22);

\path[fill=fillColor,fill opacity=0.60] (234.53, 42.22) rectangle (237.00, 42.22);

\path[fill=fillColor,fill opacity=0.60] (237.00, 42.22) rectangle (239.47, 42.22);

\path[fill=fillColor,fill opacity=0.60] (239.47, 42.22) rectangle (241.95, 42.22);

\path[fill=fillColor,fill opacity=0.60] (241.95, 42.22) rectangle (244.42, 42.22);

\path[fill=fillColor,fill opacity=0.60] (244.42, 42.22) rectangle (246.89, 42.22);

\path[fill=fillColor,fill opacity=0.60] (246.89, 42.22) rectangle (249.37, 42.22);

\path[fill=fillColor,fill opacity=0.60] (249.37, 42.22) rectangle (251.84, 42.22);

\path[fill=fillColor,fill opacity=0.60] (251.84, 42.22) rectangle (254.31, 42.22);

\path[fill=fillColor,fill opacity=0.60] (254.31, 42.22) rectangle (256.79, 42.22);

\path[fill=fillColor,fill opacity=0.60] (256.79, 42.22) rectangle (259.26, 42.22);

\path[fill=fillColor,fill opacity=0.60] (259.26, 42.22) rectangle (261.73, 42.22);

\path[fill=fillColor,fill opacity=0.60] (261.73, 42.22) rectangle (264.21, 42.22);

\path[fill=fillColor,fill opacity=0.60] (264.21, 42.22) rectangle (266.68, 42.22);

\path[fill=fillColor,fill opacity=0.60] (266.68, 42.22) rectangle (269.15, 42.22);

\path[fill=fillColor,fill opacity=0.60] (269.15, 42.22) rectangle (271.62, 42.22);

\path[fill=fillColor,fill opacity=0.60] (271.62, 42.22) rectangle (274.10, 42.22);

\path[fill=fillColor,fill opacity=0.60] (274.10, 42.22) rectangle (276.57, 42.22);

\path[fill=fillColor,fill opacity=0.60] (276.57, 42.22) rectangle (279.04, 42.22);

\path[fill=fillColor,fill opacity=0.60] (279.04, 42.22) rectangle (281.52, 42.22);

\path[fill=fillColor,fill opacity=0.60] (281.52, 42.22) rectangle (283.99, 42.22);

\path[fill=fillColor,fill opacity=0.60] (283.99, 42.22) rectangle (286.46, 42.22);

\path[fill=fillColor,fill opacity=0.60] (286.46, 42.22) rectangle (288.94, 42.22);

\path[fill=fillColor,fill opacity=0.60] (288.94, 42.22) rectangle (291.41, 42.22);

\path[fill=fillColor,fill opacity=0.60] (291.41, 42.22) rectangle (293.88, 42.22);

\path[fill=fillColor,fill opacity=0.60] (293.88, 42.22) rectangle (296.36, 42.22);

\path[fill=fillColor,fill opacity=0.60] (296.36, 42.22) rectangle (298.83, 42.22);

\path[fill=fillColor,fill opacity=0.60] (298.83, 42.22) rectangle (301.30, 42.22);

\path[fill=fillColor,fill opacity=0.60] (301.30, 42.22) rectangle (303.77, 42.22);

\path[fill=fillColor,fill opacity=0.60] (303.77, 42.22) rectangle (306.25, 42.22);

\path[fill=fillColor,fill opacity=0.60] (306.25, 42.22) rectangle (308.72, 42.22);

\path[fill=fillColor,fill opacity=0.60] (308.72, 42.22) rectangle (311.19, 42.22);

\path[fill=fillColor,fill opacity=0.60] (311.19, 42.22) rectangle (313.67, 42.22);

\path[fill=fillColor,fill opacity=0.60] (313.67, 42.22) rectangle (316.14, 42.22);

\path[fill=fillColor,fill opacity=0.60] (316.14, 42.22) rectangle (318.61, 42.22);

\path[fill=fillColor,fill opacity=0.60] (318.61, 42.22) rectangle (321.09, 42.22);

\path[fill=fillColor,fill opacity=0.60] (321.09, 42.22) rectangle (323.56, 42.22);

\path[fill=fillColor,fill opacity=0.60] (323.56, 42.22) rectangle (326.03, 42.22);

\path[fill=fillColor,fill opacity=0.60] (326.03, 42.22) rectangle (328.51, 42.22);

\path[fill=fillColor,fill opacity=0.60] (328.51, 42.22) rectangle (330.98, 42.22);

\path[fill=fillColor,fill opacity=0.60] (330.98, 42.22) rectangle (333.45, 42.22);

\path[fill=fillColor,fill opacity=0.60] (333.45, 42.22) rectangle (335.93, 42.22);

\path[fill=fillColor,fill opacity=0.60] (335.93, 42.22) rectangle (338.40, 42.22);

\path[fill=fillColor,fill opacity=0.60] (338.40, 42.22) rectangle (340.87, 42.22);

\path[fill=fillColor,fill opacity=0.60] (340.87, 42.22) rectangle (343.34, 42.22);

\path[fill=fillColor,fill opacity=0.60] (343.34, 42.22) rectangle (345.82, 42.22);

\path[fill=fillColor,fill opacity=0.60] (345.82, 42.22) rectangle (348.29, 42.22);

\path[fill=fillColor,fill opacity=0.60] (348.29, 42.22) rectangle (350.76, 42.22);

\path[fill=fillColor,fill opacity=0.60] (350.76, 42.22) rectangle (353.24, 42.22);

\path[fill=fillColor,fill opacity=0.60] (353.24, 42.22) rectangle (355.71, 42.22);

\path[fill=fillColor,fill opacity=0.60] (355.71, 42.22) rectangle (358.18, 42.22);

\path[fill=fillColor,fill opacity=0.60] (358.18, 42.22) rectangle (360.66, 42.22);

\path[fill=fillColor,fill opacity=0.60] (360.66, 42.22) rectangle (363.13, 42.22);

\path[fill=fillColor,fill opacity=0.60] (363.13, 42.22) rectangle (365.60, 42.22);

\path[fill=fillColor,fill opacity=0.60] (365.60, 42.22) rectangle (368.08, 42.22);

\path[fill=fillColor,fill opacity=0.60] (368.08, 42.22) rectangle (370.55, 42.22);

\path[fill=fillColor,fill opacity=0.60] (370.55, 42.22) rectangle (373.02, 42.22);

\path[fill=fillColor,fill opacity=0.60] (373.02, 42.22) rectangle (375.50, 42.22);

\path[fill=fillColor,fill opacity=0.60] (375.50, 42.22) rectangle (377.97, 42.22);

\path[fill=fillColor,fill opacity=0.60] (377.97, 42.22) rectangle (380.44, 42.22);

\path[fill=fillColor,fill opacity=0.60] (380.44, 42.22) rectangle (382.91, 42.22);

\path[fill=fillColor,fill opacity=0.60] (382.91, 42.22) rectangle (385.39, 42.22);

\path[fill=fillColor,fill opacity=0.60] (385.39, 42.22) rectangle (387.86, 42.22);

\path[fill=fillColor,fill opacity=0.60] (387.86, 42.22) rectangle (390.33, 42.22);

\path[fill=fillColor,fill opacity=0.60] (390.33, 42.22) rectangle (392.81, 42.22);

\path[fill=fillColor,fill opacity=0.60] (392.81, 42.22) rectangle (395.28, 42.22);

\path[fill=fillColor,fill opacity=0.60] (395.28, 42.22) rectangle (397.75, 42.22);

\path[fill=fillColor,fill opacity=0.60] (397.75, 42.22) rectangle (400.23, 42.22);

\path[fill=fillColor,fill opacity=0.60] (400.23, 42.22) rectangle (402.70, 42.22);

\path[fill=fillColor,fill opacity=0.60] (402.70, 42.22) rectangle (405.17, 42.22);

\path[fill=fillColor,fill opacity=0.60] (405.17, 42.22) rectangle (407.65, 42.22);

\path[fill=fillColor,fill opacity=0.60] (407.65, 42.22) rectangle (410.12, 42.22);

\path[fill=fillColor,fill opacity=0.60] (410.12, 42.22) rectangle (412.59, 42.22);

\path[fill=fillColor,fill opacity=0.60] (412.59, 42.22) rectangle (415.06, 42.22);

\path[fill=fillColor,fill opacity=0.60] (415.06, 42.22) rectangle (417.54, 42.22);

\path[fill=fillColor,fill opacity=0.60] (417.54, 42.22) rectangle (420.01, 42.22);

\path[fill=fillColor,fill opacity=0.60] (420.01, 42.22) rectangle (422.48, 42.22);

\path[fill=fillColor,fill opacity=0.60] (422.48, 42.22) rectangle (424.96, 42.22);

\path[fill=fillColor,fill opacity=0.60] (424.96, 42.22) rectangle (427.43, 42.22);

\path[fill=fillColor,fill opacity=0.60] (427.43, 42.22) rectangle (429.90, 42.22);

\path[fill=fillColor,fill opacity=0.60] (429.90, 42.22) rectangle (432.38, 42.22);

\path[fill=fillColor,fill opacity=0.60] (432.38, 42.22) rectangle (434.85, 42.22);

\path[fill=fillColor,fill opacity=0.60] (434.85, 42.22) rectangle (437.32, 42.22);

\path[fill=fillColor,fill opacity=0.60] (437.32, 42.22) rectangle (439.80, 42.22);

\path[fill=fillColor,fill opacity=0.60] (439.80, 42.22) rectangle (442.27, 42.22);

\path[fill=fillColor,fill opacity=0.60] (442.27, 42.22) rectangle (444.74, 42.22);

\path[fill=fillColor,fill opacity=0.60] (444.74, 42.22) rectangle (447.22, 42.22);

\path[fill=fillColor,fill opacity=0.60] (447.22, 42.22) rectangle (449.69, 42.22);

\path[fill=fillColor,fill opacity=0.60] (449.69, 42.22) rectangle (452.16, 42.22);

\path[fill=fillColor,fill opacity=0.60] (452.16, 42.22) rectangle (454.63, 42.22);

\path[fill=fillColor,fill opacity=0.60] (454.63, 42.22) rectangle (457.11, 42.22);

\path[fill=fillColor,fill opacity=0.60] (457.11, 42.22) rectangle (459.58, 42.22);

\path[fill=fillColor,fill opacity=0.60] (459.58, 42.22) rectangle (462.05, 42.22);

\path[fill=fillColor,fill opacity=0.60] (462.05, 42.22) rectangle (464.53, 42.22);

\path[fill=fillColor,fill opacity=0.60] (464.53, 42.22) rectangle (467.00, 42.22);

\path[fill=fillColor,fill opacity=0.60] (467.00, 42.22) rectangle (469.47, 42.22);

\path[fill=fillColor,fill opacity=0.60] (469.47, 42.22) rectangle (471.95, 42.22);

\path[fill=fillColor,fill opacity=0.60] (471.95, 42.22) rectangle (474.42, 42.22);

\path[fill=fillColor,fill opacity=0.60] (474.42, 42.22) rectangle (476.89, 42.22);

\path[fill=fillColor,fill opacity=0.60] (476.89, 42.22) rectangle (479.37, 42.22);

\path[fill=fillColor,fill opacity=0.60] (479.37, 42.22) rectangle (481.84, 42.22);

\path[fill=fillColor,fill opacity=0.60] (481.84, 42.22) rectangle (484.31, 42.22);

\path[fill=fillColor,fill opacity=0.60] (484.31, 42.22) rectangle (486.78, 42.22);

\path[fill=fillColor,fill opacity=0.60] (486.78, 42.22) rectangle (489.26, 42.22);

\path[fill=fillColor,fill opacity=0.60] (489.26, 42.22) rectangle (491.73, 42.22);

\path[fill=fillColor,fill opacity=0.60] (491.73, 42.22) rectangle (494.20, 42.22);

\path[fill=fillColor,fill opacity=0.60] (494.20, 42.22) rectangle (496.68, 42.22);

\path[fill=fillColor,fill opacity=0.60] (496.68, 42.22) rectangle (499.15, 42.22);

\path[fill=fillColor,fill opacity=0.60] (499.15, 42.22) rectangle (501.62, 42.22);

\path[fill=fillColor,fill opacity=0.60] (501.62, 42.22) rectangle (504.10, 42.22);

\path[fill=fillColor,fill opacity=0.60] (504.10, 42.22) rectangle (506.57, 42.22);

\path[fill=fillColor,fill opacity=0.60] (506.57, 42.22) rectangle (509.04, 42.22);

\path[fill=fillColor,fill opacity=0.60] (509.04, 42.22) rectangle (511.52, 42.22);

\path[fill=fillColor,fill opacity=0.60] (511.52, 42.22) rectangle (513.99, 42.22);

\path[fill=fillColor,fill opacity=0.60] (513.99, 42.22) rectangle (516.46, 42.22);

\path[fill=fillColor,fill opacity=0.60] (516.46, 42.22) rectangle (518.94, 42.22);

\path[fill=fillColor,fill opacity=0.60] (518.94, 42.22) rectangle (521.41, 42.22);

\path[fill=fillColor,fill opacity=0.60] (521.41, 42.22) rectangle (523.88, 42.22);

\path[fill=fillColor,fill opacity=0.60] (523.88, 42.22) rectangle (526.35, 42.22);

\path[fill=fillColor,fill opacity=0.60] (526.35, 42.22) rectangle (528.83, 42.22);

\path[fill=fillColor,fill opacity=0.60] (528.83, 42.22) rectangle (531.30, 42.22);

\path[fill=fillColor,fill opacity=0.60] (531.30, 42.22) rectangle (533.77, 42.22);

\path[fill=fillColor,fill opacity=0.60] (533.77, 42.22) rectangle (536.25, 42.22);

\path[fill=fillColor,fill opacity=0.60] (536.25, 42.22) rectangle (538.72, 42.22);

\path[fill=fillColor,fill opacity=0.60] (538.72, 42.22) rectangle (541.19, 42.22);

\path[fill=fillColor,fill opacity=0.60] (541.19, 42.22) rectangle (543.67, 42.22);

\path[fill=fillColor,fill opacity=0.60] (543.67, 42.22) rectangle (546.14, 42.22);
\definecolor{fillColor}{RGB}{204,121,167}

\path[fill=fillColor,fill opacity=0.60] ( 63.87, 42.22) rectangle ( 66.33, 42.22);

\path[fill=fillColor,fill opacity=0.60] ( 66.33, 42.22) rectangle ( 68.79, 42.22);

\path[fill=fillColor,fill opacity=0.60] ( 68.79, 42.22) rectangle ( 71.24, 42.22);

\path[fill=fillColor,fill opacity=0.60] ( 71.24, 42.22) rectangle ( 73.70, 42.22);

\path[fill=fillColor,fill opacity=0.60] ( 73.70, 42.22) rectangle ( 76.15, 42.22);

\path[fill=fillColor,fill opacity=0.60] ( 76.15, 42.22) rectangle ( 78.61, 42.22);

\path[fill=fillColor,fill opacity=0.60] ( 78.61, 42.22) rectangle ( 81.06, 42.22);

\path[fill=fillColor,fill opacity=0.60] ( 81.06, 42.22) rectangle ( 83.52, 42.22);

\path[fill=fillColor,fill opacity=0.60] ( 83.52, 42.22) rectangle ( 85.97, 42.22);

\path[fill=fillColor,fill opacity=0.60] ( 85.97, 42.22) rectangle ( 88.43, 42.22);

\path[fill=fillColor,fill opacity=0.60] ( 88.43, 42.22) rectangle ( 90.88, 42.22);

\path[fill=fillColor,fill opacity=0.60] ( 90.88, 42.22) rectangle ( 93.34, 42.22);

\path[fill=fillColor,fill opacity=0.60] ( 93.34, 42.22) rectangle ( 95.79, 42.22);

\path[fill=fillColor,fill opacity=0.60] ( 95.79, 42.22) rectangle ( 98.25, 42.22);

\path[fill=fillColor,fill opacity=0.60] ( 98.25, 42.22) rectangle (100.70, 42.22);

\path[fill=fillColor,fill opacity=0.60] (100.70, 42.22) rectangle (103.16, 42.22);

\path[fill=fillColor,fill opacity=0.60] (103.16, 42.22) rectangle (105.61, 42.22);

\path[fill=fillColor,fill opacity=0.60] (105.61, 42.22) rectangle (108.07, 42.22);

\path[fill=fillColor,fill opacity=0.60] (108.07, 42.22) rectangle (110.53, 42.22);

\path[fill=fillColor,fill opacity=0.60] (110.53, 42.22) rectangle (112.98, 42.22);

\path[fill=fillColor,fill opacity=0.60] (112.98, 42.22) rectangle (115.44, 42.22);

\path[fill=fillColor,fill opacity=0.60] (115.44, 42.22) rectangle (117.89, 42.22);

\path[fill=fillColor,fill opacity=0.60] (117.89, 42.22) rectangle (120.35, 42.22);

\path[fill=fillColor,fill opacity=0.60] (120.35, 42.22) rectangle (122.80, 42.22);

\path[fill=fillColor,fill opacity=0.60] (122.80, 42.22) rectangle (125.26, 42.22);

\path[fill=fillColor,fill opacity=0.60] (125.26, 42.22) rectangle (127.71, 42.22);

\path[fill=fillColor,fill opacity=0.60] (127.71, 42.22) rectangle (130.17, 42.22);

\path[fill=fillColor,fill opacity=0.60] (130.17, 42.22) rectangle (132.62, 42.22);

\path[fill=fillColor,fill opacity=0.60] (132.62, 42.22) rectangle (135.08, 42.22);

\path[fill=fillColor,fill opacity=0.60] (135.08, 42.22) rectangle (137.53, 42.22);

\path[fill=fillColor,fill opacity=0.60] (137.53, 42.22) rectangle (139.99, 42.22);

\path[fill=fillColor,fill opacity=0.60] (139.99, 42.22) rectangle (142.44, 42.22);

\path[fill=fillColor,fill opacity=0.60] (142.44, 42.22) rectangle (144.90, 42.22);

\path[fill=fillColor,fill opacity=0.60] (144.90, 42.22) rectangle (147.36, 42.22);

\path[fill=fillColor,fill opacity=0.60] (147.36, 42.22) rectangle (149.81, 42.22);

\path[fill=fillColor,fill opacity=0.60] (149.81, 42.22) rectangle (152.27, 42.22);

\path[fill=fillColor,fill opacity=0.60] (152.27, 42.22) rectangle (154.72, 42.22);

\path[fill=fillColor,fill opacity=0.60] (154.72, 42.22) rectangle (157.18, 42.22);

\path[fill=fillColor,fill opacity=0.60] (157.18, 42.22) rectangle (159.63, 42.22);

\path[fill=fillColor,fill opacity=0.60] (159.63, 42.22) rectangle (162.09, 42.22);

\path[fill=fillColor,fill opacity=0.60] (162.09, 42.22) rectangle (164.54, 42.22);

\path[fill=fillColor,fill opacity=0.60] (164.54, 42.22) rectangle (167.00, 42.22);

\path[fill=fillColor,fill opacity=0.60] (167.00, 42.22) rectangle (169.45, 42.22);

\path[fill=fillColor,fill opacity=0.60] (169.45, 42.22) rectangle (171.91, 42.22);

\path[fill=fillColor,fill opacity=0.60] (171.91, 42.22) rectangle (174.36, 42.22);

\path[fill=fillColor,fill opacity=0.60] (174.36, 42.22) rectangle (176.82, 42.22);

\path[fill=fillColor,fill opacity=0.60] (176.82, 42.22) rectangle (179.27, 42.22);

\path[fill=fillColor,fill opacity=0.60] (179.27, 42.22) rectangle (181.73, 42.22);

\path[fill=fillColor,fill opacity=0.60] (181.73, 42.22) rectangle (184.18, 42.22);

\path[fill=fillColor,fill opacity=0.60] (184.18, 42.22) rectangle (186.64, 42.22);

\path[fill=fillColor,fill opacity=0.60] (186.64, 42.22) rectangle (189.10, 42.22);

\path[fill=fillColor,fill opacity=0.60] (189.10, 42.22) rectangle (191.55, 42.22);

\path[fill=fillColor,fill opacity=0.60] (191.55, 42.22) rectangle (194.01, 42.22);

\path[fill=fillColor,fill opacity=0.60] (194.01, 42.22) rectangle (196.46, 42.22);

\path[fill=fillColor,fill opacity=0.60] (196.46, 42.22) rectangle (198.92, 42.22);

\path[fill=fillColor,fill opacity=0.60] (198.92, 42.22) rectangle (201.37, 42.22);

\path[fill=fillColor,fill opacity=0.60] (201.37, 42.22) rectangle (203.83, 42.22);

\path[fill=fillColor,fill opacity=0.60] (203.83, 42.22) rectangle (206.28, 42.22);

\path[fill=fillColor,fill opacity=0.60] (206.28, 42.22) rectangle (208.74, 42.22);

\path[fill=fillColor,fill opacity=0.60] (208.74, 42.22) rectangle (211.19, 42.22);

\path[fill=fillColor,fill opacity=0.60] (211.19, 42.22) rectangle (213.65, 42.22);

\path[fill=fillColor,fill opacity=0.60] (213.65, 42.22) rectangle (216.10, 42.22);

\path[fill=fillColor,fill opacity=0.60] (216.10, 42.22) rectangle (218.56, 42.22);

\path[fill=fillColor,fill opacity=0.60] (218.56, 42.22) rectangle (221.01, 42.22);

\path[fill=fillColor,fill opacity=0.60] (221.01, 42.22) rectangle (223.47, 42.22);

\path[fill=fillColor,fill opacity=0.60] (223.47, 42.22) rectangle (225.93, 42.22);

\path[fill=fillColor,fill opacity=0.60] (225.93, 42.22) rectangle (228.38, 42.22);

\path[fill=fillColor,fill opacity=0.60] (228.38, 42.22) rectangle (230.84, 42.22);

\path[fill=fillColor,fill opacity=0.60] (230.84, 42.22) rectangle (233.29, 42.22);

\path[fill=fillColor,fill opacity=0.60] (233.29, 42.22) rectangle (235.75, 42.22);

\path[fill=fillColor,fill opacity=0.60] (235.75, 42.22) rectangle (238.20, 42.22);

\path[fill=fillColor,fill opacity=0.60] (238.20, 42.22) rectangle (240.66, 42.22);

\path[fill=fillColor,fill opacity=0.60] (240.66, 42.22) rectangle (243.11, 42.22);

\path[fill=fillColor,fill opacity=0.60] (243.11, 42.22) rectangle (245.57, 42.22);

\path[fill=fillColor,fill opacity=0.60] (245.57, 42.22) rectangle (248.02, 42.22);

\path[fill=fillColor,fill opacity=0.60] (248.02, 42.22) rectangle (250.48, 42.22);

\path[fill=fillColor,fill opacity=0.60] (250.48, 42.22) rectangle (252.93, 42.22);

\path[fill=fillColor,fill opacity=0.60] (252.93, 42.22) rectangle (255.39, 42.22);

\path[fill=fillColor,fill opacity=0.60] (255.39, 42.22) rectangle (257.84, 42.22);

\path[fill=fillColor,fill opacity=0.60] (257.84, 42.22) rectangle (260.30, 42.22);

\path[fill=fillColor,fill opacity=0.60] (260.30, 42.22) rectangle (262.75, 42.22);

\path[fill=fillColor,fill opacity=0.60] (262.75, 42.22) rectangle (265.21, 42.22);

\path[fill=fillColor,fill opacity=0.60] (265.21, 42.22) rectangle (267.67, 42.22);

\path[fill=fillColor,fill opacity=0.60] (267.67, 42.22) rectangle (270.12, 42.22);

\path[fill=fillColor,fill opacity=0.60] (270.12, 42.22) rectangle (272.58, 42.22);

\path[fill=fillColor,fill opacity=0.60] (272.58, 42.22) rectangle (275.03, 42.22);

\path[fill=fillColor,fill opacity=0.60] (275.03, 42.22) rectangle (277.49, 42.22);

\path[fill=fillColor,fill opacity=0.60] (277.49, 42.22) rectangle (279.94, 42.22);

\path[fill=fillColor,fill opacity=0.60] (279.94, 42.22) rectangle (282.40, 42.22);

\path[fill=fillColor,fill opacity=0.60] (282.40, 42.22) rectangle (284.85, 42.22);

\path[fill=fillColor,fill opacity=0.60] (284.85, 42.22) rectangle (287.31, 42.22);

\path[fill=fillColor,fill opacity=0.60] (287.31, 42.22) rectangle (289.76, 42.22);

\path[fill=fillColor,fill opacity=0.60] (289.76, 42.22) rectangle (292.22, 42.22);

\path[fill=fillColor,fill opacity=0.60] (292.22, 42.22) rectangle (294.67, 42.22);

\path[fill=fillColor,fill opacity=0.60] (294.67, 42.22) rectangle (297.13, 42.22);

\path[fill=fillColor,fill opacity=0.60] (297.13, 42.22) rectangle (299.58, 42.22);

\path[fill=fillColor,fill opacity=0.60] (299.58, 42.22) rectangle (302.04, 42.22);

\path[fill=fillColor,fill opacity=0.60] (302.04, 42.22) rectangle (304.50, 42.22);

\path[fill=fillColor,fill opacity=0.60] (304.50, 42.22) rectangle (306.95, 42.22);

\path[fill=fillColor,fill opacity=0.60] (306.95, 42.22) rectangle (309.41, 42.22);

\path[fill=fillColor,fill opacity=0.60] (309.41, 42.22) rectangle (311.86, 42.22);

\path[fill=fillColor,fill opacity=0.60] (311.86, 42.22) rectangle (314.32, 42.22);

\path[fill=fillColor,fill opacity=0.60] (314.32, 42.22) rectangle (316.77, 42.22);

\path[fill=fillColor,fill opacity=0.60] (316.77, 42.22) rectangle (319.23, 42.22);

\path[fill=fillColor,fill opacity=0.60] (319.23, 42.22) rectangle (321.68, 42.22);

\path[fill=fillColor,fill opacity=0.60] (321.68, 42.22) rectangle (324.14, 42.22);

\path[fill=fillColor,fill opacity=0.60] (324.14, 42.22) rectangle (326.59, 42.22);

\path[fill=fillColor,fill opacity=0.60] (326.59, 42.22) rectangle (329.05, 42.22);

\path[fill=fillColor,fill opacity=0.60] (329.05, 42.22) rectangle (331.50, 42.22);

\path[fill=fillColor,fill opacity=0.60] (331.50, 42.22) rectangle (333.96, 42.22);

\path[fill=fillColor,fill opacity=0.60] (333.96, 42.22) rectangle (336.41, 42.22);

\path[fill=fillColor,fill opacity=0.60] (336.41, 42.22) rectangle (338.87, 42.22);

\path[fill=fillColor,fill opacity=0.60] (338.87, 42.22) rectangle (341.32, 42.22);

\path[fill=fillColor,fill opacity=0.60] (341.32, 42.22) rectangle (343.78, 42.22);

\path[fill=fillColor,fill opacity=0.60] (343.78, 42.22) rectangle (346.24, 42.22);

\path[fill=fillColor,fill opacity=0.60] (346.24, 42.22) rectangle (348.69, 42.22);

\path[fill=fillColor,fill opacity=0.60] (348.69, 42.22) rectangle (351.15, 42.22);

\path[fill=fillColor,fill opacity=0.60] (351.15, 42.22) rectangle (353.60, 42.22);

\path[fill=fillColor,fill opacity=0.60] (353.60, 42.22) rectangle (356.06, 42.22);

\path[fill=fillColor,fill opacity=0.60] (356.06, 42.22) rectangle (358.51, 42.22);

\path[fill=fillColor,fill opacity=0.60] (358.51, 42.22) rectangle (360.97, 42.22);

\path[fill=fillColor,fill opacity=0.60] (360.97, 42.22) rectangle (363.42, 42.22);

\path[fill=fillColor,fill opacity=0.60] (363.42, 42.22) rectangle (365.88, 42.22);

\path[fill=fillColor,fill opacity=0.60] (365.88, 42.22) rectangle (368.33, 42.22);

\path[fill=fillColor,fill opacity=0.60] (368.33, 42.22) rectangle (370.79, 42.22);

\path[fill=fillColor,fill opacity=0.60] (370.79, 42.22) rectangle (373.24, 42.22);

\path[fill=fillColor,fill opacity=0.60] (373.24, 42.22) rectangle (375.70, 42.22);

\path[fill=fillColor,fill opacity=0.60] (375.70, 42.22) rectangle (378.15, 42.22);

\path[fill=fillColor,fill opacity=0.60] (378.15, 42.22) rectangle (380.61, 42.22);

\path[fill=fillColor,fill opacity=0.60] (380.61, 42.22) rectangle (383.07, 42.22);

\path[fill=fillColor,fill opacity=0.60] (383.07, 42.22) rectangle (385.52, 42.22);

\path[fill=fillColor,fill opacity=0.60] (385.52, 42.22) rectangle (387.98, 42.22);

\path[fill=fillColor,fill opacity=0.60] (387.98, 42.22) rectangle (390.43, 42.22);

\path[fill=fillColor,fill opacity=0.60] (390.43, 42.22) rectangle (392.89, 42.22);

\path[fill=fillColor,fill opacity=0.60] (392.89, 42.22) rectangle (395.34, 42.22);

\path[fill=fillColor,fill opacity=0.60] (395.34, 42.22) rectangle (397.80, 42.22);

\path[fill=fillColor,fill opacity=0.60] (397.80, 42.22) rectangle (400.25, 42.22);

\path[fill=fillColor,fill opacity=0.60] (400.25, 42.22) rectangle (402.71, 42.22);

\path[fill=fillColor,fill opacity=0.60] (402.71, 42.22) rectangle (405.16, 42.22);

\path[fill=fillColor,fill opacity=0.60] (405.16, 42.22) rectangle (407.62, 42.22);

\path[fill=fillColor,fill opacity=0.60] (407.62, 42.22) rectangle (410.07, 42.22);

\path[fill=fillColor,fill opacity=0.60] (410.07, 42.22) rectangle (412.53, 42.22);

\path[fill=fillColor,fill opacity=0.60] (412.53, 42.22) rectangle (414.98, 42.22);

\path[fill=fillColor,fill opacity=0.60] (414.98, 42.22) rectangle (417.44, 42.22);

\path[fill=fillColor,fill opacity=0.60] (417.44, 42.22) rectangle (419.89, 42.22);

\path[fill=fillColor,fill opacity=0.60] (419.89, 42.22) rectangle (422.35, 42.22);

\path[fill=fillColor,fill opacity=0.60] (422.35, 42.22) rectangle (424.81, 42.22);

\path[fill=fillColor,fill opacity=0.60] (424.81, 42.22) rectangle (427.26, 42.22);

\path[fill=fillColor,fill opacity=0.60] (427.26, 42.22) rectangle (429.72, 42.22);

\path[fill=fillColor,fill opacity=0.60] (429.72, 42.22) rectangle (432.17, 42.22);

\path[fill=fillColor,fill opacity=0.60] (432.17, 42.22) rectangle (434.63, 42.22);

\path[fill=fillColor,fill opacity=0.60] (434.63, 42.22) rectangle (437.08, 42.22);

\path[fill=fillColor,fill opacity=0.60] (437.08, 42.22) rectangle (439.54, 42.22);

\path[fill=fillColor,fill opacity=0.60] (439.54, 42.22) rectangle (441.99, 42.22);

\path[fill=fillColor,fill opacity=0.60] (441.99, 42.22) rectangle (444.45, 42.22);

\path[fill=fillColor,fill opacity=0.60] (444.45, 42.22) rectangle (446.90, 42.22);

\path[fill=fillColor,fill opacity=0.60] (446.90, 42.22) rectangle (449.36, 42.25);

\path[fill=fillColor,fill opacity=0.60] (449.36, 42.22) rectangle (451.81, 42.33);

\path[fill=fillColor,fill opacity=0.60] (451.81, 42.22) rectangle (454.27, 42.63);

\path[fill=fillColor,fill opacity=0.60] (454.27, 42.22) rectangle (456.72, 43.33);

\path[fill=fillColor,fill opacity=0.60] (456.72, 42.22) rectangle (459.18, 45.10);

\path[fill=fillColor,fill opacity=0.60] (459.18, 42.22) rectangle (461.64, 49.04);

\path[fill=fillColor,fill opacity=0.60] (461.64, 42.22) rectangle (464.09, 56.12);

\path[fill=fillColor,fill opacity=0.60] (464.09, 42.22) rectangle (466.55, 69.70);

\path[fill=fillColor,fill opacity=0.60] (466.55, 42.22) rectangle (469.00, 90.12);

\path[fill=fillColor,fill opacity=0.60] (469.00, 42.22) rectangle (471.46,119.40);

\path[fill=fillColor,fill opacity=0.60] (471.46, 42.22) rectangle (473.91,155.31);

\path[fill=fillColor,fill opacity=0.60] (473.91, 42.22) rectangle (476.37,190.11);

\path[fill=fillColor,fill opacity=0.60] (476.37, 42.22) rectangle (478.82,219.43);

\path[fill=fillColor,fill opacity=0.60] (478.82, 42.22) rectangle (481.28,234.62);

\path[fill=fillColor,fill opacity=0.60] (481.28, 42.22) rectangle (483.73,230.82);

\path[fill=fillColor,fill opacity=0.60] (483.73, 42.22) rectangle (486.19,208.99);

\path[fill=fillColor,fill opacity=0.60] (486.19, 42.22) rectangle (488.64,176.16);

\path[fill=fillColor,fill opacity=0.60] (488.64, 42.22) rectangle (491.10,139.47);

\path[fill=fillColor,fill opacity=0.60] (491.10, 42.22) rectangle (493.55,105.46);

\path[fill=fillColor,fill opacity=0.60] (493.55, 42.22) rectangle (496.01, 79.91);

\path[fill=fillColor,fill opacity=0.60] (496.01, 42.22) rectangle (498.46, 62.03);

\path[fill=fillColor,fill opacity=0.60] (498.46, 42.22) rectangle (500.92, 52.10);

\path[fill=fillColor,fill opacity=0.60] (500.92, 42.22) rectangle (503.38, 46.38);

\path[fill=fillColor,fill opacity=0.60] (503.38, 42.22) rectangle (505.83, 43.90);

\path[fill=fillColor,fill opacity=0.60] (505.83, 42.22) rectangle (508.29, 42.85);

\path[fill=fillColor,fill opacity=0.60] (508.29, 42.22) rectangle (510.74, 42.44);

\path[fill=fillColor,fill opacity=0.60] (510.74, 42.22) rectangle (513.20, 42.27);

\path[fill=fillColor,fill opacity=0.60] (513.20, 42.22) rectangle (515.65, 42.23);

\path[fill=fillColor,fill opacity=0.60] (515.65, 42.22) rectangle (518.11, 42.22);

\path[fill=fillColor,fill opacity=0.60] (518.11, 42.22) rectangle (520.56, 42.22);

\path[fill=fillColor,fill opacity=0.60] (520.56, 42.22) rectangle (523.02, 42.22);

\path[fill=fillColor,fill opacity=0.60] (523.02, 42.22) rectangle (525.47, 42.22);

\path[fill=fillColor,fill opacity=0.60] (525.47, 42.22) rectangle (527.93, 42.22);

\path[fill=fillColor,fill opacity=0.60] (527.93, 42.22) rectangle (530.38, 42.22);

\path[fill=fillColor,fill opacity=0.60] (530.38, 42.22) rectangle (532.84, 42.22);

\path[fill=fillColor,fill opacity=0.60] (532.84, 42.22) rectangle (535.29, 42.22);

\path[fill=fillColor,fill opacity=0.60] (535.29, 42.22) rectangle (537.75, 42.22);

\path[fill=fillColor,fill opacity=0.60] (537.75, 42.22) rectangle (540.20, 42.22);

\path[fill=fillColor,fill opacity=0.60] (540.20, 42.22) rectangle (542.66, 42.22);

\path[fill=fillColor,fill opacity=0.60] (542.66, 42.22) rectangle (545.12, 42.22);

\path[fill=fillColor,fill opacity=0.60] (545.12, 42.22) rectangle (547.57, 42.22);
\end{scope}
\begin{scope}
\path[clip] (  0.00,  0.00) rectangle (578.16,289.08);
\definecolor{drawColor}{RGB}{0,0,0}

\path[draw=drawColor,line width= 0.6pt,line join=round] ( 38.36, 30.72) --
	( 38.36,283.58);
\end{scope}
\begin{scope}
\path[clip] (  0.00,  0.00) rectangle (578.16,289.08);
\definecolor{drawColor}{gray}{0.30}

\node[text=drawColor,anchor=base east,inner sep=0pt, outer sep=0pt, scale=  0.88] at ( 33.41, 39.19) {0.00};

\node[text=drawColor,anchor=base east,inner sep=0pt, outer sep=0pt, scale=  0.88] at ( 33.41,115.81) {0.05};

\node[text=drawColor,anchor=base east,inner sep=0pt, outer sep=0pt, scale=  0.88] at ( 33.41,192.43) {0.10};

\node[text=drawColor,anchor=base east,inner sep=0pt, outer sep=0pt, scale=  0.88] at ( 33.41,269.06) {0.15};
\end{scope}
\begin{scope}
\path[clip] (  0.00,  0.00) rectangle (578.16,289.08);
\definecolor{drawColor}{gray}{0.20}

\path[draw=drawColor,line width= 0.6pt,line join=round] ( 35.61, 42.22) --
	( 38.36, 42.22);

\path[draw=drawColor,line width= 0.6pt,line join=round] ( 35.61,118.84) --
	( 38.36,118.84);

\path[draw=drawColor,line width= 0.6pt,line join=round] ( 35.61,195.46) --
	( 38.36,195.46);

\path[draw=drawColor,line width= 0.6pt,line join=round] ( 35.61,272.09) --
	( 38.36,272.09);
\end{scope}
\begin{scope}
\path[clip] (  0.00,  0.00) rectangle (578.16,289.08);
\definecolor{drawColor}{RGB}{0,0,0}

\path[draw=drawColor,line width= 0.6pt,line join=round] ( 38.36, 30.72) --
	(572.66, 30.72);
\end{scope}
\begin{scope}
\path[clip] (  0.00,  0.00) rectangle (578.16,289.08);
\definecolor{drawColor}{gray}{0.20}

\path[draw=drawColor,line width= 0.6pt,line join=round] ( 62.65, 27.97) --
	( 62.65, 30.72);

\path[draw=drawColor,line width= 0.6pt,line join=round] (159.79, 27.97) --
	(159.79, 30.72);

\path[draw=drawColor,line width= 0.6pt,line join=round] (256.94, 27.97) --
	(256.94, 30.72);

\path[draw=drawColor,line width= 0.6pt,line join=round] (354.08, 27.97) --
	(354.08, 30.72);

\path[draw=drawColor,line width= 0.6pt,line join=round] (451.23, 27.97) --
	(451.23, 30.72);

\path[draw=drawColor,line width= 0.6pt,line join=round] (548.37, 27.97) --
	(548.37, 30.72);
\end{scope}
\begin{scope}
\path[clip] (  0.00,  0.00) rectangle (578.16,289.08);
\definecolor{drawColor}{gray}{0.30}

\node[text=drawColor,anchor=base,inner sep=0pt, outer sep=0pt, scale=  0.88] at ( 62.65, 19.71) {0};

\node[text=drawColor,anchor=base,inner sep=0pt, outer sep=0pt, scale=  0.88] at (159.79, 19.71) {3};

\node[text=drawColor,anchor=base,inner sep=0pt, outer sep=0pt, scale=  0.88] at (256.94, 19.71) {6};

\node[text=drawColor,anchor=base,inner sep=0pt, outer sep=0pt, scale=  0.88] at (354.08, 19.71) {9};

\node[text=drawColor,anchor=base,inner sep=0pt, outer sep=0pt, scale=  0.88] at (451.23, 19.71) {12};

\node[text=drawColor,anchor=base,inner sep=0pt, outer sep=0pt, scale=  0.88] at (548.37, 19.71) {15};
\end{scope}
\begin{scope}
\path[clip] (  0.00,  0.00) rectangle (578.16,289.08);
\definecolor{drawColor}{RGB}{0,0,0}

\node[text=drawColor,anchor=base,inner sep=0pt, outer sep=0pt, scale=  1.10] at (305.51,  7.44) {Euclidian distance};
\end{scope}
\begin{scope}
\path[clip] (  0.00,  0.00) rectangle (578.16,289.08);
\definecolor{drawColor}{RGB}{0,0,0}

\node[text=drawColor,rotate= 90.00,anchor=base,inner sep=0pt, outer sep=0pt, scale=  1.10] at ( 13.08,157.15) {Frequency};
\end{scope}
\end{tikzpicture}
}
\caption{Distribution of the euclidean distance between two random points of $[0,1]^d$ w.r.t.\ the dimension of the space $d \in \{ 2, 5, 10, 20, 50, 100, 1000 \}$.}
\label{fig:distance}
\end{figure}

%Coming back to the supervised classification setting, .

\subsection{The blessings of dimensionality}

The relative ``emptiness'' of the feature space $\gls{Xspace}$ in high dimension is not solely a curse: recall that we motivated the use of \gls{lr}, quantization and \gls{lr} trees for their interpretability. We would like simple, linear boundaries between good and bad borrowers. As should now be apparent from the few Gini figures given in this manuscript, such boundaries do not exist in \textit{Credit Scoring} in small dimension which motivates the use of additional data. The higher the dimension $d$, the higher the likelihood of existence of a linear boundary between good and bad borrowers~\cite{gorban2018blessing}. However, if some feature are just ``noise'' w.r.t.\ the class or contain redundant information, this linear boundary is highly likely to not generalize well, \textit{i.e.}\ overfit~\cite{bertrandagnan_2017}. To avoid this pitfall but still benefit from the vast amount of available data, a simple solution is to reduce the dimensionality $d$ of the data to only ``useful'' dimensions (in the sense of the predictive task).

\subsection{Dimension reduction}

A straightforward way of avoiding the curse(s) of dimensionality for \gls{lr} is to get back to a small dimension $d'$ relative to $n$ by pre-processing the $d$ features. In Chapter~\ref{chap4}, and particularly Section~\ref{sec:motivation}, it was argued that quantization could be thought of as a dimensionality reduction technique, because information was compressed in intervals and ``meta''-groups for continuous and categorical features respectively without affecting predictive power (on the contrary!). Two way more classical ways of performing dimensionality reduction are presented here: combining original features in principal components, which was already discussed in Chapter~\ref{chap6} when building segments of clients, and feature selection, which are the subjects of the two subsequent sections respectively.

\subsubsection{By combining input features}

Various algorithms were designed to map features onto a new ``representation'' which can have interesting properties. In Chapter~\ref{chap6}, we discussed at great length of \gls{famd} which consists in constructing orthogonal principal components, ranked by their respective eigen value which corresponds to the proportion of the total explained variance. In the same fashion as \textit{Credit Scoring} practitioners only care about the first two axes when looking for potential segments (see Chapter~\ref{chap6}), one could discard many dimensions of its original data by considering only the principal components that explain at least a given proportion of the total variance.

However, as was sensed in Chapter~\ref{chap6} for segmentation, the goal remains to predict a class, not to account for most of the variance of the predictors! These two goals being possibly disjoint, we introduced the \gls{pls} and \gls{spc} algorithms, which aim to take into account the target feature. Moreover, the \gls{spc} algorithm is an iterative procedure to select only principal components that have predictive power.


\subsubsection{By selecting input features}

Practitioners are often not convinced by the preceding approach since the new representation of the data is hard to grasp. Subsequently, feature selection approaches, which remain in the canonical space, are usually preferred. Such algorithms are out of the scope of the present manuscript, although the LASSO was mentioned earlier. This penalization method performs feature selection as a side effect and one of its refinement, the adaptive LASSO~\cite{zou2006adaptive}, has strong oracle properties (\textit{e.g.}\ it finds the subset of the features that participate in the true model, if it exists) which are appealing to both academics and industrial practitioners.

\section{New data types in a supervised classification setting}

In essence, it is not solely the volume of data that has to be addressed, but the variety of its format, as is apparent from the motivational section. These unstructured data require specific modelling techniques, ideally to automatically extract their predictive information into inputs that can be processed by simple interpretable models like \gls{lr}. Internally at \gls{cacf}, some simple features are extracted, typically from the credit card transactions, and seems to be the case in other financial institutions~\cite{verkade_2018}. In this applied work, the author compares the ``traditional'' approach to a model exploiting only similarities between clients' credit card transactions, achieving a good overall performance, in particular in conjunction with traditional data. This is not the case at \gls{cacf} where attempts to use only web logs showed poor performance. This empirical study motivated \gls{cacf} to study ways to structure these data and extract only the most important predictive information.

Is this structuring work a result of some statistical procedure or shall it remain a manual feature engineering work done by field experts? This question has found a clear answer in favour of automatic statistical procedures in fields like Computer Vision and Speech Recognition where neural networks, which are motivated by their automatic feature engineering, have made tremendous improvements over previous approaches relying on manual feature engineering~\cite{goodfellow2016deep}. These models are not directly applicable to \textit{Credit Scoring} since we require interpretability through simple models like \gls{lr}. Applying techniques from metric learning~\cite{2015BelletHS} and functional principal components~\cite{functional} are future areas of research, since credit card transactions can be reduced to a similarity measure as in~\cite{verkade_2018} and web visitation patterns can be seen as functional categorical data (where categories are web pages) that can be summarized by (functional) principal components.


\selectlanguage{french}

\section{Conclusion générale}

Cette thèse a permis d'explorer cinq sujets directement inspirés de problématiques industrielles de \textit{Credit Scoring}, sans doute graduellement du plus opérationnel, dont le questionnement était parfaitement posé, la ``réintégration des refusés'', au plus ouvert, l'utilisation de données non structurées en grande dimension, pour laquelle il ne semble pas y avoir d'approche universelle existante. On passe rapidement en revue ces problèmes en donnant les idées clés du problème, de sa résolution et des contributions de cette thèse.

\medskip

Le chapitre~\ref{chap2} consacré à la ``réintégration des refusés'' a permis de poser un problème ancien de l'industrie du crédit à la consommation : l'ensemble d'apprentissage de la règle de classement bons / mauvais payeurs est un échantillon de la population ayant déjà été financée. Ce financement est fortement corrélé à plusieurs règles existantes destinées à ne financer que des clients supposés bons. Cela induit-il un biais dans l'estimation des modèles de classification supervisée, en particulier la régression logistique ? En réinterprétant la classe des clients non financés comme des données manquantes, et en distinguant les cas du vrai (\textit{well-specified}) et du mauvais (\textit{misspecified}) modèle, on a montré que le paramètre de la régression logistique peut en effet être biaisé. Néanmoins, en reformulant les techniques \textit{ad hoc} d'utilisation des informations des clients non financés comme des tentatives de modélisation du mécanisme de financement, on a montré que la méthode actuelle consistant à n'utiliser que les clients financés pour lesquels $Z = \text{f}$ était satisfaisante.

\medskip

Rassuré sur la pertinence de l'échantillon d'apprentissage, le praticien poursuit ses travaux par certains pré-traitements, qui ont une justification pratique mais aussi théorique : apprendre une ``meilleure'' représentation des données au sens de l'interprétation du modèle mais aussi de sa qualité prédictive. La quantification (\textit{quantization}) regroupe la discrétisation de prédicteurs continus (la transformation d'un âge en une tranche d'âge par exemple) et le regroupement de modalités de prédicteurs catégoriels (le regroupement de modèles de véhicule en segments comme les citadines, routières, etc.). Ce pré-traitement manuel est à faible valeur ajoutée pour le statisticien et lui prend un temps considérable, qui tend à augmenter (du fait de l'augmentation du nombre de prédicteurs) ; de plus, en reposant sur des méthodes \textit{ad hoc} et univariées, la qualité prédictive du modèle résultant est diminuée. Il s'agissait alors de formaliser ce problème et de proposer une automatisation qui faisait néanmoins sens du point de vue statistique. Une nouvelle méthode, \textit{glmdisc}, est proposée, ainsi que deux stratégies d'estimation différentes, dont les résultats sur données réelles sont meilleurs que les approches \textit{ad hoc} susmentionnées et les méthodes d'état de l'art.

\medskip

De manière similaire, pour des raisons pratiques et théoriques, il est courant d'étudier des croisements (\textit{pairwise interactions}) de variables~: on suppose que l'effet combiné de deux prédicteurs sur le risque du client est différent de la somme des effets de ces prédicteurs. Encore une fois, des techniques \textit{ad hoc}, sous-optimales, étaient employées, nécessitant des données préalablement quantifiées. Or, la quantification et l'introduction d'interactions, en agissant sur l'espace des modèles considérés, doivent être effectuées simultanément. Une approche de type MCMC, utilisant l'algorithme de Metropolis-Hastings, a été proposée pour l'introduction d'interactions et dont l'intérêt principal est l'utilisation aisée en combinaison de l'algorithme \textit{glmdisc} construit pour le problème de quantification. Il est alors possible d'obtenir une régression logistique performante et interprétable en quelques heures de temps machine, ce qui nécessitait un à deux mois de temps humain.

\medskip

Nous avions ensuite pris du recul sur le quotidien du praticien en \textit{Credit Scoring} qui se voit confier des scores et / ou des améliorations ``locales'' du système d'acceptation, c'est-à-dire ne concernant qu'une partie de la population totale des demandeurs de crédit. En effet, ledit système est bien souvent composé de nombreuses règles ``métier'' (écartées de cette étude) mais surtout de nombreux scores, c'est-à-dire des régressions logistiques utilisant des variables différentes, des quantifications et croisements différents, et utilisées sur des clientèles différentes. En ne remettant jamais en cause la structure d'arbre du système d'acceptation total, la qualité prédictive est nécessairement sous-optimale, ce qui nous a conduit à présenter les méthodes actuelles utilisées en industrie pour construire des segments sur lesquelles différentes régressions logistiques sont ensuite construites. Plusieurs méthodes alternatives de l'état de l'art, très simples à mettre en oeuvre et qui produisent de meilleurs résultats que l'approche actuelle sur des données simulées, ainsi qu'une piste de résolution, sous la forme d'un algorithme~\gls{sem} comparable à celui exploité dans \textit{glmdisc} ont été passées en revue et comparées sur des données simulées et réelles et montrent de bons résultats préliminaires.
%L'application de cette piste de recherche sur les données réelles de \gls{cacf} est un premier futur axe de travail.

\medskip

Enfin, tous ces travaux exploitaient des données dites ``classiques'' en \textit{Credit Scoring}, c'est-à-dire majoritairement issues de formulaires remplis par le client ou par le vendeur (en magasin). \gls{cacf} dispose par ailleurs d'autres données, dont l'intérêt, la capacité prédictive additionnelle en tête de liste, reste à démontrer, comme par exemple les données transactionnelles de cartes de crédit, les données de log de connexion au site internet, les données marketing, etc. La présente conclusion a été l'occasion de voir les problèmes classiques liés à l'augmentation de dimension~: des espaces vides où la notion de ``voisin'' ne fait pas toujours sens, mais où il est plus facile de trouver de simples hyperplans séparant les classes bons et mauvais payeurs, pourvu que toutes ces nouvelles dimensions apportent de l'information. On s'est ensuite attardé sur les données non structurées à la disposition des banques dont la granularité fine (des centaines de transactions de carte bancaire pour un seul client) conduit le praticien, une fois de plus, à s'engouffrer dans des techniques \textit{ad hoc}, manuelles et chronophages d'agrégation, sans garantie statistique. La littérature relative à ces nouvelles données a été synthétisée de manière à favoriser la diffusion de ces bonnes pratiques~; l'application de ces techniques sur les données réelles de \gls{cacf} est un futur axe de travail.

\medskip

Pour conclure, cette thèse a permis d'apporter des réponses théoriques à des problèmes récurrents connexes au \textit{Credit Scoring} et nécessitant un tel travail de formalisation. Elle a également permis, s'agissant d'une thèse CIFRE, d'apporter une solution pratique aux problèmes de quantification et de croisements de variables sous la forme de deux solutions logicielles. Le chapitre~\ref{chap1} a permis d'introduire plusieurs problèmes ouverts liés au \textit{Credit Scoring}, parmi lesquels la ``segmentation'' et l'utilisation de données massives non structurées. Ces deux sujets ont fait l'objet des derniers travaux et ont abouti à une bibliographie épurée ainsi qu'à des simulations donnant une base solide à de futurs travaux. Les perspectives de travaux de recherche applicables au \textit{Credit Scoring} ne se tariront pas de si tôt, tant le contexte de disponibilité de nombreuses sources de données et les enjeux économiques importants sont catalyseurs des besoins de traitements statistiques rigoureux.

\printbibliography[heading=subbibliography, title=Références de la conclusion]
%
% Liste des références bibliographiques
\printbibliography
%
%%%%%%%%%%%%%%%%%%%%%%%%%%%%%%%%%%%%%%%%%%%%%%%%%%%%%%%%%%%%%%%%%%%%%%%%%%%%%%%
% Début de la partie annexe éventuelle
%%%%%%%%%%%%%%%%%%%%%%%%%%%%%%%%%%%%%%%%%%%%%%%%%%%%%%%%%%%%%%%%%%%%%%%%%%%%%%%
\appendix
%
% Premier chapitre annexe (éventuel)
\chapter{Algorithms}

\selectlanguage{english}

\section{Reject Inference methods}

\subsection{Fuzzy Augmentation} \label{fuzzy}

Fuzzy Augmentation can be found in \cite{economix}; it is the following procedure:
\begin{enumerate}
\item Construct Scorecard "Known Good Bad" (KGB) $S^{\text{f}}$ with financed clients' data (Figure \ref{fuzzy:sfig1})
\item Calculate $p(1|x,\hat{\theta}^{\text{f}}) = \text{logit}(S^{\text{f}}(x))$ for rejects (Figure \ref{fuzzy:sfig2})
\item Infer rejected client $i$ as good with weight $p(1|x,\hat{\theta}^{\text{f}})$ and as bad with weight {\begin{sloppypar} $1-p(1|x,\hat{\theta}^{\text{f}})$ (Figures \ref{fuzzy:sfig2} and \ref{fuzzy:sfig3}) \end{sloppypar} }
\item Calibrate a new scorecard with the ‘‘augmented'' dataset (Figure \ref{fuzzy:sfig4})
\end{enumerate}

\begin{figure}
{\setlength{\parindent}{0cm}
\begin{multicols}{4}
\small

\begin{subfigure}[t]{0.22\textwidth}
\begin{center}
\begin{adjustbox}{max width=0.95\textwidth}
\begin{tabular}{l l}
\toprule
\textbf{${\bm{y}}^{\text{f}}$} & \textbf{${\bm{x}}^{\text{f}}$}\\
\midrule
1 & 0.562 \\
1 & 0.910 \\
0 & 0.430 \\
\bottomrule
\end{tabular}
\end{adjustbox}
\end{center}

\subcaption{Scorecard $S^{\text{f}}$ on financed loans}
\label{fuzzy:sfig1}
\end{subfigure}

\columnbreak

\begin{subfigure}[t]{0.22\textwidth}
\begin{center}
\begin{adjustbox}{max width=0.95\textwidth}
\begin{tabular}{l l l}
\toprule
\textbf{Weight} & \textbf{$\hat{\bm{y}}^{\text{nf}}$} & \textbf{${\bm{x}}^{\text{nf}}$}\\
\midrule
0.68 & 1 & 0.347 \\
0.10 & 1 & 0.140 \\
0.35 & 1 & 0.295 \\
\bottomrule
\end{tabular}
\end{adjustbox}
\end{center}

\caption{Inferred good not financed loans and their weights}
\label{fuzzy:sfig2}
\end{subfigure}

\columnbreak

\begin{subfigure}[t]{0.22\textwidth}
\begin{center}
\begin{adjustbox}{max width=0.95\textwidth}
\begin{tabular}{l l l}
\toprule
\textbf{Weight} & \textbf{$\hat{\bm{y}}^{\text{nf}}$} & \textbf{${\bm{x}}^{\text{nf}}$}\\
\midrule
0.32 & 0 & 0.347 \\
0.90 & 0 & 0.140 \\
0.65 & 0 & 0.295 \\
\bottomrule
\end{tabular}
\end{adjustbox}
\end{center}

\caption{Inferred bad not financed loans and their weights}
\label{fuzzy:sfig3}
\end{subfigure}

\columnbreak

\begin{subfigure}[t]{0.22\textwidth}
\begin{center}
\begin{adjustbox}{max width=0.95\textwidth}
\begin{tabular}{l l l}
\toprule
\textbf{Weight} & \textbf{${\bm{y}}$} & \textbf{${\bm{x}}$}\\
\midrule
1 & 0 & 0.562 \\
1 & 1 & 0.910 \\
1 & 0 & 0.430 \\
0.68 & 1 & 0.347 \\
0.10 & 1 & 0.140 \\
0.35 & 1 & 0.295 \\
0.32 & 0 & 0.347 \\
0.90 & 0 & 0.140 \\
0.65 & 0 & 0.295 \\
\bottomrule
\end{tabular}
\end{adjustbox}
\end{center}
\caption{Fuzzy augmented learning dataset}
\label{fuzzy:sfig4}
\end{subfigure}

\end{multicols}
}
\caption{Example of implementation of the Fuzzy Augmentation method on a small dataset}
\label{fuzzyexample}
\end{figure}

 Clearly:

 \[ \forall j = 1, \ldots, d, \: \frac{\partial \sum_{i=n+1}^{m+n} \sum_{y_i = 0}^{1} p(y_i^{\text{inf}}| x_i, \hat{\theta}^{\text{f}})\ln (p(y_i| x_i, \theta))}{\partial \theta_j} = 0 \Leftrightarrow \theta = \hat{\theta}^{\text{f}} \]

 It can be shown that:
 \[\argmax_{\theta \in \Theta}  \sum_{i=n+1}^{m+n} \sum_{y_i = 0}^{1} p_{\hat{\theta}^{\text{f}}}(y_i| x_i)\ln (p_\theta(y_i| x_i)) = \hat{\theta}^{\text{f}}\]

 And finally:
 \[\argmax_{\theta \in \Theta} \ell(\theta;{\bm{x}},{\bm{y}}^{\text{f}},{\bm{y}}^{\text{inf}}) = \argmax_{\theta \in \Theta} \ell(\theta;{\bm{x}},{\bm{y}}^{\text{f}}) = \hat{\theta}^{\text{f}}\]

 To conclude, this method will not change the estimated parameters of any discriminant model, asymptotically and with a finite set of observations, regardless of any assumption on the missingness mechanism or the true model hypothesis. In other words, Fuzzy Augmentation has no effect on the Kullback-Leibler divergence, making this method useless because it is no different than the accepted clients model minimizing asymptotically expression~$H^{\text{f}}_{\theta}$ and not expression~$H_{\theta}$.


\subsection{Reclassification} \label{reclassification}

Reclassification can be found in \cite{RI6}, also sometimes referred to as extrapolation as in \cite{banasik}; it is the following procedure:
\begin{enumerate}
\item Construct Scorecard "Known Good Bad" (KGB) $S^{\text{f}}$ with financed clients' data (Figure~\ref{reclass:sfig1})
\item Calculate $p(1|x,\hat{\theta}^{\text{f}}) = \text{logit}(S^{\text{f}}(x))$ for rejects
\item Infer default status of rejected client $i$ if $S^{\text{f}}(x) > \text{threshold}$; typically threshold $=0.5$ (Figure~\ref{reclass:sfig2})
\item Calibrate a new scorecard with the ‘‘augmented'' dataset (Figure~\ref{reclass:sfig3})
\end{enumerate}

\begin{figure}
{\setlength{\parindent}{0cm}
\begin{multicols}{3}

\begin{subfigure}[t]{0.31\textwidth}
\begin{center}
\begin{adjustbox}{max width=\textwidth}
\begin{tabular}{l l}
\toprule
\textbf{${\bm{y}}^{\text{f}}$} & \textbf{${\bm{x}}^{\text{f}}$}\\
\midrule
1 & 0.562 \\
1 & 0.910 \\
0 & 0.430 \\
\bottomrule
\end{tabular}
\end{adjustbox}
\end{center}

\caption{Development of scorecard $S^{\text{f}}$ on financed clients}
\label{reclass:sfig1}
\end{subfigure}


\columnbreak

\begin{subfigure}[t]{0.31\textwidth}
\begin{center}
\begin{adjustbox}{max width=\textwidth}
\begin{tabular}{l l l}
\toprule
\textbf{$p(1|x,\hat{\theta}^{\text{f}})$} & \textbf{$\hat{\bm{y}}^{\text{nf}}$} & \textbf{${\bm{x}}^{\text{nf}}$}\\
\midrule
0.68 & 1 & 0.347 \\
0.10 & 0 & 0.140 \\
0.35 & 0 & 0.295 \\
\bottomrule
\end{tabular}
\end{adjustbox}
\end{center}

\caption{We force $y^{\text{nf}}=1$ if $\text{logit}(S^{\text{f}}(x)) \geq 0.5$}
\label{reclass:sfig2}
\end{subfigure}

\columnbreak

\begin{subfigure}[t]{0.31\textwidth}
\begin{center}
\begin{adjustbox}{max width=\textwidth}
\begin{tabular}{l l}
\toprule
\textbf{${\bm{y}}$} & \textbf{${\bm{x}}$}\\
\midrule
0 & 0.562 \\
1 & 0.910 \\
0 & 0.430 \\
1 & 0.347 \\
0 & 0.140 \\
0 & 0.295 \\
\bottomrule
\end{tabular}
\end{adjustbox}
\end{center}
\caption{Reclassified learning dataset}
\label{reclass:sfig3}
\end{subfigure}

\end{multicols}
}
\caption{Example of implementation of the Reclassification method on a small dataset}
\label{reclassexample}
\end{figure}


\subsection{Augmentation} \label{augmentation}

\begin{enumerate}
\item Construct Scorecard "Accept Reject" (ACRJ) $R$ with financed clients' data on target variable $Z$ (Figure~\ref{augment:sfig1})
\item Create $K$ score bands $B_1, \ldots, B_K$ according to $R$
\item Compute in each score band $\hat{p}_{\text{true}}(\text{f}|x \in B_k) = \dfrac{|B_k|}{|z=\text{f}|}$ (Figure~\ref{augment:sfig2})
\item Construct Scorecard $S$ on target variable Good/Bad with financed clients' data re-weighted (Figure~\ref{augment:sfig3})
\end{enumerate}

\begin{figure}
{\setlength{\parindent}{0cm}
\begin{multicols}{3}

\begin{subfigure}[t]{0.31\textwidth}
\begin{center}
\begin{adjustbox}{max width=0.95\textwidth}

\begin{tabular}{l l l}
\toprule
\textbf{${\bm{y}}$} & \textbf{${\bf{z}}$} & \textbf{Score-band}\\
\midrule
1 & \text{f} & 1 \\
1 & \text{f} & 1 \\
0 & \text{f} & 1 \\
NA & \text{nf} & 1 \\
NA & \text{nf} & 1 \\
NA & \text{nf} & 1 \\
... & ... & ... \\
\bottomrule
\end{tabular}
\end{adjustbox}
\end{center}

\caption{Calculation of $K$ score-bands on the ACRJ score}
\label{augment:sfig1}
\end{subfigure}

\columnbreak

\begin{subfigure}[t]{0.31\textwidth}
\begin{center}
\begin{adjustbox}{max width=0.95\textwidth}

\begin{tabular}{l l l}
\toprule
\textbf{Score-band} & \textbf{Weight}\\
\midrule
1 & 2 \\
... & ... \\
K & 1.1 \\
\bottomrule
\end{tabular}
\end{adjustbox}
\end{center}

\caption{Aggregate the data to estimate the inverse of the probability of being accepted in each score band}
\label{augment:sfig2}
\end{subfigure}

\columnbreak

\begin{subfigure}[t]{0.31\textwidth}
\begin{center}
\begin{adjustbox}{max width=0.95\textwidth}

\begin{tabular}{l l l l}
\toprule
\textbf{Weight} & \textbf{Score-band} & \textbf{${\bm{y}}$} & \textbf{${\bm{x}}$}\\
\midrule
2 & 1 & 1 & 0.123 \\
2 & 1 & 0 & 0.432 \\
2 & 1 & 1 & 0.562 \\
... & ... & ... & ... \\
1.1 & K & 0 & 0.962 \\
1.1 & K & 0 & 0.812 \\
\bottomrule
\end{tabular}
\end{adjustbox}
\end{center}

\caption{Merge weights and data on financed clients to construct the new scorecard}
\label{augment:sfig3}
\end{subfigure}
\end{multicols}
}
\caption{Example of implementation of the Augmentation method on a small dataset}
\label{augmentexample}
\end{figure}

\subsection{Twins} \label{Twins}

The Twins Method is an internal method at Crédit Agricole documented by Crédit \cite{groupe} (confidential); it consists in the following procedure:
\begin{enumerate}
\item Develop KGB (Known Good/Bad) scorecard $S^{\text{f}}$ on financed clients' data predicting $Y$; this gives us $\hat{\theta}^{\text{f}}$ (Figure~\ref{twins:sfig1})
\item Develop ACRJ (Accept/Reject) scorecard $S_Z$ on all applicants predicting $Z$; this gives us $\hat{\zeta}$ (Figure~\ref{twins:sfig2})
\item Develop a scorecard $S$ on financed clients' data predicting $Y$ based solely on $S^{\text{f}}$ and $S_Z$; this gives us $\hat{\theta}^{\text{twins}}$ (Figure~\ref{twins:sfig3})
\item Calculate $S$ on rejected applicants and reintegrate them twice in the training dataset like we did with Fuzzy Augmentation in section \ref{subsec:reweighting} (Figure~\ref{twins:sfig4})
\item Develop scorecard $S_{\text{twins}}$ on all applicants' data
\end{enumerate}

\begin{figure}
{\setlength{\parindent}{0cm}
\begin{multicols}{4}

\begin{subfigure}[t]{0.22\textwidth}
\begin{center}
\begin{adjustbox}{max width=0.95\textwidth}
\begin{tabular}{l l l}
\toprule
\textbf{${\bm{y}}$} & \textbf{${\bm{x}}$}\\
\midrule
1 & 0.562 \\
1 & 0.910 \\
0 & 0.430 \\
NA & 0.361 \\
NA & 0.402 \\
NA & 0.294 \\
\bottomrule
\end{tabular}
\end{adjustbox}
\end{center}

\caption{Development of scorecard $S^{\text{f}}$ on financed clients}
\label{twins:sfig1}
\end{subfigure}

\columnbreak

\begin{subfigure}[t]{0.22\textwidth}
\begin{center}
\begin{adjustbox}{max width=0.95\textwidth}
\begin{tabular}{l l l}
\toprule
\textbf{${\bm{z}}$} &  \textbf{${\bm{x}}$} \\
\midrule
\text{f} & 0.562 \\
\text{f} & 0.910 \\
\text{f} & 0.430 \\
\text{nf} & 0.361 \\
\text{nf} & 0.402 \\
\text{nf} & 0.294 \\
\bottomrule
\end{tabular}
\end{adjustbox}
\end{center}

\caption{Development of scorecard $S_2$ on all clients}
\label{twins:sfig2}
\end{subfigure}

\columnbreak

\begin{subfigure}[t]{0.22\textwidth}
\begin{center}
\begin{adjustbox}{max width=0.95\textwidth}
\begin{tabular}{l l l}
\toprule
\textbf{${\bm{y}}$} & \textbf{$S^{\text{f}}({\bm{x}})$} & \textbf{$S_Z({\bm{x}})$}\\
\midrule
1 & 1.3 & 2.5\\
1 & 3.1 & 4.5 \\
0 & -0.3 & 0.4 \\
NA & -1.2 & -0.5 \\
NA & -0.4 & 0.3 \\
NA & -2.0 & -2.5 \\
\bottomrule
\end{tabular}
\end{adjustbox}
\end{center}

\caption{Development of scorecard $S_3$ on financed clients}
\label{twins:sfig3}
\end{subfigure}

\columnbreak

\begin{subfigure}[t]{0.22\textwidth}
\begin{center}
\begin{adjustbox}{max width=0.95\textwidth}
\begin{tabular}{l l l}
\toprule
\textbf{Weight} & \textbf{$\hat{\bm{y}}^{\text{nf}}$} & \textbf{${\bm{x}}^{\text{nf}}$}\\
\midrule
1 & 1 & 0.562 \\
1 & 1 & 0.910 \\
1 & 0 & 0.430 \\
0.64 & 0 & 0.361 \\
0.73 & 0 & 0.402 \\
0.44 & 0 & 0.294 \\
0.36 & 1 & 0.361 \\
0.27 & 1 & 0.402 \\
0.37 & 1 & 0.294 \\
\bottomrule
\end{tabular}
\end{adjustbox}
\end{center}

\caption{Inference for not financed clients}
\label{twins:sfig4}
\end{subfigure}

\end{multicols}
}
\caption{Example of implementation of the Twins method on a small dataset}
\label{twins}
\end{figure}


\subsection{Parceling} \label{Parceling}

Parcelling is a process of reweighing according to the probability of default by score-band that is adjusted by the credit modeler. It has been documented in \cite{saporta,banasik,RI6}.

\begin{enumerate}
\item Construct Scorecard "Known Good Bad" (KGB) $S^{\text{f}}$ with financed clients' data (Figure~\ref{parcel:sfig1})
\item Create $K$ score bands $B_1, \ldots, B_K$ according to $S^{\text{f}}$.
\item Compute the observed default rate for each band $T(k) = \dfrac{|\text{Bad financed in } B_k|}{|B_k|}$, $1 \leq k  \leq K$.
\item Infer for each band the not financed default rate $U(j) = \epsilon_k T(k)$ where $\epsilon_1 > \ldots > \epsilon_k > \ldots > \epsilon_K > 1$ (Figure~\ref{parcel:sfig2}).
\item Reintegrate 2 times each rejected applicant from $B_k$ with weight $U(k)$ as bad and weight $1-U(k)$ as good, like the Fuzzy Augmentation method in section \ref{subsec:reweighting} (Figure~\ref{parcel:sfig3}).
\item Construct final Scorecard.
\end{enumerate}

\begin{figure}
{\setlength{\parindent}{0cm}
\begin{multicols}{3}

\begin{subfigure}[t]{0.31\textwidth}
\begin{center}
\begin{adjustbox}{max width=\textwidth}
\begin{tabular}{l l l}
\toprule
\textbf{Weight} & \textbf{${\bm{y}}^{\text{f}}$} & \textbf{${\bm{x}}^{\text{f}}$}\\
\midrule
1 & 1 & 0.562 \\
1 & 1 & 0.910 \\
1 & 0 & 0.430 \\
\bottomrule
\end{tabular}
\end{adjustbox}
\end{center}

\caption{Development of scorecard $S^{\text{f}}$ on financed clients}
\label{parcel:sfig1}
\end{subfigure}

\columnbreak

\begin{subfigure}[t]{0.31\textwidth}
\begin{center}
\begin{adjustbox}{max width=\textwidth}
\begin{tabular}{l l l}
\toprule
\textbf{Score-band} & \textbf{$T$} &  \textbf{$U$} \\
\midrule
1 & 0.5 & 0.8 \\
... & ... & ... \\
K & 0.01 & 0.04 \\
\bottomrule
\end{tabular}
\end{adjustbox}
\end{center}

\caption{Calculation of $T(k)$ and $U(k)$}
\label{parcel:sfig2}
\end{subfigure}

\columnbreak

\begin{subfigure}[t]{0.31\textwidth}
\begin{center}
\begin{adjustbox}{max width=\textwidth}
\begin{tabular}{l l l}
\toprule
\textbf{Weight} & \textbf{${\bm{y}}$} & \textbf{${\bm{x}}$}\\
\midrule
1 & 0 & 0.562 \\
1 & 1 & 0.910 \\
1 & 0 & 0.430 \\
1 & 1 & 0.347 \\
1 & 0 & 0.140 \\
1 & 0 & 0.295 \\
\bottomrule
\end{tabular}
\end{adjustbox}
\end{center}

\caption{Inference for not financed clients}
\label{parcel:sfig3}
\end{subfigure}

\end{multicols}
}
\caption{Example of implementation of the Parcelling method on a small dataset}
\label{parcel}
\end{figure}



\section{Discretization methods}

\subsection{Méthodes non supervisées : \textit{equal-width} et \textit{equal-length}}


\begin{algorithm}[H]
 \KwData{$n,\glssymbol{bbx},(m_j)_1^d$}
 \KwResult{$\hat{\q}$}
 \For{$j=1$ to $d$}{
Sort $\glssymbol{bbx}^j$ by ascending order\;
Let $c_0=-\infty$, $c_{m_j} = + \infty$ and $c_{j,h} = x_{\left\lceil{{\frac{h \cdot n}{m_j}}}\right\rceil,j}$\;
Let $C_{j,h} = ]c_{j,h-1};c_{j,h}]$ and $\hat{\q_j(\cdot)} = (\hat{q}_{j,h}(\cdot))_1^{m_j}$\;
Set $\hat{q}_{j,h}(\cdot)=\mathds{1}_{C_{j,h}}(\cdot)$.
%\For{$i=1$ to $n$}{
%Set $q_i^j(x_j) = \begin{cases} 1 \text{ si } x_i^j \leq x_{\left\lceil{{\frac{n}{m_j}}}\right\rceil}^j \\ o \text{ si } x_{\left\lceil{{\frac{(o-1)*n}{m_j}}}\right\rceil}^j < x_i^j \leq x_{\left\lceil{{\frac{o*n}{m_j}}}\right\rceil}^j \\ m_j \text{ si } x_{\left\lceil{{\frac{(m_j-1)*n}{m_j}}}\right\rceil}^j < x_i^j \end{cases}$
%}
}
 \caption{\label{equal-freq-disc} \textit{equal-freq} discretization: an equal number of training observations are in each bin.}
\end{algorithm}


\begin{algorithm}[H]
 \KwData{$n,\glssymbol{bbx},(m_j)_1^d$}
 \KwResult{$\hat{\q}$}
 \For{$j=1$ to $d$}{
Let $w_j = \max{i} x_{i,j} - \min{i} x_{i,j}$\;
Let $c_0=-\infty$, $c_{m_j} = + \infty$ and $c_{j,h} = \frac{w_j \cdot h}{m_j} + \min{i} x_{i,j}$\;
Let $C_{j,h} = ]c_{j,h-1};c_{j,h}]$ and $\hat{\q_j(\cdot)} = (\hat{q}_{j,h}(\cdot))_1^{m_j}$\;
Set $\hat{q}_{j,h}(\cdot)=\mathds{1}_{C_{j,h}}(\cdot)$.
}
 \caption{\label{equal-freq-disc} \textit{equal-length} discretization: each bin has the width of the training set's total support divided by the number of bins.}
\end{algorithm}



\subsection{Méthodes supervisées univariées}

\subsubsection{\textit{ChiMerge}}



\subsubsection{\textit{MDLP}}



\subsection{Proposal: \textit{glmdisc}}


\subsubsection{\textit{glmdisc} with neural networks}

\textcolor{red}{insert animation here}

\begin{algorithm}[H]
 \KwData{$(\bm{x},\bm{y})$}
 \KwResult{$\bm{e},(\beta_k)_1^{d_1}$,(T$^k)_{d_1}^{d_1+d_2}$}
 $r = 0$\;
 Initialization of $\bm{e}^{(r)}$ at random\;
 \While{$r < $ Maximum number of iteration not reached}{
  Adjust logistic regression $\hat{\gamma}^{(r)} = \argmax_\gamma \ell(\gamma;\bm{y},\bm{e^{(r)}})$\;
  ${\bm{e}^{(r+1)}} \leftarrow {\bm{e}^{(r)}}$\;
  \For{$k \leftarrow 1$ \KwTo $d_1$}{
   Adjust multinomial logistic regression $\hat{\beta}^{k(r)} = \argmax_\beta \ell(\beta;\bm{e^{k(r)}},\bm{x^k})$\;
   ${\bm{e}^{k(r+1)}} \leftarrow$ Mult$(p_{\hat{\gamma}^{(r)}}(\bm{y}|\bm{{e}^{(r+1)}}) p_{\hat{\beta}^{k(r)}}(\bm{e}^{k} | \bm{x}^k))$\;
   ${\bm{e}_{MAP}^{k(r+1)}} \leftarrow \argmax_j p_{\hat{\beta}^{k(r)}}(\bm{E}^{k}=j | \bm{x}^k))$\;
   }
  \For{$k \leftarrow d_1+1$ \KwTo $d_1+d_2$}{
   From the contingency table T$^{k(r)}$ of $\bm{e}^{k(r)}$ against $\bm{x}^k$, calculate the frequencies of each value of $({e}^{k},x^k)$\;
	${\bm{e}^{k(r+1)}} \leftarrow$ Mult$(p_{\hat{\gamma}^{(r)}}(\bm{y}|\bm{{e}^{(r+1)}}) p_{T^{k(r)}}(\bm{e}^{k} | \bm{x}^k))$\;
	${\bm{e}_{MAP}^{k(r+1)}} \leftarrow \argmax_j p_{T^{k(r)}}(\bm{E}^{k}=j | \bm{x}^k))$\;
	}
   $r \leftarrow r+1$\;
 }
 Choose the best logistic regression model from $(p_{\hat{\eta}^{(r)}}(\bm{y} | \bm{e}_{MAP}^{(r)}))_1^{Max}$\;
 \caption{\label{NN-disc} \textit{glmdisc}-NN: supervised multivariate discretization for logistic regression with neural networks.}
\end{algorithm}

\begin{animateinline}[poster=first, controls=all, palindrome, autopause, autoresume, width=\textwidth, height=7cm]{3}
\multiframe{200}{i=1+1}{\input{R_CODE_FIGURES/appendix/animation_disc_tensorflow/False_simulated_data/feature_0_iteration_\i.tex}}%
\end{animateinline}


\textcolor{red}{à décommenter // ajouter caption et numéro itération}

\subsubsection{\textit{glmdisc} with an \gls{sem} algorithm}

\begin{algorithm}[H]
 \KwData{$(\bm{x},\bm{y})$}
 \KwResult{$\bm{e},(\beta_k)_1^{d_1}$,(T$^k)_{d_1}^{d_1+d_2}$}
 $r = 0$\;
 Initialization of $\bm{e}^{(r)}$ at random\;
 \While{$r < $ Maximum number of iteration not reached}{
  Adjust logistic regression $\hat{\gamma}^{(r)} = \argmax_\gamma \ell(\gamma;\bm{y},\bm{e^{(r)}})$\;
  ${\bm{e}^{(r+1)}} \leftarrow {\bm{e}^{(r)}}$\;
  \For{$k \leftarrow 1$ \KwTo $d_1$}{
   Adjust multinomial logistic regression $\hat{\beta}^{k(r)} = \argmax_\beta \ell(\beta;\bm{e^{k(r)}},\bm{x^k})$\;
   ${\bm{e}^{k(r+1)}} \leftarrow$ Mult$(p_{\hat{\gamma}^{(r)}}(\bm{y}|\bm{{e}^{(r+1)}}) p_{\hat{\beta}^{k(r)}}(\bm{e}^{k} | \bm{x}^k))$\;
   ${\bm{e}_{MAP}^{k(r+1)}} \leftarrow \argmax_j p_{\hat{\beta}^{k(r)}}(\bm{E}^{k}=j | \bm{x}^k))$\;
   }
  \For{$k \leftarrow d_1+1$ \KwTo $d_1+d_2$}{
   From the contingency table T$^{k(r)}$ of $\bm{e}^{k(r)}$ against $\bm{x}^k$, calculate the frequencies of each value of $({e}^{k},x^k)$\;
	${\bm{e}^{k(r+1)}} \leftarrow$ Mult$(p_{\hat{\gamma}^{(r)}}(\bm{y}|\bm{{e}^{(r+1)}}) p_{T^{k(r)}}(\bm{e}^{k} | \bm{x}^k))$\;
	${\bm{e}_{MAP}^{k(r+1)}} \leftarrow \argmax_j p_{T^{k(r)}}(\bm{E}^{k}=j | \bm{x}^k))$\;
	}
   $r \leftarrow r+1$\;
 }
 Choose the best logistic regression model from $(p_{\hat{\eta}^{(r)}}(\bm{y} | \bm{e}_{MAP}^{(r)}))_1^{Max}$\;
 \caption{\label{SEM-disc} \textit{glmdisc}-NN: supervised multivariate discretization for logistic regression with an \gls{sem} algorithm.}
\end{algorithm}

\begin{animateinline}[poster=first, controls=all, palindrome, autopause, autoresume, width=\textwidth, height=7cm]{5}
\multiframe{200}{i=1+1}{\input{R_CODE_FIGURES/appendix/animation_disc_SEM/sem_simulated_data/sem_feature_1_iter_\i.tex}}%
\end{animateinline}
Vil59650
\textcolor{red}{à décommenter // ajouter caption}


\section{Factor levels grouping methods}


\section{Interaction discovery methods}


\section{Logistic regression-based trees}


%
% Deuxième chapitre annexe (éventuel)
\chapter{Softwares}

\section{The R Statistical Software}

The vast majority of the code used throughout this manuscript has been done in \textbf{R}. Most experiments can be rerun from the Github repository of the manuscirpt at \url{https://www.github.com/adimajo/manuscrit_these}.

More information about the \textbf{R} Statistical Software, RStudio, and git, which I used extensively during the PhD, can be found respectively in~\cite{}.

\subsection{The glmdisc package} \label{app2:glmdisc}

% Fonctions principales
The glmdisc package can be found on Github at \url{}. It consists in the \textbf{R} implementation of the \textit{glmdisc} algorithm for discretizing continuous attributes, merging factor levels and introducing sparse pairwise interactions proposed in Chapters~\ref{chap5} and~\ref{chap6}.

\paragraph{Quick installation guide}

\paragraph{Main functions}

Once installed, the \textbf{R} help and vignette detail the functioning of the package. Nevertheless, I should mention a few 



\subsection{Miscellaneous}

Apart from the glmdisc package, I produced a package named scoring for the purpose of \textit{Credit Scoring} practitioners, which contains the glmdisc package, the Reject Inference methods discussed thoroughly in Chapter~\ref{chap2} and detailed in Appendix~\ref{app1}, enhances the discretization package containing, among others, the MDLP and $chi^2$ discretization methods to which \textit{glmdisc} is compared in Chapter~\ref{cha5}, and and the model to perform automatic segmentation discussed in Chapter~\ref{chap6}. The scoring package can be found at \url{}.

The figures that were generated by the combined used of \textbf{R} code and the tikzDevice package can be rerun and are located in the R\_CODE\_FIGURES folder of the repository.

\section{The Python programming language}

Some experiments .

% PyPi

\subsection{The glmdisc package}


\paragraph{Quick installation guide}


\paragraph{Main functions}




\subsection{The glmdisc-NN notebooks}

As mentioned in Chapter~\ref{chap4}, the experiments of .

%
% Troisième chapitre annexe (éventuel)
\chapter{Publications}

\section{Poster}

Les travaux de discrétisation, regroupement et introduction d'interactions pour le modèle de régression logistique discutés aux chapitres~\ref{chap4} et~\ref{chap5} ont fait l'objet d'un poster :



\section{Présentations à des conférences avec comité de relecture}

Les travaux concernant la réintégration des refusés discutés au chapitre~\ref{chap2} ont fait l'objet de deux communications orales :


Les travaux de discrétisation, regroupement et introduction d'interactions pour le modèle de régression logistique discutés aux chapitres~\ref{chap4} et~\ref{chap5} ont fait l'objet d'une communication orale :


\section{Articles scientifiques}

Les travaux concernant la réintégration des refusés discutés au chapitre~\ref{chap2} ont fait l'objet d'un article scientifique :

Les travaux de discrétisation, regroupement et introduction d'interactions pour le modèle de régression logistique discutés aux chapitres~\ref{chap4} et~\ref{chap5} ont fait l'objet d'un article scientifique :

%
%%%%%%%%%%%%%%%%%%%%%%%%%%%%%%%%%%%%%%%%%%%%%%%%%%%%%%%%%%%%%%%%%%%%%%%%%%%%%%%
% Début de la partie finale
%%%%%%%%%%%%%%%%%%%%%%%%%%%%%%%%%%%%%%%%%%%%%%%%%%%%%%%%%%%%%%%%%%%%%%%%%%%%%%%
\backmatter
%
%
% (Facultatif) Index :
% \printindex
%
% Table des matières
\tableofcontents
%
% (Facultatif) Production de la 4e de couverture :
\makebackcover
%
\end{document}

