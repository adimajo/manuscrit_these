% missingness mechanisms

\newacronym{mcar}{MCAR}{missing completely at random}

\newacronym{mar}{MAR}{missing at random}

\newacronym{mnar}{MNAR}{missing not at random}


% models

\newacronym{svm}{SVM}{support vector machine}

\newacronym{lr}{LR}{régression logistique}

\newacronym{pdf}{pdf}{densité de probabilité}

\newacronym{sem}{SEM}{Stochastic Expectation Maximization}

\newacronym{cem}{CEM}{Classification Expectation Maximization}

\newacronym{lotus}{LOTUS}{Logistic Tree with Unbiased Selection}

\newacronym{lmt}{LMT}{Logistic Model Trees}

\newacronym{mob}{MOB}{Model-Based Recursive Partitioning}

\newacronym{em}{EM}{Expectation Maximization}

\newacronym{pca}{PCA}{Principal Component Analysis}

\newacronym{mca}{MCA}{Multiple Correspondence  Analysis}

\newacronym{famd}{FAMD}{Factor Analysis of Mixed Data}

\newacronym{pls}{PLS}{Partial Least Squares}

\newacronym{satimix}{Satimix}{Méthode Satimix}

\newacronym{spc}{SPC}{Supervised Principal Components}

\newacronym{mle}{MLE}{Maximum Likelihood Estimation}


% companies

\newacronym{cacf}{CACF}{Crédit Agricole Consumer Finance}


\newglossaryentry{score}
{
	name = score,
	description = {Le score désigne la ``fonction'' qui attribue une note, témoignant de la propension à rembourser le crédit, d'un demandeur en fonction de ses caractéristiques},
	plural = scores
}



\newglossaryentry{creditclassique}
{
	name = crédit classique,
	description = {Les conditions du prêt sont fixées à l’avance, lors de la signature du contrat. Le taux, la durée, et les mensualités du prêt sont fixes. Le coût total du financement est ainsi connu dès le début du prêt},
	plural = crédits classiques
}

\newglossaryentry{creditaffecte}
{
	name = crédit affecté,
	description = {Le crédit affecté est accordé par un établissement de crédit ou une banque. Il est utilisé pour un achat déterminé : un bien mobilier (crédit automobile par exemple) ou une prestation. Il est souvent contracté directement sur le lieu de vente. Généralement, le défaut du crédit entraîne la récupération du bien sous-jacent par un huissier},
	plural = crédits affectés
}

\newglossaryentry{creditrenouvelable}
{
	name = crédit renouvelable,
	description = {Le crédit renouvelable, encore appelé crédit permanent, crédit revolving ou crédit reconstituable, consiste à mettre à la disposition d'un emprunteur une réserve d'argent qu'il pourra utiliser et reconstituer selon son gré. Ce crédit est proposé par un établissement financier ou une enseigne commerciale. Il peut être couplé avec une carte de crédit et peut être couvert par une assurance},
	plural = crédits renouvelables
}

\newglossaryentry{location}
{
	name = location,
	description = {La location est elle-même commercialisée sous deux formes : la location avec option d'achat (L.O.A.), pour laquelle le client peut décider d'acquérir le bien loué en fin d'échéancier pour un montant d'option d'achat fixé à l'avance et la location longue-durée (L.L.D.) pour laquelle c'est le magasin / concessionnaire qui dispose d'une option d'achat},
	plural = locations
}

\newglossaryentry{cut}
{
	name = cut,
	description = {Le cut d'un score est un seuil qui représente la note (ou, de façon équivalente, la probabilité) à partir de laquelle un client est accepté ; en-dessous de celui-ci, le client est jugé trop risqué et il est refusé},
	plural = cuts
}

% math

\newglossaryentry{R}{
        description={nombres réels},
        name={\ensuremath{\mathbb{R}}},
        type=space,
  	   sort={aa},  
}

\newglossaryentry{N}{
        description={entiers naturels},
	   name={\ensuremath{\mathbb{N}}},
        type=space,
  	   sort={ab},  
}

\newglossaryentry{NO}{
        description={entiers naturels de $1$ à $l_j$},
        name={\ensuremath{\mathbb{N}_{l_j}}},
        type=space,
  	   sort={ac},  
}

\newglossaryentry{X}{
        description={variable aléatoire},
        name={\ensuremath{X}},
        type=variable,
  	   sort={ad},  
}

\newglossaryentry{x}{
        description={réalisation de $X$},
        name={\ensuremath{{x}}},
        type=scalaire,
  	   sort={ae},  
}

\newglossaryentry{bX}{
        description={vecteur aléatoire de caractéristiques à $d$ composantes continues ou catégorielles},
        name={\ensuremath{\bm{X}}},
        type=variable,
  	   sort={af},  
}

\newglossaryentry{Xspace}{
        description={espace du vecteur de caractéristiques (on se limite au produit de l'ensemble des réelles -variables continues- et des entiers naturels -variables catégorielles)},
        name={\ensuremath{\mathcal{X}}},
        type=space,
  	   sort={afa},  
}

\newglossaryentry{bx}{
        description={réalisation du vecteur de caractéristiques d'un client},
        name={\ensuremath{\bm{x}}},
        type=scalaire,
  	   sort={ag},  
}

\newglossaryentry{Xj}{
        description={$j^{\text{ème}}$ variable aléatoire de $\bm{X}$},
        name={\ensuremath{X_j}},
        type=variable,
  	   sort={ag},  
}

\newglossaryentry{lj}{
        description={nombre de modalités associés à la variable qualitative $X_j$},
        name={\ensuremath{{{l_j}}}},
        type=scalaire,
  	   sort={aga},  
}

\newglossaryentry{Xnotj}{
        description={vecteur aléatoire $\bm{X}$ privé de sa $j^{\text{ème}}$ composante},
        name={\ensuremath{X_{-\{j\}}}},
        type=variable,
  	   sort={ag},  
}

\newglossaryentry{xij}{
        description={réalisation de $X_j$ pour le $i^{\text{ème}}$ client (en colonne)},
        name={\ensuremath{x_{i,j}}},
        type=scalaire,
  	   sort={ah},  
}

\newglossaryentry{bbx}{
        description={matrice de design de caractéristiques (en colonnes) d'un ensemble de clients (en lignes)},
        name={\ensuremath{\mathbf{x}}},
        type=scalaire,
  	   sort={ai},  
}

\newglossaryentry{bxi}{
        description={Ligne de caractéristiques du client $i$ dans la matrice de design $\mathbf{x}$},
        name={\ensuremath{\bm{x}_i}},
        type=scalaire,
  	   sort={aj},  
}

\newglossaryentry{bbxj}{
        description={Colonne d'observations de la caractéristique $j$ de la matrice de design $\mathbf{x}$},
        name={\ensuremath{\bm{x}_i}},
        type=scalaire,
  	   sort={ak},  
}

\newglossaryentry{Y}{
        description={variable aléatoire binaire dissociant ``bons'' et ``mauvais'' clients},
        name={\ensuremath{Y}},
        type=variable,
  	   sort={al},  
}

\newglossaryentry{y}{
        description={observation du caractère ``bon'' ($1$) ou ``mauvais'' ($0$) d'un client},
        name={\ensuremath{y}},
        type=scalaire,
  	   sort={am},  
}

\newglossaryentry{bby}{
        description={colonne de réponses associées à la matrice de design $\mathbf{x}$},
        name={\ensuremath{\mathbf{y}}},
        type=scalaire,
  	   sort={an},  
}

\newglossaryentry{Z}{
        description={variable aléatoire binaire dissociant clients ``financés'' et ``non financés''},
        name={\ensuremath{{Z}}},
        type=variable,
  	   sort={ao},  
}

\newglossaryentry{z}{
        description={observation du caractère ``financé'' (f) ou ``non financé'' (nf) d'un client},
        name={\ensuremath{z}},
        type=scalaire,
  	   sort={ap},  
}

\newglossaryentry{bbz}{
        description={colonne de décisions d'acceptation / rejet associées à la matrice de design $\mathbf{x}$},
        name={\ensuremath{\mathbf{z}}},
        type=scalaire,
  	   sort={aq},
}

\newglossaryentry{bq}{
        description={fonction de quantification multivariée},
        name={\ensuremath{{\bm{q}}}},
        type=fonction,
  	   sort={aqa},  
}

\newglossaryentry{bQ}{
        description={espace des quantifications multivariées},
        name={\ensuremath{{\bm{Q}}}},
        type=space,
  	   sort={aqa},  
}

\newglossaryentry{bm}{
        description={vecteur du nombre de modalités associés à chaque quantification},
        name={\ensuremath{{\bm{m}}}},
        type=scalaire,
  	   sort={aqaa},  
}

\newglossaryentry{bQm}{
        description={espace des quantifications multivariées à $\bm{m}$ modalités},
        name={\ensuremath{{\mathcal{Q}_{\bm{m}}}}},
        type=space,
  	   sort={aqab},  
}

\newglossaryentry{qj}{
        description={fonction de quantification univariée de la $j^{\text{ème}}$ composante},
        name={\ensuremath{{\bm{q}_j}}},
        type=fonction,
  	   sort={aqb},
}

\newglossaryentry{bQj}{
        description={espace des quantifications univariées de la variable $j$},
        name={\ensuremath{{\mathcal{Q}_{j}}}},
        type=space,
  	   sort={aqba},  
}

\newglossaryentry{mj}{
        description={nombre de modalités associés à la quantification de la $j^{\text{ème}}$ composante},
        name={\ensuremath{{{m_j}}}},
        type=scalaire,
  	   sort={aqbb},  
}

\newglossaryentry{bQjmj}{
        description={espace des quantifications univariées à $m_j$ modalités},
        name={\ensuremath{{\mathcal{Q}_{j,m_j}}}},
        type=space,
  	   sort={aqc},  
}

\newglossaryentry{ehmj}{
        description={$h^{\text{ième}}$ vecteur de base de $\mathbb{R}^{m_j}$: $(0,\dots,0,1,0,\dots,0)$},
        name={\ensuremath{\bm{e}_h^{m_j}}},
        type=scalaire,
  	   sort={aqd},
}

\newglossaryentry{bdelta}{
        description={matrice d'interactions dont l'entrée $\delta_{k,\ell}$ encode l'interaction (1) ou l'absence d'interaction (0) entre les variables $k$ et $\ell$},
        name={\ensuremath{{\bm{\delta}}}},
        type=scalaire,
  	   sort={aqe},  
}

\newglossaryentry{bth}{
        description={vecteur de paramètres, généralement pour la régression logistique, du modèle prédictif de $Y$ conditionnellement aux caractéristiques $\bm{X}$},
        name={\ensuremath{\bm{\theta}}},
        type=parametre,
  	   sort={ar},  
}

\newglossaryentry{bthstar}{
        description={vrai (bon modèle) ou ``meilleur'' (mauvais modèle) vecteur de paramètres au sens de la divergence de Kullback-Leibler},
        name={\ensuremath{\bm{\theta}^\star}},
        type=parametre,
  	   sort={ara},  
}

\newglossaryentry{phi}{
        description={paramètre du mécanisme d'acceptation / rejet ($Z$) des demandeurs de crédit conditionnellement aux données $\bm{X}$ et $Y$},
        name={\ensuremath{\bm{\phi}}},
        type=parametre,
  	   sort={arba},
}

\newglossaryentry{ag}{
        description={paramètre de relaxation du problème de quantification},
        name={\ensuremath{\bm{\alpha}}},
        type=parametre,
  	   sort={arbb},
}

\newglossaryentry{beta}{
        description={paramètre de relaxation du problème de segmentation},
        name={\ensuremath{\bm{\beta}}},
        type=parametre,
  	   sort={arc},
}

\newglossaryentry{KL}{
        description={divergence de Kullback-Leibler},
        name={\ensuremath{\text{KL}}},
        type=fonction,
  	   sort={as},  
}

\newglossaryentry{T}{
        description={ensemble d'observations d'apprentissage},
        name={\ensuremath{\mathcal{T}}},
        type=scalaire,
  	   sort={at},  
}

\newglossaryentry{Tf}{
        description={ensemble d'observations de clients financés},
        name={\ensuremath{\mathcal{T}_{\text{f}}}},
        type=scalaire,
  	   sort={au},  
}

\newglossaryentry{K}{
        description={nombre de segments de clients},
        name={\ensuremath{\text{K}}},
        type=scalaire,
  	   sort={aua},  
}

\newglossaryentry{c}{
        description={observation du segment d'un client},
        name={\ensuremath{c}},
        type=scalaire,
  	   sort={aub},  
}

\newglossaryentry{C}{
        description={variable aléatoire latente du segment auquel appartient un client},
        name={\ensuremath{C}},
        type=variable,
  	   sort={aub},  
}

\newglossaryentry{Tnf}{
        description={ensemble d'observations de clients non financés},
        name={\ensuremath{\mathcal{T}_{\text{nf}}}},
        type=scalaire,
  	   sort={av},  
}

\newglossaryentry{F}{
        description={ensemble des indices de clients financés},
        name={\ensuremath{{\text{F}}}},
        type=space,
  	   sort={av},  
}

\newglossaryentry{NF}{
        description={ensemble des indices de clients non financés},
        name={\ensuremath{{\text{NF}}}},
        type=space,
  	   sort={av},  
}