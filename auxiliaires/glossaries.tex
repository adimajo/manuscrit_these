% models

\newacronym{svm}{SVM}{support vector machine}

\newacronym{lr}{LR}{régression logistique}

\newacronym{pdf}{pdf}{densité de probabilité}

\newacronym{sem}{SEM}{Stochastic Expectation Maximization}

\newacronym{lmt}{LMT}{Logistic Model Trees}

\newacronym{mob}{MOB}{Model-Based Recursive Partitioning}

\newacronym{em}{EM}{Expectation Maximization}

% companies

\newacronym{cacf}{CACF}{Crédit Agricole Consumer Finance}


\newglossaryentry{score}
{
	name = score,
	description = {Le score désigne la ``fonction'' qui attribue une note, témoignant de la propension à rembourser le crédit, d'un demandeur en fonction de ses caractéristiques.},
	plural = scores
}



\newglossaryentry{creditclassique}
{
	name = crédit classique,
	description = {Les conditions du prêt sont fixées à l’avance, lors de la signature du contrat. Le taux, la durée, et les mensualités du prêt sont fixes. Le coût total du financement est ainsi connu dès le début du prêt.},
	plural = crédits classiques
}

\newglossaryentry{creditaffecte}
{
	name = crédit affecté,
	description = {Le crédit affecté est accordé par un établissement de crédit ou une banque. Il est utilisé pour un achat déterminé : un bien mobilier (crédit automobile par exemple) ou une prestation. Il est souvent contracté directement sur le lieu de vente. Généralement, le défaut du crédit entraîne la récupération du bien sous-jacent par un huissier.},
	plural = crédits affectés
}

\newglossaryentry{creditrenouvelable}
{
	name = crédit renouvelable,
	description = {Le crédit renouvelable, encore appelé crédit permanent, crédit revolving ou crédit reconstituable, consiste à mettre à la disposition d'un emprunteur une réserve d'argent qu'il pourra utiliser et reconstituer selon son gré. Ce crédit est proposé par un établissement financier ou une enseigne commerciale. Il peut être couplé avec une carte de crédit et peut être couvert par une assurance.},
	plural = crédits renouvelables
}

\newglossaryentry{location}
{
	name = location,
	description = {La location est elle-même commercialisée sous deux formes : la location avec option d'achat (L.O.A.), pour laquelle le client peut décider d'acquérir le bien loué en fin d'échéancier pour un montant d'option d'achat fixé à l'avance et la location longue-durée (L.L.D.) pour laquelle c'est le magasin / concessionnaire qui dispose d'une option d'achat.},
	plural = locations
}

\newglossaryentry{cut}
{
	name = cut,
	description = {Le cut d'un score est un seuil qui représente la note (ou, de façon équivalente, la probabilité) à partir de laquelle un client est accepté ; en-dessous de celui-ci, le client est jugé trop risqué et il est refusé.},
	plural = cuts
}


% math

\newglossaryentry{R}{
        name={nombres réels},
        description={},
        symbol={\ensuremath{\mathbb{R}}},
        type=symbols
}

\newglossaryentry{N}{
        name={entiers naturels},
        description={},
	   symbol={\ensuremath{\mathbb{N}}},
        type=symbols
}

\newglossaryentry{NO}{
        name={entiers naturels de $1$ à $o_j$},
        description={},
        symbol={\ensuremath{\mathbb{N}_{o_j}}},
        type=symbols
}

\newglossaryentry{X}{
        name={variable aléatoire},
        description={},
        symbol={\ensuremath{X}},
        type=symbols
}

\newglossaryentry{x}{
        name={réalisation de $X$},
        description={},
        symbol={\ensuremath{{x}}},
        type=symbols
}

\newglossaryentry{bX}{
        name={vecteur aléatoire de caractéristiques},
        description={},
        symbol={\ensuremath{\bm{X}}},
        type=symbols
}

\newglossaryentry{Xj}{
        name={$j^{\text{ème}}$ variable aléatoire de $\bm{X}$},
        description={},
        symbol={\ensuremath{X_j}},
        type=symbols
}

\newglossaryentry{xij}{
        name={réalisation de $X_j$ pour le $i^{\text{ème}}$ client},
        description={},
        symbol={\ensuremath{x_{i,j}}},
        type=symbols
}

\newglossaryentry{bx}{
        name={vecteur de caractéristiques d'un client},
        description={},
        symbol={\ensuremath{\bm{x}}},
        type=symbols
}

\newglossaryentry{bbx}{
        name={matrice de design de caractéristiques d'un ensemble de clients},
        description={},
        symbol={\ensuremath{\mathbf{x}}},
        type=symbols
}

\newglossaryentry{bxi}{
        name={Ligne de caractéristiques du client $i$ dans la matrice de design $\mathbf{x}$},
        description={},
        symbol={\ensuremath{\bm{x}_i}},
        type=symbols
}

\newglossaryentry{bbxj}{
        name={Colonne d'observations de la caractéristique $j$ de la matrice de design $\mathbf{x}$},
        description={},
        symbol={\ensuremath{\bm{x}_i}},
        type=symbols
}


\newglossaryentry{Z}{
        name={vecteur aléatoire de caractéristiques discrètes / groupées},
        description={},
        symbol={\ensuremath{{Z}}},
        type=symbols
}

\newglossaryentry{z}{
        name={vecteur de caractéristiques discrètes / groupées d'un client},
        description={},
        symbol={\ensuremath{z}},
        type=symbols
}

\newglossaryentry{Y}{
        name={variable aléatoire binaire dissociant ‘‘bons'' et ‘‘mauvais'' clients},
        description={},
        symbol={\ensuremath{Y}},
        type=symbols
}

\newglossaryentry{y}{
        name={observation du caractère ‘‘bon'' ($0$) ou ‘‘mauvais'' ($1$) d'un client},
        description={},
        symbol={\ensuremath{y}},
        type=symbols
}

\newglossaryentry{bby}{
        name={colonne de réponses associées à la matrice de design $\mathbf{x}$},
        description={},
        symbol={\ensuremath{\mathbf{y}}},
        type=symbols
}

\newglossaryentry{bth}{
        name={vecteur de paramètres},
        description={},
        symbol={\ensuremath{\bm{\theta}}},
        type=symbols
}

\newglossaryentry{KL}{
        name={divergence de Kullback-Leibler},
        description={},
        symbol={\ensuremath{\text{KL}}},
        type=symbols
}
