\chapter{Remerciements}

Aboutissement d'un travail personnel, cette thèse n'en est pas moins une réussite collective et la contribution de nombreuses personnes, injustement absente de la page de couverture de ce manuscrit, doit ici être extensivement mentionnée.

% Collègues
Tout d'abord, je suis persuadé que le principal facteur de succès d'une thèse CIFRE est l'implication de l'entreprise d'accueil, de la conception du sujet à l'usage des fruits du travail de recherche. A ce titre, je remercie Crédit Agricole Consumer Finance de m'avoir permis de réaliser cette thèse dans de très bonnes conditions. En particulier, j'ai eu la chance d'interagir avec des managers réceptifs à la démarche de recherche et qui m'ont fait confiance : un grand merci à Jérôme Beclin et Nicolas Borde. Je me dois également de saluer la probité intellectuelle de Sébastien Beben ; nos riches échanges de début de thèse constituent sans doute le carburant de ce doctorat.

% Labo
Haut-lieu de la recherche publique française, Inria m'a permis, en acceptant d'être le laboratoire d'accueil de cette CIFRE, de compléter ma formation d'ingénieur généraliste centralien en tentant de combler le vide technique ressenti en fin de cursus, ce qui m'avait motivé à poursuivre en thèse. Je vous laisse le soin, chers lecteurs, d'apprécier l'éventuelle réussite de cet objectif initial. Je remercie chaleureusement le centre de Lille et plus particulièrement l'équipe-projet M$\Theta$DAL pour m'avoir permis de (re)connaître la beauté des mathématiques. Contributeurs directs et véritables artisans de ce travail de recherche, mes trois co-directeurs de thèse ont constitué le moteur de cette thèse ; merci à Christophe Biernacki dont j'espère garder la rigueur scientifique ; merci à Philippe Heinrich, pour m'avoir montré qu'un problème bien posé est déjà à moitié résolu ; merci à Vincent Vandewalle, dont les éclairages passionés, à grands coups de feutre virevoltant sur le tableau ou scripts \textsf{R} envoyés au milieu de la nuit, ont pour la plupart donné la vitesse initiale à chaque partie de ce manuscrit.

% Famille
Aussi puissant et bien alimenté qu'il soit, un véhicule est peu de choses sans ses quatre roues. Les quelques mots qui suivront sont bien peu de choses en comparaison de la stabilité 
