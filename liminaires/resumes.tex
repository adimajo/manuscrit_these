% Résumés (de 1700 caractères maximum, espaces compris) dans la
% langue principale (1re occurrence de l'environnement « abstract »)
% et, facultativement, dans la langue secondaire (2e occurrence de
% l'environnement « abstract »)
\begin{abstract}
%Si les travaux de thèse et de rédaction du manuscrit sont indéniablement complexes, l'écriture du résumé en constitue sans doute la partie la plus aisée, tant la question du sujet de ma thèse m'a été posée dans ma famille ou dans l'entreprise d'accueil de cette thèse CIFRE.
%Crédit Agricole Consumer Finance est un établissement bancaire spécialiste du crédit à la consommation ; à ce titre, il cherche à sélectionner, parmi les emprunteurs demandeurs, les particuliers les plus à-mêmes de rembourser cet emprunt. De ce constat est née la discipline du \textit{Credit Scoring} à partir des années 1940


Le ratio risque/récompense désigne en finance la logique selon laquelle un investissement peu risqué ne pourra être que faiblement rentable tandis qu'un investissement risqué a un rendement plus élevé mais est exposé à une perte. Les établissements financiers spécialisés en crédit à la consommation transposent ce principe en deux heuristiques : premièrement, le taux d'intérêt des crédits est adapté en fonction des clients et des produits ; deuxièmement, les clients demandeurs sont sélectionnés selon leur solvabilité. Ce mécanisme d'acceptation/rejet de la clientèle est composé de plusieurs règles de décision dont un score, c'est-à-dire une notation témoignant de la probabilité de défaut d'un nouveau client. La construction de ce score, qu'on désigne généralement par \textit{Credit Scoring}, repose sur des techniques statistiques relativement anciennes et des heuristiques industrielles dont certaines ont été examinées dans cette thèse.

Après une première partie décrivant l'évolution et le contexte industriels actuels ainsi que la littérature académique associée, on s'intéressera dans une deuxième partie à une contribution importante de cette thèse : la ‘‘réintégration des refusés'' ou comment tirer partie des informations collectées sur les clients refusés mais non utilisées. On verra ensuite en troisième partie l'apport de la méthode proposée dans cette thèse pour la discrétisation (resp. regroupement de modalités) des variables quantitatives (resp. qualitatives) constitutives du score ainsi que l'introduction d'interactions sur sa qualité. Enfin, la quatrième partie .

L'ensemble des travaux est illustré par des données réelles de Crédit Agricole Consumer Finance, établissement bancaire spécialiste du crédit à la consommation à l'origine de cette thèse CIFRE.



\end{abstract}



\begin{abstract}
The risk-reward is a well known finance paradigm: the higher the risk of an investment, the higher the expected reward . When it comes to consumer loans, 
\end{abstract}


%
% Production de la page de résumés
\makeabstract

