% Résumés (de 1700 caractères maximum, espaces compris) dans la
% langue principale (1re occurrence de l'environnement « abstract »)
% et, facultativement, dans la langue secondaire (2e occurrence de
% l'environnement « abstract »)
\begin{abstract}
Cette thèse se place dans le cadre des modèles d’apprentissage automatique de classification binaire. Le cas d’application est le scoring de risque de crédit. En particulier, les méthodes proposées ainsi que les approches existantes sont illustrées par des données réelles de Crédit Agricole Consumer Finance, acteur majeur en Europe du crédit à la consommation, à l’origine de cette thèse grâce à un financement CIFRE.

Premièrement, on s’intéresse à la problématique dite de ``réintégration des refusés''. L’objectif est de tirer parti des informations collectées sur les clients refusés, donc par définition sans étiquette connue, quant à leur remboursement de crédit. L’enjeu a été de reformuler cette problématique industrielle classique dans un cadre rigoureux, celui de la modélisation pour données manquantes. Cette approche a permis de donner tout d’abord un nouvel éclairage aux méthodes standards de réintégration, et ensuite de conclure qu’aucune d’entre elles n’était réellement à recommander tant que leur modélisation, lacunaire en l’état, interdisait l’emploi de méthodes de choix de modèles statistiques.

Une autre problématique industrielle classique correspond à la discrétisation des variables continues et le regroupement des modalités de variables catégorielles avant toute étape de modélisation. La motivation sous-jacente correspond à des raisons à la fois pratiques (interprétabilité) et théoriques (performance de prédiction). Pour effectuer ces quantifications, des heuristiques, souvent manuelles et chronophages, sont cependant utilisées. Nous avons alors reformulé cette pratique courante de perte d’information comme un problème de modélisation à variables latentes, revenant ainsi à une sélection de modèle. Par ailleurs, la combinatoire associé à cet espace de modèles nous a conduit à proposer des  stratégies d’exploration, soit basées sur un réseau de neurone avec un gradient stochastique, soit basées sur un algorithme de type EM stochastique.

Comme extension du problème précédent, il est également courant d’introduire des interactions entre variables afin, comme toujours, d’améliorer la performance prédictive des modèles. La pratique classiquement répandue est de nouveau manuelle et chronophage, avec des risques accrus étant donnée la surcouche combinatoire que cela engendre. Nous avons alors proposé un algorithme de Metropolis-Hastings permettant de rechercher les meilleures interactions de façon quasi-automatique tout en garantissant de bonnes performances grâce à ses propriétés de convergence standards.

La dernière problématique abordée vise de nouveau à formaliser une pratique répandue, consistant à définir le système d’acceptation non pas comme un unique score mais plutôt comme un arbre de scores. Chaque branche de l’arbre est alors relatif à un segment de population particulier. Pour lever la sous-optimalité des méthodes classiques utilisées dans les entreprises, nous proposons une approche globale optimisant le système d’acceptation dans son ensemble. Les résultats empiriques qui en découlent sont particulièrement prometteurs, illustrant ainsi la flexibilité d’un mélange de modélisation paramétrique et non paramétrique.

Enfin, nous anticipons sur les futurs verrous qui vont apparaître en Credit Scoring et qui sont pour beaucoup liés la grande dimension (en termes de prédicteurs). En effet, l’industrie financière investit actuellement dans le stockage de données massives et non structurées, dont la prochaine utilisation dans les règles de prédiction devra s’appuyer sur un minimum de garanties théoriques pour espérer atteindre les espoirs de performance prédictive qui ont présidé à cette collecte.

\medskip

\end{abstract}

\begin{abstract}
This manuscript deals with model-based statistical learning in the binary classification setting. As an application, credit scoring is widely examined with a special attention on its specificities. Proposed and existing approaches are illustrated on real data from Crédit Agricole Consumer Finance, a financial institute specialized in consumer loans which financed this PhD through a CIFRE funding.

First, we consider the so-called reject inference problem, which aims at taking advantage of the information collected on rejected credit applicants for which no repayment performance can be observed (\textit{i.e.}\ unlabelled observations). This industrial problem led to a research one by reinterpreting unlabelled observations as an information loss that can be compensated by modelling missing data. This interpretation sheds light on existing reject inference methods and allows to conclude that none of them were to be recommended since they lack proper modelling assumptions that make them suitable for classical statistical model selection tools.

Next, yet another industrial problem, corresponding to the discretization of continuous features or grouping of levels of categorical features before any modelling step, was tackled. This is motivated by practical (interpretability) and theoretical reasons (predictive power). To perform these quantizations, \textit{ad hoc} heuristics are often used, which are empirical and time-consuming for practitioners. They are seen here as a latent variable problem, setting us back to a model selection problem. The high combinatorics of this model space necessitated a new cost-effective and automatic exploration strategy which involves either a particular neural network architecture or a Stochastic-EM algorithm and gives precise statistical guarantees.

Third, as an extension to the preceding problem, interactions of covariates may be introduced in the problem in order to improve the predictive performance. This task, up to now again manually processed by practitioners and highly combinatorial, presents an accrued risk of misselecting a ``good'' model. It is performed here with a Metropolis-Hastings sampling procedure which finds the best interactions in an automatic fashion while ensuring its standard convergence properties, thus good predictive performance is guaranteed.

Finally, contrary to the preceding problems which tackled a particular scorecard, we look at the scoring system as a whole. It generally consists of a tree-like structure composed of many scorecards (each relative to a particular population segment), which is often not optimized but rather imposed by the company's culture and / or history. Again, \textit{ad hoc} industrial procedures are used, which lead to suboptimal performance. We propose some lines of approach to optimize this logistic regression tree which result in good empirical performance and new research directions illustrating the predictive strength and interpretability of a mix of parametric and non-parametric models.

This manuscript is concluded by a discussion on potential scientific obstacles, among which the high dimensionality (in the number of features). The financial industry is indeed investing massively in unstructured data storage, which remains to this day largely unused for \textit{Credit Scoring} applications. Doing so will need statistical guarantees to achieve the additional predictive performance that was hoped for.

\medskip

\end{abstract}

\makeabstract