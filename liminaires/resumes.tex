% Résumés (de 1700 caractères maximum, espaces compris) dans la
% langue principale (1re occurrence de l'environnement « abstract »)
% et, facultativement, dans la langue secondaire (2e occurrence de
% l'environnement « abstract »)
\begin{abstract}
%Si les travaux de thèse et de rédaction du manuscrit sont indéniablement complexes, l'écriture du résumé en constitue sans doute la partie la plus aisée, tant la question du sujet de ma thèse m'a été posée dans ma famille ou dans l'entreprise d'accueil de cette thèse CIFRE.
%Crédit Agricole Consumer Finance est un établissement bancaire spécialiste du crédit à la consommation ; à ce titre, il cherche à sélectionner, parmi les emprunteurs demandeurs, les particuliers les plus à-mêmes de rembourser cet emprunt. De ce constat est née la discipline du \textit{Credit Scoring} à partir des années 1940


Le ratio risque/récompense désigne en finance la logique selon laquelle un investissement peu risqué ne pourra être que faiblement rentable tandis qu'un investissement risqué a un rendement plus élevé mais est exposé à une perte. Les établissements financiers spécialisés en crédit à la consommation transposent ce principe en deux heuristiques : premièrement, le taux d'intérêt des crédits est adapté en fonction des clients et des produits ; deuxièmement, les clients demandeurs sont sélectionnés selon leur solvabilité. Ce mécanisme d'acceptation/rejet de la clientèle est composé de plusieurs règles de décision dont un score, c'est-à-dire une notation liée aux caractéristiques socio-démographiques de clients passés témoignant de la probabilité de défaut d'un nouveau client. La construction de ce score, qu'on désigne généralement par \textit{Credit Scoring}, repose sur des techniques statistiques et des heuristiques industrielles dont certaines ont été examinées dans cette thèse.

Après une première partie décrivant l'évolution et le contexte industriels actuels ainsi que la littérature académique associée à la classification supervisée, on s'intéressera dans une deuxième partie à une contribution importante de cette thèse : la ‘‘réintégration des refusés'' ou comment tirer partie des informations collectées sur les clients refusés mais non utilisées. On verra ensuite en troisième partie l'apport de la méthode proposée dans cette thèse pour la discrétisation (resp. regroupement de modalités) des variables quantitatives (resp. qualitatives) constitutives du score. La ré-interprétation comme un problème à variables manquantes a permis de proposer une nouvelle approche de résolution dont les résultats sont significatifs en termes de performance et de gain de temps pour le statisticien. Outre la discrétisation et le regroupement de modalités, il est courant en \textit{Credit Scoring} d'introduire également des interactions, c'est-à-dire de choisir des produits de variables prédictives plutôt qu'un effet additif, comme par exemple la présence simultanée d'une catégorie de revenus et d'un emploi particulier. Nous ajouterons en quatrième partie une résolution du problème de sélection de ces interactions à l'algorithme développé en troisième partie, rendant le mécanisme de construction des scores quasi-automatique. Nous prendrons ensuite du recul dans la cinquième partie pour constater que le système d'acceptation est rarement constitué d'un seul score mais plutôt d'un arbre de scores, ou d'un mélange d'experts dans la littérature statistique, chacun relatif à un segment de population particulier. Cette structure est posée \textit{a priori} et est donc certainement sous-optimale. Nous proposerons des pistes de réflexion pour optimiser le système d'acceptation dans son entièreté. Enfin, nous poursuivrons en sixième et dernière partie sur les enjeux et défis de la grande dimension (en termes de prédicteurs) dans la pratique du \textit{Credit Scoring}.


%ainsi que l'introduction d'interactions sur sa qualité. Enfin, la quatrième partie .

L'ensemble des travaux est illustré par des données réelles de Crédit Agricole Consumer Finance, établissement bancaire spécialiste du crédit à la consommation à l'origine de cette thèse CIFRE.



\end{abstract}



\begin{abstract}
The risk-reward is a well known finance paradigm: the higher the risk of an investment, the higher the expected reward . When it comes to consumer loans, two heuristics are usually employed: first, the interest rate of the loan depends on the client's risk of defaulting and second, applicants might simply get rejected given their creditworthiness (or lack thereof). This acceptation / rejection mechanism is composed of several business rules, among which the score, \textit{i.e.}\ a numeric value depending upon sociodemographic characteristics of past clients and their repayment behaviour which tends to reflect the propensity of new clients to pay back. The construction of such scores relies on both machine learning and \textit{ad hoc}, industrial recipes, which are partly tackled in this manuscript.

The first part focuses on the industrial context of \textit{Credit Scoring} and its associated academic literature of supervised classification. In the second part, we tackle the so-called ‘‘Reject Inference'' problem, which aims at taking into account information about previously rejected clients (for which no repayment performance was observed). The industrial practice of \textit{Credit Scoring} often involves discretizing (resp. grouping) continuous features (resp. levels of categorical features) before performing logistic regression. To do so, in-house heuristics are usually used, which take a lot of the practitioner's time. The third part reinterpretes this practice as a latent variable problem and proposes a new resolution which results are satisfactory both in performance and in saving the practitioner's time. Moreover, interactions among covariates, \textit{e.g.}\ the simultaneous presense of a category of wages and a type of job, might be introduced in the logistic regression model. In that case as well, part four builds on what was proposed for discretization and grouping of factor levels to automatically search for the best interactions among covariates. In part five, we take a step back and look at the acceptance system as a whole: it is generally composed of many scorecards, in a tree-like structure which is often not optimized over but rather imposed by the company's culture and / or history. We propose some lines of approach to optimize the whole scoring system. In the six and last part of this manuscript, we discuss how curse and blessings of dimensionality (in the number of features) might affect the practice of \textit{Credit Scoring}.

Proposed methods and comparisons with existing approaches are illustrated on real data from Crédit Agricole Consumer Finance, a financial institute specialized in consumer loans which financed this PhD through a CIFRE funding.

\end{abstract}


%
% Production de la page de résumés
\makeabstract

